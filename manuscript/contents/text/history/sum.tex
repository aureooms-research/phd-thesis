\section{3SUM \& \(k\)-SUM}%
\label{sec:history:sum}

The 3SUM problem is defined as follows: given $n$ distinct real numbers, decide
whether any three of them sum to zero.
%
The 3SUM problem also falls in the category of degeneracy testing problems.
This problem asks to decide whether a given set of \(n\) numbers contains a triple
whose sum is zero. We can solve this problem by brute-force in \(O(n^3)\) time,
and in \(O(n^2)\) time with a slightly more clever algorithm.

However toyish 3SUM may look like, it is considered one of several key problems
in P: many geometric problems reduce from it in subquadratic time. Hence, a
conjectured quadratic lower bound on 3SUM implies a conditional lower bound on
all those more practical problems~\cite{GO95}.

Like for GPT, there exist lower bounds for 3SUM in restricted models of
computation: 3SUM cannot be decided in \(o(n^2)\) time if the only way we
inspect the input is by testing for the sign of weighted sums of three
input numbers~\cite{Er99a}.

Even before this lower bound was known, it was conjectured that a quadratic lower
bound would hold in other models of computation like the real-RAM model.

A first stab at the conjecture was made when it was proven that for integer
input numbers, it is possible to beat the conjectured lower bound by a few
logarithmic factors~\cite{BDP08}.
However, it remained open whether such improvements were
possible for real inputs.

Eventually, in a breakthrough paper, Gr\o nlund and Pettie gave a subquadratic
uniform algorithm that shaves a root of a logarithmic factor from quadratic
time~\cite{GP18}.
%
Since then more roots of logarithmic factors have been shaved~\cite{Fr15,GS15}.
%
To this day, it is still conjectured that, for all \(\delta > 0\), 3SUM
requires \(\Omega(n^{2 - \delta})\) time to solve in the real-RAM model.

%
The seminal paper by Gajentaan and Overmars showed the crucial role
of 3SUM in understanding the complexity of several problems in computational
geometry~\cite{GO95}.

A popular conjecture is that no $n^{2-\Omega(1)}$-time real-RAM algorithm
solves 3SUM.
%
\begin{conjecture}[label=conj:3sum,restate=ConjectureSUM]
	There is no \(n^{2-\Omega(1)}\)-time real-RAM algorithm for 3SUM.
\end{conjecture}

%
This conjecture has been used to show conditional
lower bounds for problems in \(\textsc{P}\),
notably in computational geometry with problems
such as
GeomBase, general position testing (GPT)~\cite{GO95}
and
Polygonal Containment~\cite{BH01},
and more recently for string problems such as
Local Alignment~\cite{AVW14}
and
Jumbled Indexing~\cite{ACLL14},
as well as
dynamic versions of graph problems~\cite{Pa10,KPP16,AV14},
triangle enumeration and Set Disjointness~\cite{KPP16}.


The \(k\)-SUM problem is a straightforward generalization of the 3SUM problem:
given \(n\) distinct real numbers, decide whether \(k\) of them sum to zero.
%
The paper of Gr\o nlund and Pettie also discusses the following generalization of
the 3SUM problem: ``For a fixed \(k\), given a set of \(n\) real numbers,
decide whether there exists a \(k\)-subset whose elements sum to zero.''
This problem is called the \(k\)-SUM problem.

Obviously, the 3SUM problem is the \(k\)-SUM problem where \(k=3\).
Moreover, there is a simple reduction from \(k\)-SUM to 2SUM when \(k\) is even
and to 3SUM when \(k\) is odd. Those reductions yield a
\(O(n^{\frac{k}{2}} \log n)\) time real-RAM algorithm for \(k\) even and a
\(O(n^{\frac{k+1}{2}})\) time real-RAM algorithm for \(k\) odd.

In their paper, in addition to the slightly subquadratic uniform algorithm for
3SUM, Gr\o nlund and Pettie give a strongly subquadratic nonuniform
algorithm for 3SUM. The algorithms runs in time \(\tilde{O}(n^{3/2})\), and,
because of the aforementioned reduction, immediately yields an improved
\(\tilde{O}(n^{\frac{k}{2}})\) nonuniform time complexity for \(k\)-SUM when
\(k\) is odd.

As for uniform time complexity we do not know whether this nonuniform
improvement can be transferred to the real-RAM model: we do not know of any
real-RAM \(o(n^{\frac{k+1}{2}})\) time algorithm for \(k\)-SUM when
\(k\) is odd.

The \(k\)-SUM problem reduces to the following point location problem: ``Given
a input point \(q \in \mathbb{R}^n\), locate \(q\) in the arrangement of
\(n \choose k\) hyperplanes of equation \(x_{i_1} + x_{i_2} + \cdots +
x_{i_k} = 0\).'' Applying the best nonuniform algorithms for point location in
arrangements of hyperplanes by Meyer auf der Heide~\cite{M84} and
Meiser~\cite{M93} yields linear decision trees of depth \(n^{O(1)}\) for
\(k\)-SUM, where the constant of proportionality in the big-oh does not depend
on \(k\).


It has been known for long that \(k\)-SUM is $\textsc{W}[1]$-hard and proved
recently to be $\textsc{W}[1]$-complete~\cite{ALW14}.
The \(k\)-SUM problem is also a fixed-parameter version of the subset-sum
problem, a standard \textit{NP}-complete problem.

Similarly to 3SUM, the \(k\)-SUM problem
has proved to be a computational bottleneck in high dimensional convex hull
construction or general
position testing~\cite{Er99b}.

For all those reasons, 3SUM and \(k\)-SUM are considered as key subjects of an
emerging theory of complexity-within-\(\textsc{P}\) (or ``fine-grained''
complexity), on par with other problems such as
all-pairs shortest paths,
orthogonal vectors,
boolean matrix multiplication,
and conjectures such as
the Strong Exponential Time Hypothesis~\cite{MO01,PW10,AVY15,HKNS15,CGIMPS16}.

%Cite that paper that links \(n^2\)-hard problems to P vs NP?


\subsection{Variants}

For the sake of simplicity, we will sometimes consider the following definition of 3SUM\@:
\begin{problem}[3SUM variant]
	Given three sets \(A\), \(B\), and \(C\) of
	\(n\) real numbers, decide
	whether any triple of numbers in \(A \times B \times C\) sums to zero.
\end{problem}
%\begin{problem}[3SUM]
%Given 3 sets $A$, $B$, and $C$, each containing $n$ real numbers, decide
%whether there exist $a \in A$, $b \in B$, and $c \in C$ such that $c=a+b$.
%\end{problem}


A similar variant can be defined for \(k\)-SUM\@:
\begin{problem}[\(k\)-SUM variant]
	Given \(k\) sets \(S_i\) of \(n\) real numbers,
	decide whether there exists
	\((s_1, s_2, \ldots, s_k) \in S_1 \times S_2 \times \cdots \times S_k\)
	such that \(\sum_{i=1}^{k} s_i = 0 \).
\end{problem}


The \(k\)-SUM problem can be further generalized to the
linear degeneracy testing problem (\(k\)-LDT) where we allow
coefficients other than zero and one.
%
\begin{problem}[\(k\)-LDT]
 Given an input vectors \(q\in\mathbb{R}^n\) and
 $\alpha \in \mathbb{Q}^n$ and constant $c \in \mathbb{Q}$
 decide whether there exists a
 $k$-tuple \((i_1, i_2,\ldots ,i_k) \in {[n]}^k\) such that
 \(c + \sum_{j=1}^k \alpha_j q_{i_j} = 0\).
 \end{problem}

%
Our algorithms apply to this more general problem with only minor changes.

\todo{Add reduction found in GP18.}

\todo{Add reduction found in KLM18.}

\todo{Sorting \(X + Y\)?}

\subsection{Point Location}

The Point Location problem is a classic problem in Computational Geometry.

The Point Location problem is a classic problem in Computational Geometry.
The Point Location problem is a classic problem in Computational Geometry.
The Point Location problem is a classic problem in Computational Geometry.
\input{text/definition/point-location}

Because the bounds in \Cref{thm:buck} are attained when \(\mathcal{H}\) is in
general position, there is a lower bound of \(\Omega(d \log m)\) on the depth
of decision trees solving this problem.

For our purpose, it is useful to see the \(k\)-SUM problem as a point location
problem in \(\mathbb{R}^n\) where the coordinates of \(q\) are the input
numbers and where \(\mathcal{H}\) is the set of all \(n \choose k\) hyperplanes
of equation \(\sum_{j=1}^{k} x_{i_j} = 0\).

Sorting can also be seen as a point location problem in \(\mathbb{R}^n\).
In this case \(\mathcal{H}\) is the set of all \(n \choose 2\) hyperplanes of
equation \(x_i = x_j\). The arrangement \(\mathcal{A}(\mathcal{H})\) has
exactly \(n!\) \(n\)-dimensional cell that correspond to the \(n!\)
permutations the input might have.


Because the bounds in \Cref{thm:buck} are attained when \(\mathcal{H}\) is in
general position, there is a lower bound of \(\Omega(d \log m)\) on the depth
of decision trees solving this problem.

For our purpose, it is useful to see the \(k\)-SUM problem as a point location
problem in \(\mathbb{R}^n\) where the coordinates of \(q\) are the input
numbers and where \(\mathcal{H}\) is the set of all \(n \choose k\) hyperplanes
of equation \(\sum_{j=1}^{k} x_{i_j} = 0\).

Sorting can also be seen as a point location problem in \(\mathbb{R}^n\).
In this case \(\mathcal{H}\) is the set of all \(n \choose 2\) hyperplanes of
equation \(x_i = x_j\). The arrangement \(\mathcal{A}(\mathcal{H})\) has
exactly \(n!\) \(n\)-dimensional cell that correspond to the \(n!\)
permutations the input might have.


Because the bounds in \Cref{thm:buck} are attained when \(\mathcal{H}\) is in
general position, there is a lower bound of \(\Omega(d \log m)\) on the depth
of decision trees solving this problem.

For our purpose, it is useful to see the \(k\)-SUM problem as a point location
problem in \(\mathbb{R}^n\) where the coordinates of \(q\) are the input
numbers and where \(\mathcal{H}\) is the set of all \(n \choose k\) hyperplanes
of equation \(\sum_{j=1}^{k} x_{i_j} = 0\).

Sorting can also be seen as a point location problem in \(\mathbb{R}^n\).
In this case \(\mathcal{H}\) is the set of all \(n \choose 2\) hyperplanes of
equation \(x_i = x_j\). The arrangement \(\mathcal{A}(\mathcal{H})\) has
exactly \(n!\) \(n\)-dimensional cell that correspond to the \(n!\)
permutations the input might have.


Because the bounds in \Cref{thm:buck} are attained when \(\mathcal{H}\) is in
general position, there is a lower bound of \(\Omega(d \log m)\) on the depth
of decision trees solving this problem.

For our purpose, it is useful to see the \(k\)-SUM problem as a point location
problem in \(\mathbb{R}^n\) where the coordinates of \(q\) are the input
numbers and where \(\mathcal{H}\) is the set of all \(n \choose k\) hyperplanes
of equation \(\sum_{j=1}^{k} x_{i_j} = 0\).
%
Locating \(q\) in \(\mathcal{A}(\mathcal{H})\)
amounts to deciding in $n$-dimensional space, for each hyperplane
\(H \in \mathcal{H}\), whether \(q\)
lies on, above, or below \(H\). Since \(q\) lies on some \(H\) if and only if
the \(k\)-SUM instance is degenerate, this constitutes a valid reduction.
%
We emphasize that in this reduction the set of hyperplanes depends only on $k$
and $n$ and not on the actual input vector $q$.

Sorting can also be seen as a point location problem in \(\mathbb{R}^n\).
In this case \(\mathcal{H}\) is the set of all \(n \choose 2\) hyperplanes of
equation \(x_i = x_j\). The arrangement \(\mathcal{A}(\mathcal{H})\) has
exactly \(n!\) \(n\)-dimensional cell that correspond to the \(n!\)
permutations the input might have. Identifying the cell containing the input
point reveals its permutation, hence solving the sorting problem.

See~\cite[Section~34.6]{Sn04} for more on point location in high dimension.

\subsection{Information Theoretic Lower Bound}%
\label{sec:history:sum:itlb}

The ITLB for sorting is the logarithm of the number of possible
permutations of the input, that is \(\lceil \log n! \rceil = \Omega(n \log
n)\), for \emph{any} decision tree. This does not hold for element uniqueness:
in an arbitrarily powerful decision tree model, using a single query, we can
find out whether the input contains duplicates. This matches the useless lower
bound of \(\log 2 = 1\).

This observation holds for any decision problem: if the problem amounts to
retrieve a single bit of information, the ``yes/no'' answer, we cannot we derive
any useful lower bound for arbitrary decision tree models.

Restricting our model to only allow linear queries,
we can derive a more useful lower bound for element uniqueness.
\begin{lemma}
	In the linear decision tree model,
	any algorithm solving the
	Element Uniqueness problem
	must
	sort its input (if it has no duplicates).
\end{lemma}
\begin{proof}
	Assume that the input \(q\) has no duplicates.
	%
	Then, looking at \(q\) as a point in \(\mathbb{R}^n\), \(q\) is contained
	in a \(d\)-cell of the arrangement of hyperplanes of equation
	\(x_i = x_j\).
	%
	Sorting the input amounts to identifying this cell.
	%
	Assume we have solved the element uniqueness problem for this instance
	and further assume, by contradiction,
	that we have not identified the cell containing \(q\).
	%
	In the model we consider, each query/answer pair corresponds to a halfspace
	containing \(q\).
	%
	Because we have not identified the \(d\)-cell containing \(q\), it must be
	that points from at least two \(d\)-cells of the arrangement are contained
	in the intersection of the halfspaces.
	%
	Because halfspaces are convex sets, their intersection is convex too, and
	thus connected. In this connected set, a path
	from \(q\) to any point from the other \(d\)-cell must at some point
	intersect one of the hyperplanes \(x_i = x_j\) that separates the two
	cells.
	%
	An adversary can therefore update \(q\) to be this intersection point
	without changing any of the previous answers, even though this new point
	contains duplicates, a contradiction.
\end{proof}
%
Because of the decision tree lower bound on Sorting, we have
%
\begin{corollary}
	Any linear decision tree solving the
	Element Uniqueness problem
	has depth \(\Omega(n \log n)\).
\end{corollary}
%
The same conclusion can be drawn for the 3SUM and \(k\)-SUM problems since
Element Uniqueness reduces to them.
%
\begin{corollary}
	Any linear decision tree solving the
	\(k\)-SUM problem
	has depth \(\Omega(n \log n)\).
\end{corollary}


\subsection{Higher Lower Bounds}

In 1999, Erickson showed that we cannot solve \(k\)-SUM or
\(k\)-LDT in subquadratic time in the \(k\)-linear decision tree
model~\cite{Er99a}:
%
\begin{theorem}[Erickson~\cite{Er99a}]
The depth of a \(3\)-linear decision tree for 3SUM is $\Omega(n^2)$.
\end{theorem}

%
The proof uses an adversary argument which can be explained geometrically. As
we already observed, we can solve \(k\)-LDT problems by modeling them as point
location problems in an arrangement of hyperplanes. Solving one such problem
amounts to determining which cell of the arrangement contains the input point.
The adversary argument of Erickson~\cite{Er99a} is that there exists a cell having
$\Omega(n^{\lceil\frac{k}{2}\rceil})$ boundary facets and in this model point
location in such a cell requires testing each facet.

%The lower bound of Erickson generalizes to \( \Omega (n^{\lceil k/2
%\rceil }) \) for \( k \)-linear decision trees that solve \( k \)-SUM\@.

%Erickson proved a near-matching lower bound in the $k$-linear
%decision tree model~\cite{Er99a}.

In 2005, Ailon and Chazelle slightly extended the range of query sizes for
which a nontrivial lower bound could be established, elaborating on Erickson's
technique~\cite{AC05}.
%
They study \(s\)-linear decision trees to solve the \(k\)-SUM problem when
\(s > k\). In particular, they give an additional proof for the
$\Omega(n^{\ceil{\frac{k}{2}}})$ lower bound of Erickson and
generalize the lower bound for the \(s\)-linear decision tree model when \(s >
k\). Note that the exact lower bound given by Erickson for \(s = k\) is
$\Omega({(nk^{-k})}^{\ceil{\frac{k}{2}}})$ while the one given by
Ailon and Chazelle is $\Omega({(nk^{-3})}^{\ceil{\frac{k}{2}}})$. Their result
improves therefore the lower bound for \(s = k\) when \(k\) is large.
The lower bound they prove for \(s > k\) is the following
%
\begin{theorem}[Ailon and Chazelle~\cite{AC05}]
The depth of a $s$-linear decision tree solving the \(k\)-LDT\ problem is
\begin{displaymath}
{\Omega\mleft(nk^{-3}\mright)}^{
	(1-\epsilon_k)
	\,
	{\left(\frac{2k-s}{2}\right)}
	/
	{\left\lceil\frac{s-k+1}{2}\right\rceil}
},
\end{displaymath}
where \(\epsilon_k > 0\) tends to \(0\) as \(k \to\infty\).
\end{theorem}

%
This lower bound breaks down when
\(k = \Omega(n^{\frac{1}{3}})\) or \(s \ge 2 k\) and the cases where \(k < 6\)
give trivial lower bounds. For example, in the case
of 3SUM with \(s = 4\) we only get an $\Omega(n)$ lower bound.

\subsection{Uniform Algorithms}%
\label{sec:history:sum:uniform-algorithms}

On the real-RAM and word-RAM,
it is easy to solve any \(k\)-SUM instance in time \(O(n^{k/2} \log n)\) for
\(k\) even and \(O(n^{\frac{k+1}{2}})\) for \(k\) odd,
and hence 3SUM in time \(O(n^2)\). Those algorithms inspect the input using
\(k\)-linear queries exclusively and the lower bounds of Erickson and Ailon and
Chazelle in this model of computation essentially match their complexity.

Because fixing two of the numbers $a$ and $b$ in a triple only allows for one
solution to the equation $a + b + x = 0$, an instance of 3SUM has at most
$n^2$ degenerate triples. An instance giving a matching lower bound is for
example the set
of \(n\) integers
$\{\,\frac{1-n}{2},\frac{3-n}{2},\ldots,\frac{n-1}{2}\,\}$ (for odd $n$)
with $\frac{3}{4} n^2 + \frac 14$ degenerate triples.
%
One might be tempted to think that the number of ``solutions'' to the problem
would lower bound the complexity of algorithms for the decision version of the
problem, as it is the case for this problem, and other problems, in restricted
models of computation~\cite{Er96,Er99a}.
%
This intuition is incorrect.

In 2008, Baran, Demaine, and P{\u a}tra{\cb s}cu
gave the first subquadratic algorithms for
3SUM~\cite{BDP08}. They design subquadratic Las Vegas
algorithms for 3SUM on integer and
rational numbers in the circuit RAM, word RAM, external memory, and
cache-oblivious models of computation. The idea of their approach is to exploit
the parallelism of the models, using linear and universal hashing. However,
since their algorithms do not handle real inputs,
this did not settle the question of subquadratic algorithms in full
generality.

In 2014, Gr{\o}nlund and Pettie gave
the first subquadratic algorithms for real-input 3SUM~\cite{GP18}.
Using an old trick due to Fredman~\cite{Fr76},
they prove the existence of a linear decision tree
solving the 3SUM problem using a strongly subquadratic number of linear queries.
The classical quadratic algorithm for 3SUM uses \(3\)-linear queries
while the decision tree of Gr{\o}nlund and Pettie uses \(4\)-linear queries and
requires $O(n^{\frac{3}{2}} \sqrt{\log n})$ of them.
\begin{theorem}[Gr\o nlund and Pettie~\cite{GP18}]
There is a $4$-linear decision tree of depth
$O(n^{3/2} \sqrt{\log n})$ for 3SUM\@.
\end{theorem}


They show how to adapt this decision tree to run in subquadratic time in the
real-RAM model. They design two subquadratic 3SUM
real-RAM algorithms. A deterministic one running in
$O(n^2/{(\log n/\log \log n)}^{\frac{2}{3}})$
time and a randomized one running in
$O(n^2 {(\log \log n)}^2 / \log n)$ time with high probability.

In 2015, Chan and Lewenstein designed strongly subquadratic word-RAM algorithms
for a high-dimensional variant of integer 3SUM with applications to jumbled
indexing~\cite{CL15}. This result generalizes the folk wisdom that 3SUM on small
integers can be solved in nearlinear time using FFT's.

The same year, Freund~\cite{Fr15} and Gold and Sharir~\cite{GS15}
improved on the results of Gr{\o}nlund and Pettie~\cite{GP18}.
Freund~\cite{Fr15} gave a deterministic algorithm for 3SUM running in \(O(
{n^2\log \log n}/{\log n})\) time. Gold and Sharir~\cite{GS15} gave another
deterministic algorithm for 3SUM with the same running time and shaved off the
$\sqrt{\log n}$ factor in the decision tree complexities of 3SUM and \(k\)-SUM
given by Gr{\o}nlund and Pettie.
%
\begin{theorem}[Freund~\cite{Fr15}, Gold~and~Sharir~\cite{GS15}]
There is a $O(n^2 \log \log n / \log n)$-time real-RAM algorithm
for 3SUM\@.%
\footnote{Chan~\cite{Ch18} shows that an additional logarithmic factor can be shaved
by augmenting the real-RAM model with constant time nonstandard operations on
$\Theta(\log n)$ bits words. His improvements extend to 3POL.}
\end{theorem}


\subsection{Nonuniform Algorithms}%
\label{sec:history:sum:nonuniform-algorithms}

The nonuniform algorithm of Gr\o nlund and
Pettie for 3SUM in the \(4\)-linear decision tree model
generalizes to \(k\)-SUM~\cite{GP18}.
In the $(2k-2)$-linear decision tree model,
only $O(n^\frac{k}{2}\sqrt{\log n})$ queries are required for odd values of $k$.
Putting this in perspective with the lower bounds of Erickson and Ailon and
Chazelle, this indicates that increasing the size of the queries, thus making
the model more powerful, does yield
improvements on the depth of the minimal-height decision tree.

It has been well established that there exist nonuniform
polynomial-time algorithms for the subset-sum and knapsack problems, even
though those problems are \textsc{NP}-complete.
%
In 1984,
Meyer auf der Heide was the first to use a point location algorithm to solve
the knapsack problem in the linear decision tree model in polynomial time~\cite{M84}.
He
thereby answered a question raised by Dobkin and Lipton~\cite{DL74,DL78},
Yao~\cite{Y82} and others. However, if one uses this algorithm to locate a
point in an arbitrary arrangement of hyperplanes the running time is increased
by a factor linear in the greatest coefficient in the equations of all
hyperplanes.
%
In 1993,
a second algorithm was described by Meiser~\cite{M93}, and is derived from a
point location data structure for arrangements of hyperplanes using
\emph{bottom vertex triangulation} and \emph{\(\varepsilon\)-nets} (see
\S\ref{sec:arrangements:triangulation} and
\S\ref{sec:divide-and-conquer:epsilon-nets}).
%
The complexity of Meiser's point location algorithm is polynomial in the
dimension, logarithmic in the number of hyperplanes and does not depend on the
coefficients in the equations of the hyperplanes. A useful
complete description of this algorithm is given by Bürgisser et
al.~\cite[Section~3.4]{BCS97}.

When applied to \(k\)-SUM, those algorithms can be cast as the construction of
a \(n\)-linear decision tree, even though \(k\) is constant.
They both yield \(n^{O(1)}\)-time nonuniform algorithms for
\(k\)-SUM where the constant of proportionality does not depend on \(k\), thus
exhibiting the potential superiority of $n$-linear queries.

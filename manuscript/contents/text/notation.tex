\chapternonum{On Notation}

This short chapter hopes to lift any ambiguity in the notation used.

\section*{Big-Oh}

To express the asymptotic behaviour of functions representing resource
complexities (time and space) we use the standard \emph{big-oh} notation
(see for instance~\cite[Chapter~3]{CLRS09}). For brevity, we add a few
tweaks:

\begin{itemize}

\item The notation $\tilde{O}(\cdot)$ ignores polylogarithmic factors. For
	instance, we have
	\(n^3 \log^2 n = \tilde{O}(n^3)\).

\item The symbol $\epsilon$ (not to be mistaken for \(\varepsilon\))
	denotes a positive real number that can be made as small as desired.
	Writing \(T(n) = O(n^{12/7 + \epsilon})\) means that for any fixed
	\(\delta > 0\), \(T(n) = O(n^{12/7 + \delta})\), where the constant of
	proportionality may depend on \(\delta\).
	In particular, \(n^2 \log^2 n = O(n^{2 + \epsilon})\).

\item The notation \(O_{p_1,p_2, \ldots}(f(n))\) means that the constant of
	proportionality depends on the parameters \(p_i\).

\item When we write \(O(f(n_1,n_2, \ldots))\) we assume that one of the variables
	\(n_i\) is the one going towards infinity in the big-oh definition, while the
	others are monotone functions of \(n_i\) (increasing or decreasing
	depending on the context).

\end{itemize}

Because this asymptotic notation takes little care of constant factors,
logarithms are in base two unless otherwise indicated.

\section*{Sets}

We denote by \(\mathbb{R}^d\) the \(d\)-dimensional Euclidean space
and try to be consistent with the set notation used to represent the subsets
of this space.
\begin{itemize}
	\item A point is indicated by a lowercase letter, for instance a vertex \(p\).
	\item A curve is indicated by a greek letter, for instance
		a planar curve \(\gamma\).
	\item Other sets of points is indicated by an uppercase letter, for instance
		a line \(L\),
		an hyperplane \(H\),
		a cell \(C\),
		or a simplex \(S\).
	\item Sets of curves are indicated by an uppercase greek letter, for
		instance
		a set of curves \(\Gamma\).
	\item Other sets of sets of points are indicated by a calligraphic uppercase
		letter, for instance
		a set of hyperplanes \(\mathcal{H}\),
		a net \(\mathcal{N}\), or an arrangement \(\mathcal{A}(\mathcal{H})\).
\end{itemize}

For finite sets,
we sometimes use the short notation \([n] = \{\,1,2,\ldots ,n\,\}\) and
describe a set of cardinality \(n\) as a \(n\)-set.


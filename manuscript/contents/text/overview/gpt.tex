Let us illustrate by giving a first example of a degeneracy testing problem. We
begin with a definition:

\begin{definition}
A set of \(n\) points in \(\mathbb{R}^d\)
is in general position if and only if every (\(d+1\))-subset spans the entire
space. A point set that is not in general position is called \emph{degenerate}.
\end{definition}

The General Position Testing problem (GPT) asks to decide if a given set of
\(n\) point is in general position. We can solve this problem by brute-force in
\(O(n^{d+1})\) time.
We can do it an order of magnitude faster
by constructing the dual arrangement of hyperplanes in
\(O(n^d)\) time.
On the one hand,
improving on this slightly better solution appears to be non-trivial: there
exists a class of algorithms that cannot do better even though they
exploit one of the core structures of the problem, the chirotope axioms.
On the other hand, information theory only gives a decision tree
lower bound of \(\Omega(n \log n)\).
A popular conjecture is that no \(o(n^d)\) time real-RAM
algorithm exists for this problem.

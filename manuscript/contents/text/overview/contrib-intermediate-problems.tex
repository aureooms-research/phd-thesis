In paper B, we generalize Gr\o nlund and Pettie's approach to solve 1D-3POL (or
more simply, 3POL) in subquadratic time. Our approach is essentially the same
in that it reduces 3POL to a sorting phase and a searching phase, the sorting
phase being an instance of 2D-2POL.\footnote{Note that according to the implied
definition, \(d\)D-\(k\)-SUM is equivalent to \(k\)-SUM. Therefore, the 2SUM
instance in the sorting phase of their 3SUM algorithm is an instance of 2D-2SUM
in disguise.} Again, the implementation of the uniform algorithm solves this
instance of 2D-2POL using dominance reporting (a generalization of it).

This result illustrates why a better understanding of the landscape of problems
surrounding GPT helps to identify intermediate problems whose resolution marks
progress towards the question of whether GPT admits subquadratic algorithms.
%
Figure~\ref{fig:landscape} depicts this landscape.

\begin{figure}
  \centering{}
  \includegraphics[width=\linewidth]{figures/landscape}
  \caption{%
	The problems surrounding GPT.%
  }%
  \label{fig:landscape}
\end{figure}

Another model of computation in which no \(o(n^d)\) time algorithm for GPT
is known is the algebraic computation tree model. In this model,
an algorithm is a rooted tree whose internal nodes are either arithmetic operations or
sign tests on real variables, and whose leaves are the result of the computation.
%Arithmetic operation nodes have one children and sign test nodes have two.
%An execution of such an algorithm corresponds to a root-to-leaf path in the
%corresponding tree.

This model is
more generous than the real-RAM model in the sense that all computations that
can be carried out by only knowing the input size incur no cost.
%
Because a computation tree has a fixed size, we need a different tree for each
input size.
%
Therefore,
we say that this model is \emph{nonuniform} since it allows to have a
distinct algorithm for each input size.

This thesis considers both uniform algorithms in the real-RAM and word-RAM
models of computation and nonuniform algorithms in the algebraic computation
tree, algebraic decision tree, and linear decision tree models of computation.
For a given task, the complexity of the nonuniform algorithm is less than
the complexity of the uniform algorithm. While a nonuniform algorithm is rarely
practical, designing those at least means making progress on the question of
whether a sensible computation tree lower bound can be derived.

A perspicacious reader may have frowned at the previous paragraphs wondering
whether this embryo of classification brings more insight than confusion.
``Sure!'', one may say, ``Many problems involve sorting, and therefore,
according to this dubious definition, are generalizations of it.''

However, the author begs to differ.

As a first example, take one of the best known algorithms for 3SUM. This
algorithm reduces an instance of 3SUM to a sorting phase and a searching
phase~\cite{GP18}. The sorting phase consists in answering all questions of the type
\( x_i + x_{i'} \le x_j + x_{j'} \) with some restriction on the indices.
Well, it
turns out that this phase is exactly an instance of
the sorting version of the 2SUM problem where the input numbers are the
differences \(x_i - x_j\).\footnote{This observation generalizes to the \(k\)-SUM
problem: any \(k\)-SUM instance can be reduced to a larger (\(k-1\))-SUM instance
followed by a searching phase.} Moreover, to make the uniform algorithm
practical, they implement the resolution of the 2SUM instance via dominance
reporting.
Note that this sorting problem is similar to the Sorting \(X+Y\) problem where
the answers to most questions are not cared for.

As a second example, we have the GPT problem. For this problem, the fact that
the one-dimensional version is Sorting does not seem to give any insight on how
to apprehend the more general problem, even in two dimensions. In order to make
some progress in that direction, we need to capture more precisely to which
family of problems GPT belongs. For that, we look at the algebra behind the
problem.

GPT asks whether for any choice of \(n \choose d+1\) input points \(p_i\) with
coordinates \((p_{i,1} , p_{i,2} , \ldots, p_{i,d} )\),
the determinant
%
\begin{displaymath}
  \det\begin{pmatrix}
    1 & p_{1,1} & p_{1,2} & \hdots & p_{1,d} \\
    1 & p_{2,1} & p_{2,2} & \hdots & p_{2,d} \\
    \vdots & \vdots & \vdots & \ddots & \vdots \\
    1 & p_{d+1,1} & p_{d+1,2} & \hdots & p_{d+1,d}
  \end{pmatrix}
\end{displaymath}
is zero. This determinant is a degree-\(d\) (\(d^2 + d\))-variate polynomial.
In particular, when \(d=2\), the determinant is a degree-\(2\) \(6\)-variate
polynomial. The GPT testing problem then amounts to testing whether the
coordinates of any combination of the input points yields a root of that
polynomial. We therefore consider the more general \(d\)-dimensional-\(k\)-POL
(\(d\)D-\(k\)-POL) problem:
%
Given a \(dk\)-variate constant degree polynomial \(F\) and a set \(S\) of \(n\) points
in \(\mathbb{R}^d\), decide whether \(F(S^k)\) contains any zeroes.
%
For instance, 2D-3POL generalizes GPT with \(d=2\) where the constant degree
polynomial is the \(3 \times 3\) determinant mentioned above. Moreover, 1D-3POL
generalizes 3SUM where the polynomial is simply the sum function. Equally
interesting is the fact that 2D-2POL generalizes Hopcroft's problem with
\(d=2\) where the polynomial is the dot product.

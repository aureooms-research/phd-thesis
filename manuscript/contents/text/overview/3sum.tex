The 3SUM problem also falls in the category of degeneracy testing problems.
This problem asks to decide whether a given set of \(n\) numbers contains a triple
whose sum is zero. We can solve this problem by brute-force in \(O(n^3)\) time,
and in \(O(n^2)\) time with a slightly more clever algorithm.

However toyish 3SUM may look like, it is considered one of several key problems
in P: many geometric problems reduce from it in subquadratic time. Hence, a
conjectured quadratic lower bound on 3SUM implies a conditional lower bound on
all those more practical problems.

Like for GPT, there exist lower bounds for 3SUM in restricted models of
computation: 3SUM cannot be decided in \(o(n^2)\) time if the only way we
inspect the input is by testing for the sign of weighted sums of three
input numbers.

Even before this lower bound was known, it was conjectured that a quadratic lower
bound would hold in other models of computation like the real-RAM model.

A first stab at the conjecture was made when it was proven that for integer
input numbers, it is possible to beat the conjectured lower bound by a few
logarithmic factors~\cite{BDP08}.
However, it remained open whether such improvements were
possible for real inputs.

Eventually, in a breakthrough paper, Gr\o nlund and Pettie gave a subquadratic
uniform algorithm that shaves a root of a logarithmic factor from quadratic
time~\cite{GP18}.
%
Since then more roots of logarithmic factors have been shaved~\cite{Fr15,GS15}.
%
To this day, it is still conjectured that, for all \(\delta > 0\), 3SUM
requires \(\Omega(n^{2 - \delta})\) time to solve in the real-RAM model.

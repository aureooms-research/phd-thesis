\section{GPT \& 3POL}\label{sec:problem:pol}

In the plane, the General Position Testing problem (GPT) asks whether, given
\(n\) points, three of them are collinear. Because of some
elementary algebra, the following form is equivalent:
%
\begin{problem}[GPT in the plane]
	Given \(n\) points \((x_1,y_1), \ldots, (x_n,y_n) \in \mathbb{R}^2\), decide
	whether there exist \(i < j < k \in [n]\) such that
	\begin{displaymath}
		\det
		\left(
		\begin{matrix}
		1 & x_i & y_i \\
		1 & x_j & y_j \\
		1 & x_k & y_k
		\end{matrix}
		\right)
		= 0.
	\end{displaymath}
\end{problem}


Note that this determinant is a degree-two polynomial in six variables
\begin{displaymath}
	x_i y_j - y_i x_j - x_i y_k + y_i x_k + x_j y_k - y_j x_k.
\end{displaymath}

In Computational Geometry it is better known as the clockwise/counterclockwise
predicate because for given points \(i\), \(j\), and \(k\) the sign of this
expression gives the orientation of the triangle those points define.
%
It is more commonly written as
\begin{displaymath}
	(x_j - x_i)(y_k - y_i) - (y_j - y_i)(x_k - x_i),
\end{displaymath}
with only five subtractions and two multiplications.

The problem generalizes to higher dimension: given \(n\) points in
\(\mathbb{R}^d\), decide whether any \(d+1\) of them lie on a common
hyperplane. Again, using algebra:
%
\begin{problem}[GPT in \(\mathbb{R}^d\)]
	Given \(n\) points \(p_i = (p_{i,1},p_{i,2}, \ldots, p_{i,d}) \in \mathbb{R}^d\), decide
	whether there exist \(i_1 < i_2 < \cdots < i_{d+1} \in [n]\) such that
	\begin{displaymath}
		\det
		\left(
		\begin{matrix}
			1 & p_{1,1} & p_{1,2} & \hdots & p_{1,d} \\
			1 & p_{2,1} & p_{2,2} & \hdots & p_{2,d} \\
			\vdots & \vdots & \vdots & \ddots & \vdots \\
			1 & p_{d+1,1} & p_{d+1,2} & \hdots & p_{d+1,d}
		\end{matrix}
		\right)
		= 0.
	\end{displaymath}
\end{problem}


Again, this determinant is a degree \(d\) polynomial in
\(d^2 + d\) variables.

Later in this document we mention GPT without specifying the dimension.
The problem is then assumed to be in the plane.
When we consider a different parameterization we explicitly mention it.

\subsection{Variants}

Similarly to the 3SUM, \(k\)-SUM, and \(k\)-LDT problems, we can study a
variant of GPT where the tested triples come from different sets.
%
\begin{problem}[GPT variant in the plane]
	Given three sets \(A\), \(B\), and \(C\) of
	\(n\) points in \(\mathbb{R}^2\), decide
	whether any triple of points in \(A \times B \times C\) is collinear.
\end{problem}


This variant is useful for a particular reduction from 3SUM.
A similar variant can be defined for the \(d\)-dimensional problem. However, we
do not study it here.

\subsection{Reductions}

% TODO Reduction from 3SUM to GPT (folklore)

Given an instance of 3SUM \(\langle A,B,C \rangle\), we can reduce it to the
following instance of GPT

% DONE Reduction from \(k\)-SUM to GPT

When \(d=1\), GPT is exactly the element uniqueness problem.
%
\begin{problem}[GPT on the real line]
	Given \(n\) points \(q_1, \ldots, q_n \in \mathbb{R}\), decide
	whether there exist \(i < j \in [n]\) such that
	\begin{displaymath}
		\det
		\left(
		\begin{matrix}
		1 & q_i \\
		1 & q_j
		\end{matrix}
		\right)
		= q_j - q_i
		= 0.
	\end{displaymath}
\end{problem}


One seemingly stupid way to solve this problem is to
consider the Vandermonde matrix
%
\begin{displaymath}
V(x) = \begin{pmatrix}
    1       & x_1     & x_1^2     & \dots  & x_1^{n-1}     \\
    1       & x_2     & x_2^2     & \dots  & x_2^{n-1}     \\
    \vdots  & \vdots  & \vdots    & \ddots & \vdots        \\
    1       & x_n & x_n^2 & \dots  & x_n^{n-1}
\end{pmatrix},
\end{displaymath}
%
and to compute its determinant
%
\begin{displaymath}
|V(q)| = \det\begin{pmatrix}
    1       & q_1     & q_1^2     & \dots  & q_1^{n-1}     \\
    1       & q_2     & q_2^2     & \dots  & q_2^{n-1}     \\
    \vdots  & \vdots  & \vdots    & \ddots & \vdots        \\
    1       & q_n & q_n^2 & \dots  & q_n^{n-1}
\end{pmatrix}
=
\prod_{1 \leq i < j \leq n} (q_j - q_i).
\end{displaymath}

This can be done in \(O(n \log n)\) ring operations by defining
%
\(p(x) = \prod_{i=0}^{n-1} (x-q_i)\)
%
and computing
%
\begin{align*}
{\det(V(q))}^2 &= \mathrm{disc}_x(p(x)) \\
&= {(-1)}^{\binom{n}{2}} \mathrm{res}_x(p(x), \mathrm{diff}_x(p(x))) \\
&= {(-1)}^{\binom{n}{2}} \prod_{i=0}^{n-1} \mathrm{diff}_x(p(x))(q_i),
\end{align*}
%
using fast polynomial interpolation to first find the coefficients of
\(p(x)\) and then using fast polynomial evaluation to evaluate
\(\mathrm{diff}_x(p(x))\) at all \(q_i\)~\cite{St73,Ku73a,Ku73b,ASU75}.

Consider points on the moment curve $m_d(t) = (t, t^2, \ldots, t^d)$. Given a
GPT instance consisting of points $m_d(q_i)$ on this curve the determinant
in Problem~\ref{problem:gpt-d} becomes
%
\begin{displaymath}
\det\begin{pmatrix}
1      & q_1^1 & q_1^2 & \cdots & q_1^d \\
1      & q_2^1 & q_2^2 & \cdots & q_2^d \\
\vdots & \vdots  & \vdots  & \ddots & \vdots  \\
1      & q_{d+1}^1 & q_{d+1}^2 & \cdots & q_{d+1}^d
\end{pmatrix}
=
|V(q)|.
\end{displaymath}

Consider points on the \emph{weird} moment curve $\omega_d(t) = (t, t^2, \ldots,
t^{d-1}, t^{d+1})$~\cite{Er99b}.
If our input only consists of points $\omega_d(q_i)$ on this curve then the
determinant becomes
%
\begin{displaymath}
\det\begin{pmatrix}
1      & q_0^1 & q_0^2 & \cdots & q_0^{d-1} & q_0^{d+1} \\
1      & q_1^1 & q_1^2 & \cdots & q_1^{d-1} & q_1^{d+1} \\
\vdots & \vdots  & \vdots  & \ddots & \vdots & \vdots  \\
1      & q_d^1 & q_d^2 & \cdots & q_d^{d-1} & q_d^{d+1}
\end{pmatrix} = |V(q)| \cdot \sum_{i=1}^{d+1} q_i,
\end{displaymath}
%
which is the predicate we want to test for $(d+1)$-SUM (modulo the $|V(q)|$
factor which is nonzero for distinct input numbers).

In the plane, this reduction from 3SUM to the cubic curve \(\omega_2(t) = (t,
t^3)\) is also folklore.
%Note that the weird moment curve in two dimensions is precisely the cubic curve
%from the reduction from 3SUM to GPT.

\subsection{Lower Bounds}

Lower bounds (Goodman Pollack Seidel? Ailon)

Ben-or applies Milnor Thom

\subsection{Algorithms}

Naive algorithm

Duality and constructing the dual arrangement

No subquadratic algorithm known, uniform or nonuniform.

\subsection{The Intermediate Problem}

We consider an algebraic generalization of the 3SUM problem: we replace the sum
function by a constant-degree polynomial in three variables $F \in
\mathbb{R}[x,y,z]$ and ask to determine whether there exists a
\emph{degenerate} triple $(a,b,c)$ of input numbers such that $F(a,b,c)=0$. We
call this new problem the \emph{3POL problem}.

\begin{problem}[%
	name=3POL,%
	label=problem:3pol,%
	restate=ProblemPOLImplicit%
]
Let $F \in \mathbb{R}[x,y,z]$ be a trivariate polynomial of constant degree,
given three sets $A$, $B$, and $C$, each containing $n$ real numbers, decide
whether there exist $a \in A$, $b \in B$, and $c \in C$ such that
$F(a,b,c)=0$.
\end{problem}


\subsection{Combinatorics}

In a series of results spanning fifteen years,
Elekes and Rónyai~\cite{ER00},
Elekes and Szabó~\cite{ES12},
Raz, Sharir and Solymosi~\cite{RSS14}, and
Raz, Sharir and de Zeeuw~\cite{RSZ15}
give upper bounds on the number of degenerate triples for the 3POL problem.
%
For the particular case $F(x,y,z) = f(x,y) - z$ where $f \in \mathbb{R}[x,y]$
is a constant-degree bivariate polynomial, Elekes and Rónyai~\cite{ER00} show
that the number of degenerate triples is $o(n^2)$ unless $f$ is
\emph{special}. Special for $f$ means that $f$ has one of the two special forms
\begin{displaymath}
f(u,v)=h(\varphi(u)+\psi(v))
\qquad
\text{or}
\qquad
f(u,v)=h(\varphi(u)\cdot\psi(v)),
\end{displaymath}
where $h,\varphi,\psi$ are univariate polynomials of constant degree.
It must be noted that the 3SUM problem falls in the special category since, in
that case, \( f \) is the sum function.
%
Elekes and Szabó~\cite{ES12} later generalized this result to a broader range
of functions $F$ using a wider definition of specialness.
%
Raz, Sharir and Solymosi~\cite{RSS14} and Raz, Sharir and de Zeeuw~\cite{RSZ15}
improved both bounds to $O(n^{11/6})$.
%
They translated the problem into an incidence problem between points and
constant-degree algebraic curves. Then, they showed that unless $f$ (or $F$) is
special, these curves have low multiplicities. Finally, they applied a theorem
due to Pach and Sharir~\cite{PS98} bounding the number of incidences between
the points and the curves.
%
\begin{theorem}[Raz, Sharir and de Zeeuw~\cite{RSZ15}]
	Let $A$, $B$, $C$ be $n$-sets of real numbers and $F \in \mathbb{R}[x,y,z]$
	be a polynomial of constant degree, then
	\begin{displaymath}
		%| \{\, (a,b,c) \in ( A \times B \times C ) \st F(a,b,c) = 0 \,\} | = O(n^{11/6}),
		| Z(F) \cap ( A \times B \times C ) | = O(n^{11/6}),
	\end{displaymath}
	unless $F$ has some group related form.\footnote{Because our results do not
	depend on the meaning of \emph{group related form}, we do not bother
	defining it here. We refer the reader to Raz, Sharir and de Zeeuw~\cite{RSZ15}
	for the exact definition.}
\end{theorem}


Raz, Sharir and de Zeeuw~\cite{RSZ15} also look at the number of degenerate
triples for the General Position Testing problem when the input is restricted
to points lying on a constant number of constant-degree algebraic curves.
%
\begin{theorem}[Raz, Sharir and de Zeeuw~\cite{RSZ15}]\label{thm:rsz15:col}
Let $C_1, C_2, C_3$ be three (not necessarily distinct) irreducible algebraic curves
of degree at most $d$ in $\mathbb{C}^2$, and let $S_1 \subset C_1, S_2 \subset C_2, S_3
\subset C_3$ be finite subsets. Then the
number of proper collinear triples in $S_1 \times S_2 \times S_3$ is
\begin{displaymath}
	O_d( |S_1|^{1/2} |S_2|^{2/3} |S_3|^{2/3} + |S_1|^{1/2} (|S_1|^{1/2} + |S_2| +
|S_3| ) ),
\end{displaymath}
unless $C_1 \cup C_2 \cup C_3$ is a line or a cubic curve.
\end{theorem}


Nassajian Mojarrad, Pham, Valculescu and de Zeeuw~\cite{MPVd16} and
Raz, Sharir and de Zeeuw~\cite{RSZ16} proved bounds for versions of the
problem where $F$ is a $4$-variate polynomial.


\subsection{Encodings}

Pseudoline arrangements

Encodings

Canonical Labelings

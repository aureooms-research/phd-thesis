\begin{definition}[%
	name={Chirotope of a Point Set in \(\mathbb{R}^d\)},
	label=definition:chirotope%
]
	Given a set of
	\(n\) points \(p_i = (p_{i,1},p_{i,2}, \ldots, p_{i,d}) \in \mathbb{R}^d\),
	its (rank-(\(d+1\))) chirotope is the function \(\chi \colon\,
	{[n]}^{d+1} \to \{\, - , 0 , +\,\}\) such that
	\begin{displaymath}
		\chi(i_1, i_2, \ldots, i_{d+1}) =
		\det
		\left(
		\begin{matrix}
			1 & p_{i_1,1} & p_{i_1,2} & \hdots & p_{i_1,d} \\
			1 & p_{i_2,1} & p_{i_2,2} & \hdots & p_{i_2,d} \\
			\vdots & \vdots & \vdots & \ddots & \vdots \\
			1 & p_{i_{d+1},1} & p_{i_{d+1},2} & \hdots & p_{i_{d+1},d}
		\end{matrix}
		\right),
	\end{displaymath}
	up to isomorphism.
\end{definition}
%
\begin{definition}[name={Realizable Chirotope},label={def:realizable-chirotope}]
When considering an arbitrary function \(f \colon\, {[n]}^{d+1} \to \{\, -, 0,
+\,\}\) we say that \(f\) is a realizable chirotope if there is a
\(n\)-set \(P \subset \mathbb{R}^2\)
such that its chirotope is \(f\).
\end{definition}

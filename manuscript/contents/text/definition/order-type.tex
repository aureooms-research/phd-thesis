\begin{definition}[name={Order Type of a Point Set},label={def:order-type}]
Given a set of \(n\) labeled points \(P = \{\, p_1, p_2, \ldots, p_n\,\}\), we
define the \emph{order type} of \(P\) to be the function
\(\chi \colon\, {[n]}^3 \to \{\, -, 0, +\,\}
\colon\, (a,b,c) \mapsto \nabla(p_a, p_b, p_c)\)
that maps each triple of point labels to the orientation of the corresponding
points, up to isomorphism.
\end{definition}
%
\begin{definition}[name={Realizable Order Type},label={def:realizable-order-type}]
When considering an arbitrary function \(f \colon\, {[n]}^3 \to \{\, -, 0,
+\,\}\) we say that \(f\) is a realizable order type if there is a point set
\(P = \{\, p_1, p_2, \ldots, p_n\,\}\)
such that
\(f(a,b,c) = \nabla(p_a, p_b, p_c)\).
that realizes it.
\end{definition}

\section{Data Structures}

\subsection{Succinct Data Structures}

\subsection{Construction}

\subsection{Queries}

\subsection{Encodings}

\subsection{Connection to Decision Trees}

Give example with k-SUM, CIO, ES and KLM.

Those yield succinct encoding from which the signs can be recovered.

This is not necessarily true with algebraic decision trees because the answer
set is not necessarily connected. With linear decision trees it is because the
answer set is convex, and thus connected.

Given a realizable order type, instead of encoding the answers to all possible
of its order type queries, consider the problem of deciding whether one of
those queries answers \(0\).
This problem is the general position testing problem (GPT) and corresponds to
deciding whether an input set of points in the plane contains a collinear
triple.

It is an open question whether solving this problem can be done in subquadratic
time. In particular, no subquadratic constant-degree algebraic decision tree is
known for that problem.

A constant-degree algebraic decision tree is a rooted tree whose internal nodes are
labeled with inequalities of the form \(f(\text{input}) \leq 0\), where
\(f\) is a constant-degree polynomial,
and whose leaves are labeled with \emph{yes} or \emph{no}.

Observe the following: Assume we are given an instance of some real-input
decision problem. Given a (computable???) decision tree of depth \(D\) for this problem for which
each input query and answer can be encoded using at most \(Q\) bits, we can
encode the instance using at most \(DQ\) bits by encoding the path traversed when
executing the decision tree on this instance. The general position testing
problem (GPT) asks if an input set of \(n\) points in the plane contains a
collinear triple. It is an example of a real-input decision problem for which
no subquadratic real-RAM algorithm is known even though the best known lower
bound is \(\Omega(n \log n)\). Since shallow decision trees yield short encodings,
we see the design of a subquadratic encoding for realizable order types as a
stepping stone towards nonuniform and uniform subquadratic algorithms for
GPT\@.

Unfortunately, even though our encodings achieve subquadratic space for
realizable order types, they cannot be used to test for isomorphism in
subquadratic time. This is partly because the preprocessing time to construct
the encoding is already quadratic.
%
However, observe that the preprocessing time we achieve in this contribution
matches the best known upper bound for GPT in the algebraic decision tree
model:
\begin{theorem}
  If there is an encoding with construction cost \(C(N)\) for
  realizable order types in the algebraic decision tree model, then
  there is a nonuniform algorithm for general position testing that runs
  in time \(C(N)\) in the algebraic decision tree model.
\end{theorem}
%
\begin{proof}
  Construct the encoding. Then at zero cost in the nonuniform
  model, run all \(O(n^3)\) queries on the encoding.
\end{proof}
%
Goodman and Pollack saw GPT as a multidimensional generalization of
sorting~\cite{GP83}.
%
We again stress the need for a better understanding of this fundamental
problem.

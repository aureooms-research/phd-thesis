\section{Data Structures}% \label{sec:models-of-computation:data-structures}

Data Structures are ubiquitous in algorithmics. Given some data, we want to
store it in computer memory in a way that 1) allows us to efficiently access
the information we want to extract from this data and 2) does not require too
much storage space. The term \emph{structure} is by opposition to a randomly
ordered stream of data.

Usually, a third important feature of those structures is that they can be also
updated efficiently if the underlying data changes. We do not study or use that
aspect in this thesis, instead we only require that those data structures can
be constructed from scratch in an efficient way. Data
structures allowing efficient updates are usually named \emph{dynamic}
while the ones we study are \emph{static} data structures.

Those intuitive concepts are summarized as follows:

\paragraph{Preprocessing Time} Given some data, the time it takes to construct
the corresponding data structure in the given computation model.

\paragraph{Space} The amount of space the data structure consumes in the given
computation model. Can be expressed in bits, words, memory cells, \dots

\paragraph{Query Time} Given a data structure, the time consumed by a
single query in the given computation model.
\\

In \S\ref{sec:models-of-computation:data-structures:encodings} we give a precise
definition of a particular type of data structure we study: encodings.

\subsection{Encodings}%
\label{sec:models-of-computation:data-structures:encodings}

A data structure is called \emph{succinct} if its space usage is close to the ITLB.
%
This concept was first introduced by Jacobson in his PhD thesis~\cite{Ja88}.
%
Since then, it has been extensively studied.
%
Raman et al. studied the dynamic implementation of such data
structures and gave the first rank-select succinct
data structures~\cite{RRS01,RRS07}.
%
%P{\u a}tra{\cb s}cu et al. studied the succinct encoding of arrays of
%trits~\cite{Pa08,DPT10}.
%
Those data structures are even efficient in practice as show by Vigna with
their implementations~\cite{Vi08}.

In this thesis, we also try to get good bounds on the space used by the data
structures we design. However, in most cases we are still very far from the
ITLB. We make all our data structure design problems fit in a single framework
dubbed \emph{instance encoding} so that their space and query time
requirements can be easily compared.
%
\begin{definition}[label=def:encoding,restate=DefinitionEncoding]
  For fixed \(k\) and given a function \(f : {[n]}^k \to [O(1)]\), we define
  a \((S(n),Q(n))\)-encoding of \(f\) to be a string of \(S(n)\) bits such
  that, given this string and any \(i \in {[n]}^k\), we can compute \(f(i)\)
  in \(Q(n)\) time in the word-RAM model with word size \(w \geq \log
  n\).
\end{definition}


%In Papers~\ref{paper:order-type-encoding}~and~\ref{paper:3sum-encoding} we
%design such encodings for order types of point configurations and 3SUM types of
%lists of numbers. In the case of abstract order type we manage to
%obtain a \emph{compact} data structure, that is, a \(O(\text{ITLB})\)-bits encoding
%that allows efficient queries.

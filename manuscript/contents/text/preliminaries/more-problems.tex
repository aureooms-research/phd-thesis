\part{Problems}

%\chapter{Move}

%\section{Sorting}


%\section{Sorting \(X+Y\)}

%\section{SUBSET-SUM}

%\section{Hopcroft's Problem}

%\begin{problem}[Hopcroft's problem]
	%Given a set of $n$ points and $m$ lines in $\mathbb{R}^2$,
	%does any point lie on any line?
%\end{problem}
%There are combinatorial upper bounds~\cite{ST83} on the number of point-line
%incidences an instance can have and there are algorithmic lower
%bounds~\cite{Er96} essentially matching the complexity of Matou\v{s}ek's algorithm.

%\paragraph{Szemeredi-Trotter theorem}
%Szemeredi~and~Trotter~\cite{ST83} give an upper bound on the number of
%incidences between points and lines in $\mathbb{R}^2$.
%\begin{theorem}[Szemeredi and Trotter~\cite{ST83}]
	The number of incidences between $n$ points and $m$ lines in the plane, is
	$O(m^{2/3}n^{2/3}+n+m)$.
\end{theorem}


%\begin{problem}[Hopcroft's problem (any dimension)]
	%Given a set of $n$ points and $m$ hyperplanes in $\mathbb{R}^d$,
	%is any point contained in any hyperplane?
%\end{problem}

%Combinatorial upper bounds~\cite{AA92,CEGSW90}.
%Algorithmic lower bounds~\cite{BK03}.

%\section{Dominance Reporting}

%\section{General Position Testing}

%\section{Point Location}

%\section{3POL}

%\section{Polynomial Dominance Reporting}

\chapter{3SUM}

\chapter{\(k\)-SUM}
 and Linear Degeneracy Testing

\chapter{GPT}


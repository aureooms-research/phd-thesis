\section{\iftitlecase%
Sublogarithmic Query Complexity\else%
Sublogarithmic query complexity\fi}\label{sec:query-time}
%\subsection{\iftitlecase%
%Pushing Query Time below Logarithmic\else%
%Pushing query time below logarithmic\fi}\label{sec:query-time}

We further refine the encoding introduced in the previous section so as to
reduce the query time by a \( \log\log n \) factor. We do so using
specificities of the word-RAM model that allow us to preprocess computations
on inputs of small but superconstant size.
%
This refinement is applicable to both abstract and realizable order types,
and leads to an improvement of our main theorems for the two-dimensional case:
%
\begin{contribution}[label=thm:abstract-loglog,restate=TheoremGPTAbstractLogLog]
  All abstract order types have an encoding
  using \(O(n^2)\) bits of space
  and allowing for queries in \(O(\frac{\log n}{\log \log n})\) time.
  %(\(O(n^2)\), \(O(\frac{\log n}{\log \log n})\))-encoding.
  %\(O(n^2)\)-bits
  %\(O(\frac{\log{n}}{\log{\log{n}}}))\)-querytime
  %encoding.
\end{contribution}

\begin{contribution}[
  name={
    \(o(n^2)\)-bits
    \(o(\log n)\)-querytime
    encodings for realizable OT
  },%
  label=thm:realizable-loglog,%
  restate=TheoremGPTRealizableLogLog%
]
  All realizable order types have
  a \(O(\frac{n^2 \log^\epsilon n}{\log n})\)-bits encoding
  allowing for queries in \(O(\frac{\log n}{\log \log n})\) time.
  %All realizable order types have a
  %\(O(\frac{n^2\log^\epsilon n}{\log n})\)-bits
  %\(O(\frac{\log{n}}{\log{\log{n}}}))\)-querytime encoding.
\end{contribution}

\begin{theorem}\label{thm:preprocessing-loglog}
  In the real-RAM model and the constant-degree algebraic decision tree model,
  given \(n\) real-coordinate input points in \(\mathbb{R}^2\) we can compute
  the encoding of their order type as in
  Theorems~\ref{thm:abstract-loglog}~and~\ref{thm:realizable-loglog} in
  \(O(n^2)\) time.
\end{theorem}


\paragraph{Idea}

A natural idea is to pick \(r\) to be small but superconstant to reduce the
number of levels of the hierarchy and thus the query time. As already pointed
out, this has the drawback of increasing the time complexity of the intersection
location primitive from constant to \(\Theta(r^2)\).
Since this primitive is used at each level of the hierarchy, the
\(\Theta(\log r)\) factor saved by having less levels is lost.

To get past this difficulty,
the trick is to encode approximations of the traces
\(\textsc{Tr}(\mathcal{C}, p')\) to still allow constant intersection location.
We call those approximations \emph{signatures} and denote them by
\(\textsc{Sig}(\mathcal{C}, p')\).
%
We define those signatures so that they approximately encode the cyclic
permutation of the intersections around each subcell. By carefully choosing some
parameters, we are able to fit two of those signatures in a single word of
memory. We can then precompute the output to all
possible inputs for the intersection location primitive and put them in a small
table.

Because of this size reduction, distinct pseudolines could
be mapped to identical signatures.
%
Those ambiguous situations can be deterministically handled using
an additional lookup table.
%
Because those situations rarely arise, this table also is small.

Once we have located the intersection \(p_a' \cap p_b'\), we still need to deal
with \(p_c'\). We change the layout of a trace to locate the subcell containing
\(p_a' \cap p_b'\) with respect to \(p_c'\) in constant time.
%
In case \(p_c'\) properly intersects the subcell, we need to recurse on the
subcell. To do that, we need to map the pseudoline indices \(a\), \(b\), and
\(c\) to the indices those pseudolines have in that subcell. This is
done with one indirection:
%
For each of the three pseudolines we
identify the index of the subcell in the list of subcells intersected by that
pseudoline.
This is implemented as a rank operation on a list of \(O(r^2)\) bits.
%
Given that index, we can find the intersection indices of that pseudoline in
the cyclic permutation of the subcell in constant time.

In what follows, we describe five new structures:
the signatures,
the intersection oracle,
the disambiguation table,
the augmented traces,
and the subcell mapper.
%
The first reduces the bitsize of the original traces so that the second can
be implemented in constant time. The third handles bad cases that arise because
of this size reduction. The fourth defines a new layout for the traces that
includes the signature. The fifth allows to implement the parent-to-subcell
pseudoline index mapping in constant time using this new layout.
%
\paragraph*{\iftitlecase%
Signatures\else%
Signatures\fi}
Fix a small constant \(\alpha\) and define \(r = \Theta(\log^{\alpha} n)\).%
\footnote{%
  The exact bound for how small \(\alpha\)
  must be depends on hidden constant factors in the Zone Theorem and in the definition
  of the word size. In particular, we must have \(\alpha \leq \frac 12\) when
  \(w = \Theta(\log n)\).%
}
%
We encode a \(\ell\)-levels hierarchical cutting of parameter \(r\).
%
Note that we can construct a hierarchical cutting with
superconstant \(r\) by constructing a hierarchical cutting with some
appropriate constant parameter \(r'\), and then skip levels that we do not
need.
%
As in Section~\ref{sec:lines-and-pseudolines}, we store a combinatorial representation of
this hierarchical cutting. We make some tweaks to this representation.

We augment the traces of Section~\ref{sec:lines-and-pseudolines} with a \emph{signature}.
The trace \(\textsc{Tr}(\mathcal{C}, p')\) of a cell-pseudoline pair
is composed of two parts:
The incidence bits
that tell us for each subcell of the cell whether the pseudoline lies above, lies below, intersects
or contains it, and the cyclic permutation bits used to locate the intersection
of two pseudolines inside the cell.
The first part uses \(\Theta(r^2)\) bits.
The second part uses \(\Theta(r \log{\frac{N}{r^{i+1}}})\) bits for a cell of
level \(i\).

To construct the signature \(\textsc{Sig}(\mathcal{C}, p')\),
we keep the \(\Theta(r^2) = \Theta(\log^{2\alpha} n)\) incidence bits because
they fit in sublogarithmic space for sufficiently small \(\alpha\).
The second part would use superlogarithmic space if handled as before.
We thus replace the \(\Theta(r \log{\frac{n}{r^{i+1}}})\) bits of the
cyclic permutation by a well chosen approximation.

Let \(\beta = 2^{\Theta(\log^{\alpha} n)}\) and denote by \(n_i = n/r^i\)
an upper bound on the number of pseudolines intersecting a cell of level \(i\). For
each subcell of level \(i\), partition its cyclic permutation into
\(\beta_i \leq \min \{\, \beta , n_{i+1} \,\} \) blocks
of at most \(\lceil n_{i+1} / \beta_i \rceil\) intersections.
%
For each pseudoline intersecting a cell we store the block indices that
that pseudoline touches instead of storing the cyclic permutation indices.%
\footnote{For \(\beta\) a power of two, this can be implemented by truncating
the original cyclic permutation indices.}
Hence, the second part
of each signature only uses \(O(r \log \beta) = O(\log^{2\alpha})\) bits.

\paragraph*{\iftitlecase%
Intersection Oracle\else%
Intersection oracle\fi}
We construct a lookup table to compute in constant time,
for any given cell of any given level, the subcell in which \(q_i \cap q_j\)
lies.
%Computing it via scanning with so many subcells to check would waste any
%further savings.
For that we need a general observation on the precomputation of functions on
small universes.
%
\begin{observation}\label{obs:small-universe-functions}
  In the word-RAM model with word size \(w \geq \log n\), for any word-to-word
  function \(f:[2^w] \to [2^w]\), we can build a lookup table of total bitsize
  \(2^g h\) for all \(2^g\) inputs \(x \in [2^g]\) of bitsize \(g \leq w\),
  mapping to images \(y \in [2^h]\) of bitsize \(h \leq w\),
  in time \(2^g T(g)\) where \(T(g)\) is the complexity of computing \(f(x)\), 
  \(x \in [2^g]\). The image of bitsize \(h\) of any input of bitsize \(g\) can then be
  retrieved in \(O(1)\) time by a single lookup (since inputs and outputs fit in
  a single word). In particular, the preprocessing time \(2^g T(g)\) and the
  space \(2^g h\) are sublinear as long as \(T(g)=g^{O(1)}\) and \(g + \log h
  = o(\log n)\).
\end{observation}
%
In other words, any polynomial time computable word-to-word function can be
precomputed in sublinear time and space for all inputs and outputs of
sublogarithmic size.
%
\aurelien{Maybe move to preliminaries?}

Since each pseudoline signature fits in
\(O(r^2 + r \log \beta) = O(\log^{2\alpha}{n})\)
bits,
and since the number of subcells of each cell is \(O(\log^{2\alpha}{n})\),
we can choose an appropriate \(\alpha\) so as to satisfy the requirements given
above:
take \(\alpha < \frac 12\) so that two pseudoline signatures
have a combined bitsize of \(g = o(\log n)\).
The output size is the bitsize of a subcell identifier, which is \(h \leq 2
\alpha \log \log n = o(\log n)\).
%
In some cases, the block indices stored in the signatures will not contain
enough information to point to a unique output. In those cases we store a
special value that indicates ambiguity of the input.

We can thus precompute the function that sends two pseudoline
signatures to either the subcell containing their intersection or to some
special value in case of an ambiguous input. Since we compute the function for
all members of its universe, we can implement the lookup table using
direct addressing into an array.

Note that the output of this oracle is the same no matter what level or cell we
consider: for non-ambiguous inputs, all the information required to locate
\(q_i \cap q_j\) is included in the input.
We thus only need a single lookup table, and the space needed is proportional to
\begin{displaymath}
  2^g h
  =
  2^{\Theta(\log^{2\alpha} n)}.
\end{displaymath}

\paragraph*{Disambiguation}
An input for the intersection oracle is ambiguous if and only if at least one
boundary intersection of each input pseudoline appears in the same cyclic
permutation block of the cell that contains their intersection.
%
Thus, ambiguous inputs rarely occur:
%
less than the number of blocks times the number of pairs of boundary
intersections in a block,
%
that is, less than
\(
\beta \cdot {({n_{i+1}} / \beta)}^2
=
\frac{n^2}{r^{2(i+1)}} / \beta
\)
times per subcell of level \(i\) of the hierarchy.
%
When \(\beta \geq n_{i+1}\) all ambiguity is lifted so we can ignore
those cases.

The disambiguation table is a hash table storing the answer to all ambiguous
inputs for all cells of each level \(i \in \{\, 0,1, \ldots, \ell -1\,\}\).
%
This table maps a triple of a cell
\(\mathcal{C}_{0,j_1,j_2,\ldots,j_i}\),
a pseudoline \(p_a'\),
and a pseudoline \(p_b'\) to the index \( j_{i+1} \in \{\, 0,1, \ldots, O(r^2)\,\}\) of
the subcell \(\mathcal{C}_{0,j_1,j_2,\ldots,j_i,j_{i+1}}\) containing the
intersection \(p_a' \cap p_b'\).
%
Summing over all subcells of each level, we obtain that the
number of entries in this table is bounded above by
\begin{displaymath}
  \sum_{i=0}^{\ell - 1}  r^{2i} \cdot r^2 \cdot \frac{n^2}{r^{2(i+1)}} / \beta
  =
  \frac{n^2 \ell}{2^{\Theta(\log^{\alpha} n)}}.
\end{displaymath}
%
The number of bits of each entry is at most \(\ell \lceil \log cr^2 \rceil + 2 \log n =
O(\log n)\) for the key and
\(\lceil \log cr^2 \rceil\) for the value. Both get absorbed by the
\(2^{-\Theta(\log^{\alpha} n)}\) factor in the number of entries, so we can
keep the same expression for the number of bits used by the disambiguation table.

Since the keys and values fit in a constant number of words,
we can guarantee worst case constant query time using
cuckoo~hashing~\cite{PR04} or
perfect~hashing~\cite{FKS84}.
%
In both cases the construction of the table is randomized
and takes expected linear time in the number of entries.
Perfect hashing has the advantage that we can drop the entry keys since we
only query this table for existing entries.
Note that this is the only part of the construction that is randomized.

\paragraph*{\iftitlecase%
Augmented Traces\else%
Augmented traces\fi}

The augmented traces are simply the concatenation of the signature and the
first intersection indices as depicted in Figure~\ref{fig:new-trace}.
%
\begin{figure}
  \centering{}
  \includegraphics[scale=0.7]{figures/new-trace}
  \caption{%
	  An augmented trace \(\textsc{Tr}'(\mathcal{C}, p')\)
      for the same cell-pseudoline pair as in Figure~\ref{fig:trace}.
  }\label{fig:new-trace}
\end{figure}
%
This layout allows for constant time access to the signature.
Given a subcell index we can test its incidence bits in constant time.
Given an intersected subcell index we can access the first intersection index
of the pseudoline in that subcell in constant time.

As discussed earlier, the first part of the signature uses \(O(r^2)\) bits.
We already noted that the second part of the signature
uses \(O(r \log \beta) = O(\log^{2 \alpha} n)\) bits. However, a better bound
to use for the analysis of the total space is
\(O(r \log \beta_i) = O(r \log n_{i+1})\) bits, which is proportional to the
number of bits needed for the first intersection indices.

Summing over all pseudolines and all intersected cells of each level,
the space used for the augmented traces is proportional to
\begin{displaymath}
n \sum_{i=0}^{\ell-1} r^i \cdot \left( r^2 + r \log n_{i+1} \right)
=
O\left(\frac{n^2}{t} (\log t + r)\right),
%n \sum_{i=0}^{\ell-1} r^i \cdot \left( r^2 + r \log \beta \right)
%=
%n \cdot ( r + \log \beta ) \cdot \sum_{i=0}^{\ell - 1} r^{i+1}
%=
%O\left(\frac{n^2}{t} (r + \log \beta)\right)
%=
%O\left(\frac{n^2}{t} \log^{\alpha} n\right).
\end{displaymath}
%
as in Lemma~\ref{lem:space-2-hierarchy}.
Since \(r = \Theta(\log^{\alpha} n)\) the bound becomes
\(O\left(\frac{n^2}{t} (\log t + \log^{\alpha} n)\right)\).


\paragraph*{\iftitlecase%
Subcell Mapper\else%
Subcell mapper\fi}

As before, let \(c_h\) be the constant hidden by the \(O(r^2)\) of the
hierarchical cutting, and let \(c_z = 9.5\) be the constant in
Theorem~\ref{thm:zone-theorem-2}.
We need a way to map a subcell index \(0 \leq j_i \leq c_h r^2 - 1\) to the
index \(0 \leq j_i' \leq \lfloor c_z r \rfloor - 2\) this subcell has in the
list of subcells intersected by a given pseudoline. If we can achieve this in
constant time, then we can also access the first intersection index this
pseudoline has in this subcell.

It is not hard to see that this operation boils down to computing
\(\operatorname{rank}_{\texttt{10}}(j_i)\) where the data array is composed of
the incidence bits of the given pseudoline. Hence this can be solved by adding
an auxiliary rank-select data structure to each trace~\cite{RRS07,BH17}.
Another solution is to reuse Observation~\ref{obs:small-universe-functions} to
construct a lookup table for all \(2^{\Theta(r^2)}\) possible incidence
vectors.
%
\aurelien{Maybe add rank-select data structure to preliminaries?}

With the first solution, the space used by the traces stays the same up to a
constant factor. With the second solution, the space used is dominated by the
space taken by the intersection oracle. Hence, we do not bother including it in
the space analysis.


All that is left to do is to properly solve the subproblems spawned by the last
level of the hierarchy. This is done exactly as in the previous section.
%
\paragraph*{\iftitlecase%
Leaves of the Hierarchy\else%
Leaves of the hierarchy\fi}
%
As before, we have \(O(\frac{n^2}{t^2})\) subproblems of size \(t\) to encode.
We follow the solution previously described to obtain:
\(O(\frac{n^2}{t^2})\) pointers of size \(\lceil \log{\nu(t)} \rceil\),
\(O(\frac{n^2}{t^2})\) canonical labelings of size \(\Theta(t \log t)\),
and \(\nu(t)\) lookup tables of size \(O(t^3)\).
%
This is sufficient to take care of each of those subproblems in constant time.
The total space usage for those leaves is unchanged and stays proportional to
\begin{displaymath}
  \frac{n^2}{t^2} \cdot (t \log t + \log{\nu(t)}) + t^3 \nu(t).
\end{displaymath}

\paragraph*{\iftitlecase%
Space Complexity\else%
Space complexity\fi}
Summing all terms together and adding the space taken by the
parameters of the hierarchy \(n\), \(r\) and \(t\), we obtain:
\begin{lemma}\label{lem:space-2-all-query}
  The space taken by the encoding described in Section~\ref{sec:query-time} is
  proportional to
    \begin{multline*}
    \overbrace{\log ntr}^{\text{Parameters}}
    +
    \overbrace{\frac{n^2}{t} (\log t + \log^{\alpha} n)}^{\text{Traces}}
    +
    \overbrace{2^{\Theta(\log^{2\alpha} n)}}^{\text{Intersection Oracle}}\\
    +
    \underbrace{\frac{n^2 \ell}{2^{\Theta(\log^{\alpha} n)}}}_{\text{Disambiguation Table}}
    +
    \underbrace{\frac{n^2}{t^2} ( \log \nu(t) + t \log t) + t^3 \nu(t)}_{\text{Leaves}}.
    \end{multline*}
\end{lemma}
As before,
we take \(t = \sqrt{\delta \log n}\) for abstract order types and \(t = \delta
\log n / \log\log n\) for realizable ones.
Taking \(\delta\) to be sufficiently small,
the space taken by the leaves of the hierarchy is thus \(\Theta(n^2)\) for
abstract order types and dominated by the term \(\frac{n^2}{t} \log t\)
in the case of realizable order types.
%
Setting \(\alpha < \frac 12\)
guarantees
that the space taken by the intersection oracle is subpolynomial,
and
that the space taken by the traces is subquadratic.
%
The space taken by the
disambiguation table is in \(O(\frac{n^2}{\log^c n})\) for all \(c\) and is
thus dominated by the other terms.

For abstract order types, all those terms are subquadratic, except for the
pointers at the leaves. The total space
usage for abstract order types is thus dominated by this term and
is quadratic.
%
For realizable order types, the total space is dominated by the term
\(\frac{n^2}{t} \log^{\alpha} n\).
Indeed, we can take \(\alpha\) as small as desired
to make the factor \(\log^{\alpha} n = O(\log^{\varepsilon} n)\).
%
This proves the space constraints in
Theorems~\ref{thm:abstract-loglog}~and~\ref{thm:realizable-loglog}.
%
Unfortunately, the present solution incurs a nonabsorbable
extra \(\log^{\varepsilon} n\) factor in the realizable case. Note that a
\(\log{\log{\log{n}}}\) factor can be squeezed from the query time without
increasing the space usage by choosing \(r = \Theta(\log \log n)\) instead.


\paragraph*{\iftitlecase%
Correctness and Query Complexity\else%
Correctness and query complexity\fi}
Given a query \(p_a', p_b', p_c'\) and a cell \(\mathcal{C}\),
%at each level,
the subcell \(\mathcal{C}'\) containing \(p_a' \cap p_b'\) is found in constant
time via the
intersection oracle and, if necessary, the disambiguation table.
The location of that subcell with respect to \(p_c'\) can then be retrieved by
a single lookup in the incidence bits of \(\textsc{Tr}'(\mathcal{C}, p_c')\).
In case of recursion, we can compute the address of the traces
\(\textsc{Tr}'(\mathcal{C}', p_a')\),
\(\textsc{Tr}'(\mathcal{C}', p_b')\), and
\(\textsc{Tr}'(\mathcal{C}', p_c')\) in constant time using the subcell mapper.
The base case is handled in constant time as before: using the pointers,
canonical labelings and order type lookup tables.

We now have a shallower decision tree of depth
\(\log_r{\frac{n}{t}} = O_\alpha(\frac{\log{n}}{\log{\log{n}}})\)
and the work at each level takes constant time.
This proves the query time constraints in
Theorems~\ref{thm:abstract-loglog}~and~\ref{thm:realizable-loglog}.

\paragraph*{\iftitlecase%
Preprocessing Time\else%
Preprocessing time\fi}
We prove Theorem~\ref{thm:preprocessing-loglog}.
\begin{proof}
  As before, the hierarchical cutting and all traces \(\textsc{Tr}'(C, p')\)
  can be computed in \(O(nr^\ell)\) time (with or without rank-select data
  structures). The lookup tables and leaf-table pointers can be computed in
  \(O(n^2)\) time. The intersection oracle, the disambiguation table, and the
  optional subcell mapper can be computed in subquadratic time.
\end{proof}

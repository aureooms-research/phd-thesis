Storing the permutation at each subcell would suffice to answer all
queries that do not reach the last level of the hierarchy. However,
for those queries to be answered efficiently,
we need to have access to all bits belonging to a given pseudoline
without having to read the bits of the others.
%
One solution is to augment each subcell with a hash table that translates
pseudoline indices of the parent cell into pseudoline indices of the subcell.
%
Another cleaner solution is to use the Zone Theorem
(%
\ifjournal%
Theorem~\ref{thm:zone-theorem-2}%
\else%
\cite{BEPY90,CGL85,Go04}%
\fi%
): by constructing the hierarchical cutting
via decompositions of subsets of the input pseudolines, we can bound
the number of subcells of a given cell a given pseudoline intersects by
\(O(r)\).%
\footnote{A reason to prefer this solution is that it enables the query time
reduction in the next section.}
%
Hence,
the number of bits stored for a single intersecting cell-pseudoline
pair at level \(i\) is bounded by
\(|\textsc{Tr}_i| = O(r^2 + r \log{\frac{n}{r^{i+1}}})\).
This bound allows us to store all bits belonging to a given cell-pseudoline pair
\((\mathcal{C},L)\)
in a contiguous block of memory, denoted by \(\textsc{Tr}(\mathcal{C},L)\),
whose address in memory is easy to compute (as detailed later on).
%
The overall number of bits stored stays the same up to a constant factor.
%
We call
\(\textsc{Tr}(\mathcal{C},L)\) the \emph{trace} of \(L\) in \(\mathcal{C}\).
Figure~\ref{fig:trace} depicts an example trace.
%
\begin{figure}
  \centering{}
  \includegraphics[width=\linewidth]{figures/trace}
  \caption{%
	  A trace \(\textsc{Tr}(\mathcal{C}, L)\). The cell \(\mathcal{C}\) has
	  eight subcells. Each subcell is intersected by at most four pseudolines.
	  The pseudoline \(L\) lies above two of them, lies below two of them,
	  contains one of them, and intersects three of them at indices
	  \((2,6)\), \((5,7)\), and \((0, 3)\).
  }\label{fig:trace}
\end{figure}
%
%Furthermore, the Zone Theorem implies that the number of cells of level \(i\)
%intersected by a given pseudoline is bounded by \(O(r^i)\). Hence, we can
%concatenate all traces of a given pseudoline and store them in a continuous
%block of memory in a space-efficient way.
%\aurelien{This would require to use another cell to subcell translation table
%that would blow up the space or rely on other complicated tools}

For queries that reach the last level of the hierarchy, storing an individual
lookup table for each leaf would cost too much as soon as \(t = \omega(1)\).
However, as long as \(t\) is small enough, each order type is shared by
many leaves, and we can thus save space. Formally, let \(\nu(t)\) denote the
number of order types of size \(t\), which is \(\nu(t) = 2^{\Theta(t^2)}\) for
abstract order types~\cite{Fe96} and \(\nu(t) = 2^{\Theta(t \log{t})}\) for realizable
order types~\cite{Al86,GP86}. At most \(\nu(t)\) distinct lookup tables are needed
to answer
the queries on the subcells of the last level of the hierarchy. Hence, the
pointers have size \(|\textsc{Pointer}| = O(\log \nu(t))\) and the total space needed for the lookup
tables is \(O(t^3 \nu(t))\). For each leaf, we store a canonical
labeling of size \(|\textsc{Labeling}| = O(t \log t)\) on the pseudolines that intersect it,
as in Lemma~\ref{lem:canonical-labeling}. We use
that labeling to order the queries in the associated lookup table.%
\ifjournal\footnote{%
Note that the use of this canonical labeling is not necessary if
\(t\) is constant, because then we can afford to use one lookup table per leaf.
For superconstant \(t\), the canonical labeling is necessary to get
the construction time down to \(O(n^2)\) as explained later.
Space and query complexity are not affected by this design choice (other than
constant factors).%
}\fi

\paragraph*{\iftitlecase%
Space Complexity\else%
Space complexity\fi}
We first prove a general bound on the space taken by the encoding of a
\(\ell\)-level hierarchical cutting of parameter \(r \geq 2\).
For the space taken by the lookup tables, their associated pointers and canonical
labelings at the leaves, and the parameters of the hierarchy \(n\), \(r\) and
\(t\), the analysis is immediate.

Let \(H_r^\ell(n) \in \mathbb{N}\) be the maximum amount of space (bits), over
all arrangements of \(n\) pseudolines, taken by the \(\ell \in \mathbb{N}\)
levels of a hierarchy with parameter \(r \in (1,+\infty)\).

\begin{lemma}\label{lem:space-2-hierarchy}
For \( r \geq 2 \) and \( t = \frac{n}{r^\ell}\) we have
\begin{displaymath}
H_r^\ell(n)
=
O\left(\frac{n^2}{t} ( \log t + r )\right).
\end{displaymath}
\end{lemma}

\begin{proof}
By definition, summing over all subcells, we have
\begin{displaymath} % This is equivalent but sums over all subcells
H_r^\ell(n)
= O \left(
    \sum_{i=0}^{\ell-1} \left(
	r^{2i} \cdot r^2 \cdot \left(
    \frac{n}{r^i} + \frac{n}{r^{i+1}} \log \frac{n}{r^{i+1}}
    \right)
\right)
\right).
\end{displaymath}
%
Note that we obtain the same bound by
summing over all traces of intersecting cell-pseudoline pairs
%
\begin{displaymath}
H_r^\ell(n)
= O \left(
    n
    \sum_{i=0}^{\ell-1} \left(
      r^i \cdot
      \left(
        r^2 + r \log \frac{n}{r^{i+1}}
      \right)
    \right)
\right).
\end{displaymath}
%
Multiply any of the previous equations by \(\frac{n}{tr^\ell} = 1\)
to obtain
%
\begin{displaymath}
H_r^\ell(n)
= O \left(
\frac{n^2}{t}
    \sum_{i=0}^{\ell-1} \left(
      \frac{1}{r^{\ell - i - 1}} \cdot \left(
	r + \log \frac{n}{r^{i+1}}
    \right)
\right)
\right).
\end{displaymath}
%
We use the equivalence \(\frac{n}{r^{i+1}} = t r^{\ell - i - 1}\) to replace
the last term in the previous equation
%
\begin{displaymath}
H_r^\ell(n)
= O \left(
\frac{n^2}{t}
    \sum_{i=0}^{\ell-1} \left(
      \frac{1}{r^{\ell - i - 1}} \cdot \left(
	r + \log t + (\ell - i - 1) \log r
    \right)
\right)
\right).
\end{displaymath}
%
We reverse the summation by redefining \(i \leftarrow \ell - i - 1\) and group the terms
%
\begin{displaymath}
H_r^\ell(n)
=
O \left(
\frac{n^2}{t} \left(
	(\log t + r)  \sum_{i=0}^{\ell-1} \frac{1}{r^{i}}
	+
	\log r \sum_{i=0}^{\ell-1} \frac{i}{r^{i}}
\right)
\right).
\end{displaymath}
%
Using the following inequalities (the geometric series and a multiple of its derivative):
%
\begin{displaymath}
\sum_{i = 0}^{k} x^i
\le
\frac{1}{1-x}
\qquad
\text{and}
\qquad
\sum_{i = 0}^{k} i x^i
\le
\frac{x}{{(1-x)}^2},
\qquad
\forall k \in \mathbb{N},
\forall x \in (0,1),
\end{displaymath}
%
we conclude that
%
\begin{displaymath}
H_r^\ell(n)
=
O\left(
\frac{n^2}{t} \left(
	\left(1 + \frac{1}{r-1} \right) (\log t + r)
	+
	\left(1 + \frac{2r-1}{r^2-2r+1}\right) \frac{\log r}{r}
\right)
\right),
\end{displaymath}
%
and the statement follows from \(r \geq 2\).
\end{proof}

%\paragraph*{Remark} The following could be stated for recursively defined
%families of order types of size \(\nu\).

\ifeurocg\else%
\begin{figure}
  \centering{}
  \includegraphics[scale=.8]{figures/space-analysis}
  \caption{Space analysis.}\label{fig:space-analysis}
\end{figure}
\fi

Figure~\ref{fig:space-analysis} sketches the different components of the
encoding and shows the space taken by each of them.
To that we must add the space taken by the
parameters of the hierarchy \(n\), \(r\) and \(t\) if those are not implicitly
known (here we assume the dimension \(d=2\) is implicitly known).
We have thus the following bound:
%
\begin{lemma}\label{lem:space-2-all}
  The space taken by the encoding described in Section~\ref{sec:lines-and-pseudolines} is
  proportional to
    \begin{displaymath}
      \underbrace{\log ntr}_{\text{Parameters}}
      +
      \underbrace{\frac{n^2}{t} ( \log t + r )}_{\text{Traces}}
      +
      \underbrace{\frac{n^2}{t^2} ( \log \nu(t) + t \log t)}_{\text{Pointers and Canonical Labelings}}
      +
      \underbrace{t^3 \nu(t)}_{\text{Lookup Tables}}.
    \end{displaymath}
\end{lemma}

We pick \(r\) constant for both abstract and realizable order types.
%
We have \(\nu(t) = 2^{\Theta(n^2)}\) for abstract order types, hence we
choose \(t = \sqrt{\delta \log n}\) for small enough \(\delta\) and the third
term in Lemma~\ref{lem:space-2-all} dominates with \(n^2\). Note how the
quadratic bottleneck of this encoding is the storage of the order type pointers
at the leaves of the hierarchy.
%
We have \(\nu(t) = 2^{\Theta(n \log n)}\) for realizable order types, hence we
choose \(t = \delta \log n / \log \log n\) for small enough \(\delta\) and the
second and third term in Lemma~\ref{lem:space-2-all} dominate with \(n^2 {(\log
\log n)}^2 / \log n\). This proves the space constraints in
Theorems~\ref{thm:abstract}~and~\ref{thm:realizable}.

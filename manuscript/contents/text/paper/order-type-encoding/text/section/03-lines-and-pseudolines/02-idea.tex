\paragraph*{Idea} We want to preprocess \(n\) pseudolines \( \{\, p_1' ,
p_2' , \ldots, p_n'\,\} \) in the plane so that, given three indices \(a\),
\(b\), and \(c\), we can compute their orientation, that is, whether the
intersection \(p_a' \cap p_b'\) lies above, below or on \(p_c'\). Our
data structure builds on cuttings as follows: Given a cutting \(\Xi\) and the
three indices, we can locate the intersection of \(p_a'\) and \(p_b'\)
with respect to \(\Xi\). The location of this intersection is a cell of
\(\Xi\). The next step is to decide whether \(p_c'\) lies above, lies below,
contains or intersects that cell. In the first three cases, we are done.
Otherwise, we can answer the query by recursing on the subset of pseudolines
intersecting the cell containing the intersection. We build on hierarchical
cuttings to control the size of each such subproblem.

\paragraph*{Layout}
For completeness, we detail precisely how bits of the encoding are
laid out in memory to allow an efficient decoding.
%
Indeed, the data structure we encode is a tree and many space-efficient layouts
exists for those when their nodes each have the same size. Here however, node
size shrinks as we go down the hierarchy.
We spend a few paragraphs showing it
is still possible to
address all components in a
time- and space-efficient way.
%
As this is fairly straightforward to adapt for the encodings in
\S\ref{sec:query-time} and \S\ref{sec:hyperplanes},
we do not give the details for those sections.
%
%Figure~\ref{fig:layout} gives a sketch of this layout.

%Figure is OUTDATED
%\ifeurocg\else%
%\begin{figure}
  %\centering{}
  %\includegraphics[scale=.8]{figures/layout}
  %\caption{Memory layout of the encoding.
    %Here, \(\textsc{Tr}(a, C_{i,j}(a))\) is used as a shorthand
    %for \(\textsc{Tr}(L_a, C_{i,j}(a))\) and \(-1\) is used as a shorthand
    %for ``last index''. The symbols \(\pi_j\) and
    %\(\textsc{Ptr}_j\) represent the canonical labeling and the pointer for leaf
    %\(j\), respectively. The symbol \(\textsc{Table}_{\tau}\) represents the lookup
    %table for order type \(\tau\).}\label{fig:layout}
%\end{figure}
%\fi

The encoding is the concatenation of
the parameters \(n\), \(r\), and \( t\),
the cells of the hierarchy,
and
the lookup tables.
%
We order the cells of the hierarchy in a depth-first manner: a cell of level
\(i\) is denoted by \(\mathcal{C}_{0, j_1, j_2, \ldots, j_i}\)
with \(j_1, j_2, \ldots, j_i \in \{\, 0, 1, \ldots, O(r^2)\,\}\).
The root cell is
\(\mathcal{C}_0\) and the cell \(\mathcal{C}_{0, j_1, j_2, \ldots, j_i, j_{i+1}}\) is
the (\(j_{i+1} + 1\))-th subcell of the cell
\(\mathcal{C}_{0, j_1, j_2, \ldots, j_i}\).
%
Cells are then ordered lexicographically.
%
For each leaf cell we store its pointer and canonical labeling.
%
For each internal cell \(\mathcal{C}\) we store the traces
\(\textsc{Tr}(\mathcal{C}, L)\) in a certain permutation of the pseudolines
\(L\).
%
To order the pseudolines, we use the first index of each pseudoline
in the cyclic permutation of the cell (for the root cell \(\mathcal{C}_0\) this
is the index given by the input permutation).
%
Note that those indices are between \(0\) and \(2 \lfloor \frac{N}{r^{i}}
\rfloor - 1\) for a cell of level \(i\). We allocate twice the required space
for traces and canonical labelings to avoid defining a mapping between the
indices of a cell and the indices of each of its subcells.

%To translate between cell's subcell indices and pseudoline's
%\aurelien{We avoid doing that by enumerating first cell-wise, then
%pseudoline-wise}

For the root cell of the hierarchy
\(\mathcal{C}_0\) representing the entire space and containing all the
intersections of the arrangement, \(\textsc{Addr}(\mathcal{C}_0)\) is the
first free address after the encoding of the parameters \(n\), \(r\), and \(t\).

The address \(\textsc{Addr}(\mathcal{C}_{0,j_1,j_2, \ldots, j_{i-1}, 0})\)
of the first subcell of \(\mathcal{C}_{0,j_1,j_2, \ldots, j_{i-1}}\)
is offset by twice the space taken by the traces for that cell
%
\begin{displaymath}
  \textsc{Addr}(\mathcal{C}_{0,j_1,j_2, \ldots, j_{i-1}, 0})
  =
  \underbrace{\textsc{Addr}(\mathcal{C}_{0,j_1,j_2, \ldots, j_{i-1}})}_{\text{Address of the parent cell}}
  +
  \underbrace{2 \left\lfloor \frac{n}{r^{i-1}} \right\rfloor |\textsc{Tr}_{i-1}|}_{\text{Traces of the parent cell}}
  .
\end{displaymath}

Let \(c_h\) be the constant hidden by the \(O(r^2)\) of the hierarchical cutting.
%
The space taken by a subtree rooted at a node of level \(i\) is bounded by
\begin{multline*}
  | \textsc{Subtree}_i |
  =
  \overbrace{2 \left\lfloor \frac{n}{r^{i}} \right\rfloor |\textsc{Tr}_{i}|}^{\text{Traces at the root}}
  +
  \overbrace{2c_h \sum_{k=1}^{\ell-i-1} r^{2k} \left\lfloor \frac{n}{r^{i+k}}
  \right\rfloor |\textsc{Tr}_{i+k}|}^{\text{Traces of the subtrees}}\\
  +
  \underbrace{c_h r^{2(\ell-i)} | \textsc{Pointer} |}_{\text{Pointers}}
  +
  \underbrace{2c_h r^{2(\ell-i)} | \textsc{Labeling} |}_{\text{Canonical labelings}}.
\end{multline*}

The address of any other subcell
\(\textsc{Addr}(\mathcal{C}_{0,j_1,j_2, \ldots, j_{i-1}, j_{i}})\)
is offset by the space taken by the subtrees of its siblings \(0, 1, \ldots,
j_{i} - 1\)
%
\begin{displaymath}
  \textsc{Addr}(\mathcal{C}_{0,j_1,j_2, \ldots, j_{i-1}, j_{i}})
  =
  \underbrace{\textsc{Addr}(\mathcal{C}_{0,j_1,j_2, \ldots, j_{i-1}, 0})}_{\text{Address of the first sibling}}
  +
  \underbrace{j_{i} \cdot | \textsc{Subtree}_{i} |}_{\text{Left siblings subtrees}}
  .
\end{displaymath}

The address of the trace \(\textsc{Tr}(\mathcal{C}, L_a)\),
where
\(0 \leq a \leq 2 \lfloor \frac{N}{r^{i}} \rfloor - 1\) is the first intersection
index of \(L_a\) in the cyclic permutation of the level-\(i\) cell
\(\mathcal{C}\),
is
%
\begin{displaymath}
  \textsc{Addr}(\mathcal{C}, L_a)
  =
  \textsc{Addr}(\mathcal{C})
  + a \cdot |\textsc{Tr}_{i}|.
\end{displaymath}
%
Since we do not store any trace for the leaves,
define \(| \textsc{Tr}_{\ell} | = 0\).
The pointer and canonical labeling of the leaf
\(\mathcal{C}_{0, j_1, j_2, \ldots, j_{\ell}}\)
are concatenated at position
\(\textsc{Addr}(\mathcal{C}_{0, j_1, j_2, \ldots, j_{\ell}})\)
%
and
%
the lookup tables are concatenated at position
\(
\textsc{Addr}(\mathcal{C}_{0})
+
| \textsc{Subtree}_{0} |
\).

This layout makes traversing the hierarchy from root to leaf efficient:
%
The address of the root cell is discovered after parsing the parameters \(n\),
\(r\), and \(t\).
%
The address of the first subcell of a parent cell is computed in constant time
from the address of the parent cell.
%
The size of the hierarchy is computed in time proportional its height.
%
The size of a subtree of level \(i+1\) is derived in constant time from the
size of a subtree of level \(i\).
%
The address of any subcell of level \(i\) is computed in constant time from its
index, the address of the parent cell, and the size of a subtree of level
\(i\).
%
The address of a trace is computed in constant time from the address of its
cell and the index of its pseudoline.
%
The address of the pointer and canonical labeling of a leaf
is the address of the corresponding cell.
%
The address of a lookup table is computed in constant time given the address of
the root cell, the size of the hierarchy, the size of a lookup table, and the
leaf pointer.

During a traversal, pseudoline indices of the parent cell are mapped to
indices in the subcell in constant time by using the first intersection index
of each pseudoline in that subcell. A final index mapping happens when translating leaf indices
to lookup table indices using the canonical labeling of the leaf.

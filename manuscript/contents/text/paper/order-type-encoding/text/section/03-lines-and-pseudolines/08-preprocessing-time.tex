\paragraph*{\iftitlecase%
Preprocessing Time\else%
Preprocessing time\fi}
For a set of \(n\) points in the plane, or an arrangement of \(n\) lines in the
dual, we can construct the encoding of their order type in quadratic time in
the real-RAM and constant-degree algebraic computation tree models. We prove
Lemma~\ref{thm:preprocessing}.
\aurelien{Should also work for pseudolines using the wiring diagram as input.}
\begin{proof}
  Using Lemma~\ref{lem:hierarchical-cutting-2},
  a hierarchical cutting can be computed in \(O(nr^\ell)\) time in the dual
  plane. All traces \(\textsc{Tr}(\mathcal{C}, L)\) can be computed from the cutting
  in the same time. The lookup tables and leaf-table pointers can be computed
  in \(O(n^2 + t^3 \nu(t))\) time as follows: For each subcell \(\mathcal{C}\)
  among the \(\frac{n^2}{t^2}\) subcells of the last level of the hierarchy,
  compute a canonical labeling and representation of the lines intersecting
  \(\mathcal{C}\) in \(O(t^2)\) time as in Lemma~\ref{lem:canonical-labeling}.
  Insert the canonical representation in some trie in \(O(t^2)\)
  time. If the canonical representation was not already in the trie, create a lookup
  table with the answers to all \(O(t^3)\) queries on those lines and attach a
  pointer to that table in the trie. This happens at most \(\nu(t)\) times.
  In the encoding, store the canonical labeling and this new pointer or the
  pointer that was already in the trie for the subcell \(\mathcal{C}\). All
  parts of the encoding can be concatenated together in time proportional to
  the size of the encoding.
\end{proof}

\ifeurocg\else
To make our statements clear, we use the concept of an encoding.
We recall its definition
%
\DefinitionEncoding*
%
In this section, we use this definition with \(f\) being some order
type,\footnote{%
  Technically, we encode the orientation predicate of some realizing
  arrangement of the order type and skip the isomorphism. If desired, a
  canonical labeling of the arrangement can be produced in \(O(n^2)\) time for
  abstract and realizable order types (see Lemma~\ref{lem:canonical-labeling}).
}
\(k=3\) and the codomain of \(f\) being \(\{\, -,0,+\,\}\). For the rest of the
discussion, we assume the word-RAM model with word size \(w \geq \log n\) and
the standard arithmetic and bitwise operators.
%
We prove our main theorems for the two-dimensional case:
%
\TheoremGPTAbstract*
\TheoremGPTRealizable*
\begin{theorem}\label{thm:preprocessing}
  In the real-RAM model and the constant-degree algebraic decision tree model,
  given \(n\) real-coordinate input points in \(\mathbb{R}^2\) we can compute
  the encoding of their order type as in
  Theorems~\ref{thm:abstract}~and~\ref{thm:realizable} in \(O(n^2) \) time.
\end{theorem}

%
For instance, Theorem~\ref{thm:realizable} implies that for any set of points
\(\{\, p_1, \ldots, p_n\,\}\), there exists a string of \(O(n^2 {(\log
\log n)}^2 / \log n)\) bits such that given this string and any triple of
indices \((a,b,c) \in {[n]}^3\) we can compute the value of \(\chi(a,b,c) =
\nabla(p_a, p_b, p_c)\) in \(O(\log n)\) time.
\fi

\ifeurocg
We \else
Throughout the rest of this paper, we \fi
assume that we can access some arrangement of pairwise intersecting lines or
pseudolines that realizes the order type we want to encode. We thus exclusively
focus on the problem of encoding the order type of a given arrangement. This
does not pose a threat against the existence of an encoding.
\ifeurocg\else%
However, we have to be more careful when we bound the preprocessing
time required to compute such an encoding. This is why, in
Theorem~\ref{thm:preprocessing}, we specify the model of computation and how
the input is given.
\fi%
\ifeurocg%
In this extended abstract, we sketch the general idea for a simple subquadratic
encoding. For full details, proofs, and improvements, we refer to the arXiv
version~\cite{CCILO19}.
\fi%

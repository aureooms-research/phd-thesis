\paragraph*{Intersection Location}
%
When the arrangement consists of straight lines,
locating the intersection \(p_a' \cap p_b'\) in \(\Xi\) is trivial if we
know the real parameters of \(p_a'\) and \(p_b'\) and of the descriptions
of the subcells of \(\Xi\). However, in our model we are not allowed to store
real numbers. To circumvent this annoyance, and to handle arrangements of
pseudolines, we make an observation illustrated by
Figure~\ref{fig:permutation}.
%
\begin{figure}
\centering{}
\begin{subfigure}[t]{0.5\textwidth}
\centering{}
\includegraphics[scale=.\ifeurocg7\else9\fi]{figures/permutation-0}
\caption{\(p_a' \cap p_b' \cap \mathcal{C} = \emptyset\).}
\end{subfigure}%
\begin{subfigure}[t]{0.5\textwidth}
\centering{}
\includegraphics[scale=.\ifeurocg7\else9\fi]{figures/permutation-1}
\caption{\(p_a' \cap p_b' \cap \mathcal{C} \neq \emptyset\).}
\end{subfigure}
\caption{Cyclic permutations (\(\pi\)).}\label{fig:permutation}
\end{figure}
%
\begin{definition}
    Given a set of pseudolines, a pseudocircle is a simple closed curve such that
    each pseudoline properly intersects it in exactly two points.
    %
    A pseudodisk is a region bounded by a pseudocircle.
\end{definition}
%
\begin{observation}
    Two pseudolines \(p_a'\) and \(p_b'\)
    intersect in the interior of a pseudodisk \(\mathcal{C}\) if and only if
    the intersections of \(p_a'\) and \(p_b'\) with \(\mathcal{C}\)
    alternate on the boundary of \(\mathcal{C}\).
\end{observation}
%
%
We construct \(\Xi\) so that its cells are pseudodisks for the pseudolines that
intersect them.
%
Hence, this observation gives us a way to encode the location of the
intersection of \(p_a'\) and \(p_b'\) in \(\Xi\) using only bits.
%
We formalize this observation with the following definition:
\begin{definition}[Cyclic Permutation]
  The \emph{cyclic permutation} of a full-dimensional cell of a cutting
  is the cyclic ordering of the pseudolines crossing its boundary.
\end{definition}
%
For an arrangement of pseudolines, we use the standard vertical decomposition
(\S\ref{sec:arrangements:vertical-decomposition})
to construct a hierarchical cutting
(\S\ref{sec:divide-and-conquer:hierarchical-cuttings}).
%
This decomposition partitions the space into trapezoidal cells.
%
For an arrangement of lines, we can
use the standard bottom-vertex triangulation
(\S\ref{sec:arrangements:triangulation})
instead, which allows us to
generalize our results to higher dimensions in \S\ref{sec:hyperplanes}.
%
In the plane, the bottom-vertex triangulation partitions the space into
triangular cells.
%
Note that non-full-dimensional cells are easier to encode. For a 0-dimensional
cell and a pseudoline, we store whether the pseudoline lies above, lies below,
or contains the 0-dimensional cell. For a 1-dimensional cell, a pseudoline
could also intersect the interior of the cell, but in only one point. The
pseudolines intersecting that cell define an (acyclic) permutation with potentially
several intersections at the same position. This information suffices to answer
location queries for those cells, and the space taken is not more than that
necessary for full-dimensional cells. When two pseudolines intersect in a
1-dimensional cell or contain the same 0-dimensional cell, they
appear simultaneously in the cyclic permutation of an adjacent 2-dimensional
cell if they intersect its interior. If that is the case, the location of
the intersection of those two pseudolines in the cutting is the
non-full-dimensional cell. A constant number of bits can be added to the
encoding each time we need to know the dimension of the cell we encode.

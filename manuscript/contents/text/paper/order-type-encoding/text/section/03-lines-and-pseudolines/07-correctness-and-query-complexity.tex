\paragraph*{\iftitlecase%
Correctness and Query Complexity\else%
Correctness and query complexity\fi} Given our encoding and three
pseudoline indices \(a,b,c\) we answer a query as follows: We start by decoding the
parameters \(n\), \(r\), and \(t\). In our model, this can be done in
\(O(\log^* n + \log^* r + \log^* t)\) time, where \(\log^*\) is the iterated
logarithm (as in~\cite{Ma93}).\footnote{%
  Logarithmic space and constant decoding time is trivial when \(w =
  \Theta(\log n)\).
  If \(w\) is too large, encode \(n\) in binary using \(\lceil \log n + 1\rceil\)
  bits, \(\lceil \log n + 1\rceil\) using \(\lceil \log \lceil \log n + 1
  \rceil +1 \rceil\) bits,
  \(\lceil \log \lceil \log n + 1\rceil +1\rceil\) using \(\lceil \log \lceil
  \log \lceil \log n + 1\rceil +1\rceil +1\rceil\) bits,
  etc.\ until the number to encode is smaller than a constant which we encode
  in unary with \texttt{1}'s. Prepend a \texttt{1} to the largest number and
  \texttt{0} to all the others. Concatenate those numbers
  from smallest to largest. Total space is \(O(\log n)\) bits and decoding
  \(n\) can be done in \(O(\log^* n)\) time in the word-RAM model with \(w \geq
  \log n\).
  As an alternative, logarithmic space and logarithmic decoding time is also
  trivially achievable with no constraint on \(w\).%
}
Let \(\mathcal{C} = \mathcal{C}_{0}\).
First, find the subcell \(\mathcal{C}'\) of \(\mathcal{C}\) containing \(L_a
\cap L_b\) by testing for each subcell whether
\(L_a\) and \(L_b\) alternate in its cyclic permutation.
This can be done in \(O(r^2)\) time by scanning \(\textsc{Tr}(\mathcal{C},
L_a)\) and \(\textsc{Tr}(\mathcal{C}, L_b)\) in parallel. Note that non-full
dimensional cells and subcells are easier to test. Next, if \(L_c\) does
not properly intersect \(\mathcal{C}'\), answer the query accordingly. If on
the other hand \(L_c\) does properly intersect the subcell we recurse on
\(\mathcal{C}'\). This can be tested by scanning \(\textsc{Tr}(\mathcal{C},
L_c)\) in \(O(r^2)\) time. Note that in case that the subcell is
non-full-dimensional we can already answer the query. When we
reach the relative interior of a subcell of the last level of the hierarchy
without having found a satisfactory answer, we can answer the query by table
lookup in constant time. This works as long as each order type identifier for
at most \(t\) pseudolines fits in a constant number of words, which is the case
for the values of \(t\) we defined.
%
The layout described earlier makes all memory address computations of this
query algorithm take constant time.
%
The total query time is thus proportional
to \(r^2 \log_r n\) in the worst case, which is logarithmic since \(r\) is
constant.
%
This proves the query time constraints in
Theorems~\ref{thm:abstract}~and~\ref{thm:realizable}.

With the hope of getting faster queries we could pick \(r = \Theta(\log
t)\) to reduce the depth of the hierarchy, without changing the
space requirements by more than a constant factor.
However, if no additional care is taken, this would slow the queries down by a
\(\Theta(\log^2 t / \log \log t)\) factor because of the scanning approach
taken when locating the intersection \(L_a \cap L_b\). We show how to handle
small but superconstant \(r\) properly in the next section.

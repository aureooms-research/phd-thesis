%\section{abstract}
	For many algorithms dealing with sets of points in the plane, the only
	relevant information carried by the input is the combinatorial
	configuration of the points: the orientation of each triple of points in
	the set (clockwise, counterclockwise, or collinear). This information is
	called the \emph{order type} of the point set.
%
	In the dual, realizable order types and abstract order types are
	combinatorial analogues of line arrangements and pseudoline arrangements.
%
	Too often in the literature we analyze algorithms in the
	real-RAM model for simplicity, putting aside the fact that computers as we
	know them cannot handle arbitrary real numbers without some sort of
	encoding.
%
	Encoding an order type by the integer coordinates of a realizing point
	set is known to yield doubly exponential coordinates in some cases. Other
	known encodings can achieve quadratic space or fast orientation queries,
	but not both.
%
	In \S\ref{sec:lines-and-pseudolines},
	we give a compact encoding for abstract order types
	that allows an efficient query of the orientation of any triple: the encoding
	uses \( O(n^2) \) bits and an orientation query takes \(O(\log n)\) time in
	the word-RAM model with word size \(w \geq \log n\).
%
	This encoding is space-optimal for abstract order types. We show how to
	shorten the encoding to \(O(n^2 {(\log\log n)}^2 / \log n)\) bits for
	realizable order types, giving the first subquadratic encoding for those
	order types with fast orientation queries.
%
	In \S\ref{sec:query-time}, we further refine our encoding to attain
	\(O(\log n/\log\log n)\) query time at the expense of a negligibly larger
	space requirement.
%
	In the realizable case, we show that all those encodings can be computed
	efficiently.
%
	In \S\ref{sec:hyperplanes},
	we generalize our results to the encoding of point configurations
	in higher dimension.
%\end{abstract}
%\clearpage
%\pagebreak
%\setcounter{page}{1}

\section{Higher-Dimensional Encodings}\label{sec:hyperplanes}
We generalize our point configuration encoding to any dimension \(d\). The
chirotope of a point set in \(\mathbb{R}^d\) consists of all orientations of
simplices defined by \(d+1\) points of the set~\cite{RZ04}.
The orientation of the simplex with \(d+1\) ordered vertices \(p_i\) with
coordinates \((p_{i,1} , p_{i,2} , \ldots, p_{i,d} )\) is given by the sign of
the determinant
%
\begin{displaymath}
  \begin{vmatrix}
    1 & p_{1,1} & p_{1,2} & \hdots & p_{1,d} \\
    1 & p_{2,1} & p_{2,2} & \hdots & p_{2,d} \\
    \vdots & \vdots & \vdots & \ddots & \vdots \\
    1 & p_{d+1,1} & p_{d+1,2} & \hdots & p_{d+1,d}
  \end{vmatrix}.
\end{displaymath}

We obtain the following generalized result:
%
\TheoremGPTRealizableD*
\TheoremGPTPreprocessingD*

\paragraph*{Idea}

In the primal, the orientation of a simplex whose vertices are ordered can be
interpreted as the location of its last vertex with respect to the (oriented)
hyperplane spanned by its first \(d\) vertices. In the dual, this orientation
corresponds to the location of the intersection of the first \(d\) dual
hyperplanes with respect to the last (oriented) dual hyperplane.
%
\ifjournal%
  In the primal, degenerate simplices have orientation \(0\). In the dual, this
  corresponds to linearly dependent subsets of \(d+1\) hyperplanes.
\fi%

We gave an encoding for the two-dimensional case in \S\ref{sec:lines-and-pseudolines}.
With this encoding, a query is answered by traversing the levels of
some hierarchical cutting, branching on the location of the intersection of two
of the three query lines. We generalize this idea to \(d\) dimensions. Now the
cell considered at the next level of the hierarchy depends on the location
of the intersection of \(d\) of the \(d+1\) query hyperplanes.
\ifjournal%
  We will also have to take care of degenerate cases.
\fi%

\paragraph*{\iftitlecase%
Intersection Location\else%
Intersection location\fi}

In \S\ref{sec:lines-and-pseudolines}, we solved the following two-dimensional subproblem:

\begin{problem}
  Given a triangle and \(n\) lines in the plane, build a data structure that,
  given two of those lines, allows to decide whether their intersection
  lies in the interior of the triangle.
\end{problem}

In retrospective, we showed that there exists such a data structure using
\(O(n \log n)\) bits that allows for queries in \(O(1)\) time. We generalize
this result. Consider the following generalization of the problem in
\(d\) dimensions:

\begin{problem}
  Given a convex body and \(n\) hyperplanes in \(\mathbb{R}^d\), build a data
  structure that, given \(d\) of those hyperplanes, allows to decide whether their
  intersection is a vertex that lies in the interior of the convex body.
\end{problem}

Of course this problem can be solved using \(O(n^d)\) space by explicitly
storing the answers to all possible queries. If the input hyperplanes are
given in an arbitrary order, this is best possible for
\(d=1\). For \(d \geq 2\), we show how to reduce the space to \(O(n^{d-1} \log
n)\) by recursing on the dimension, taking \(d = 2\) as the base case.
\aurelien{Proof of a lower bound would be a nice addition.}

We encode the function \(\mathcal{I}_{C, H}\)
that maps a \(d\)-tuple of indices of input dual
hyperplanes \(H_i\) to \(1\) if their intersection is a vertex that lies in the
interior of a fixed convex body \(C\), and to \(0\) otherwise.
%
\begin{displaymath}
  \mathcal{I}_{C,H} \colon\, {[n]}^d \to \{\, 0,1\,\} \colon\,
  (i_1,i_2,\ldots,i_d) \mapsto
  (H_{i_1} \cap H_{i_2} \cap \cdots \cap H_{i_d})
  \in  C.
\end{displaymath}
%
We call this function the \emph{intersection function} of \((C,H)\).
%
We prove the following:
\begin{theorem}\label{thm:intersection-d}
  All intersection functions have an \((O(n^{d-1} \log n),O(d))\)-encoding.
\end{theorem}


\begin{proof}
  Consider a convex body \(C\) and \(n\) hyperplanes \(H_i\).
  At first, for simplicity, assume \(C\) is \(d\)-dimensional and
  assume that any \(d\) hyperplanes \(H_{i_j}\) meet in a single point.
  With those assumptions,
  we want a data structure that can answer any query of the type
  \begin{displaymath}
    (H_{i_1} \cap H_{i_2} \cap H_{i_3} \cap H_{i_4} \cap \cdots \cap
    H_{i_{d-1}} \cap H_{i_d})
    \cap C
    \neq \emptyset.
  \end{displaymath}

  Note that this is equivalent to deciding whether
  \begin{displaymath}
    (H_{i_1} \cap H_{i_2} \cap H_{i_3} \cap H_{i_4} \cap \cdots \cap H_{i_{d-1}})
    \cap (H_{i_d} \cap C)
    \neq \emptyset,
  \end{displaymath}
  where \(H_{i_d} \cap C\) is a convex body of dimension \(d-1\) (or empty), and the
  number of hyperplanes we want to intersect it with is \(d-1\).

  We unroll the recursion until the convex body is of dimension two (or empty),
  and only two hyperplanes are left to intersect. We then notice that the decision we
  are left with is equivalent to
  \begin{multline*}
    (H_{i_1} \cap (H_{i_3} \cap H_{i_4} \cap \cdots \cap H_{i_d}))
    \cap
    (H_{i_2} \cap (H_{i_3} \cap H_{i_4} \cap \cdots \cap H_{i_d}))\\
    \cap (C \cap (H_{i_3} \cap H_{i_4} \cap \cdots \cap H_{i_d}))
    \neq \emptyset,
  \end{multline*}
  which, if the three objects are non-empty, reads:
  %
  ``Given two lines and a convex body in some plane, do they intersect?''.
  %
  We can answer this query if we have the encoding for \(d = 2\) which is
  obtained by replacing \emph{triangle} by \emph{convex body} in the
  two-dimensional original problem. The total space taken is multiplied by
  \(n\) for each time we unroll the recursion times the space taken in two
  dimensions, which is proportional to \(n^{d-2} \cdot n \log{n} = O(n^{d-1} \log{n}) \).
  Queries can then be answered in \(O(d)\) time.

  Note that degenerate cases are likely to arise:
  %
  convex bodies that are empty because of nonintersecting hyperplanes, convex
  bodies that are higher dimensional because of linearly dependent hyperplanes,
  and convex bodies that are lower dimensional because \(C\) was not
  full-dimensional to start with.
  %
  However, all those cases can be dealt with appropriately:
  %
  If the query suffix
  \(H_{i_3} \cap H_{i_4} \cap \cdots \cap H_{i_d}\)
  leads to an empty convex body
  \(C \cap (H_{i_3} \cap H_{i_4} \cap \cdots \cap H_{i_d})\)
  then the query point is not in \(C\) and we encode
  \(0\) for all the queries in \(C\) ending in this suffix.
  This information can be encoded in a table of size \(O(n^{d-2})\).
  %
  If the query suffix
  \(H_{i_3} \cap H_{i_4} \cap \cdots \cap H_{i_d}\)
  leads to a convex body
  \(C \cap (H_{i_3} \cap H_{i_4} \cap \cdots \cap H_{i_d})\)
  of dimension \(\geq 3\) then the intersection of all objects
  is not a \(0\)-flat and we encode a \(0\) to follow the definition
  of \(\mathcal{I}_{C,H}\).
  Again, this information can be encoded in a table of size \(O(n^{d-2})\).
  %
  If the query suffix
  \(H_{i_3} \cap H_{i_4} \cap \cdots \cap H_{i_d}\)
  leads to a convex body
  \(C \cap (H_{i_3} \cap H_{i_4} \cap \cdots \cap H_{i_d})\)
  of dimension zero, one, or two,
  then we use the encoding of size \(O(n \log n)\) described in
  \S\ref{sec:lines-and-pseudolines} as a base case.
\end{proof}

%Note that there is a linear lower bound for encoding this subproblem in
%dimension \(d=1\) and a \(\Omega(n \log n)\) lower bound for \(d=2\). We might be able
%to strengthen the lower bound for higher dimensions.

In the paragraphs that follow, we show how to plug this result in those of
the previous sections to obtain analogous results for the \(d\)-dimensional
version of the problem (Theorem~\ref{thm:realizable-d} and
Theorem~\ref{thm:preprocessing-d}).

\paragraph*{Encoding}
We build a hierarchical cutting as in
\S\ref{sec:lines-and-pseudolines} (this time in dimension
\(d\)).
%
Given \(n\) hyperplanes in \(\mathbb{R}^d\) and some fixed
parameters \(r\) and \(\ell\), compute a \(\ell\)-levels hierarchical cutting
of parameter \(r\) for those
hyperplanes as in Lemma~\ref{lem:hierarchical-cutting-d}.
This hierarchical cutting consists of \( \ell \) levels
labeled \(0,1,\ldots,\ell-1\). Level \(i\) has \(O(r^{di})\)
cells. Each of those cells is further partitioned into
\(O(r^d)\) subcells. The \(O(r^{d(i+1)})\) subcells of level \(i\)
are the cells of level \(i+1\). Each cell of level \(i\) is intersected by at
most \(\frac{n}{r^i}\) hyperplanes, and hence each subcell is intersected by at
most \(\frac{n}{r^{i+1}}\) hyperplanes.

We compute and store a combinatorial representation of the hierarchical cutting
as follows: For each level of the hierarchy, for each cell in that level, for
each hyperplane intersecting that cell, for each subcell of that cell, we store
two bits to indicate the location of the hyperplane with respect to that
subcell\ifeurocg\else, that is, whether the hyperplane lies above
(\texttt{00}), lies below (\texttt{01}), intersects the interior of that
subcell (\texttt{10}), or contains the subcell (\texttt{11})\fi.
%
When a hyperplane \(H_{i}\) intersects the interior of a subcell \(\mathcal{C}'\),
we also store the two \(O(\log \frac{n}{r^{i+1}})\)-bits two-dimensional
intersections indices this hyperplane has in each of the
\(O\left({\left(\frac{n}{r^{i+1}}\right)}^{d-2}\right)\)
query suffixes \((H_{i_3} \cap H_{i_4} \cap \ldots \cap H_{i_d})\) with each of the
\(H_{i_j}\) properly intersecting \(\mathcal{C}'\).

This representation takes
\(O\left(
  \frac{n}{r^i}
  +
  {\left(\frac{n}{r^{i+1}}\right)}^{d-1} \log{\frac{n}{r^{i+1}}}
\right)\)
bits per subcell of level \(i\) by storing for each
hyperplane its location and, when needed, the bits they hold in
the encoding of the intersection function of the subcell.
At the last level of the hierarchy, let \(t =
\frac{n}{r^\ell}\) denote an upper bound on the number of hyperplanes
intersecting each subcell. For each of those \(O(r^{d \ell}) =
O(\frac{n^d}{t^d})\) subcells we store a pointer to a lookup table of size
\(O(t^{d+1})\) that allows to answer the query of the orientation of any
triple of hyperplanes intersecting that subcell.

Using the Zone Theorem in higher dimensions (Theorem~\ref{thm:zone-theorem-d}),
we can have all bits belonging to a single cell-hyperplane pair in a contiguous
block of memory with the same space bound.

\paragraph*{\iftitlecase%
Space Complexity\else%
Space complexity\fi}
As before, for the space taken by the lookup tables, their
associated pointers and canonical labelings at the leaves, and the parameters
of the hierarchy \(n\), \(r\) and \(t\), the analysis is immediate. If not
implicitly known, the dimension \(d\) can also trivially be added to the
encoding.

For the space taken by the hierarchy,
we generalize Lemma~\ref{lem:space-2-hierarchy} of
\S\ref{sec:lines-and-pseudolines}. Let \(H_r^\ell(n,d) \in \mathbb{N}\) be the maximum
amount of space (bits), over all arrangements of \(n\) hyperplanes in \(\mathbb{R}^d\),
taken by the \(\ell \in \mathbb{N}\) levels of a hierarchy with parameter \(r \in
(1,+\infty)\).
%
\begin{lemma}\label{lem:space-d-hierarchy}
For \( r \geq 2 \) we have
\begin{displaymath}
H_r^\ell(n,d)
=
O\left(\frac{n^d}{t} \left(\log t + \frac{r}{t^{d-2}}\right)\right).
\end{displaymath}
\end{lemma}

\begin{proof}
By definition we have
\begin{displaymath}
H_r^\ell(n,d)
= O \left(
  \sum_{i=0}^{\ell-1} \left(
	r^{di} \cdot r^d \cdot \left(
	  \frac{n}{r^i} + {\left(\frac{n}{r^{i+1}}\right)}^{d-1} \cdot \log \frac{n}{r^{i+1}}
    \right)
  \right)
\right).
\end{displaymath}
%
Using \(t = \frac{n}{r^{\ell}}\), reversing the summation with
\(i \leftarrow \ell - i - 1\), and grouping the terms, we have
\begin{displaymath}
H_r^\ell(n,d)
=
O \left(
\frac{n^d}{t} \left(
	\frac{r}{t^{d-2}}
	\sum_{i=0}^{\ell-1} \frac{1}{{(r^{d-1})}^i}
	+
	\log t
	\sum_{i=0}^{\ell-1} \frac{1}{r^i}
	+
	\log r
	\sum_{i=0}^{\ell-1} \frac{i}{r^i}
\right)
\right).
\end{displaymath}
%
Using the geometric inequalities
(see the proof of Lemma~\ref{lem:space-2-hierarchy})
the statement follows from \(r \geq 2\).
%
\end{proof}


The final picture is almost the same as in Figure~\ref{fig:space-analysis}.
%
Summing all terms, we obtain
%
\begin{lemma}\label{lem:space-d-all}
    The space taken by the encoding described in \S\ref{sec:hyperplanes}
    is proportional to
    \begin{displaymath}
      \underbrace{\log dntr}_{\text{Parameters}}
      +
      \underbrace{\frac{n^d}{t} \left( \log t + \frac{r}{t^{d-2}} \right)}_{\text{Traces}}
      +
      \underbrace{\frac{n^d}{t^d} ( \log \nu_d(t) + t \log t )}_{\text{Leaves}}
      +
      \underbrace{t^{d+1}\nu_d(t)}_{\text{Lookup Tables}},
    \end{displaymath}
    where \(\nu_d(t) = 2^{\Theta(d^2 t \log t)}\) denotes the number of
    realizable rank-\((d+1)\) chirotopes of size \(t\).
\end{lemma}

We pick \(r\) constant and choose \(t = \delta \log n / \log \log n\) for small
enough \(\delta\). The second term in Lemma~\ref{lem:space-d-all} dominates
with \(n^d {(\log \log n)}^2 / \log n\). This proves the space constraint in
Theorem~\ref{thm:realizable-d}.

\paragraph*{\iftitlecase%
Correctness and Query Complexity\else%
Correctness and query complexity\fi}
As before, a query is answered by traversing the hierarchy, which takes
\(O(\log n)\) time. The query time can be further improved using the method from
\S\ref{sec:query-time} with \(r = \Theta(t^{d-2} \log t)\). This proves the
query time constraint in Theorem~\ref{thm:realizable-d}.

\paragraph*{\iftitlecase%
Preprocessing Time\else%
Preprocessing time\fi}
We prove Theorem~\ref{thm:preprocessing-d}.
\begin{proof}
  The hierarchical cuttings can be computed in \(O(n{(r^\ell)}^{d-1})\) time.
  The lookup table and leaf-table pointers can be computed in \(O(n^d)\) time
  using the canonical labeling and representation for rank-\((d+1)\) chirotopes
  given in~\cite{AILOW14}. The intersection oracle, the disambiguation table,
  and the subcell mapper can be computed in \(o(n^{d})\) time.
\end{proof}

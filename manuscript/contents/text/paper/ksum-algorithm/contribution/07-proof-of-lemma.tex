\subsection{Proof of Lemma~\ref*{lem:bound}}
\label{app:bound}

\begin{theorem}[Cramer's rule]\label{thm:cramer}
	If a system of $n$ linear equations for $n$ unknowns, represented in matrix
	multiplication form $Ax=b$,
	has a unique solution $x=(x_1,x_2,\ldots,x_n)^T$ then, for all $i \in [n]$,
	$$
		x_i = \frac{\det(A_i)}{\det(A)}
	$$
	where $A_i$ is $A$ with the $i$th column replaced by the column vector $b$.
\end{theorem}


\begin{lemma}[Meyer auf der Heide\cite{M84}]\label{lem:detZ}
	The absolute value of the determinant of an $n\times n$ matrix $M =
	M_{i=1\ldots n,j=1\ldots n}$ with integer entries is an integer that is at
	most $C^n n^{\frac n2}$, where $C$ is the maximum absolute value in $M$.
\end{lemma}
\begin{proof}
	The determinant of $M$ must be an integer
	and is the volume of the
	hyperparalleliped spanned by the row vectors of $M$, hence
	$$
	|\det(M)| \le \prod_{i=1}^n \sqrt{\sum_{j=1}^{n} M_{i,j}^2} \le {(\sqrt{n C^2})}^{n} \le C^n n^\frac n2.
	$$
\end{proof}

\begin{lemma}\label{lem:detQ}
	The determinant of an $n\times n$ matrix $M = M_{i=1\ldots n,j=1\ldots n}$ with
	rational entries can be represented as a fraction whose numerators and
	denominators absolute values are bounded above by
	${(ND^{n-1})}^n n^{\frac n2}$ and $D^{n^2}$
	respectively, where $N$ and $D$
	are respectively the maximum absolute value of a numerator and a
	denominator in $M$.
\end{lemma}
\begin{proof}
	Le $\delta_{i,j}$ denote the denominator of $M_{i,j}$.
	Multiply each row $M_i$ of $M$ by $\prod_j \delta_{i,j}$.
	Apply Lemma~\ref{lem:detZ}.
\end{proof}

We can now proceed to the proof of Lemma~\ref{lem:bound}.
\begin{proof}
	Coefficients of the hyperplanes of the arrangement are constant rational
	numbers, those can be changed to constant integers (because each
	hyperplane has at most $k$ nonzero coefficients). Let $C$ denote the
	maximum absolute value of those coefficients.

	Because of Theorem~\ref{thm:cramer} and Lemma~\ref{lem:detZ}, vertices of
	the arrangement have rational coordinates whose numerators and
	denominators absolute values are bounded above by $C^n n^{\frac n2}$.

	Given simplices whose vertices are vertices of the arrangement, hyperplanes
	that define the faces of those simplices have rational coefficients whose
	numerators and denominators absolute values are bounded above by $C^{2n^3}
	n^{n^3+\frac n2}$ by Theorem~\ref{thm:cramer} and Lemma~\ref{lem:detQ}.
	(Note that some simplices might be not fully dimensional, but we can handle
	those by adding vertices with coordinates that are not much larger than
	that of already existing vertices).

	By applying Theorem~\ref{thm:cramer} and Lemma~\ref{lem:detQ} again, we
	obtain that vertices of the arrangement of those new hyperplanes (and thus
	vertices of $\simplex$) have rational coefficients whose numerators and
	denominators absolute values are bounded above by $C^{4n^5} n^{2n^5+n^3+\frac n2}$.
\end{proof}

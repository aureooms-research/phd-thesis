\subsection{Query Size}%
\label{paper:ksum-algorithm:contrib:query-size}

In this section, we consider a simple blocking scheme that allows us to explore
a tradeoff between the number of queries and the size of the queries.

\begin{lemma}
\label{lem:block}
For any integer $b>0$, an instance of the \(k\)-SUM problem on $n>b$ numbers can be split into
$O(b^{k-1})$ instances on at most $k\lceil \frac{n}{b}\rceil$ numbers, so that every $k$-tuple
forming a solution is found in exactly one of the subproblems.
The transformation can be carried out in time $O(n\log n + b^{k-1})$.
\end{lemma}
\begin{proof}
Given an instance on $n$ numbers, we can sort them in time $O(n\log n)$, then partition
the sorted sequence into \(b\) consecutive blocks \(B_1, B_2,\ldots ,B_b\) of equal size.
This partition can be associated with a partition of the real line
into $b$ intervals, say $I_1, I_2,\ldots ,I_b$. Now consider the partition of $\R^k$
into grid cells defined by the $k$th power of the partition $I_1, I_2,\ldots ,I_b$. The
hyperplane of equation $x_1 + x_2 +\cdots +x_k = 0$ hits $O(b^{k-1})$ such grid cells.
Each grid cell $I_{i_1}\times I_{i_2}\times \cdots \times I_{i_k}$ corresponds to a
\(k\)-SUM problem on the numbers in the set $B_{i_1}\cup B_{i_2}\cup \ldots \cup B_{i_k}$ (note that
the indices $i_j$ need not be distinct). Hence each such instance has size at most $k\lceil \frac{n}{b}\rceil$.
\end{proof}

Combining Lemma~\ref{lem:block} and Theorem~\ref{thm:cube} directly yields our
second contribution.
%
\TheoremKSUMQuerySize*

The following two corollaries are obtained by taking
$b=\Theta(\polylog(n))$, and $b=\Theta(n^{\alpha})$, respectively
\begin{corollary}\label{cor:logn}
There exists an $o(n)$-linear decision tree of
depth $\tilde{O}(n^3)$ solving the \(k\)-SUM problem.
Moreover, this decision tree can be implemented as an
$\tilde{O}(n^{\ceil{\frac{k}{2}}+8})$ Las Vegas algorithm.
\end{corollary}
\begin{corollary}\label{cor:ne}
For any $\alpha$ such that $0<\alpha<1$,
there exists an $O(n^{1-\alpha})$-linear decision tree of
depth $\tilde{O} (n^{3+(k-4)\alpha})$ solving the \(k\)-SUM problem.
Moreover, this decision tree can be implemented as an
$\tilde{O}(n^{(1+\alpha)\frac{k}{2} + 8.5})$
Las Vegas algorithm.
\end{corollary}

Note that the latter query complexity improves on $\tilde{O}(n^{\frac{k}{2}})$
whenever \(\alpha < \frac{k-6}{2k-8}\) and $k\ge 7$.
By choosing $\alpha=\frac{k-6}{2k-8}-\frac{\beta}{k-4}$
we obtain $O(n^{1-\frac{k-6}{2k-8}+\frac{\beta}{k-4}})$-linear decision trees
of depth
$\tilde{O}(n^{\frac k2 - \beta})$
for any $k \ge 7$.
Hence for instance, we obtain
$O(n^{\frac{3}{4}+\frac{\beta}{4}})$-linear decision trees of depth
$\tilde{O}(n^{4-\beta})$ for the \dsum[8] problem.


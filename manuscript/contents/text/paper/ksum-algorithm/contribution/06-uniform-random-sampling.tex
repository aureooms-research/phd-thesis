
\subsection{Uniform Random Sampling}%
\label{app:sampling}

Theorem~\ref{thm:enet} requires us to pick a sample of the hyperplanes
uniformly at random. Actually the theorem is a little stronger; we can draw
each element of $\net$ uniformly at random, only keeping distinct elements.
This is not too difficult to achieve for $k$-LDT when the $\alpha_i,i\in[k]$
are all distinct: to pick a hyperplane of the form $\alpha_0 + \alpha_1 x_{i_1}
+ \alpha_2 x_{i_2} + \cdots + \alpha_k x_{i_k} = 0$ uniformly at random, we can
draw each $i_j\in[n]$ independently and there are $n^k$ possible outcomes.
However, in the case of $k$-SUM, we only have $\binom{n}{k}$ distinct
hyperplanes. A simple dynamic programming approach solves the problem for
\(k\)-SUM. For \(k\)-LDT we can use the same approach, once for each class of equal
$\alpha_i$.

\begin{lemma}
	Given $n\in\N$ and $(\alpha_0,\alpha_1,\ldots,\alpha_k) \in \R^{k+1}$,
	$m$ independent uniform random draws of
	hyperplanes in $\R^n$ with equations of the form
	$\alpha_0 + \alpha_1 x_{i_1} + \alpha_2 x_{i_2} + \cdots + \alpha_k x_{i_k} = 0$
	can be computed in time $O(mk^2 \log n)$
	and preprocessing time $O(k^2 n)$.
\end{lemma}
\begin{proof}
	We want to pick an assignment
	$a=\enum{(\alpha_1,x_{i_1}),\ldots,(\alpha_k,x_{i_k})}$
	uniformly at random. Note that all $x_i$ are distinct while the $\alpha_j$
	can be equal.

	Without loss of generality, suppose $\alpha_1 \le \alpha_2 \le \cdots \le
	\alpha_k$. There is a bijection between assignments and lexicographically
	sorted $k$-tuples
	$((\alpha_1,x_{i_1}),\ldots,(\alpha_k,x_{i_k}))$.

	Observe that $x_{i_j}$ can be drawn independently of $x_{i_{j'}}$ whenever
	$\alpha_j \neq \alpha_{j'}$. Hence, it suffices to generate a lexicographically sorted
	$\card{\chi}$-tuple of $x_i$ for each class $\chi$ of equal $\alpha_i$.

	Let $\omega(m,l)$ denote the number of lexicographically sorted $l$-tuples,
	where each element comes from a set of $m$ distinct $x_i$.
	We have
	$$
		\omega(m,l) = \begin{cases}
			1 & \text{if}\ l = 0\\
			\sum_{i=1}^{m}\omega(i,l-1) & \text{otherwise.}
		\end{cases}
	$$

	To pick such a tuple $(x_{i_1},x_{i_2},\ldots,x_{i_l})$ uniformly at random
	we choose $x_{i_l} = x_o$ with probability
	$$
		P(x_{i_l} = x_o) = \begin{cases}
			0 & \text{if}\ o > m\\
			\frac{\omega(o,l-1)}{\omega(m,l)} & \text{otherwise}
		\end{cases}
	$$
	that we append to a prefix $(l-1)$-tuple (apply the procedure recursively),
	whose elements come from a set of $o$ symbols. If $l=0$ we just
	return the empty tuple.

	Obviously, the probability for a given $l$-tuple to be picked is equal to
	$\frac{1}{\omega(m,l)}$.

	Let $X$ denote the partiton of the $\alpha_i$ into equivalence classes,
	then the number of
	assignments is equal to $\prod_{\chi \in X}\omega(n,\card{\chi})$.
	(Note that for $k$-SUM this is simply $\omega(n,k)$ since there is only a
	single class of equivalence.)
	For each equivalence class $\chi$ we draw independently a lexicographically
	sorted $\card{\chi}$-tuple on $n$ symbols using the procedure
	above. This yields a given assignment with probability
	$\frac{1}{\prod_{\chi \in X}\omega(n,\card{\chi})}$.
	Hence, this corresponds to a uniform random draw over the assignments.

	It is a well known fact that $\omega(n,k)=\binom{n+k-1}{k-1}$, hence each
	number we manipulate fits in $O(k \log n)$ bits, that is, $O(k)$ words.
	Moreover $\omega(n,k) = \omega(n-1,k) + \omega(n-1,k-1)$ so each
	$\omega(m,l)$ can be computed using a single addition on numbers of $O(k)$
	words.

	For given $n$ and $k$, there are at most $nk$ values $\omega(m,l)$ to
	compute, and for a given $k$-LDT instance, it must be computed only once.
	One way to perform the random draws is to compute the cumulative
	distribution functions of the discrete distributions defined above, then to
	draw $x_{i_l}$, we use binary search to find a generated random integer of
	$O(k)$ words in the cumulative distribution function. Computing the
	values $\omega(m,l)$ and all cumulative distributions functions can be done
	as a preprocessing step in $O(k^2 n)$ time. Assuming the generation
	of a random sequence of words takes linear time, performing a random draw
	takes time $O(k^2 \log n)$.

\end{proof}


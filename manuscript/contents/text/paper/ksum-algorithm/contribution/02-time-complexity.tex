\subsection{Time complexity}
\label{paper:ksum-algorithm:contrib:time-complexity}

Proving the second part of Theorem~\ref{thm:cube} involves efficient implementations of
the two most time-consuming steps of Algorithm~\ref{algo:meiser}.
In order to efficiently implement the pruning
step, we define an intermediate problem, that we call the \emph{double $k$-SUM}
problem.
\begin{problem}[double $k$-SUM]
	Given two vectors $\nu_1, \nu_2 \in {[-1,1]}^n$, where the coordinates of
	$\nu_i$ can be written down as fractions whose numerator and denominator
	lie in the interval $[-M,M]$, enumerate all
	$i\in {[n]}^k$ such that
	$$
	\left(\sum_{j=1}^{k} \nu_{1,i_j}\right)
	\left(\sum_{j=1}^{k} \nu_{2,i_j}\right)
	< 0.
	$$
\end{problem}


In other words, we wish to list all hyperplanes of $\Hy$
intersecting the open line segment $\nu_1\nu_2$.
We give an efficient output-sensitive algorithm for this problem.

\begin{problem}[double $k$-SUM]
	Given two vectors $\nu_1, \nu_2 \in {[-1,1]}^n$, where the coordinates of
	$\nu_i$ can be written down as fractions whose numerator and denominator
	lie in the interval $[-M,M]$, enumerate all
	$i\in {[n]}^k$ such that
	$$
	\left(\sum_{j=1}^{k} \nu_{1,i_j}\right)
	\left(\sum_{j=1}^{k} \nu_{2,i_j}\right)
	< 0.
	$$
\end{problem}

\begin{proof}
	If $k$ is even, we consider all possible $\frac{k}{2}$-tuples of numbers in
	$\nu_1$ and $\nu_2$ and sort their sums in increasing order. This takes time
	$O(n^{\frac{k}{2}} \log n)$ and yields two permutations $\pi_1$ and $\pi_2$
	of $[n^\frac{k}{2}]$.
	If $k$ is odd, then we sort both the $\lceil\frac{k}{2}\rceil$-tuples and
	the $\lfloor\frac{k}{2}\rfloor$-tuples. For simplicity, we will only
	consider the even case in what follows. The odd case carries through.

	We let $N = n^\frac{k}{2}$. For $i \in [N]$ and $m \in \{1,2\}$, let
	$\Sigma_{m,i}$ be the sum of the $\frac{k}{2}$ components of the $i$th
	$\frac{k}{2}$-tuple in $\nu_m$, in the order prescribed by $\pi_m$.

	We now consider the two $N \times N$ matrices $M_1$ and $M_2$ giving all
	possible sums of two $\frac{k}{2}$-tuples, for both $\nu_1$ with the ordering
	$\pi_1$ and $\nu_2$ with the ordering $\pi_2$.

	We first solve the $k$-SUM problem on $\nu_1$, by finding the sign of all
	pairs $\Sigma_{1,i} + \Sigma_{1,j}$, $i, j \in [N]$. This can be done in
	time $O(N)$ by parsing the matrix $M_1$, just as in the standard \(k\)-SUM algorithm.
        We do the same with $M_2$.

	The set of all indices $i, j \in [N]$ such that $\Sigma_{1,i} +
	\Sigma_{1,j}$ is positive forms a staircase in $M_1$. We sweep $M_1$ column
	by column in order of increasing $j \in [N]$, in such a way that the number
	of indices $i$ such that $\Sigma_{1,i} + \Sigma_{1,j} > 0$ is growing.
	For each new such value $i$ that is encountered during the sweep, we insert
	the corresponding $i' = \pi_2(\pi_1^{-1}(i))$ in a balanced binary search
	tree.

	After each sweep step in $M_1$ --- that is, after incrementing $j$ and
	adding the set of new indices $i'$ in the tree --- we search the tree to
	identify all the indices $i'$ such that $\Sigma_{2,i'} + \Sigma_{2,j'} <
	0$, where $j' = \pi_2(\pi_1^{-1}(j))$. Since those indices form an interval
	in the ordering $\pi_2$ when restricted to the indices in the tree, we can
	search for the largest $i_0'$ such that $\Sigma_{2,i_0'} < -\Sigma_{2,j'}$
	and retain all indices $i' \le i_0'$ that are in the tree. If we denote by
	$z$ the number of such indices, this can be done in $O(\log N + z) = O(\log
	n + z)$ time. Now all the pairs $i', j'$ found in this way are such that
	$\Sigma_{1,i} + \Sigma_{1,j}$ is positive and $\Sigma_{2,i'} +
	\Sigma_{2,j'}$ is negative, hence we can output the corresponding
	$k$-tuples. To get all the pairs $i', j'$ such that
	$\Sigma_{1,i} + \Sigma_{1,j}$ is negative and $\Sigma_{2,i'} +
	\Sigma_{2,j'}$ positive, we repeat the sweeping algorithm after swapping the
	roles of $\nu_1$ and $\nu_2$.

	Every matching $k$-tuple is output exactly once, and every
	$\frac{k}{2}$-tuple is inserted at most once in the binary search tree.
	Hence the algorithm runs in the claimed time.

	Note that we only manipulate rational numbers that are the sum of at most
	$k$ rational numbers of size $O(\log M)$.
\end{proof}

Now observe that a hyperplane intersects the interior of a simplex if and only
if it intersects the interior of one of its edges. Hence given a simplex
$\simplex$ we can find all hyperplanes of $\Hy$ intersecting its interior by
running the above algorithm ${n\choose 2}$ times, once for each pair of
vertices $(\nu_1,\nu_2)$ of $\simplex$, and take the union of the solutions.
The overall running time for this implementation will therefore be
$\tilde{O}(n^2 (n^{\lceil \frac k2 \rceil} \log M + Z))$, where $Z$ is at most the
number of intersecting hyperplanes and $M$ is to be determined later.
This provides an implementation of the pruning step in Meiser's algorithm, that
is, step~\step{4} of Algorithm~\ref{algo:meiser}.

\begin{corollary}\label{cor:double}
	Given a simplex $\simplex$, we can compute all $k$-SUM hyperplanes
	intersecting its interior in $\tilde{O}(n^2 (n^{\lceil \frac k2 \rceil}
	\log M + Z))$ time, where $\log M$ is proportional to the number of bits
	necessary to represent $\simplex$.
\end{corollary}

In order to detect solutions in step~\step{3} of Algorithm~\ref{algo:meiser}, we also
need to be able to quickly solve the following problem.
\begin{problem}[multiple $k$-SUM]
	Given $d$ points $\nu_1,\nu_2,\ldots,\nu_d \in \R^n$, where the coordinates of
	$\nu_i$ can be written down as fractions whose numerator and denominator
	lie in the interval $[-M,M]$,
	decide whether there exists a hyperplane with equation of the form $x_{i_1}
	+ x_{i_2} + \cdots + x_{i_k} = 0$ containing all of them.
\end{problem}

Here the standard $k$-SUM algorithm can be applied, taking advantage of
the fact that the coordinates lie in a small discrete set.
\begin{lemma}\label{lem:ksum}
	$k$-SUM on $n$ integers $\in [-V,V]$ can be solved in time
	$\tilde{O}(n^{\ckt}\log V)$.
\end{lemma}

\begin{lemma}\label{lem:multiple}
	Multiple $k$-SUM can be solved in time $\tilde{O}(dn^{\ckt+2}\log M)$.
\end{lemma}
\begin{proof}
	Let $\mu_{i,j}$ and $\delta_{i,j}$ be the numerator and denominator of
	$\nu_{i,j}$ when written as an irreducible fraction. We define
	$$
		\zeta_{i,j} = \nu_{i,j}\prod_{\mathclap{(i,j) \in [d]\times[n]}} \delta_{i,j} =
		\frac{\mu_{i,j}\displaystyle\prod_{\mathclap{(i',j') \in [d]\times[n]}} \delta_{i',j'}}{\delta_{i,j}}.
	$$
	By definition $\zeta_{i,j}$ is an integer and its absolute value is bounded
	by $U = M^{n^2}$, that is, it can be
	represented using $O(n^2 \log M)$ bits. Moreover, if one of the hyperplanes
	contains the point $(\zeta_{i,1},\zeta_{i,2},\ldots,\zeta_{i,n})$, then it
	contains $\nu_i$. Construct $n$ integers of $O(dn^2 \log M)$ bits that can
	be written $\zeta_{1,j}+U,\zeta_{2,j}+U,\ldots,\zeta_{d,j}+U$ in base
	$2Uk+1$. The answer to our decision problem is ``yes'' if and only if there
	exists $k$ of those numbers whose sum is $kU,kU,\ldots,kU$. We simply
	subtract the number $U,U,\ldots,U$ to all $n$
	input numbers to obtain a standard $k$-SUM instance on $n$ integers of
	$O(dn^2 \log M)$ bits.
\end{proof}

We now have efficient implementations of steps \step{3} and \step{4}
of Algorithm~\ref{algo:meiser} and can proceed to the proof of the second part
of Theorem~\ref{thm:cube}.

\begin{proof}
The main idea consists in modifying the first iteration of Algorithm~\ref{algo:meiser}, by
letting $\varepsilon = \Theta(n^{-\frac{k}{2}})$.
Hence we pick a random subset $\net$ of
$O(n^{k/2 + 2} \log n)$ hyperplanes in $\Hy$ and use this as an \enet.
This can be done efficiently, as shown in~Appendix~\ref{app:sampling}.

Next, we need to locate the input $q$ in the arrangement induced by
$\net$. This can be done by running Algorithm~\ref{algo:meiser} on the set
$\net$. From the previous considerations on Algorithm~\ref{algo:meiser}, the
running time of this step is
$$
O(n|\net|) = \tilde{O}(n^{k/2 + 4}),
$$
and the number of queries is $O(n^3\log^2 n)$.

Then, in order to prune the hyperplanes in $\Hy$, we have to
compute a simplex $\simplex$ that does not intersect any hyperplane of
$\net$. For this, we observe that the above call to Algorithm~\ref{algo:meiser}
involves computing a sequence of simplices for the successive pruning
steps. We save the description of those simplices. Recall that there are
$O(\log n)$ of them, all of them contain the input $q$ and have vertices
coinciding with vertices of the original arrangement $\arrangement(\Hy)$. In
order to compute a simplex $\simplex$ avoiding all hyperplanes of
$\net$, we can simply apply Algorithm~\ref{algo:simplex} on the set of
hyperplanes bounding the intersection of these simplices. The running time
and number of queries for this step are bounded respectively by
$n^{O(1)}$ and $O(n^2\log n)$.

Note that the vertices of $\simplex$ are not vertices
of $\arrangement(\Hy)$ anymore. However, their coordinates lie in a finite set
(see~Appendix~\ref{app:bound})
\begin{lemma}\label{lem:bound}
Vertices of $\simplex$ have rational coordinates whose fraction representations
have their numerators and denominators absolute values bounded above by
$C^{4n^5} n^{2n^5+n^3+\frac n2}$, where $C$ is a constant.
\end{lemma}

We now are in position to perform the pruning of the hyperplanes in $\Hy$ with
respect to $\simplex$. The number of remaining hyperplanes after the pruning is
at most $\varepsilon n^k = O(n^{k/2})$. Hence from Corollary~\ref{cor:double}, the
pruning can be performed in time proportional to $\tilde{O}(n^{\ceil{k/2} + 7})$.

Similarly, we can detect any hyperplane of $\Hy$ containing $\simplex$ using the
result of Lemma~\ref{lem:multiple} in time $\tilde{O}(n^{\lceil k/2\rceil +
8})$. Note that those last two steps do not require any query.

Finally, it remains to detect any solution that may lie in the remaining
set of hyperplanes of size $O(n^{k/2})$. We can again fall back
on Algorithm~\ref{algo:meiser}, restricted to those hyperplanes. The running
time is $\tilde{O}(n^{k/2 + 2})$, and the number of queries is still
$O(n^3 \log^2 n)$.

Overall, the maximum running time of a step is $\tilde{O}(n^{\ceil{\frac{k}{2}} + 8})$,
while the number of queries is always bounded by $O(n^3\log^2 n)$.

\end{proof}


The 3SUM problem asks if an input $n$-set of real numbers contains a
triple whose sum is zero.
%
We qualify such a triple as \emph{degenerate} because the probability of
finding one in a random input is zero.
%
We consider the 3POL problem, an algebraic generalization of 3SUM where we
replace the sum function by a constant-degree polynomial in three variables.
%
The motivations are threefold.
%
Raz, Sharir, and de Zeeuw gave an $O(n^{11/6})$ upper bound on the number of
degenerate triples for the 3POL problem. We give algorithms for the
corresponding problem of counting them.
%
Gr\o nlund and Pettie designed subquadratic algorithms for 3SUM\@. We
prove that 3POL admits bounded-degree algebraic decision trees of depth
$O(n^{12/7+\epsilon})$, and we prove that 3POL can be solved in $O(n^2
{(\log \log n)}^{3/2} / {(\log n)}^{1/2})$ time in the real-RAM model,
generalizing their results.
%
Finally, we shed light on the General
Position Testing~(GPT) problem: ``Given $n$ points in the plane, do three of
them lie on a line?'', a key problem in computational geometry: we show how to
solve GPT in subquadratic time when the input points lie on a small number of
constant-degree polynomial curves. Many other geometric degeneracy testing
problems reduce to 3POL\@.
%
%This constitutes the first step towards closing the major open question of
%whether GPT can be solved in subquadratic time.

%\keywords{3SUM \and subquadratic algorithms \and general position testing \and
%range searching \and dominance reporting \and algebraic geometry \and
%degeneracy testing}
%\subclass{68Pa10 \and 68Q25 \and 68R05 \and 68W40}

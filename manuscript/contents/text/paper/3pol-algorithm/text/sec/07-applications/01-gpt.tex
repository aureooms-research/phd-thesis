\subsection{General position testing for points on curves}%
\label{sec:paper:3pol-algorithm:application:gpt}

%Raz, Sharir and de Zeeuw~\cite{RSZ15} give three corollaries to
%Theorem~\ref{thm:rsz15:col}.
%\begin{corollary}[Raz, Sharir and de Zeeuw~\cite{RSZ15}]
	%Any $n$ points on an irreducible algebraic curve of degree $d$ in
	%$\mathbb{C}^2$ determine
	%$\tilde{O_d}(n^{\frac{11}{6}})$ proper collinear triples, unless the curve is a line or a cubic.
%\end{corollary}
%\begin{corollary}[Raz, Sharir and de Zeeuw~\cite{RSZ15}]
	%Any $n$ points on an algebraic curve of degree $d$ in $\mathbb{C}^2$ determine
	%$\tilde{O_d}(n^{\frac{11}{6}})$
	%proper collinear quadruples, unless the curve contains a line.
%\end{corollary}
%\begin{corollary}[Raz, Sharir and de Zeeuw~\cite{RSZ15}]
	%Any $n$ points on an irreducible algebraic curve of degree $d$ in
	%$\mathbb{C}^2$ determine
	%$\Omega(n^{\frac 43})$ distinct directions, unless the curve is a conic.
%\end{corollary}

The following is a corollary of Theorem~\ref{thm:rsz15:col} in Raz, Sharir and de Zeeuw~\cite{RSZ15}
\begin{corollary}[Raz, Sharir and de Zeeuw~\cite{RSZ15}]
	Any $n$ points on an irreducible algebraic curve of degree $d$ in
	$\mathbb{C}^2$ determine
	$\tilde{O}_d(n^{\frac{11}{6}})$ proper collinear triples, unless the curve is a line or a cubic.
\end{corollary}

An interesting application of our results is the existence of subquadratic
nonuniform and uniform algorithms for the computational version of
this corollary.
%their first corollary.
\begin{problem}[GPT on curves]
	Let $C_1, C_2, C_3$ be three (not necessarily distinct) parameterized
	constant-degree polynomial curves in $\mathbb{R}^2$, so that each
	$C_i$ can be written $(g_i(t),h_i(t))$ for some polynomials of
	constant degree $g_i,h_i$.
	Given three $n$-sets $S_1 \subset C_1, S_2 \subset C_2, S_3
	\subset C_3$, decide whether there exist any collinear
	triple of points in $S_1 \times S_2 \times S_3$.
\end{problem}
\begin{theorem}\label{thm:gpt-to-3pol}
	 GPT on curves reduces linearily to 3POL\@.
\end{theorem}
\begin{proof}
For each set $S_i$, construct the set $T_i = \{\,t
\colon\, p \in S_i, p = (g_{i}(t),h_{i}(t))\,\}$.
Testing whether there exists a collinear triple
%$((g_1(t_1),h_1(t_1)),(g_2(t_2),h_2(t_2)),(g_3(t_3),h_3(t_3))) \in
in $
S_1 \times S_2 \times S_3$ amounts to testing whether any determinant
\begin{displaymath}
\begin{vmatrix}
g_1(t_1)&h_1(t_1)&1\\
g_2(t_2)&h_2(t_2)&1\\
g_3(t_3)&h_3(t_3)&1
\end{vmatrix}
\end{displaymath}
equals zero.
This determinant is a trivariate constant-degree polynomial in
$\mathbb{R}[t_1,t_2,t_3]$.
Solving the original problem amounts thus to deciding whether this polynomial
cancels for any triple $(t_1,t_2,t_3) \in T_1 \times T_2 \times T_3$.
\end{proof}

Note that a similar polynomial predicate exists for testing collinearity in
higher dimension.
\begin{lemma}
Let $p = (p_1,p_2,\ldots,p_d)$, $q = (q_1,q_2,\ldots,q_d)$,
and $r=(r_1,r_2,\ldots,r_d)$ be three points in $\mathbb{R}^d$, then $p$, $q$, and
$r$ are collinear if and only if
\begin{displaymath}
{\left[\sum_{i=1}^{d}(p_i-r_i)(q_i-p_i)\right]}^2 -
\left[\sum_{i=1}^{d}{(p_i-r_i)}^2\right]\left[\sum_{i=1}^{d}{(q_i-p_i)}^2\right] =
0.
\end{displaymath}
\end{lemma}
\begin{proof}
Let $a = (p_1,p_2,\ldots,p_d)$, $b = (q_1,q_2,\ldots,q_d)$, and
$c=(r_1,r_2,\ldots,r_d)$ be three points in $\mathbb{R}^d$. The points $p$, $q$,
and $r$ are collinear if and only if $r = p + \lambda (q-p)$ for some unique
$\lambda \in \mathbb{R}$, that is
\begin{align*}
	&&(p-r) + \lambda (q-p) = 0,\\
	\Rightarrow&& \forall i \in [d] \colon\, (p_i-r_i) + \lambda (q_i-p_i) = 0,\\
	\Rightarrow&& \sum_{i=1}^{d}{\left[(p_i-r_i) + \lambda (q_i-p_i)\right]}^2
	= 0,\\
	\Rightarrow&& \sum_{i=1}^{d}\left[{(q_i-p_i)}^2 \lambda^2 + 2(p_i-r_i)(q_i-p_i) \lambda +
		{(p_i-r_i)}^2\right] = 0,\\
	\Rightarrow&& \underbrace{\left[\sum_{i=1}^{d}{(q_i-p_i)}^2\right]}_{A} \lambda^2 +
		\underbrace{\left[2\sum_{i=1}^{d}(p_i-r_i)(q_i-p_i)\right]}_{B} \lambda +
		\underbrace{\left[\sum_{i=1}^{d}{(p_i-r_i)}^2\right]}_{C} = 0,\\
	\Rightarrow&& \lambda = \frac{-B \pm \sqrt{B^2-4AC}}{2A}.
\end{align*}

For $\lambda$ to exist and be unique $B^2-4AC$ must be zero.
Hence, $p$, $q$, and $r$ are collinear if and only if
\begin{displaymath}
	{\left[2\sum_{i=1}^{d}(p_i-r_i)(q_i-p_i)\right]}^2
	- 4 \left[\sum_{i=1}^{d}{(p_i-r_i)}^2\right]\left[\sum_{i=1}^{d}{(q_i-p_i)}^2\right]
	= 0.
\end{displaymath}
\end{proof}

Moreover, the improvement that we obtain in the time complexity of 3POL
can be exploited to boost the number of curves we pick the points from.
\begin{theorem}
	Let $C_1,C_2,\ldots,C_k$ be $k=o\mleft({(\log n)}^\frac{1}{6}/{(\log \log
	n)}^\frac{1}{2}\mright)$
	(not necessarily distinct)
	constant-degree polynomial curves in $\mathbb{R}^d$.
	Given $k$ $n$-sets $S_{i_j} \subset C_{i_j}$,
	deciding whether there exists any collinear
	triple of points in any triple of sets $S_{i_1} \times S_{i_2} \times
	S_{i_3}$ can be solved in subquadratic time.
\end{theorem}
\begin{proof}
	Solve a 3POL instance for each choice of $S_{i_1} \times S_{i_2} \times
	S_{i_3}$. Since there are
	$o\mleft({(\log n)}^\frac{1}{2}/{(\log \log n)}^\frac{3}{2}\mright)$
	such choices, the theorem follows.
\end{proof}
%\remark{Compare that to the clusterability results of~\cite{CL15}.}


\subsection{Incidences on Unit Circles}%
\label{sec:paper:3pol-algorithm:application:circles}

Raz, Sharir and Solymosi~\cite{RSS15} mention the following problem as a
special case of the framework they introduce.
Let $p_1,p_2,p_3$ be three distinct points in the plane, and, for $i=1,2,3$,
let $\mathcal{C}_i$ be a family of $n$ unit circles (a circle of radius $1$)
that pass through $p_i$.  Their goal is to obtain an upper bound on the number
of \emph{triple points}, which are points that are incident to a circle of each
family.
%
They prove:
\begin{theorem}
	Let $p_1,p_2,p_3$ be three distinct points in the plane, and, for $i=1,2,3$,
	let $\mathcal{C}_i$ be a family of $n$ unit circles that pass through $p_i$.
	Then the number of points incident to a circle of each family is $O(n^{11/6})$.
\end{theorem}


They observe that the following dual formulation is equivalent to their
original problem:
\begin{theorem}
	Let $C_1,C_2,C_3$ be three unit circles in $\mathbb{R}^2$, and, for each
	$i=1,2,3$, let $S_i$ be a set of $n$ points lying on $C_i$. Then the number of
	unit circles, spanned by triples of points in $S_1 \times S_2 \times S_3$, is
	$O(n^{11/6})$.
\end{theorem}


Our new algorithms indeed allow us to solve the decision version of their
problems in subquadratic time.
\begin{problem}[%
	label=problem:UCSPTUC%
]
	Let $C_1,C_2,C_3$ be three unit circles in $\mathbb{R}^2$ with centers
	$c_i$, and, for each $i=1,2,3$, let $S_i =
	\{\,(x_{i,1},y_{i,1}),(x_{i,2},y_{i,2}),\ldots,(x_{i,n},y_{i,n})\,\}$ be a
	set of $n$ points lying on $C_i$. Decide whether any triple of points
	$(p_1,p_2,p_3) \in S_1 \times S_2 \times S_3$ spans a unit circle.
\end{problem}

\begin{theorem}
	\Cref{problem:UCSPTUC} can be solved in
	$O(n^2 {(\log \log n)}^\frac{3}{2} / {(\log n)}^\frac{1}{2})$ time.
\end{theorem}

\begin{proof}
	Without loss of generality, assume all input points lie on the right
	$y$-monotone arc of their respective circle. All other seven cases can be
	handled similarly. We can also assume that no input point is the top or
	bottom vertex of its circle, rotating the plane if necessary.

	Given three points $p_1$, $p_2$, and $p_3$, let
	\begin{displaymath}
		x=\lVert p_1 - p_2 \rVert,\,\,
		X=x^2,\,\,
		y=\lVert p_1 - p_3 \rVert,\,\,
		Y=y^2,\,\,
		z=\lVert p_2 - p_3 \rVert,\,\,
		Z=z^2.
	\end{displaymath}
	Testing if the three points $p_1$, $p_2$, and $p_3$ span a unit circle
	amounts to testing whether
	\begin{displaymath}
		X^2 + Y^2 + Z^2 - 2XY - 2XZ - 2YZ + XYZ = 0.
	\end{displaymath}

	The fact that the input points lie on the right $y$-monotone arc of unit
	circles of centers $c_1,c_2,c_3$ allows us to get down to a single variable
	per point. Let $c_i = (c_i^x,c_i^y)$ and
	$t_{i,j} = \sqrt{\frac{1 - x_{i,j} + c_i^x}{1 + x_{i,j} - c_i^x}}$. Then
	the \(j\)th input point of the \(i\)th circle can be expressed as
	\begin{displaymath}
		p_{i,j} = (x_{i,j},y_{i,j}) = c_i + \left(\frac{1-t_{i,j}^2}{1+t_{i,j}^2},\frac{2
		t_{i,j}}{1+t_{i,j}^2}\right).
	\end{displaymath}

	Combining those two observations with some algebraic manipulations, one can
	show that there exists some trivariate polynomial $F$ of degree at most
	$24$ that cancels on $t_1,t_2,t_3$ when the points $c_1
	+\left(\frac{1-t_1^2}{1+t_1^2},\frac{2t_1}{1+t_1^2}\right)$, $c_2
	+\left(\frac{1-t_2^2}{1+t_2^2},\frac{2t_2}{1+t_2^2}\right)$, and $c_3
	+\left(\frac{1-t_3^2}{1+t_3^2},\frac{2t_3}{1+t_3^2}\right)$ span a unit
	circle.

	Hence,
	the polynomial \(F\)
	together with
	the sets
	$\{\,t_{1,1},t_{1,2},\ldots,t_{1,n}\,\}$,
	$\{\,t_{2,1},t_{2,2},\ldots,t_{2,n}\,\}$, and
	$\{\,t_{3,1},t_{3,2},\ldots,t_{3,n}\,\}$
	give an instance of 3POL we can solve in subquadratic
	time with our new algorithms.

	Unfortunately, the computation $\sqrt{\cdot}$ is not allowed in our model, and
	so, we cannot compute $t_{i,j}$.
	However, we can generalize the 3POL problem to make it fit:
	\begin{problem}[Modified 3POL]
		Let $F \in \mathbb{R}[x,y,z]$ be a trivariate polynomial of constant degree,
		given three sets $A$, $B$, and $C$, each containing $n$ real numbers, decide
		whether there exist $a \in A$, $b \in B$, and $c \in C$ such that
		\begin{displaymath}
			\exists t_1,t_2,t_3
			(t_1^2 = a \land
			t_2^2 = b \land
			t_3^2 = c \land
			F(t_1,t_2,t_3) = 0).
		\end{displaymath}
	\end{problem}

	Hence, the polynomial \(F\) together with
	the sets
	$\{\,t_{1,1}^2,t_{1,2}^2,\ldots,t_{1,n}^2\,\}$,
	$\{\,t_{2,1}^2,t_{2,2}^2,\ldots,t_{2,n}^2\,\}$, and
	$\{\,t_{3,1}^2,t_{3,2}^2,\ldots,t_{3,n}^2\,\}$
	give an instance of this variant of 3POL\@.
	Note that those sets are computable in our models.

	We can tweak our algorithms so that they work for this new version of
	3POL\@. We prefix each decision we make on the first-order theory of the reals
	with an existential quantifier and a condition of the type $t_i^2=x$, with
	$x$ the square of $t_i$, when we reference $t_i$ in the formula we test.
%
	This new algorithm answers positively if and only if the original problem
	contains a triple of points spanning a unit circle.

	In general, any constant-degree polynomial curve can be decomposed in
	a constant number of pieces as above. Each point on this curve can be given a
	parameterization that might involve roots of its coordinates. Those can be
	taken care of by appropriately augmenting the Tarski sentences in our
	algorithm with equations that encode those roots for free.
\end{proof}

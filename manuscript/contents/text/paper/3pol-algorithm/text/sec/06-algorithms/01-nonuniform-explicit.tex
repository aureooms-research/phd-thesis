\subsection{Nonuniform Algorithm for Explicit 3POL}%
\label{sec:algo:explicit:nonuniform}

We begin with the description of a nonuniform algorithm for explicit 3POL which
we use later as a basis for other algorithms.
%
We recall the definition of the explicit 3POL problem
%
\restate{\ProblemPOLExplicit*}

We prove the following
\restate{\TheoremPOLNonuniformExplicit*}

\paragraph{Idea}
We partition the sets $A$ and $B$ into small groups of consecutive
elements. That way, we can divide the $A\times B$ grid into cells with the
guarantee that each curve $c = f(x,y)$ intersects a small number
of those cells. For each such curve and each cell it intersects, we
search $c$ among the values $f(a,b)$ for all $(a,b)$ in a given intersected
cell. We generalize Fredman's trick~\cite{Fr76} --- and how it is used in Gr\o
nlund and Pettie's paper~\cite{GP18} --- to quickly obtain a sorted order on
those values, which provides us a logarithmic search time for each cell.
Below is a sketch of the algorithm.
\begin{algorithm}[Nonuniform algorithm for explicit 3POL]\label{algo:ne}
\item[input] $A = \{\,a_1 < \cdots < a_n\,\},B = \{\,b_1<\cdots<b_n\,\},\\
    C = \{\,c_1<\cdots<c_n\,\}
    \subset \mathbb{R}$.
\item[output] \emph{accept} if $\exists\, (a,b,c) \in A \times B \times C$ such that $c
    = f(a,b)$, \emph{reject} otherwise.%
    \footnote{Note that it is easy to modify the algorithm to count or report the
    solutions. In the latter case, the algorithm becomes output sensitive.}
\item[1.] Partition the intervals $[a_1,a_n]$ and $[b_1,b_n]$ into blocks
    $A_i^*$ and $B_j^*$ such that $A_i = A \cap A_i^*$ and $B_j = B
    \cap B_j^*$ have size $g$.
\item[2.] Sort the sets $f(A_i \times B_j) = \{\, f(a,b) \st (a,b) \in A_i
    \times B_j\,\}$ for all $A_i,B_j$. This is the only step that
    is nonuniform.
\item[3.] For each $c \in C$,
\item[3.1.] For each cell $A_i^* \times B_j^*$ intersected by the curve
$c=f(x,y)$,
\item[3.1.1.] Binary search for $c$ in the sorted set $f(A_i \times B_j)$.
If $c$ is found, \emph{accept} and halt.
\item[4.] \emph{reject} and halt.
\end{algorithm}
%
Like in Gr\o nlund and Pettie's $\tilde{O}(n^{3/2})$ decision tree for
3SUM~\cite{GP18}, the key is to give an efficient implementation
of step~\textbf{2}.

\paragraph{$A \times B$ grid partitioning}
Let $A = \{\,a_1<a_2<\cdots<a_n\,\}$ and $B = \{\,b_1<b_2<\cdots<b_n\,\}$.
For some positive integer $g$ to be determined later, partition the interval
\([a_1,a_n]\) into \(n/g\)
blocks $A_1^*,A_2^*,\ldots,A_{n/g}^*$ such that each block
contains \(g\) numbers in \(A\). Do the same for the interval
\([b_1,b_n]\) with the numbers in \(B\) and name the blocks of this
partition $B_1^*,B_2^*,\ldots,B_{n/g}^*$.
For the sake of simplicity, and without loss of generality, we assume
here that $g$ divides $n$.  We continue to make this assumption in the
following sections.
To each of the \({ ( n/g ) }^2\) pairs of blocks $A_i^*$ and $B_j^*$
corresponds a cell $A_i^* \times B_j^*$.
By definition, each cell contains $g^2$ pairs in \(A \times
B\).
For the sake of notation,
we define
$A_i = A \cap A_i^* = \{\,a_{i,1} < a_{i,2} < \cdots < a_{i,g}\,\}$
and
$B_j = B \cap B_j^* = \{\,b_{j,1} < b_{j,2} < \cdots < b_{j,g}\,\}$.
Figure~\ref{fig:partition} depicts this construction.

\begin{figure*}
\centering
    \includegraphics{figures/partition}
    \caption{The partitioning of \(A \times B\). There are \( n/g \) columns \( A_i^*
    \), \( n/g \) rows \( B_j^* \), and \( {(n/g)}^2 \) cells \( A_i^* \times
    B_j^* \). There are \( n^2 \) points in \( A \times B \).
    Each column contains the \( ng \) points in \( A_i \times B \), each row contains
    the \( ng \) points in \( A \times B_j \), and each cell contains the \(
    g^{2} \) points in \( A_i \times B_j \).}%
    \label{fig:partition}
\end{figure*}
\begin{figure*}
\centering
    \includegraphics{figures/intersected-cells}
    \caption{An $xy$-monotone arc of the two-dimensional polynomial curve of
    equation $c=f(x,y)$. This arc intersects a staircase of at most $2
    \frac ng - 1$ cells in the grid. When $f$ has constant degree, the defined
    curve can be partitioned into $O(1)$ such arcs.}%
    \label{fig:intersected-cells}
\end{figure*}
%\caption{Properties of the $A\times B$ grid.}\label{fig:axb}

The following two lemmas result from this construction:
\begin{lemma}\label{lem:intersections}
    For a fixed value \(c \in C\), the curve $c=f(x,y)$ intersects $O(\frac ng)$
    cells. Moreover, those cells can be found in $O(\frac ng)$ time.
\end{lemma}
\begin{proof}
    Since $f$ has constant degree, the curve $c=f(x,y)$ can be partitioned into
    a constant number of $xy$-monotone arcs. Split the curve into
    $x$-monotone pieces, then each $x$-monotone piece into $y$-monotone arcs.
    The endpoints of the $xy$-monotone arcs are the intersections of
    $f(x,y)=c$ with its derivatives $f'_x(x,y)=0$ and $f'_y(x,y)=0$. By
    B\'ezout's theorem, there are $O({\deg(f)}^2)$ such intersections and so
    $O({\deg(f)}^2)$ $xy$-monotone arcs. Figure~\ref{fig:intersected-cells} shows
    that each such arc intersects $O(\frac ng)$ cells since the
    cells intersected by a $xy$-monotone arc form a staircase in the
    grid.
    This proves the first part of the lemma.

    To prove the second part, observe that for each connected component of
    $c=f(x,y)$ intersecting at least one cell of the grid either: (1) it
    intersects a boundary cell of the grid, or (2) it is a (singular) point or
    contains vertical and horizontal tangency points.%
    \footnote{Note that vertical and horizontal lines fall in both categories.}
    The cells intersected by
    $c=f(x,y)$ are computed by exploring the grid from $O(\frac ng)$ starting cells.
    Start with an empty set. Find and add all boundary cells containing a point of the
    curve. Finding those cells is achieved by solving
    the Tarski sentence
    $\exists x \exists y (
    c=f(x,y)
    \land
    x \in A_i^*
    \land
    y \in B_j^*
    )$,
    for each cell $A_i^* \times B_j^*$ on the boundary.
    This takes $O(\frac ng)$ time.
    Find and add the cells containing singular points and tangency points of
    $c=f(x,y)$. Finding those cells is achieved by first finding
    the constant number of vertical and horizontal slabs $A_i^* \times
    \mathbb{R}$ and $\mathbb{R} \times B_j^*$ containing such points:
    \begin{align*}
        \exists x \exists y (
	c=f(x,y)
	\land
	( f'_x(x,y)=0 \lor f'_y(x,y)=0 )
	\land
	x \in A_i^*), \\
        \exists x \exists y (
	c=f(x,y)
	\land
	( f'_x(x,y)=0 \lor f'_y(x,y)=0 )
	\land
	y \in B_j^*).
    \end{align*}
    This takes $O(\frac ng)$ time.
    Then for each pair of vertical and horizontal slab containing such a point,
    check that the cell at the intersection of the slabs also contains such a
    point:
    \begin{displaymath}
        \exists x \exists y (
	c=f(x,y)
	\land
	( f'_x(x,y)=0 \lor f'_y(x,y)=0 )
	\land
	x \in A_i^*
	\land
	y \in B_j^*
	).
    \end{displaymath}
    This takes $O(1)$ time.
%
Note that we can always assume the constant-degree polynomials we manipulate
are square-free, as making them square-free is trivial~\cite{Y76}: since
$\mathbb{R}[x]$ and $\mathbb{R}[y]$ are unique factorization domains, let $Q =
P/\text{gcd}(P,P'_x;x)$ and $\text{sf}(P) = Q/\text{gcd}(P,P'_y;y)$, where
$\text{gcd}(P,Q;z)$ is the greatest common divisor of $P$ and $Q$ when viewed
as polynomials in $R[z]$ where $R$ is a unique factorization domain and
$\text{sf}(P)$
is the square-free part of $P$.
%
    The set now contains, for each component of each
    type, at least one cell intersected by it. Initialize a list with the
    elements of the set. While the list is not empty, remove any cell from the
    list, add each of the eight neighbouring cells to the set and the list,
    if it contains a point of $c=f(x,y)$ --- this can be checked with the same
    sentences as in the boundary case --- and if
    it is not already in the set. This costs $O(1)$ per cell intersected.
    The set now contains all cells of the grid intersected by $c=f(x,y)$.
\end{proof}

\begin{lemma}\label{lem:preprocessing}
    If the sets $A,B,C$ can be preprocessed in $S_g(n)$ time so that,
    for any given cell $A_i^* \times B_j^*$ and any given $c \in C$, testing whether
    $c \in f(A_i \times B_j) = \{\,f(a,b)\st (a,b) \in A_i \times B_j\,\}$ can be done in
    $O(\log g)$ time, then, explicit 3POL can be solved in
    $S_g(n)+O(\frac{n^2}{g}\log g)$ time.
\end{lemma}
\begin{proof}
    %The total time needed is
    We need $S_g(n)$ preprocessing time plus the time required
    to search each of the $n$ numbers $c \in C$ in each of the $O(\frac ng)$
    cells intersected by $c=f(x,y)$. Each search costs $O(\log g)$ time.
    We can compute the cells intersected by $c=f(x,y)$ in $O(\frac ng)$ time
    by Lemma~\ref{lem:intersections}.
\end{proof}
\paragraph{Remark} We do not give a $S_g(n)$-time real-RAM
algorithm for preprocessing the input, but only a $S_g(n)$-depth bounded-degree
ADT\@. In fact, this preprocessing step is the only
nonuniform part of Algorithm~\ref{algo:ne}. A real-RAM implementation of this
step is given in
\S\ref{sec:algo:explicit:uniform}.

\begin{figure*}
\centering
\includegraphics{figures/ab-pairs}
\caption{For each cell, we sort the points it contains with comparisons.
The points \((a,b)\) and \((a',b')\) are compared using the comparison
\( f(a,b) \le f(a',b') \).}%
\label{fig:ab-pairs}
\end{figure*}
\begin{figure*}
\centering
\includegraphics{figures/zero-set}
\caption{%
The disk is the semi-algebraic set \( \{\, (x,y) \st f(x,b) \le f(y,b')\,\} \).
Here $(a,a')$ lies outside this semi-algebraic set which implies that $f(a,b) >
f(a',b')$.}%
\label{fig:zero-set}
\end{figure*}
%\caption{Generalization of Fredman's trick (Lemma~\ref{lem:fredman}).}\label{fig:trick}
\paragraph{Preprocessing} All that is left to prove is that $S_g(n)$ is
subquadratic for some choice of $g$. To achieve this we sort the points
inside each cell using Fredman's trick~\cite{Fr76}. Gr\o nlund and
Pettie~\cite{GP18} use this trick to sort the sets
$A_i + B_j = \{\,a + b \st (a,b) \in A_i \times B_j \,\}$
with few comparisons: sort the set
$D = (\cup_i [A_i - A_i]) \cup (\cup_j [B_j - B_j])$,
where
$A_i - A_i = \{\,a - a' \st (a,a')\in A_i \times A_i \,\}$
and
$B_j - B_j = \{\,b - b' \st (b,b')\in B_j \times B_j \,\}$,
using $O(n\log n + |D|)$ comparisons, then testing whether
$a + b \le a' + b'$
can be done using the free (already computed) comparison
$a - a' \le b' - b$.
We use a generalization of this trick to sort the sets $f(A_i\times B_j)$.
For each $B_j$, for each pair \((b,b') \in B_j \times B_j\),
define the curve
%\begin{displaymath}
$
	\gamma_{b,b'} = \{\,(x,y) \st f(x,b) = f(y,b')\,\}.
$
%\end{displaymath}
Define the sets
$\gamma^0_{b,b'} = \gamma_{b,b'}$,
$\gamma^-_{b,b'} = \{\,(x,y) \st f(x,b) < f(y,b')\,\}$, and
$\gamma^+_{b,b'} = \{\,(x,y) \st f(x,b) > f(y,b')\,\}$.
%
The following lemma --- illustrated by
Figures~\ref{fig:ab-pairs}~and~\ref{fig:zero-set} --- follows by definition:
\begin{lemma}\label{lem:fredman}
Given a cell $A_i^* \times B_j^*$ and two pairs \((a,b), (a',b') \in A_i \times B_j\),
deciding whether \(f(a,b) < f(a',b')\) (respectively
$f(a,b) = f(a',b')$
and
$f(a,b) > f(a',b')$)
amounts to deciding whether the point
$(a,a')$ is contained in \(\gamma^-_{b,b'}\) (respectively
\(\gamma^0_{b,b'}\) and
\(\gamma^+_{b,b'}\)).
\end{lemma}

There are $N \coloneqq \frac ng \cdot g^2 = ng$ pairs $(a,a') \in \cup_i [A_i \times
A_i]$ and there are $N$ pairs $(b,b') \in \cup_j [B_j \times B_j]$.  Sorting
the $f(A_i\times B_j)$ for all $(A_i, B_j)$ amounts to solving the
following problem:
\begin{problem}[Polynomial Batch Range Searching]
    Given $N$ points and $N$ polynomial curves in $\mathbb{R}^2$, locate
    each point with respect to each curve.
\end{problem}

We can now refine the description of step~\textbf{2} in Algorithm~\ref{algo:ne}
\begin{algorithm}[Sorting the $f(A_i \times B_j)$ with a nonuniform algorithm]\label{algo:sfaixbj}
\item[input] $A = \{\,a_1<a_2<\cdots<a_n\,\},B = \{\,b_1<b_2<\cdots<b_n\,\}
    \subset \mathbb{R}$
\item[output] The sets $f(A_i \times B_j)$, sorted.
\item[2.1.] Locate each point $(a,a') \in \cup_i [A_i \times A_i]$ w.r.t.\
    each $\gamma_{b,b'}, (b,b') \in \cup_j [B_j \times B_j]$.
%\footnote{See Algorithm~\ref{algo:pbrs} for the implementation.}
\item[2.2.] Sort the sets $f(A_i \times B_j)$ using the
    information retrieved in step \textbf{2.1}.
\end{algorithm}
Note that this algorithm is nonuniform: step~\textbf{2.2} costs at least
quadratic time in the real-RAM model, however, this step does not need to
query the input at all, as all the information needed to sort is retrieved during
step~\textbf{2.1}.
%Indeed,
Step~\textbf{2.2} incurs no cost in
our nonuniform model.
%the bounded-degree ADT model.

To implement step~\textbf{2.1}, we use a modified\footnote{The original
algorithm relies on hierarchical cuttings which cannot be
implemented in the bounded-degree ADT model.}
version of the $N^{\frac{4}{3}} 2^{O(\log^* N)}$ algorithm of
Matou\v{s}ek~\cite{Ma93} for Hopcroft's problem.
In \S~\ref{sec:algo:point-curves-location}, we prove the following upper bound:
\begin{lemma}\label{lem:batch}
Polynomial Batch Range Searching can be solved in
$O(N^{\frac{4}{3}+\varepsilon})$ time in the real-RAM model when the input
curves are the $\gamma_{b,b'}$.
\end{lemma}

%Note that we do not get the nicer $2^{O(\log^*N)}$ factor because of the
%limitations of the tools we use.

\paragraph{Analysis}
Combining Lemma~\ref{lem:preprocessing} and Lemma~\ref{lem:batch} yields a $O(
{(ng)}^{4/3+\varepsilon} + n^2 \log g / g)$-depth bounded-degree ADT for
explicit 3POL\@. By optimizing over $g$, we get $g = \Theta(n^{2/7-\varepsilon})$, and
the previous expression simplifies to $O(n^{12/7+\varepsilon})$, proving
\Cref{thm:explicit:nonuniform}.

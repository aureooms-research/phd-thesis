\subsection{Uniform Algorithm for 3POL}%
\label{sec:algo:implicit:uniform}
%\improve{Section 3.6 could benefit of some more structure.}
%\improve{This section needs more details.}

In this section, we combine the uniform algorithm for explicit 3POL given in
\S\ref{sec:algo:explicit:uniform} with the nonuniform algorithm for
3POL given in \S\ref{sec:algo:implicit:nonuniform} to obtain a uniform
subquadratic algorithm for 3POL\@.

We recall the definition of the 3POL problem.
\ProblemPOLImplicit*

We prove the following:
\TheoremPOLUniformImplicit*

\paragraph{Idea}
In the uniform algorithm for explicit 3POL of
\S\ref{sec:algo:explicit:uniform}, we partition the set $A \times B$
into very small sets $A_i \times B_j$, sort the sets $f(A_i \times B_j)$ using
the dominance reporting algorithm of \S\ref{sec:algo:dominance} then
binary search on those sorted sets in order to find a matching $c$.
%
Here we devise a similar scheme with the only difference that the sets to sort
are the unions of the real roots of the univariate polynomials
$F(a,b,z)\in\mathbb{R}[z]$ over all $(a,b) \in A_i \times B_j$.
%
The main difficulty resides in implementing the
equivalent of the certificates of \S\ref{sec:algo:explicit:uniform} to reuse
the dominance reporting algorithm of \S\ref{sec:algo:dominance}. We show how to
implement those certificates using the $\gamma_{b,b'}$ and $\delta_b$ curves
defined in \S\ref{sec:algo:implicit:nonuniform}.

\paragraph{$A \times B$ partition}
We use the same partitioning scheme as all previous
algorithms, hence Lemma~\ref{lem:intersections-implicit} and
Lemma~\ref{lem:preprocessing-implicit} hold. We apply the same certificate verification
scheme as in \S\ref{sec:algo:explicit:uniform}, hence, the dominance
reporting algorithm of \S\ref{sec:algo:dominance} and the analysis
in \S\ref{sec:algo:explicit:uniform} still apply.

\paragraph{Preprocessing}
The preprocessing algorithm is essentially the same as
Algorithm~\ref{algo:sfaixbj-uniform} with more complex certificates. We explain how to
construct those new certificates. The first part of the explanation consists in
generalizing the definition of a certificate. The rest of the
explanation focuses on the implementation of the verification of those
certificates via Polynomial Dominance Reporting.

\paragraph{The certificates}
For a fixed pair $(a,b)$, $F(a,b,z) \in \reals[z]$ is a polynomial in $z$ of
degree at most $\deg(F)$. Hence, $F(a,b,z)$ has at most $\deg(F)$ real roots.
For each cell $A_i^* \times B_j^*$, let
\begin{displaymath}
	A_i\times B_j=\{\,
		(a_{i,1},b_{j,1}),
		(a_{i,1},b_{j,2}),
		\ldots,
		(a_{i,2},b_{j,1}),
		(a_{i,2},b_{j,2}),
		\ldots,
		(a_{i,g},b_{j,g})
	\,\}.
\end{displaymath}
Let $\rho
\colon {[g]}^2 \to \{\,0,1,\ldots,\deg(F)\,\}$ be a function that maps a pair $(k,l)$ to the
number of real roots of $F(a_{i,k},b_{j,l},z)$. Let $\Sigma_\rho = \sum_{(i,j)
\in {[g]}^2} \rho(i,j) \le \deg(F) g^2$.
Given a function $\rho$, let
$\pi\colon\,[\Sigma_\rho]\to {[g]}^2 \times
\{\,0,1,\ldots,\deg(F)\,\}$ be a permutation of the union of the real roots of all
$g^2$ polynomials
\begin{displaymath}
F(a_{i,1},b_{j,1},z),
F(a_{i,1},b_{j,2},z),
\ldots,
F(a_{i,2},b_{j,1},z),
%F(a_{i,2},b_{j,2},z),
\ldots,
F(a_{i,g},b_{j,g},z),
\end{displaymath}
where the number of real roots of each polynomial is
prescribed by $\rho$.
Decompose $\pi = (\pi_r,\pi_c,\pi_s)$ into
row,
column and real root number functions $\pi_r,\pi_c\colon\,[\Sigma_\rho]\to[g]$,
and $\pi_s\colon\,[\Sigma_\rho]\to\{\,0,1,\ldots,\deg(F)\,\}$.
Let $\diamond(a,b,s)$ denote the $s$th real root of $F(a,b,z)$.
To fix the permutation of the union of the real roots of all
$g^2$ polynomials, we define the following
interleaving certificate with $\Sigma_\rho - 1$ inequalities, for each possible
function $\rho$ and permutation $\pi$
\begin{displaymath}
	\Phi_{\rho,\pi} =
	\diamond(a_{i,\pi_r(1)},b_{j,\pi_c(1)},\pi_s(1))
	\le
	%\diamond(a_{i,\pi_r(2)},b_{j,\pi_c(2)},\pi_s(2))
	%\le
	\cdots
	\le
	\diamond(a_{i,\pi_r(\Sigma_\rho)},b_{j,\pi_c(\Sigma_\rho)},\pi_s(\Sigma_\rho)).
\end{displaymath}
To fix the number of real roots each of the $g^2$ polynomials can have, we
define the following cardinality certificate for each function $\rho$
\begin{displaymath}
	\Psi_{\rho} =
	\bigwedge_{(k,l)\in{[g]}^2}\,\,
	F(a_{i,k},b_{j,l},z)\,\,\text{has $\rho(k,l)$ real roots}.
\end{displaymath}
For each possible function $\rho$ and permutation $\pi$ we define the
certificate $\Upsilon_{\rho,\pi} = \Psi_\rho \land \Phi_{\rho,\pi}$ that
fixes both the number of real roots each polynomial has and the permutation of
those real roots.
The total number of certificates $\Upsilon_{\rho,\pi}$ is
$\sum_{\rho\colon {[g]}^2 \to \{\,0,1,\ldots,\deg(F)\,\}} {\Sigma_\rho!}$
which is of the order of ${(g^2)}^{O(g^2)}$.

Finally, we need to handle the edge cases where a polynomial $F(a,b,z)$ is the
zero polynomial. In that case, $F(a,b,z)$ cancels for all $z \in
\mathbb{R}$. Hence, all planar curves $F(x,y,c)=0$ go through $(a,b)$ and we
can immediately accept the 3POL instance. To capture
those edge cases, we will check the following certificate before running the main
algorithm
\begin{displaymath}
	\aleph = \bigvee_{(k,l)\in{{[g]}^2}} F(a_{i,k},b_{j,l},z)\,\,\text{is the zero polynomial}.
\end{displaymath}

We can check if $\aleph$ holds for any cell $A_i \times B_j$ in $O(n \log n)$ time.
For each $b \in B$ binary search for a $a \in A$ that lies on a vertical line
component of $\delta_b$.

If this certificate is verified we accept and halt. Otherwise we can safely run
the main algorithm.

\paragraph{$A \times A$ $(b,b')$-partitions}
For each $B_j$ and for each $(b,b') \in B_j^2$ compute a partition of the
$A \times A$ grid according to the $(b,b')$-cells defined by $\Gamma_{b,b'}$
--- see \S\ref{sec:algo:implicit:nonuniform}. For each $(b,b')$-cell of
that partition, pick a sample point $(a,a')$, compute the interleaving
$\mathcal{I}((F(a,b,z),F(a',b',z)))$. Store that information for future lookup.
All this takes $O(ng)$ time.

\paragraph{PDR instance for $\Psi_\rho$}
For a fixed pair $(a,b)$, suppose $F(a,b,z)$ has $r$ real roots. Then $a$ must
lie in one of the open intervals or be one of the breaking points defined by
the VTP, SIP and DL of $\delta_b$ that fixes the number of real roots of
$F(a,b,z)$ to $r$.
Hence $\Psi_\rho$ can be rewritten as follows
\begin{displaymath}
	\Psi_{\rho} =
	\bigwedge_{(k,l)\in{[g]}^2}\,\,
	\left(\bigvee_{[u,v]\in \mathcal{I}_{\rho(k,l)}} u < a_{i,k} < v\right)
	\bigvee
	\left(\bigvee_{w\in \mathcal{B}_{\rho(k,l)}} a_{i,k} = w\right)
\end{displaymath}
where $\mathcal{I}_{\rho(k,l)}$ denotes the set of intervals fixing the number
of real roots of $F(a_{i,k},b_{j,l},z)$ to $\rho(k,l)$, and
$\mathcal{B}_{\rho(k,l)}$ denotes the set of breaking points fixing the number
of real roots of $F(a_{i,k},b_{j,l},z)$ to $\rho(k,l)$.

The PDR algorithm can only check conjunctions of polynomial inequalities.
However, we can transform $\Psi_{\rho}$ into disjunctive normal form~(DNF) by
splitting the certificate into distinct branches, each consisting of a
conjunction of polynomial inequalities.
Since the number of intervals and breaking points
considered above is constant for each pair $(k,l)$, the number of branches to
test is $2^{O(g^2)}$.

For each $A_i$ we have thus a single vector of reals
\begin{displaymath}
	p_i = (
		a_{i,1} , a_{i,1},
		a_{i,2}, a_{i,2},
		\ldots,
		a_{i,g} , a_{i,g}
	),
\end{displaymath}
and for each $B_j$ we have $2^{O(g^2)}$ vectors of linear inequalities
\begin{displaymath}
	q_j = (
		x \bowtie_{u_{1,1}} u_{1,1}, x \bowtie_{v_{1,1}} v_{1,1},
		x \bowtie_{u_{1,2}} u_{1,2}, x \bowtie_{v_{1,2}} v_{1,2},
		\ldots,
		x \bowtie_{u_{g,g}} u_{g,g}, x \bowtie_{v_{g,g}} v_{g,g},
	),
\end{displaymath}
where each $(\bowtie_{u_{k,l}}, u_{k,l}, \bowtie_{v_{k,l}}, v_{k,l})$ is an element of
\begin{displaymath}
	\{\, ( > , u , < , v ) \st (u,v) \in \mathcal{I}_{\rho(k,l)}\,\}
	\cup
	\{\, ( = , w , = , w ) \st w \in \mathcal{B}_{\rho(k,l)}\,\}.
\end{displaymath}

For a fixed function $\rho$,
the sets of vectors $p_i$ and $q_j$ is a valid PDR instance
of size $N = (n/g) \cdot 2^{O(g)} = n2^{O(g)}$ and with parameter $k = 2g^2$
that will
output all cells $A^*_i \times B^*_j$ such that $F(a_{i,k},b_{j,l},z) \in
\mathbb{R}[z]$ has exactly $\rho(k,l)$ real roots for all $(a_{i,k},a_{j,l}) \in A_i
\times B_j$.

\paragraph{PDR instance for $\Phi_{\rho,\pi}$}

For fixed pairs $(a,b)$ and $(a',b')$, suppose the $s$-th real root of
$F(a,b,z)$ is smaller or equal to the $q$-th real root of $F(a,b,z)$. Then,
$(a,a')$ must lie in a $(b,b')$-cell that orders the $s$-th root of $F(x,b,z)$
before the $q$-th root of $F(y,b',z)$ for all points $(x,y)$ in that cell.

Hence $\Phi_{\rho,\pi}$ can be rewritten as follows
\begin{displaymath}
	\Phi_{\rho,\pi} =
	\bigwedge_{t\in[\Sigma_\rho-1]}\,\,
	\bigvee_{C \in \mathcal{C}_{\rho,\pi,t}}
	( a_{i,\pi_r(t)}, a_{i,\pi_r(t+1)} ) \in C
\end{displaymath}
where $\mathcal{C}_{\rho,\pi,t}$ denotes the set of $(b,b')$-cells
fixing to $\rho(\pi_r(t),\pi_c(t))$ the number of real roots of
$F(a_{i,\pi_r(t)},b_{j,\pi_c(t)},z)$,
fixing to $\rho(\pi_r(t+1), \pi_c(t+1))$
the number of real roots of $F(a_{i,\pi_r(t+1)},b_{j,\pi_c(t+1)},z)$,
and ordering the $\pi_s(t)$-th root of $F(a_{i,\pi_r(t)},b_{j,\pi_c(t)},z)$
before the $\pi_s(t+1)$-th root of $F(a_{i,\pi_r(t+1)},b_{j,\pi_c(t+1)},z)$.

The PDR algorithm can only check conjunctions of polynomial inequalities.
However, we can transform $\Phi_{\rho,\pi}$ in DNF as we did for $\Psi_{\rho}$.
Again the number of cells considered above is constant for each $t$, the
description of each cell is constant, hence, the number of branches to test is
$2^{O(g^2)}$.

For each $A_i$ we have thus a single vector of $2$-dimensional points
\begin{displaymath}
	p_i = (
		\underbrace{%
			\ldots,
			(a_{i,\pi_r(1)},a_{i,\pi_r(2)}),
			%(a_{i,\pi_r(1)},a_{i,\pi_r(2)}),
			\ldots%,
			%(a_{i,\pi_r(1)},a_{i,\pi_r(2)})
		}_{\omega},
		%\underbrace{
			%(a_{i,\pi_r(2)},a_{i,\pi_r(3)}),
			%(a_{i,\pi_r(2)},a_{i,\pi_r(3)}),
			%\ldots,
			%(a_{i,\pi_r(2)},a_{i,\pi_r(3)})
		%}_{\omega},
		\ldots,
		\underbrace{%
			\ldots,
			(a_{i,\pi_r(\Sigma_\rho-1)},a_{i,\pi_r(\Sigma_\rho)}),
			%(a_{i,\pi_r(\Sigma_\rho-1)},a_{i,\pi_r(\Sigma_\rho)}),
			\ldots%,
			%(a_{i,\pi_r(\Sigma_\rho-1)},a_{i,\pi_r(\Sigma_\rho)})
		}_{\omega}
	),
\end{displaymath}
where $\omega$ is the size of the largest description of a $(b,b')$-cell $C$,
and for each $B_j$ we have $2^{O(g^2)}$ vectors of polynomial inequalities,
\begin{displaymath}
	q_j = (
		\underbrace{%
		\ldots,
		h_{1,\vartheta}(x,y) \bowtie_{1,\vartheta} 0,
		\ldots%
	}_{\vartheta \in [\omega]}
		,\ldots,
		%h_{1,\omega}(x,y) \bowtie_{1,\omega} 0,
		%h_{2,1}(x,y) \bowtie_{2,1} 0,
		%\ldots,
		%h_{2,\omega}(x,y) \bowtie_{2,\omega} 0,
		%\ldots,
		%h_{\Sigma_\rho-1,1}(x,y) \bowtie_{\Sigma_\rho-1,1} 0,
		\underbrace{%
		\ldots,
		h_{\Sigma_\rho-1,\vartheta}(x,y) \bowtie_{\Sigma_\rho-1,\vartheta} 0
		\ldots%
	}_{\vartheta \in [\omega]}
	),
\end{displaymath}
where each $(\ldots, h_{t,\vartheta}(x,y) \bowtie_{t,\vartheta} 0, \ldots)$
is an element of
$\{\, \text{desc}(C) \st C \in \mathcal{C}_{\rho,\pi,t}\,\}$,
where $\text{desc}(C)$ is the description of the cell $C$ given as a
certificate of belonging to $C$ in the form of a Tarski sentence.
The description of each $(b,b')$-cell is padded with
its last component so that it has length $\omega$.

For a fixed function $\rho$,
for a fixed function $\pi$,
the sets of vectors $p_i$ and $q_j$ is a valid PDR instance
of size $N = n 2^{O(g)}$ and with parameter $k = \Theta(g^2)$
that will
output all cells $A^*_i \times B^*_j$ such that
the number of real roots of $F(a_{i,\pi_r(t)},b_{j,\pi_c(t)},z)$
is $\rho(\pi_r(t),\pi_c(t))$,
the number of real roots of $F(a_{i,\pi_r(t+1)},b_{j,\pi_c(t+1)},z)$
is \\ $\rho(\pi_r(t+1),\pi_c(t+1))$,
and the $\pi_s(t)$-th root of $F(a_{i,\pi_r(t)},b_{j,\pi_c(t)},z)$
comes before the $\pi_s(t+1)$-th root of
$F(a_{i,\pi_r(t+1)},b_{j,\pi_c(t+1)},z)$,
for all $t \in [\Sigma_\rho-1]$.

\paragraph{PDR instance for $\Upsilon_{\rho,\pi}$}

We can combine the certificates given above for $\Psi_\rho$ and
$\Phi_{\rho,\pi}$ to obtain the ones for $\Upsilon_{\rho,\pi}$: concatenate the
$p_i$ and $q_j$ together (add a dummy $y$ variable for the $p_i$ and $q_j$ of
$\Psi_\rho$).
%
For a fixed function $\rho$,
for a fixed function $\pi$,
the sets of vectors $p_i$ and $q_j$ is a valid PDR instance
of size $N = n 2^{O(g)}$ and with parameter $k = \Theta(g^2)$
that will
output all cells $A^*_i \times B^*_j$ such that
$F(a_{i,k},b_{j,l},z) \in
\mathbb{R}[z]$ has exactly $\rho(k,l)$ real roots for all $(a_{i,k},a_{j,l}) \in A_i
\times B_j$,
and the $\pi_s(t)$-th root of $F(a_{i,\pi_r(t)},b_{j,\pi_c(t)},z)$
comes before the $\pi_s(t+1)$-th root of $F(a_{i,\pi_r(t+1)},b_{j,\pi_c(t+1)},z)$
for all $t \in [\Sigma_\rho-1]$.
%
The rest of the analysis in \S\ref{sec:algo:explicit:uniform} applies.
This proves Theorem~\ref{thm:implicit:uniform}.

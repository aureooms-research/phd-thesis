\subsection{Nonuniform Algorithm for 3POL}%
\label{sec:algo:implicit:nonuniform}

In this section, we extend the nonuniform algorithm given for explicit 3POL in
\S\ref{sec:algo:explicit:nonuniform} to work for the more general 3POL
problem.

We recall the definition of the 3POL problem.
\ProblemPOLImplicit*

We prove the following:
\TheoremPOLNonuniformImplicit*

\paragraph{Idea}
The idea is the same as for explicit 3POL\@. Partition the plane
into $A^*_i \times B^*_j$ cells.
Note that for a fixed $c \in C$, the curve $F(x,y,c)$ intersects $O(\frac ng)$
cells $A^*_i \times B^*_j$. The algorithm is the following: (1) for each cell
$A^*_i \times B^*_j$, sort the real roots of the $F(a,b,z)
\in \mathbb{R}[z]$ taking the union over all $(a,b) \in A_i \times B_j$,
(2) for each $c \in C$,
for each cell $A^*_i \times B^*_j$ intersected by $F(x,y,c)$, binary search on
the sorted order computed in step (1) to find $c$. Step (2) costs $O(n^2
\log g / g)$. It only remains to implement step (1) efficiently.

\paragraph{$A \times B$ partition}
We use the same partitioning scheme as before. Hence, counterparts of
Lemma~\ref{lem:intersections}
and
Lemma~\ref{lem:preprocessing}
hold
\begin{lemma}\label{lem:intersections-implicit}
    For a fixed \(c \in C\), the curve $F(x,y,c)=0$ intersects $O(\frac ng)$ cells.
    Moreover, those cells can be computed in $O(\frac ng)$ time.
\end{lemma}
\begin{lemma}\label{lem:preprocessing-implicit}
    If the sets $A,B,C$ can be preprocessed in $S_g(n)$ time so that,
    for any given cell $A^*_i \times B^*_j$ and any given $c \in C$, testing whether
    $c \in \{\, z \colon\, \exists(a,b) \in A_i \times B_j~\text{such that}~F(a,b,z) = 0\,\}$
    %$F(a,b,c)=0$ for some $(a,b) \in A_i \times B_j$
    can be done in
    $O(\log g)$ time, then, 3POL can be solved in
    $S_g(n)+O(\frac{n^2}{g}\log g)$ time.
\end{lemma}

\paragraph{Interleavings}
Let $\mathcal{P} = (P_1,P_2,\ldots,P_m)$ be a tuple of $m$ univariate nonzero
polynomials.
Let $\{\,p_{i,1} < \cdots < p_{i,\Delta_i}\,\}$
be the set of real roots of $P_i$ (\emph{without} multiplicities).
Let $I = ((i_1,j_1),\ldots,(i_{\Delta},j_\Delta))$
be a tuple of pairs of positive integers.
%
We say that $\mathcal{P}$ realizes $I$ if and only if
$I$ is a permutation of
$\{\,(i,j) \st i \in [m], j \in [\Delta_i]\,\}$, and
for all $t \in [\Delta-1]$, $p_{i_t,j_t} \le p_{i_{t+1},j_{t+1}}$.
When used in this context, we call $I$ an \emph{interleaving}.
%
Note that (1) the first condition implies $\Delta=\sum_{i=1}^{m}\Delta_i$,
(2) a tuple of polynomials realizes at least one interleaving,
(3) a tuple of polynomials realizes more than one interleaving if some of the
polynomials have common real roots.
%
We denote by $\mathcal{I}(\mathcal{P})$ the set of interleavings realized by
$\mathcal{P}$.


\paragraph{$A \times A$ $(b,b')$-partitions}
For a fixed pair $(b,b') \in B\times B$, we partition $\mathbb{R}^2$ into
$(b,b')$-cells that encode equivalence classes. Each cell $\mathcal{C}$ is
mapped to a unique interleaving $I$, and if we take any two points $(a_1,a_1')$
and $(a_2,a_2')$ inside $\mathcal{C}$,
\(I\) is realized by
both $(F(a_1,b,z),F(a_1',b',z))$ and
$(F(a_2,b,z),F(a_2',b',z))$.
%
It is possible that a degenerate case arises where we cannot associate an
interleaving to \(\mathcal{C}\) because one of the polynomials is the zero
polynomial. We can easily tackle these degeneracies,
because, if any point \((a,a')\) is contained in such a cell,
we can immediately answer the instance with the affirmative.
%
Identifying the interleaving associated with each cell of each
$(b,b')$-partition, then locating each $(a,a')$ inside each $(b,b')$-partition
provides the answers to all questions of the form ``Is the $k$th real root of
$F(a,b,z)$ greater than the $\ell$th real root of $F(a',b',z)$?'', for some $(a,b),
(a',b')\in A_i\times B_j$. Those answers are all we need to binary
search for $c$ in the union of the real roots of the $F(a,b,z) \in
\mathbb{R}[z]$ in time $O(\log g)$. Note again that in the nonuniform setting,
we do not sort the roots explicitly, but we must be able to recover the order
from the previous computation steps.

\paragraph{$\gamma_{b,b'}$ and $\delta_{b}$ curves}
We consider the set of interleavings $\mathcal{I}$ that $(F(x,b,z),F(y,b',z))$
realizes, where $z$ is a variable, and $x$ and $y$ are parameters.
We identify four types of event that can happen when the parameters $x$ and $y$
vary continuously (ignoring zero polynomials):
(1) a real root of \(P_i\) and a real root of \(P_j\) that were previously
distinct become equal, for some \(P_i\) and \(P_j\) in \(\mathcal{P}\),
(2) a real root of \(P_i\) and a real root of \(P_j\) that were previously
equal become distinct, for some \(P_i\) and \(P_j\) in \(\mathcal{P}\),
(3) a real root appears in one of the polynomials,
and (4) a real root disappears in one of the polynomials.
Note that many of those events can happen concurrently.
By definition of an interleaving, those events are the only ones that can cause
$\mathcal{I}$ to change.

To handle events of the types (1) and (2), we redefine the curves
$\gamma_{b,b'}$ from \S\ref{sec:algo:explicit:nonuniform}:
\begin{displaymath}
	\gamma_{b,b'} = \{\, (x,y) \st \exists z\ \text{such that}\
	F(x,b,z)=F(y,b',z)=0 \,\},
\end{displaymath}
that is, $(a,a') \in \gamma_{b,b'}$ if and only if $F(a,b,z)$ and $F(a',b',z)$
have at least one common root.%
\footnote{Note that Raz, Sharir, and de Zeeuw~\cite{RSZ15} use the same points
and curves.}
Note that this curve is defined
by the equation
$\res(F(x,b,z),F(y,b,z);z) = 0$,
that is, the set of pairs $(x,y)$ for which the resultant (in $z$) of
$F(x,b,z)$ and $F(y,b,z)$ vanishes. This resultant is a polynomial in
$\mathbb{R}[x,y]$ of degree $O(\deg(F)^2)$ and can be computed in constant time~\cite{CLO07}.
The following lemma follows by continuity of the manipulated curve:
\begin{lemma}\label{lem:cont}
    Let $(a_1,a'_1)$ and $(a_2,a'_2)$ be two points in the plane such
    that there does not exist an interleaving that both
    $(F(a_1,b,z),F(a'_1,b',z))$ and
    $(F(a_2,b,z),F(a'_2,b',z))$ realize. Moreover, suppose that those two
    points belong to a connected region in the plane such that for any point
    $(a,a')$ in that region, the number of real roots of $F(a,b,z)$ and
    $F(a',b',z)$ is fixed (and finite).
    Then the interior of any continuous path from $(a_1,a'_1)$ to $(a_2,a'_2)$ lying in this
    connected region must intersect $\gamma_{b,b'}$.
\end{lemma}
\begin{proof}
    Let $I_1$ be an interleaving realized by
    $(F(a_1,b,z),F(a'_1,b',z))$ and let $I_2$ be an interleaving realized
    by $(F(a_2,b,z),F(a'_2,b',z))$.
    Because the number of real roots of the polynomials $F(x,b,z)$ and
    $F(y,b',z)$ is fixed for any point $(x,y)$ lying in the connected region,
    $I_1$ and $I_2$ differ by a nonzero number of swaps.
    Moreover, by contradiction,
    there is a swap that is common to every choice of $I_1$
    and $I_2$.
    Since there is a common swap,
    for some $i,j \in [\deg(F)]$ and without loss of generality,
    the $i$th root of
    $F(a_1,b,z)$ is smaller than the $j$th root of $F(a'_1,b',z)$ whereas the
    $i$th root of $F(a_2,b,z)$ is larger than the $j$th root of
    $F(a'_2,b',z)$. By continuity, on any continuous path from
    $(a_1,a'_1)$ and $(a_2,a'_2)$ there is a point $(a,a')$ such that the
    $i$th root of $F(a,b,z)$ is equal to the $j$th root of $F(a',b',z)$.
    This point cannot be an endpoint of the path, hence,
    the interior of the path intersects $\gamma_{b,b'}$.
\end{proof}
The contrapositive states that, if there exists a continuous path from
$(a_1,a'_1)$ to $(a_2,a'_2)$ whose interior does not intersect the curve
$\gamma_{b,b'}$, then there exists an interleaving realized by both
$(F(a_1,b,z),F(a'_1,b',z))$ and $(F(a_2,b,z),F(a'_2,b',z))$.

To handle events of the types (3) and (4), we define the curve
\begin{displaymath}
	\delta_b = \{\,(x,z) \st F(x,b,z) = 0\,\},
\end{displaymath}
which lies in the $xz$-plane.
\begin{lemma}
    We can partition the $x$ axis of the $xz$-plane into a constant number of
    intervals so that for each interval the number of real roots of
    $F(a,b,z)$ is fixed for all $a$ in this interval.
\end{lemma}
\begin{proof}
    We partition the $xz$-plane into a constant number of vertical
    slabs and lines.
    %
    The $x$ coordinates of vertical tangency points and
    singular points of $\delta_b$ are the values $a$ for which
    a real root of $F(a,b,z)=0$ appears or disappears.
    %
    The number of singular and vertical tangency points of $\delta_b$ is quadratic in $\deg(F)$.
    %The number of vertical tangency points of $\delta_b$ is quadratic in $\deg(F)$.
    %The number of singular points of $\delta_b$ is at most ${\deg(F)}^2$.
    %
    For each of those points, draw a vertical line that contains the point.
    Those vertical lines partition the $xz$-plane into slabs and lines.
    %
    The number of
    vertical lines we draw is constant because the degree of $F$ is
    constant.
    %
    Figure~\ref{fig:deltab} depicts this drawing.
    %
    The projection of the vertical lines on the $x$ axis produce the desired
    partition (with roughly half of the intervals being singletons).
    %
    Let us name those lines $\delta_b$-lines for further reference.
\end{proof}
We can do a symmetric construction for $F(y,b',z)$ in the $zy$-plane and
get horizontal $\delta_{b'}$-lines.
\begin{lemma}
    We can partition the $y$ axis of the $zy$-plane into a constant number of
    intervals so that for each interval the number of real roots of
    $F(a',b',z)$ is fixed for all $a'$ in this interval.
\end{lemma}

\begin{figure*}
\centering
\includegraphics[trim={-0.11in 0 -0.11in 0},clip]{figures/deltab}
\caption{The vertical tangency points (VTP), self-intersection points
(SIP) and degenerate lines (DL) of $\delta_b$ partition the $A$
axis into intervals. For all $x$ of the same interval, the
polynomial $F(x,b,z) \in \mathbb{R}[z]$ has a fixed number of real
roots.}\label{fig:deltab}
\end{figure*}

\begin{figure*}
\centering
\includegraphics{figures/cells}
\caption{Cells obtained after partitioning the plane using
the curve $\gamma_{b,b'}$ and the
$\delta_b$ and $\delta_{b'}$-lines. The five darkened regions highlight examples of
$(b,b')$-cells.}\label{fig:cells}
\end{figure*}

%\caption{Constructions using the $\gamma_{b,b'}$ and $\delta_b$ curves.}\label{fig:delta}

\paragraph{Cells of the $(b,b')$-partition}
For a given $(b,b') \in B^2$, let $\Gamma_{b,b'}$ be the set containing
the curve $\gamma_{b,b'}$, the vertical
$\delta_b$-lines and the horizontal $\delta_{b'}$-lines.
The arrangement $\mathcal{A}(\Gamma_{b,b'})$ of those
constant-degree polynomial curves partitions $\mathbb{R}^2$ into a
constant-size set $\mathcal{C}(\Gamma_{b,b'})$ of \emph{$(b,b')$-cells}.
Let $P = \cup_{\gamma \in \Gamma_{b,b'}} \gamma$ and
$\mathcal{C} = \emptyset$. Add all
vertices of $\mathcal{A}(\Gamma_{b,b'})$ to
$\mathcal{C}$. Add each connected component of $P \setminus \mathcal{C}$ to
$\mathcal{C}$. Add each connected component of $\mathbb{R}^2 \setminus P$ to
$\mathcal{C}$. Finally $\mathcal{C}(\Gamma_{b,b'}) = \mathcal{C}$.
Those cells
can be connected regions,
pieces of the curve $\gamma_{b,b'}$, pieces of the $\delta_b$- and
$\delta_{b'}$-lines (vertical and horizontal line segments), and
intersections and self-intersection of those curves (vertices).
This partitioning scheme is depicted in figure~\ref{fig:cells}.
%
By construction, all $(b,b')$-cells have the following \emph{invariant property}
\begin{definition}
    A $(b,b')$-cell has the invariant property if, for all points $(a,a')$ in
    that cell, (1) the number of real roots of $F(a,b,z)$ is fixed, (2)
    the number of real roots of $F(a',b',z)$ is fixed,
    and (3) either, at least one of \(F(a,b,z)\) or \(F(a',b',z)\) is the zero
    polynomial, or the sorted order of the real roots of $F(a,b,z)$ and
    $F(a',b',z)$ is fixed, that is,
    $\mathcal{I}((F(a,b,z),F(a',b',z)))$ is fixed.
\end{definition}
\begin{lemma}
    All $(b,b')$-cells have the invariant property.
\end{lemma}
\begin{proof}
    First, (1) and (2) hold for all \((b,b')\)-cells because of the partition
    induced by the \(\delta_{b}\)-lines and the \(\delta_{b'}\)-lines.
%
    Second, (3) holds for all \((b,b')\)-cells that are not contained in
    \(\gamma_{b,b'}\) since (1) and (2) hold for those cells and because of the
    partition induced by \(\gamma_{b,b'}\) (see Lemma~\ref{lem:cont}).
%
    Third, (3) holds for all $(b,b')$-cells that are both contained in
    \(\gamma_{b,b'}\) and some \(\delta_b\)- or \(\delta_{b'}\)-line because
    one of the associated polynomials must be the zero polynomial.

    Finally, if a $(b,b')$-cell is contained in $\gamma_{b,b'}$ but not
    in any of the \(\delta_{b}\)- or \(\delta_{b'}\)-lines, we make a
    simple observation. This cell has two distinct neighbouring connected
    regions lying on each of its sides. We just showed that those two
    neighbouring cells have the invariant property. The union of this piece of
    $\gamma_{b,b'}$ with its two neighbouring cells is a connected region as in
    Lemma~\ref{lem:cont}. Hence, the ordering of any two real roots cannot swap
    along the piece of $\gamma_{b,b'}$, as this would otherwise
    contradict Lemma~\ref{lem:cont}. Hence, (3) holds for those pieces of
    $\gamma_{b,b'}$.
\end{proof}
This lemma implies that, provided we compute in which $(b,b')$-cells each
$(a,a')$ point lies, we only need to probe a single point per $(b,b')$-cell
to reveal the sorted permutation associated with each cell
of the $A \times B$ partition.


\paragraph{Preprocessing}
Locate all points $(a,a') \in A_i \times A_i$ for all $A_i$ with respect to all
$\gamma_{b,b'}$ curves, all vertical lines derived from $\delta_b$ and all
horizontal lines derived from $\delta_{b'}$ for all $(b,b') \in B_j \times B_j$
for all $B_j$.
%
As in \S\ref{sec:algo:explicit:nonuniform},
this can be done in a single batch using the algorithm described in
\S\ref{sec:algo:point-curves-location},
and the following generalization of Lemma~\ref{lem:dual}:
\begin{lemma}\label{lem:dual:implicit}
Define
\begin{align*}
    \hat{\gamma}_{a,a'} &= \{\, (x,y) \st \res(F(a,x,z),F(a',y,z);z) = 0 \,\},\\
    \hat{\delta}_{a}\text{-lines} &= \{\, (x,y) \st \res(F(a,x,z),F'_x(a,x,z);z) = 0 \,\},\\
    \hat{\delta}_{a'}\text{-lines} &= \{\, (x,y) \st \res(F(a',y,z),F'_y(a',y,z);z) = 0 \,\},\\
    \hat{\Gamma}_{a,a'} &= \hat{\gamma}_{a,a'} \,\,\cup\,\, \hat{\delta}_a\text{-lines} \,\,\cup\,\, \hat{\delta}_{a'}\text{-lines}.
\end{align*}
Locating $(a,a')$ with respect to $\Gamma_{b,b'}$ amounts to
locating $(b,b')$ with respect to $\hat{\Gamma}_{a,a'}$.
\end{lemma}
This gives us the information needed for the binary search in step (2).


\paragraph{Analysis}
The analysis mirrors the explicit case (described immediately after
Lemma~\ref{lem:batch}). Combining Lemma~\ref{lem:preprocessing-implicit} and
Lemma~\ref{lem:batch} yields a $O( {(ng)}^{4/3+\varepsilon} + n^2 \log g /
g)$-depth bounded-degree ADT for 3POL\@. By optimizing over $g$, we get $g =
\Theta(n^{2/7-\varepsilon})$, and the previous expression simplifies to
$O(n^{12/7+\varepsilon})$, proving Theorem~\ref{thm:implicit:nonuniform}.

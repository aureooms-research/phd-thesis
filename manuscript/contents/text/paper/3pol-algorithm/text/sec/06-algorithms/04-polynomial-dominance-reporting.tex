\subsection{Offline Polynomial Dominance Reporting}%
\label{sec:algo:dominance}

%\improve{On the analysis of PDR in Section 3.4: we
%base our analysis on that of Chan in [11]. During his talk at SoCG, he
%mentioned some improvement on this kind of analysis, stemming from recent
%papers.
%I think I remember some of the improvements consisted of going from $n^{2-1/d}$
%to $n^{2-1/log d}$ or something similar. So my question is: would it matter if we
%could improve on the fraction following the '2-' in the exponent of the first
%term (Corollary 1, page 11)? I do not think so. Maybe it is worth mentioning.}

We combine a standard dominance reporting
algorithm~\cite{PS85} with Matou\v{s}ek's algorithm~\cite{Ma93} to prove
Lemma~\ref{lem:dominance}.
For a pair of blue and red points in $\mathbb{R}^k$, we say that the blue point
\emph{dominates} the red point if for all indices $i\in[k]$ the $i$th
coordinate of the blue point is greater or equal to the $i$th coordinate of the
red point.
The standard algorithm in~\cite{PS85} solves the following problem:
%
\begin{problem}[%
	name={Offline Dominance Reporting}%
]
	Given $N$ blue
	%points
	and $M$ red points in $\mathbb{R}^k$, report all
	bichromatic dominating pairs.
\end{problem}

%
Our problem is significantly more complicated and general. Instead of blue points we
have blue $k$-tuples $p_i$ of $2$-dimensional points, instead of red points we have
red $k$-tuples $q_j$ of bivariate polynomial inequalities, and we want to report
all bichromatic pairs $(p_i,q_j)$ such that, for all $t \in [k]$,
the point $p_{i,t}$ verifies the inequality $q_{j,t}$.
The standard algorithm essentially works by a combination of divide and conquer
and prune and search, using a one-dimensional cutting (median selection) to
split a problem into subproblems. We generalize the standard algorithm by using
higher dimensional cuttings, in a way similar to Matou\v{s}ek's
algorithm~\cite{Ma93}. For the analysis, we generalize Chan's analysis
of the standard algorithm when $k$ is not constant~\cite{Cha08}.

\begin{proof}[Proof of Lemma~\ref{lem:dominance}]
We use the Veronese embedding~\cite{Har77,Har13}.
Since the polynomials have constant degree, we can trade polynomial
inequalities for linear inequalities by lifting to a space of
higher --- but constant --- dimension. The degree of each polynomial is at most
$\Delta$. There are exactly $d = \binom{\Delta+2}{2} - 1$ different bivariate
monomials of degree at most $\Delta$.%
\footnote{Not including the independent monomial, namely, $1$.}
To each monomial we associate a variable
in $\mathbb{R}^d$. By this association, points in the plane are mapped to
points in $\mathbb{R}^d$ and bivariate polynomial inequalities are mapped to
$d$-variate linear inequalities.

By abuse of notation, let $p_i$ denote the tuple $p_i$ where each
$2$-dimensional point has been replaced by its $d$-dimensional counterpart, and
let $q_i$ denote the tuple $q_i$ where each bivariate polynomial inequality
has been replaced by its $d$-variate linear counterpart.
We have $N$ $k$-tuples $p_i$ and $M$ $k$-tuples $q_j$.
%
The algorithm checks each of the $k$ components of the tuples in turn and
can be described recursively as follows for some positive integer $r > 1$:
\begin{algorithm}[Polynomial Dominance Reporting]\label{algo:pdr}
\item[input] $N$ $k$-tuples $p_i$ of $d$-dimensional points, $M$ $k$-tuples
	$q_j$ of $d$-variate linear inequalities.
\item[output] All $(p_i,q_j)$ pairs such that, for all $t \in [k]$, the point
	$p_{i,t}$ verifies the inequality $q_{j,t}$.
\item[1.] If $k=0$, then output all pairs $(p_i,q_j)$ and halt.
\item[2.] If $N < r^d$ or $M < r$, solve the problem by brute force
	in $O((N+M) k)$ time.
\item[3.]
We now only consider the $k$th component of each input $k$-tuple and call these
\emph{active} components.
%
To each active $d$-variate linear inequality corresponds a defining
hyperplane in $\mathbb{R}^d$.
Construct, as in~\cite{Ma93}, a hierarchical cutting of $\mathbb{R}^d$ using $O(r^d)$
simplicial cells such that each simplicial cell
is intersected by at most $\frac{M}{r}$ of the defining hyperplanes.
This construction also gives us for each simplicial cell of the cutting the list
of defining hyperplanes intersecting it.
This takes $O(Mr^{d-1})$ time.
%
Locate each active point inside the hierarchical cutting in time $O(N \log r)$.
Let $S$ be a simplicial cell of the hierarchical cutting.
Denote by $\Pi_S$ the set of active points in $S$. Partition each $\Pi_S$ into
$\left\lceil \frac{\lvert \Pi_S \rvert}{N r^{-2}} \right\rceil$ disjoint subsets
of size at most $\frac{N}{r^d}$.
%
For each simplicial cell, find the active inequalities whose corresponding
geometric object (hyperplane, closed or open half-space) contains the cell.
This takes $O(Mr^d)$ time.
%
The whole step takes $O(N\log r+Mr^d)$ time.
\item[4.] For each of the $O(r^d)$ simplicial cells, recurse on the at most
	$\frac{N}{r^d}$ $k$-tuples $p_i$ whose active point is inside the
	simplicial cell and the at most $\frac{M}{r}$ $k$-tuples $q_j$ whose active
	inequality's defining hyperplane intersects the simplicial cell.
\item[5.] For each of the $O(r^d)$ simplicial cells, recurse on the at most
	$\frac{N}{r^d}$ ($k-1$)-prefixes of $k$-tuples $p_i$ whose active point is
	inside the simplicial cell and the ($k-1$)-prefixes of $k$-tuples $q_j$
	whose active inequality's corresponding geometric object contains the simplicial cell.
\end{algorithm}

\paragraph{Correctness}
In each recursive call, either $k$ is decremented or $M$ and $N$ are divided by
some constant strictly greater than one, hence, one of the conditions in steps \textbf{1} and \textbf{2} is
met in each of the paths of the recursion tree and the algorithm always terminates.
Step \textbf{5} is correct because it only recurses on $(p_i,q_j)$ pairs whose
suffix pairs are dominating.
The base case in step \textbf{1} is correct because the only way for a pair
$(p_i,q_j)$ to reach this point is to have had all $k$ components checked in
step \textbf{5}.
The base case in step \textbf{2} is correct by definition.
Each dominating pair is output exactly once because the recursive calls of
step \textbf{4} and \textbf{5} partition the set of pairs $(p_i,q_j)$ that can
still claim to be candidate dominating pairs.

\paragraph{Analysis} For $k,N,M\ge0$, the total complexity $T_k(N,M)$ of
computing the inclusions for the first $k$ components, excluding the output
cost (steps~\textbf{1}~and~\textbf{2}), is bounded, in general, by
\begin{displaymath}
	T_k(N,M)
	\le
	\underbrace{O(r^d)\,T_{k-1}(N,M)}_{\text{Step}~\textbf{5}}
	+
	\underbrace{O(r^d)\,T_k\mleft(\frac{N}{r^d},\frac{M}{r}\mright)}_{\text{Step}~\textbf{4}}
	+
	\underbrace{O(N+M)}_{\text{Step}~\textbf{3}},
\end{displaymath}
and, in particular, by
\(T_k(N,M) = 0\) when \(k=0\),
\(T_k(N,M) = O(Nk)\) when \(M < r\), and
\(T_k(N,M) = O(Mk)\) when \(N < r^d\).

By point-hyperplane duality, $T_k(N,M) = T_k(M,N)$, hence, we can execute
step \textbf{4} on dual linear inequalities and dual points to balance the
recurrence. For some constant $c_1 \ge 1$,
%to be determined later:
\begin{displaymath}
	T_k(N,M)
	\le
	c_1 r^{2d} \,T_{k-1}(N,M)
	+
	c_1 r^{2d} \,T_k\mleft(\frac{N}{r^{d+1}},\frac{M}{r^{d+1}}\mright)
	+
	c_1 (N+M).
\end{displaymath}
For simplicity, we ignore some problem-size
reductions occuring in this balancing step.

Let $T_k(N) = T_k(N,N)$ denote the complexity of solving the problem when
$M=N$, excluding the output cost. Hence, we have
\begin{displaymath}
	T_k(N)
	\le
	c_1 r^{2d} \,T_{k-1}(N)
	+
	c_1 r^{2d} \,T_k\mleft(\frac{N}{r^{d+1}}\mright)
	+
	c_1 N \addtocounter{equation}{1}\tag{\theequation} \label{eq:tkn},
\end{displaymath}
and, in particular,
\(T_k(N) = 0\) when \(k=0\),
and
\(T_k(N) = O(k)\) when \(N < r^{d+1}\).

To get rid of the parameter $k$ and progress into the analysis of the
recurrence, Chan makes an ingenious change of variable~\cite{Cha08}.
With hindsight,
choose $b = r^{d+1}$ and let
\begin{equation}\label{eq:tn'}
	T(N') = \max\,
	\mleft\{\,
		T_k(N) \colon\, k \geq 0,\,\,\,\, N \geq 1,~\text{\,and\,}~b^k N\le N'
	\,\mright\}.
\end{equation}
For the rest of the discussion, we shorten the notation to
\begin{displaymath}
	T(N') = \max_{b^k N\le N'} T_k(N).
\end{displaymath}

By combining~(\ref{eq:tkn})~and~(\ref{eq:tn'}) we obtain
\begin{displaymath}
	T(N')
	=
	\max_{b^k N\le N'} T_k(N)
	\le
	c_1
	\max_{b^k N\le N'} \mleft[
	r^{2d} \,T_{k-1}\mleft(N\mright)
	+ r^{2d} \,T_k\mleft(\frac{N}{r^{d+1}}\mright)
	+ N\mright].
\end{displaymath}
The maximum of a sum is always bounded by the sum of the maxima of its terms,
hence,
\begin{displaymath}
	T(N')
	\le
	\max_{b^k N\le N'} \mleft[ c_1 r^{2d} \,T_{k-1} \mleft(N\mright)\mright]
	+
	\max_{b^k N\le N'} \mleft[ c_1 r^{2d} \,T_k \mleft(\frac{N}{r^{d+1}}\mright)\mright]
	+
	\max_{b^k N\le N'} \mleft[ c_1 N \mright].
\end{displaymath}
Looking at each term separately, by definition of $T(N')$, we have
\begin{align*}
\max_{b^k N\le N'} T_{k-1}(N)
&=
\max_{b^{k-1} N\le \frac{N'}{b}} T_{k-1}(N)
=
T\mleft(\frac{N'}{b}\mright)
=
T\mleft(\frac{N'}{r^{d+1}}\mright),\\
%
\max_{b^k N\le N'}  T_{k}\mleft(\frac{N}{r^{d+1}}\mright)
&=
\max_{b^k \frac{N}{r^{d+1}} \le \frac{N'}{r^{d + 1}}} T_{k}\mleft(\frac{N}{r^{d+1}} \mright)
=
T\mleft(\frac{N'}{r^{d+1}}\mright),\\
%
\max_{b^k N\le N'} N
&=
\max_{N\le \frac{N'}{b^k}} N
=
\frac{N'}{b^k} \le N',
\end{align*}
which, when combined with the previous inequality, produce the following recurrence
\begin{displaymath}
	T(N') \le 2 c_1 r^{2d} \,T\mleft(\frac{N'}{r^{d+1}}\mright) + c_1 N'.
\end{displaymath}

\paragraph{Powers of $r^{d+1}$}
We claim that if $N'$ is a power of $r^{d+1}$, then $T(N') \le c_2[{N'}^\alpha - N']$
for some constants $\alpha > 1$ and $c_2 \ge 1$. We prove by
induction that this (educated) guess is indeed correct.
For $N' = 1$, we have
\begin{align*}
	T(1)
	=
	\max_{b^{k}N \le 1} T_{k} (N)
	=
	T_0(1)
	=
	0
	\le
	c_2[1^{\alpha} - 1].
\end{align*}
For $N' \ge r^{d+1}$ a power of $r^{d+1}$,
and assuming the claim holds for all smaller powers:
\begin{align*}
	T(N')
	&\le
	2 c_1r^{2d}c_2\mleft[{\mleft(\frac{N'}{r^{d+1}}\mright)}^\alpha
	-\frac{N'}{r^{d+1}}\mright]
	+
	c_1 N'\\
	&\le c_2 {N'}^\alpha
	\mleft[\frac{2c_1r^{2d}}{{(r^{d+1})}^\alpha}\mright]
	- c_2 N' \mleft[2c_1r^{d-1} -
	\frac{c_1}{c_2}\mright].
\end{align*}
We want
\begin{equation*}
\frac{c_1r^{2d}}{{(r^{d+1})}^\alpha} \le \frac 12
\qquad
\text{and}
\qquad
2c_1r^{d-1} - \frac{c_1}{c_2} \ge 1.
\end{equation*}
For the first inequality, we can set the left hand side to be equal to
$c_1 r^{-\epsilon'} = \frac{1}{2}$
with some small $\epsilon' = \frac{1 + \log c_1}{\log r}$.
%
Hence,
$2d - \alpha(d+1) = -\epsilon'$, and for $\epsilon = \frac
{\epsilon'} {d+1}$, we get $\alpha = \frac{2d}{d+1} + \epsilon$.
%
The second inequality is equivalent to
$2r^{d-1} \ge \frac{1}{c_1} + \frac{1}{c_2}$,
which always holds since
$r > 1, d \ge 1, c_1 \ge 1, c_2 \ge 1$.

We now have
\begin{displaymath}
	T(N') = O\mleft({N'}^{\frac{2d}{d+1}+\epsilon}\mright),
\end{displaymath}
where
$\epsilon = \frac{1+\log c_1}{(d+1)\log r}$
can be chosen arbitrarily small by picking
an arbitrarily large
$r = {(2c_1)}^{1/\epsilon (d+1)}$.

\paragraph{Remark}
The choice $b=r^{d+1}$ gives a simpler analysis. Although giving more
freedom to the value of $b$ --- as in Chan's paper --- yields a slightly better
relation between $\epsilon$ and $r$, namely
$r>c_1^{1/\epsilon (d+1)}$, it does not get rid of the dependency
of $\epsilon$ in $r$, unless $c_1=1$.

\paragraph{General case}
When $N' \ge 2$ is not a power of $r^{d+1}$, we use the fact that
$T(N') \le T(N'+1)$ by definition,
\begin{align*}
	T(N')
	&=
	T\mleft({(r^{d+1})}^{\log_{r^{d+1}} N'}\mright)
	\\
	&\le
	T\mleft({(r^{d+1})}^{\lfloor\log_{r^{d+1}} N'\rfloor + 1}\mright)
	\\
	&=
	O\mleft({(r^{d+1})}^{(\lfloor\log_{r^{d+1}} N'\rfloor + 1
	)(\frac{2d}{d+1}+\epsilon)}\mright)
	\\
	&=
	O\mleft({(r^{d+1})}^{\frac{2d}{d+1}+\epsilon}
	{{(r^{d+1})}^{\lfloor\log_{r^{d+1}} N'\rfloor}}^{\frac{2d}{d+1}+\epsilon}\mright)
	\\
	&=
	O\mleft({{(r^{d+1})}^{\lfloor\log_{r^{d+1}}
	N'\rfloor}}^{\frac{2d}{d+1}+\epsilon}\mright)
	\\
	&=
	O\mleft({{(r^{d+1})}^{\log_{r^{d+1}}
	N'}}^{\frac{2d}{d+1}+\epsilon}\mright)
	\\
	&=
	O\mleft({N'}^{\frac{2d}{d+1}+\epsilon}\mright).
\end{align*}
\paragraph{Finally} We can now bound $T_k(N)$ using the upper bound for
$T(N')$,
\begin{displaymath}
	T_{k}(N)
	\le
	\max_{b^{k_i} N_i \le b^{k} N} T_{k_i}(N_i)
	=
	T(b^{k} N)
	=
	O\mleft({(b^{k} N)}^{\frac{2d}{d+1}+\epsilon}\mright)
	=
	2^{O(k)}{N}^{\frac{2d}{d+1}+\epsilon}.
\end{displaymath}
Hence,
$T_k(N) = 2^{O(k)}N^{\frac{2d}{d+1}+\epsilon_r}$,
and since $d=\binom{\Delta+2}{2}-1$, we have
\begin{displaymath}
	T_k(N) = 2^{O(k)}N^{2-\frac{4}{{\Delta}^2+3\Delta+2}+\epsilon_r}.
\end{displaymath}
To that complexity we add a constant time unit for each output
pair in steps \textbf{1} and \textbf{2}.
\end{proof}

%\remark{} Note that, in our case, one could use the
%$\gamma_{b,b'}$-$\hat{\gamma}_{a,a'}$ duality together with vertical
%decomposition (as in \S\ref{sec:algo:point-curves-location}) and get
%away with a complexity of $2^{O(k)}N^{\frac 43+\epsilon}$. However, using
%the Veronese embedding and point-hyperplane duality is more general, as the
%algorithm can be applied to any input polynomial inequalities.

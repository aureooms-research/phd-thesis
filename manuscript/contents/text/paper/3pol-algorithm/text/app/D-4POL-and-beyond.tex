\section{4POL and beyond}

There are various ways to define the 4POL problem. A first one is the following
\begin{problem}[easy 4POL]
Let $f,g \in \mathbb{R}[x,y]$ be polynomials of constant degree.
Given 4 sets $A$, $B$, $C$, and $D$, each containing $n$ real numbers, decide
whether there exists $(a,b,c,d) \in A \times B \times C \times D$ such that
$f(a,b)=g(c,d)$.
\end{problem}
For this problem, a naive meet-in-the-middle algorithm gives a quadratic
algorithm. Sort all $n^2$ values $f(a,b)$ for all $(a,b) \in A \times B$,
sort all $n^2$ values $g(c,d)$ for all $(c,d) \in C \times D$, then in
quadratic time walk through the $f(A,B) \times g(C,D)$ grid,
which gives a $\tilde{O}(n^2)$ time algorithm.

The two other 4POL definitions we consider are the following
\begin{problem}[explicit 4POL]
Let $f \in \mathbb{R}[x,y,z]$ be a polynomial of constant degree.
Given 4 sets $A$, $B$, $C$, and $D$, each containing $n$ real numbers, decide
whether there exists $(a,b,c,d) \in A \times B \times C \times D$ such that
$d=f(a,b,c)$.
\end{problem}
\begin{problem}[4POL]
Let $f \in \mathbb{R}[x,y,z,w]$ be a polynomial of constant degree.
Given 4 sets $A$, $B$, $C$, and $D$, each containing $n$ real numbers, decide
whether there exists $(a,b,c,d) \in A \times B \times C \times D$ such that
$f(a,b,c,d)=0$.
\end{problem}
Note that 4POL generalizes the two other problems. For 4POL we
can reuse our modified version of Matou\v{s}ek's algorithm.
\begin{theorem}
	4POL can be solved in $O(n^{\frac 83 + \varepsilon})$ time in the real-RAM
	model.
\end{theorem}
\begin{proof}
Define the curves
\begin{displaymath}
	\gamma_{c,d} = \{\,(x,y)\st f(x,y,c,d)=0\,\}
\end{displaymath}
and the dual curves
\begin{displaymath}
	\overline{\gamma}_{a,b} = \{\,(x,y)\st f(a,b,x,y)=0\,\}.
\end{displaymath}
Now solving 4POL amounts to locating all $n^2$ $(a,b)$ points with
respect to all $n^2$ $\gamma_{c,d}$ curves which is a $(n^2,n^2)$-problem.
\end{proof}

This result is close to a combinatorial bound of $O(n^{\frac 83})$ obtained by
Nassajian Mojarrad, Pham, Valculescu and de Zeeuw~\cite{MPVd16}.
\remark{Their definition of curves and dual curves is indeed the same as ours.
So in this case, what exactly is their definition of special for $F$?
This is a subtle point, and it is better to leave it as they stated, so $F$ is a
real polynomial which is $(G,K)$-cartesian and G and K are complex polynomial
(see definition 1.1). I guess it still remains to be seen if one can refine
this definition, see the paragraph after the statement of Theorem 1.3}

We can also look at the analogue of the $k$-SUM problem in our setting
\begin{problem}[$k$-POL]
Let $f \in \mathbb{R}[x_1,x_2,\ldots,x_k]$ be a polynomial of constant degree.
Given $k$ $n$-sets of real numbers $A_1, A_2, \ldots, A_k$ decide whether there
exists $(a_1,a_2,\ldots,a_k) \in A_1 \times A_2 \times \cdots \times A_k$ such
that $f(a_1,a_2,\ldots,a_k)=0$.
\end{problem}

%If we could generalize the modified Matou\v{s}ek algorithm to work for an
%arbitrary number of dimensions
%and run in time $O^*(n^\frac{2d}{d+1})$ then
%this would give a uniform algorithm with time complexity
%$O^*(n^{\frac{k^2}{k + 2}}) = O^*(n^{k - \frac{2k}{k+2}})$ for $k$-POL when $k$
%is even: we have to locate $O(n^{\frac k2})$ points $(x_1,x_2,\ldots,x_{\frac
%k2})$ with respect to $O(n^{\frac k2})$ surfaces
%\begin{displaymath}
	%\gamma_{a_{\frac k2 +1},a_{\frac k2 +2},\ldots,a_k} = \{\,
		%(x_1,x_2,\ldots,x_{\frac k2})\st
		%f(x_1,x_2,\ldots,x_{\frac k2},a_{\frac k2 +1},a_{\frac k2
		%+2},\ldots,a_k) = 0
	%\,\}.
%\end{displaymath}

By a proof similar the proof of Theorem~\ref{thm:gpt-to-3pol} given in the
previous section --- the determinant test easily generalizes --- deciding
whether an $n$-points input set in $\mathbb{R}^d$ contain a $(d+1)$-subset
lying on a hyperplane reduces linearly to $(d+1)$-POL when the input points lie
on a constant-degree polynomial curve.

Whether substantially better algorithms can be designed for 4POL and $k$-POL is
an open question.

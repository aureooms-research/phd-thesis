\chapter%
[Encoding 3SUM]%
{Encoding 3SUM\\[1ex]
\normalfont\footnotesize\itshape with
Sergio Cabello,
Jean Cardinal,
John Iacono,
Stefan Langerman,
and
Pat Morin}%
\label{paper:3sum-encoding}

Given three sets of \(n\) real numbers
\(A = \{\, a_1 < a_2 < \cdots < a_n\,\} \),
\(B = \{\, b_1 < b_2 < \cdots < b_n\,\} \),
and \(C = \{\, c_1 < c_2 < \cdots < c_n\,\}\),
we wish to build a discrete data structure (using bits, words, and pointers) such that,
given any triple \((i,j,k) \in {[n]}^3\) it is possible to compute the sign of
\(a_i + b_j + c_k\) by only inspecting the data structure (we cannot consult
\(A\), \(B\), or \(C\)).
We refer to the map $\chi : {[n]}^3\to \{-,0,+\}, (i,j,k)\mapsto\mathrm{sgn}
(a_i+b_i+c_k)$ as the {\em 3SUM type} of the instance $\langle A,B,C \rangle$.

Obviously, one can simply construct a lookup table of size \(O(n^3)\), such
that triple queries can be answered in \(O(1)\) time.
%
In \S\ref{s:numbers} we show that a minimal integer representation of a
3SUM instance may require $\Theta(n)$ bits per value, yielding
$O(n)$ query time and $O(n^2)$ space.
%
In \S\ref{s:space} we show how to use an optimal $O(n \log n)$ bits of
space with a polynomial query time. Finally, in \S\ref{s:sscqt} we show
how to use $\tilde{O}(n^{3/2})$ space to achieve $O(1)$-time queries.

\section{It is always easier in the plane}

\subsection{\iftitlecase%
Encoding Order Types via Hierarchical Cuttings\else%
Encoding order types via hierarchical cuttings\fi}\label{sec:lines-and-pseudolines}

\ifeurocg\else
To make our statements clear, we use the following definition:
%
\input{text/definition/encoding}
%
In this section, we use this definition with \(f\) being some order
type,\footnote{%
  Technically, we encode the orientation predicate of some realizing
  arrangement of the order type and skip the isomorphism. If desired, a
  canonical labeling of the arrangement can be produced in \(O(n^2)\) time for
  abstract and realizable order types (see Lemma~\ref{lem:canonical-labeling}).
}
\(k=3\) and the codomain of \(f\) being \(\{\, -,0,+\,\}\). For the rest of the
discussion, we assume the word-RAM model with word size \(w \geq \log n\) and
the standard arithmetic and bitwise operators.
%
We prove our main theorems for the two-dimensional case:
%
\input{text/theorem/abstract}
\input{text/theorem/realizable}
\input{text/theorem/preprocessing}
%
For instance, Theorem~\ref{thm:realizable} implies that for any set of points
\(\{\, p_1, p_2, \ldots, p_n\,\}\), there exists a string of \(O(n^2 {(\log
\log n)}^2 / \log n)\) bits such that given this string and any triple of
indices \((a,b,c) \in {[n]}^3\) we can compute the value of \(\chi(a,b,c) =
\nabla(p_a, p_b, p_c)\) in \(O(\log n)\) time.
\fi

\ifeurocg
We \else
Throughout the rest of this paper, we \fi
assume that we can access some arrangement of pairwise intersecting lines or
pseudolines that realizes the order type we want to encode. We thus exclusively
focus on the problem of encoding the order type of a given arrangement. This
does not pose a threat against the existence of an encoding.
\ifeurocg\else%
However, we have to be more careful when we bound the preprocessing
time required to compute such an encoding. This is why, in
Theorem~\ref{thm:preprocessing}, we specify the model of computation and how
the input is given.
\fi%
\ifeurocg%
In this extended abstract, we sketch the general idea for a simple subquadratic
encoding. For full details, proofs, and improvements, we refer to the arXiv
version~\cite{CCILO18}.
\fi%

\ifjournal\else
\input{text/tool/hierarchical-cuttings}
\fi


\paragraph*{Idea} We want to preprocess \(n\) pseudolines \( \{\, L_1,
L_2, \ldots, L_n\,\} \) in the plane so that, given three indices \(a\),
\(b\), and \(c\), we can compute their orientation, that is, whether the
intersection \(L_a \cap L_b\) lies above, below or on \(L_c\). Our
data structure builds on cuttings as follows: Given a cutting \(\Xi\) and the
three indices, we can locate the intersection of \(L_a\) and \(L_b\)
with respect to \(\Xi\). The location of this intersection is a cell of
\(\Xi\). The next step is to decide whether \(L_c\) lies above, lies below,
contains or intersects that cell. In the first three cases, we are done.
Otherwise, we can answer the query by recursing on the subset of pseudolines
intersecting the cell containing the intersection. We build on hierarchical
cuttings to control the size of each such subproblem.


\paragraph*{\iftitlecase%
Intersection Location\else%
Intersection location\fi} When the arrangement consists of straight lines,
locating the intersection \(p_a' \cap p_b'\) in \(\Xi\) is trivial if we
know the real parameters of \(p_a'\) and \(p_b'\) and of the descriptions
of the subcells of \(\Xi\). However, in our model we are not allowed to store
real numbers. To circumvent this annoyance, and to handle arrangements of
pseudolines, we make an observation illustrated by
Figure~\ref{fig:permutation}.
%
\begin{figure}
\centering{}
\begin{subfigure}[t]{0.5\textwidth}
\centering{}
\includegraphics[scale=.\ifeurocg7\else9\fi]{figures/permutation-0}
\caption{\(p_a' \cap p_b' \cap \mathcal{C} = \emptyset\).}
\end{subfigure}%
\begin{subfigure}[t]{0.5\textwidth}
\centering{}
\includegraphics[scale=.\ifeurocg7\else9\fi]{figures/permutation-1}
\caption{\(p_a' \cap p_b' \cap \mathcal{C} \neq \emptyset\).}
\end{subfigure}
\caption{Cyclic permutations (\(\pi\)).}\label{fig:permutation}
\end{figure}
%
\begin{definition}
    Given a set of pseudolines, a pseudocircle is a simple closed curve such that
    each pseudoline properly intersects it in exactly two points.
    %
    A pseudodisk is a region bounded by a pseudocircle.
\end{definition}
%
\begin{observation}
    Two pseudolines \(p_a'\) and \(p_b'\)
    intersect in the interior of a pseudodisk \(\mathcal{C}\) if and only if
    the intersections of \(p_a'\) and \(p_b'\) with \(\mathcal{C}\)
    alternate on the boundary of \(\mathcal{C}\).
\end{observation}
%
%
We construct \(\Xi\) so that its cells are pseudodisks for the pseudolines that
intersect them.
%
Hence, this observation gives us a way to encode the location of the
intersection of \(p_a'\) and \(p_b'\) in \(\Xi\) using only bits.
%
We formalize this observation with the following definition:
\begin{definition}[Cyclic Permutation]
  The \emph{cyclic permutation} of a full-dimensional cell of a cutting
  is the cyclic ordering of the pseudolines crossing its boundary.
\end{definition}
%
\ifeurocg\else%
For an arrangement of pseudolines, we use the standard vertical decomposition
to construct a hierarchical cutting.
This decomposition partitions the space into trapezoidal cells.
For an arrangement of lines, we can
use the standard bottom-vertex triangulation instead, which allows us to
generalize our results to higher dimensions\ifsocg~\cite{CCILO18}\else{} in
\S\ref{sec:hyperplanes}\fi.
In the plane, the bottom-vertex triangulation partitions the space into
triangular cells.\fi
%
\ifeurocg%
Location in a non-full-dimensional cell can be encoded similarily.%
\else
Note that non-full-dimensional cells are easier to encode. For a 0-dimensional
cell and a pseudoline, we store whether the pseudoline lies above, lies below,
or contains the 0-dimensional cell. For a 1-dimensional cell, a pseudoline
could also intersect the interior of the cell, but in only one point. The
pseudolines intersecting that cell define an (acyclic) permutation with potentially
several intersections at the same position. This information suffices to answer
location queries for those cells, and the space taken is not more than that
necessary for full-dimensional cells. When two pseudolines intersect in a
1-dimensional cell or contain the same 0-dimensional cell, they
appear simultaneously in the cyclic permutation of an adjacent 2-dimensional
cell if they intersect its interior. If that is the case, the location of
the intersection of those two pseudolines in the cutting is the
non-full-dimensional cell. A constant number of bits can be added to the
encoding each time we need to know the dimension of the cell we encode.
\fi


\paragraph*{Encoding}
Given \(n\) pseudolines in the plane and some fixed
parameters \(r\) and \(\ell\), compute a \(\ell\)-levels hierarchical cutting
of parameter \(r\) for those
pseudolines as in Lemma~\ref{lem:hierarchical-cutting-2}.
This hierarchical cutting consists of \( \ell \) levels
labeled \(0,1,\ldots,\ell-1\). Level \(i\) has \(O(r^{2i})\)
cells. Each of those cells is further partitioned into
\(O(r^2)\) subcells. The \(O(r^{2(i+1)})\) subcells of level \(i\)
are the cells of level \(i+1\). Each cell of level \(i\) is intersected by at
most \(\frac{n}{r^i}\) pseudolines, and hence each subcell is intersected by at
most \(\frac{n}{r^{i+1}}\) pseudolines.

We compute and store a combinatorial representation of the hierarchical cutting
as follows: For each level of the hierarchy, for each cell in that level, for
each pseudoline intersecting that cell, for each subcell of that cell, we store
two bits to indicate the location of the pseudoline with respect to that
subcell\ifeurocg\else, that is, whether the pseudoline lies above
(\texttt{00}), lies below (\texttt{01}), intersects the interior of that
subcell (\texttt{10}), or contains the subcell (\texttt{11})\fi.
When a pseudoline intersects the interior of a 2-dimensional subcell, we also
store the two indices of that pseudoline
in the cyclic permutation of that subcell, beginning at an
arbitrary location in, say, clockwise order.
\ifeurocg%
Location in a non-full-dimensional subcell can be encoded similarily.
\else%
If the intersected subcell is
1-dimensional instead, we store the index of the pseudoline in the acyclic
permutation of that subcell, beginning at an arbitrary endpoint.
If two pseudolines intersect in the interior of a 1-dimensional subcell or on
the boundary of a 2-dimensional subcell, they share the same index in the
permutation of that subcell.
\fi%

\ifeurocg\else
This representation takes \(O(\frac{n}{r^i} + \frac{n}{r^{i+1}}
\log{\frac{n}{r^{i+1}}})\) bits per subcell of level \(i\) by storing for each
pseudoline its location and, when needed, the permutation indices of its
intersections with the subcell. At the last level of the hierarchy, let \(t =
\frac{n}{r^\ell}\) denote an upper bound on the number of pseudolines
intersecting each subcell. For each of those \(O(r^{2 \ell}) =
O(\frac{n^2}{t^2})\) subcells we store a pointer to a lookup table of size
\(O(t^3)\) that allows to answer the query of the orientation of any
triple of pseudolines intersecting that subcell.
\fi

\ifeurocg
The hierarchy is such that each subcell of the last level is intersected by no
more than \(t = \frac{n}{r^\ell}\) pseudolines.
For those subcells, we answer the queries by table lookup.
The use of hierarchical cuttings essentially guarantees we get quadratic
preprocessing time, quadratic space, and logarithmic query time in the abstract
case. In the realizable case, we know there can only be \(2^{O(t \log t)}\)
distinct lookup tables. Choosing the right superconstant \(t\) leads to
subquadratic space in that case.
\fi


\ifeurocg\else
Storing the permutation at each subcell would suffice to answer all
queries that do not reach the last level of the hierarchy. However,
for those queries to be answered efficiently,
we need to have access to all bits belonging to a given pseudoline
without having to read the bits of the others.
%
One solution is to augment each subcell with a hash table that translates
pseudoline indices of the parent cell into pseudoline indices of the subcell.
%
Another cleaner solution is to use the Zone Theorem
(%
\ifjournal%
Theorem~\ref{thm:zone-theorem-2}%
\else%
\cite{BEPY90,CGL85,Go04}%
\fi%
): by constructing the hierarchical cutting
via decompositions of subsets of the input pseudolines, we can bound
the number of subcells of a given cell a given pseudoline intersects by
\(O(r)\).%
\footnote{A reason to prefer this solution is that it enables the query time
reduction in the next section.}
%
Hence,
the number of bits stored for a single intersecting cell-pseudoline
pair at level \(i\) is bounded by
\(|\textsc{Tr}_i| = O(r^2 + r \log{\frac{n}{r^{i+1}}})\).
This bound allows us to store all bits belonging to a given cell-pseudoline pair
\((\mathcal{C},p')\)
in a contiguous block of memory, denoted by \(\textsc{Tr}(\mathcal{C},p')\),
whose address in memory is easy to compute (as detailed later on).
%
The overall number of bits stored stays the same up to a constant factor.
%
We call
\(\textsc{Tr}(\mathcal{C},p')\) the \emph{trace} of \(p'\) in \(\mathcal{C}\).
Figure~\ref{fig:trace} depicts an example trace.
%
\begin{figure}
  \centering{}
  \includegraphics[width=\linewidth]{figures/trace}
  \caption{%
	  A trace \(\textsc{Tr}(\mathcal{C}, p')\). The cell \(\mathcal{C}\) has
	  eight subcells. Each subcell is intersected by at most four pseudolines.
	  The pseudoline \(p'\) lies above two of them, lies below two of them,
	  contains one of them, and intersects three of them at indices
	  \((2,6)\), \((5,7)\), and \((0, 3)\).
  }\label{fig:trace}
\end{figure}
%
%Furthermore, the Zone Theorem implies that the number of cells of level \(i\)
%intersected by a given pseudoline is bounded by \(O(r^i)\). Hence, we can
%concatenate all traces of a given pseudoline and store them in a continuous
%block of memory in a space-efficient way.
%\aurelien{This would require to use another cell to subcell translation table
%that would blow up the space or rely on other complicated tools}

For queries that reach the last level of the hierarchy, storing an individual
lookup table for each leaf would cost too much as soon as \(t = \omega(1)\).
However, as long as \(t\) is small enough, each order type is shared by
many leaves, and we can thus save space. Formally, let \(\nu(t)\) denote the
number of order types of size \(t\), which is \(\nu(t) = 2^{\Theta(t^2)}\) for
abstract order types~\cite{Fe96} and \(\nu(t) = 2^{\Theta(t \log{t})}\) for realizable
order types~\cite{Al86,GP86}. At most \(\nu(t)\) distinct lookup tables are needed
to answer
the queries on the subcells of the last level of the hierarchy. Hence, the
pointers have size \(|\textsc{Pointer}| = O(\log \nu(t))\) and the total space needed for the lookup
tables is \(O(t^3 \nu(t))\). For each leaf, we store a canonical
labeling of size \(|\textsc{Labeling}| = O(t \log t)\) on the pseudolines that intersect it,
as in Lemma~\ref{lem:canonical-labeling}. We use
that labeling to order the queries in the associated lookup table.%
\ifjournal\footnote{%
Note that the use of this canonical labeling is not necessary if
\(t\) is constant, because then we can afford to use one lookup table per leaf.
For superconstant \(t\), the canonical labeling is necessary to get
the construction time down to \(O(n^2)\) as explained later.
Space and query complexity are not affected by this design choice (other than
constant factors).%
}\fi


\paragraph*{Layout}
For completeness, we detail precisely how bits of the encoding are
laid out in memory to allow an efficient decoding.
%
Indeed, the data structure we encode is a tree and many space-efficient layouts
exists for those when their nodes each have the same size. Here however, node
size shrinks as we go down the hierarchy.
We spend a few paragraphs showing it
is still possible to
address all components in a
time- and space-efficient way.
%
As this is fairly straightforward to adapt for the encodings in
\S\ref{sec:query-time} and \S\ref{sec:hyperplanes},
we do not give the details for those sections.
%
%Figure~\ref{fig:layout} gives a sketch of this layout.

%Figure is OUTDATED
%\ifeurocg\else%
%\begin{figure}
  %\centering{}
  %\includegraphics[scale=.8]{figures/layout}
  %\caption{Memory layout of the encoding.
    %Here, \(\textsc{Tr}(a, C_{i,j}(a))\) is used as a shorthand
    %for \(\textsc{Tr}(L_a, C_{i,j}(a))\) and \(-1\) is used as a shorthand
    %for ``last index''. The symbols \(\pi_j\) and
    %\(\textsc{Ptr}_j\) represent the canonical labeling and the pointer for leaf
    %\(j\), respectively. The symbol \(\textsc{Table}_{\tau}\) represents the lookup
    %table for order type \(\tau\).}\label{fig:layout}
%\end{figure}
%\fi

The encoding is the concatenation of
the parameters \(n\), \(r\), and \( t\),
the cells of the hierarchy,
and
the lookup tables.
%
We order the cells of the hierarchy in a depth-first manner: a cell of level
\(i\) is denoted by \(\mathcal{C}_{0, j_1, j_2, \ldots, j_i}\)
with \(j_1, j_2, \ldots, j_i \in \{\, 0, 1, \ldots, O(r^2)\,\}\).
The root cell is
\(\mathcal{C}_0\) and the cell \(\mathcal{C}_{0, j_1, j_2, \ldots, j_i, j_{i+1}}\) is
the (\(j_{i+1} + 1\))-th subcell of the cell
\(\mathcal{C}_{0, j_1, j_2, \ldots, j_i}\).
%
Cells are then ordered lexicographically.
%
For each leaf cell we store its pointer and canonical labeling.
%
For each internal cell \(\mathcal{C}\) we store the traces
\(\textsc{Tr}(\mathcal{C}, L)\) in a certain permutation of the pseudolines
\(L\).
%
To order the pseudolines, we use the first index of each pseudoline
in the cyclic permutation of the cell (for the root cell \(\mathcal{C}_0\) this
is the index given by the input permutation).
%
Note that those indices are between \(0\) and \(2 \lfloor \frac{N}{r^{i}}
\rfloor - 1\) for a cell of level \(i\). We allocate twice the required space
for traces and canonical labelings to avoid defining a mapping between the
indices of a cell and the indices of each of its subcells.

%To translate between cell's subcell indices and pseudoline's
%\aurelien{We avoid doing that by enumerating first cell-wise, then
%pseudoline-wise}

For the root cell of the hierarchy
\(\mathcal{C}_0\) representing the entire space and containing all the
intersections of the arrangement, \(\textsc{Addr}(\mathcal{C}_0)\) is the
first free address after the encoding of the parameters \(n\), \(r\), and \(t\).

The address \(\textsc{Addr}(\mathcal{C}_{0,j_1,j_2, \ldots, j_{i-1}, 0})\)
of the first subcell of \(\mathcal{C}_{0,j_1,j_2, \ldots, j_{i-1}}\)
is offset by twice the space taken by the traces for that cell
%
\begin{displaymath}
  \textsc{Addr}(\mathcal{C}_{0,j_1,j_2, \ldots, j_{i-1}, 0})
  =
  \underbrace{\textsc{Addr}(\mathcal{C}_{0,j_1,j_2, \ldots, j_{i-1}})}_{\text{Address of the parent cell}}
  +
  \underbrace{2 \left\lfloor \frac{n}{r^{i-1}} \right\rfloor |\textsc{Tr}_{i-1}|}_{\text{Traces of the parent cell}}
  .
\end{displaymath}

Let \(c_h\) be the constant hidden by the \(O(r^2)\) of the hierarchical cutting.
%
The space taken by a subtree rooted at a node of level \(i\) is bounded by
\begin{multline*}
  | \textsc{Subtree}_i |
  =
  \overbrace{2 \left\lfloor \frac{n}{r^{i}} \right\rfloor |\textsc{Tr}_{i}|}^{\text{Traces at the root}}
  +
  \overbrace{2c_h \sum_{k=1}^{\ell-i-1} r^{2k} \left\lfloor \frac{n}{r^{i+k}}
  \right\rfloor |\textsc{Tr}_{i+k}|}^{\text{Traces of the subtrees}}\\
  +
  \underbrace{c_h r^{2(\ell-i)} | \textsc{Pointer} |}_{\text{Pointers}}
  +
  \underbrace{2c_h r^{2(\ell-i)} | \textsc{Labeling} |}_{\text{Canonical labelings}}.
\end{multline*}

The address of any other subcell
\(\textsc{Addr}(\mathcal{C}_{0,j_1,j_2, \ldots, j_{i-1}, j_{i}})\)
is offset by the space taken by the subtrees of its siblings \(0, 1, \ldots,
j_{i} - 1\)
%
\begin{displaymath}
  \textsc{Addr}(\mathcal{C}_{0,j_1,j_2, \ldots, j_{i-1}, j_{i}})
  =
  \underbrace{\textsc{Addr}(\mathcal{C}_{0,j_1,j_2, \ldots, j_{i-1}, 0})}_{\text{Address of the first sibling}}
  +
  \underbrace{j_{i} \cdot | \textsc{Subtree}_{i} |}_{\text{Left siblings subtrees}}
  .
\end{displaymath}

The address of the trace \(\textsc{Tr}(\mathcal{C}, L_a)\),
where
\(0 \leq a \leq 2 \lfloor \frac{N}{r^{i}} \rfloor - 1\) is the first intersection
index of \(L_a\) in the cyclic permutation of the level-\(i\) cell
\(\mathcal{C}\),
is
%
\begin{displaymath}
  \textsc{Addr}(\mathcal{C}, L_a)
  =
  \textsc{Addr}(\mathcal{C})
  + a \cdot |\textsc{Tr}_{i}|.
\end{displaymath}
%
Since we do not store any trace for the leaves,
define \(| \textsc{Tr}_{\ell} | = 0\).
The pointer and canonical labeling of the leaf
\(\mathcal{C}_{0, j_1, j_2, \ldots, j_{\ell}}\)
are concatenated at position
\(\textsc{Addr}(\mathcal{C}_{0, j_1, j_2, \ldots, j_{\ell}})\)
%
and
%
the lookup tables are concatenated at position
\(
\textsc{Addr}(\mathcal{C}_{0})
+
| \textsc{Subtree}_{0} |
\).

This layout makes traversing the hierarchy from root to leaf efficient:
%
The address of the root cell is discovered after parsing the parameters \(n\),
\(r\), and \(t\).
%
The address of the first subcell of a parent cell is computed in constant time
from the address of the parent cell.
%
The size of the hierarchy is computed in time proportional its height.
%
The size of a subtree of level \(i+1\) is derived in constant time from the
size of a subtree of level \(i\).
%
The address of any subcell of level \(i\) is computed in constant time from its
index, the address of the parent cell, and the size of a subtree of level
\(i\).
%
The address of a trace is computed in constant time from the address of its
cell and the index of its pseudoline.
%
The address of the pointer and canonical labeling of a leaf
is the address of the corresponding cell.
%
The address of a lookup table is computed in constant time given the address of
the root cell, the size of the hierarchy, the size of a lookup table, and the
leaf pointer.

During a traversal, pseudoline indices of the parent cell are mapped to
indices in the subcell in constant time by using the first intersection index
of each pseudoline in that subcell. A final index mapping happens when translating leaf indices
to lookup table indices using the canonical labeling of the leaf.


\paragraph*{\iftitlecase%
Space Complexity\else%
Space complexity\fi}
We first prove a general bound on the space taken by the encoding of a
\(\ell\)-level hierarchical cutting of parameter \(r \geq 2\).
For the space taken by the lookup tables, their associated pointers and canonical
labelings at the leaves, and the parameters of the hierarchy \(n\), \(r\) and
\(t\), the analysis is immediate.

Let \(H_r^\ell(n) \in \mathbb{N}\) be the maximum amount of space (bits), over
all arrangements of \(n\) pseudolines, taken by the \(\ell \in \mathbb{N}\)
levels of a hierarchy with parameter \(r \in (1,+\infty)\).

\begin{lemma}\label{lem:space-2-hierarchy}
For \( r \geq 2 \) and \( t = \frac{n}{r^\ell}\) we have
\begin{displaymath}
H_r^\ell(n)
=
O\left(\frac{n^2}{t} ( \log t + r )\right).
\end{displaymath}
\end{lemma}

\begin{proof}
By definition, summing over all subcells, we have
\begin{displaymath} % This is equivalent but sums over all subcells
H_r^\ell(n)
= O \left(
    \sum_{i=0}^{\ell-1} \left(
	r^{2i} \cdot r^2 \cdot \left(
    \frac{n}{r^i} + \frac{n}{r^{i+1}} \log \frac{n}{r^{i+1}}
    \right)
\right)
\right).
\end{displaymath}
%
Note that we obtain the same bound by
summing over all traces (of intersecting cell-pseudoline pairs)
%
\begin{displaymath}
H_r^\ell(n)
= O \left(
    n
    \sum_{i=0}^{\ell-1} \left(
      r^i \cdot
      \left(
        r^2 + r \log \frac{n}{r^{i+1}}
      \right)
    \right)
\right).
\end{displaymath}
%
Multiply any of the previous equations by \(\frac{n}{tr^\ell} = 1\)
to obtain
%
\begin{displaymath}
H_r^\ell(n)
= O \left(
\frac{n^2}{t}
    \sum_{i=0}^{\ell-1} \left(
      \frac{1}{r^{\ell - i - 1}} \cdot \left(
	r + \log \frac{n}{r^{i+1}}
    \right)
\right)
\right).
\end{displaymath}
%
We use the equivalence \(\frac{n}{r^{i+1}} = t r^{\ell - i - 1}\) to replace
the last term in the previous equation
%
\begin{displaymath}
H_r^\ell(n)
= O \left(
\frac{n^2}{t}
    \sum_{i=0}^{\ell-1} \left(
      \frac{1}{r^{\ell - i - 1}} \cdot \left(
	r + \log t + (\ell - i - 1) \log r
    \right)
\right)
\right).
\end{displaymath}
%
We reverse the summation by redefining \(i \leftarrow \ell - i - 1\) and group the terms
%
\begin{displaymath}
H_r^\ell(n)
=
O \left(
\frac{n^2}{t} \left(
	(\log t + r)  \sum_{i=0}^{\ell-1} \frac{1}{r^{i}}
	+
	\log r \sum_{i=0}^{\ell-1} \frac{i}{r^{i}}
\right)
\right).
\end{displaymath}
%
Using the following inequalities (the geometric series and a multiple of its derivative):
%
\begin{displaymath}
\sum_{i = 0}^{k} x^i
\le
\frac{1}{1-x}
\qquad
\text{and}
\qquad
\sum_{i = 0}^{k} i x^i
\le
\frac{x}{{(1-x)}^2},
\qquad
\forall k \in \mathbb{N},
\forall x \in (0,1),
\end{displaymath}
%
we conclude that
%
\begin{displaymath}
H_r^\ell(n)
=
O\left(
\frac{n^2}{t} \left(
	\left(1 + \frac{1}{r-1} \right) (\log t + r)
	+
	\left(1 + \frac{2r-1}{r^2-2r+1}\right) \frac{\log r}{r}
\right)
\right),
\end{displaymath}
%
and the statement follows from \(r \geq 2\).
\end{proof}

%\paragraph*{Remark} The following could be stated for recursively defined
%families of order types of size \(\nu\).

\ifeurocg\else%
\begin{figure}
  \centering{}
  \includegraphics[width=\linewidth]{figures/space-analysis}
  \caption{Space analysis.}\label{fig:space-analysis}
\end{figure}
\fi

Figure~\ref{fig:space-analysis} sketches the different components of the
encoding and shows the space taken by each of them.
To that we must add the space taken by the
parameters of the hierarchy \(n\), \(r\) and \(t\) if those are not implicitly
known (here we assume the dimension \(d=2\) is implicitly known).
We have thus the following bound:
%
\begin{lemma}\label{lem:space-2-all}
  The space taken by the encoding described in \S\ref{sec:lines-and-pseudolines} is
  proportional to
    \begin{displaymath}
      \underbrace{\log ntr}_{\text{Parameters}}
      +
      \underbrace{\frac{n^2}{t} ( \log t + r )}_{\text{Traces}}
      +
      \underbrace{\frac{n^2}{t^2} ( \log \nu(t) + t \log t)}_{\text{Pointers and Canonical Labelings}}
      +
      \underbrace{t^3 \nu(t)}_{\text{Lookup Tables}}.
    \end{displaymath}
\end{lemma}

We pick \(r\) constant for both abstract and realizable order types.
%
We have \(\nu(t) = 2^{\Theta(n^2)}\) for abstract order types, hence we
choose \(t = \sqrt{\delta \log n}\) for small enough \(\delta\) and the third
term in Lemma~\ref{lem:space-2-all} dominates with \(n^2\). Note how the
quadratic bottleneck of this encoding is the storage of the order type pointers
at the leaves of the hierarchy.
%
We have \(\nu(t) = 2^{\Theta(n \log n)}\) for realizable order types, hence we
choose \(t = \delta \log n / \log \log n\) for small enough \(\delta\) and the
second and third term in Lemma~\ref{lem:space-2-all} dominate with \(n^2 {(\log
\log n)}^2 / \log n\). This proves the space constraints in
Theorems~\ref{thm:abstract}~and~\ref{thm:realizable}.


\paragraph*{\iftitlecase%
Correctness and Query Complexity\else%
Correctness and query complexity\fi} Given our encoding and three
pseudoline indices \(a,b,c\) we answer a query as follows: We start by decoding the
parameters \(n\), \(r\), and \(t\). In our model, this can be done in
\(O(\log^* n + \log^* r + \log^* t)\) time, where \(\log^*\) is the iterated
logarithm (as in~\cite{Ma93}).\footnote{%
  Logarithmic space and constant decoding time is trivial when \(w =
  \Theta(\log n)\).
  If \(w\) is too large, encode \(n\) in binary using \(\lceil \log n + 1\rceil\)
  bits, \(\lceil \log n + 1\rceil\) using \(\lceil \log \lceil \log n + 1
  \rceil +1 \rceil\) bits,
  \(\lceil \log \lceil \log n + 1\rceil +1\rceil\) using \(\lceil \log \lceil
  \log \lceil \log n + 1\rceil +1\rceil +1\rceil\) bits,
  etc.\ until the number to encode is smaller than a constant which we encode
  in unary with \texttt{1}'s. Prepend a \texttt{1} to the largest number and
  \texttt{0} to all the others. Concatenate those numbers
  from smallest to largest. Total space is \(O(\log n)\) bits and decoding
  \(n\) can be done in \(O(\log^* n)\) time in the word-RAM model with \(w \geq
  \log n\).
  As an alternative, logarithmic space and logarithmic decoding time is also
  trivially achievable with no constraint on \(w\).%
}
Let \(\mathcal{C} = \mathcal{C}_{0}\).
First, find the subcell \(\mathcal{C}'\) of \(\mathcal{C}\) containing \(p_a'
\cap p_b'\) by testing for each subcell whether
\(p_a'\) and \(p_b'\) alternate in its cyclic permutation.
This can be done in \(O(r^2)\) time by scanning \(\textsc{Tr}(\mathcal{C},
p_a')\) and \(\textsc{Tr}(\mathcal{C}, p_b')\) in parallel. Note that non-full
dimensional cells and subcells are easier to test. Next, if \(p_c'\) does
not properly intersect \(\mathcal{C}'\), answer the query accordingly. If on
the other hand \(p_c'\) does properly intersect the subcell we recurse on
\(\mathcal{C}'\). This can be tested by scanning \(\textsc{Tr}(\mathcal{C},
p_c')\) in \(O(r^2)\) time. Note that in case that the subcell is
non-full-dimensional we can already answer the query. When we
reach the relative interior of a subcell of the last level of the hierarchy
without having found a satisfactory answer, we can answer the query by table
lookup in constant time. This works as long as each order type identifier for
at most \(t\) pseudolines fits in a constant number of words, which is the case
for the values of \(t\) we defined.
%
The layout described earlier makes all memory address computations of this
query algorithm take constant time.
%
The total query time is thus proportional
to \(r^2 \log_r n\) in the worst case, which is logarithmic since \(r\) is
constant.
%
This proves the query time constraints in
Theorems~\ref{thm:abstract}~and~\ref{thm:realizable}.

With the hope of getting faster queries we could pick \(r = \Theta(\log
t)\) to reduce the depth of the hierarchy, without changing the
space requirements by more than a constant factor.
However, if no additional care is taken, this would slow the queries down by a
\(\Theta(\log^2 t / \log \log t)\) factor because of the scanning approach
taken when locating the intersection \(p_a' \cap p_b'\). We show how to handle
small but superconstant \(r\) properly in the next section.


\paragraph*{\iftitlecase%
Preprocessing Time\else%
Preprocessing time\fi}
For a set of \(n\) points in the plane, or an arrangement of \(n\) lines in the
dual, we can construct the encoding of their order type in quadratic time in
the real-RAM and constant-degree algebraic computation tree models. We prove
Theorem~\ref{thm:preprocessing}. \aurelien{Should also work for pseudolines
using the wiring diagram as input.}
\begin{proof}
  Using Lemma~\ref{lem:hierarchical-cutting-2},
  a hierarchical cutting can be computed in \(O(nr^\ell)\) time in the dual
  plane. All traces \(\textsc{Tr}(\mathcal{C}, L)\) can be computed from the cutting
  in the same time. The lookup tables and leaf-table pointers can be computed
  in \(O(n^2 + t^3 \nu(t))\) time as follows: For each subcell \(\mathcal{C}\)
  among the \(\frac{n^2}{t^2}\) subcells of the last level of the hierarchy,
  compute a canonical labeling and representation of the lines intersecting
  \(\mathcal{C}\) in \(O(t^2)\) time as in Lemma~\ref{lem:canonical-labeling}.
  Insert the canonical representation in some trie in \(O(t^2)\)
  time. If the canonical representation was not already in the trie, create a lookup
  table with the answers to all \(O(t^3)\) queries on those lines and attach a
  pointer to that table in the trie. This happens at most \(\nu(t)\) times.
  In the encoding, store the canonical labeling and this new pointer or the
  pointer that was already in the trie for the subcell \(\mathcal{C}\). All
  parts of the encoding can be concatenated together in time proportional to
  the size of the encoding.
\end{proof}

\fi

%\section{\iftitlecase%
%Sublogarithmic Query Complexity\else%
%Sublogarithmic query complexity\fi}\label{sec:query-time}
\subsection{\iftitlecase%
Pushing Query Time below Logarithmic\else%
Pushing query time below logarithmic\fi}\label{sec:query-time}

We further refine the encoding introduced in the previous section so as to
reduce the query time by a \( \log\log n \) factor. We do so using
specificities of the word-RAM model that allow us to preprocess computations
on inputs of small but superconstant size.
%
This refinement is applicable to both abstract and realizable order types,
and leads to an improvement of our main theorems for the two-dimensional case:
%
\begin{contribution}[label=thm:abstract-loglog,restate=TheoremGPTAbstractLogLog]
  All abstract order types have an encoding
  using \(O(n^2)\) bits of space
  and allowing for queries in \(O(\frac{\log n}{\log \log n})\) time.
  %(\(O(n^2)\), \(O(\frac{\log n}{\log \log n})\))-encoding.
  %\(O(n^2)\)-bits
  %\(O(\frac{\log{n}}{\log{\log{n}}}))\)-querytime
  %encoding.
\end{contribution}

\begin{contribution}[
  name={
    \(o(n^2)\)-bits
    \(o(\log n)\)-querytime
    encodings for realizable OT
  },%
  label=thm:realizable-loglog,%
  restate=TheoremGPTRealizableLogLog%
]
  All realizable order types have
  a \(O(\frac{n^2 \log^\epsilon n}{\log n})\)-bits encoding
  allowing for queries in \(O(\frac{\log n}{\log \log n})\) time.
  %All realizable order types have a
  %\(O(\frac{n^2\log^\epsilon n}{\log n})\)-bits
  %\(O(\frac{\log{n}}{\log{\log{n}}}))\)-querytime encoding.
\end{contribution}

\begin{theorem}\label{thm:preprocessing-loglog}
  In the real-RAM model and the constant-degree algebraic decision tree model,
  given \(n\) real-coordinate input points in \(\mathbb{R}^2\) we can compute
  the encoding of their order type as in
  Theorems~\ref{thm:abstract-loglog}~and~\ref{thm:realizable-loglog} in
  \(O(n^2)\) time.
\end{theorem}


\paragraph{Idea}

A natural idea is to pick \(r\) to be small but superconstant to reduce the
number of levels of the hierarchy and thus the query time. As already pointed
out, this has the drawback of increasing the time complexity of the intersection
location primitive from constant to \(\Theta(r^2)\).
Since this primitive is used at each level of the hierarchy, the
\(\Theta(\log r)\) factor saved by having less levels is lost.

To get past this difficulty,
the trick is to encode approximations of the traces
\(\textsc{Tr}(\mathcal{C}, p')\) to still allow constant intersection location.
We call those approximations \emph{signatures} and denote them by
\(\textsc{Sig}(\mathcal{C}, p')\).
%
We define those signatures so that they approximately encode the cyclic
permutation of the intersections around each subcell. By carefully choosing some
parameters, we are able to fit two of those signatures in a single word of
memory. We can then precompute the output to all
possible inputs for the intersection location primitive and put them in a small
table.

Because of this size reduction, distinct pseudolines could
be mapped to identical signatures.
%
Those ambiguous situations can be deterministically handled using
an additional lookup table.
%
Because those situations rarely arise, this table also is small.

Once we have located the intersection \(p_a' \cap p_b'\), we still need to deal
with \(p_c'\). We change the layout of a trace to locate the subcell containing
\(p_a' \cap p_b'\) with respect to \(p_c'\) in constant time.
%
In case \(p_c'\) properly intersects the subcell, we need to recurse on the
subcell. To do that, we need to map the pseudoline indices \(a\), \(b\), and
\(c\) to the indices those pseudolines have in that subcell. This is
done with one indirection:
%
For each of the three pseudolines we
identify the index of the subcell in the list of subcells intersected by that
pseudoline.
This is implemented as a rank operation on a list of \(O(r^2)\) bits.
%
Given that index, we can find the intersection indices of that pseudoline in
the cyclic permutation of the subcell in constant time.

In what follows, we describe five new structures:
the signatures,
the intersection oracle,
the disambiguation table,
the augmented traces,
and the subcell mapper.
%
The first reduces the bitsize of the original traces so that the second can
be implemented in constant time. The third handles bad cases that arise because
of this size reduction. The fourth defines a new layout for the traces that
includes the signature. The fifth allows to implement the parent-to-subcell
pseudoline index mapping in constant time using this new layout.
%
\paragraph*{\iftitlecase%
Signatures\else%
Signatures\fi}
Fix a small constant \(\alpha\) and define \(r = \Theta(\log^{\alpha} n)\).%
\footnote{%
  The exact bound for how small \(\alpha\)
  must be depends on hidden constant factors in the Zone Theorem and in the definition
  of the word size. In particular, we must have \(\alpha \leq \frac 12\) when
  \(w = \Theta(\log n)\).%
}
%
We encode a \(\ell\)-levels hierarchical cutting of parameter \(r\).
%
Note that we can construct a hierarchical cutting with
superconstant \(r\) by constructing a hierarchical cutting with some
appropriate constant parameter \(r'\), and then skip levels that we do not
need.
%
As in Section~\ref{sec:lines-and-pseudolines}, we store a combinatorial representation of
this hierarchical cutting. We make some tweaks to this representation.

We augment the traces of Section~\ref{sec:lines-and-pseudolines} with a \emph{signature}.
The trace \(\textsc{Tr}(\mathcal{C}, p')\) of a cell-pseudoline pair
is composed of two parts:
The incidence bits
that tell us for each subcell of the cell whether the pseudoline lies above, lies below, intersects
or contains it, and the cyclic permutation bits used to locate the intersection
of two pseudolines inside the cell.
The first part uses \(\Theta(r^2)\) bits.
The second part uses \(\Theta(r \log{\frac{N}{r^{i+1}}})\) bits for a cell of
level \(i\).

To construct the signature \(\textsc{Sig}(\mathcal{C}, p')\),
we keep the \(\Theta(r^2) = \Theta(\log^{2\alpha} n)\) incidence bits because
they fit in sublogarithmic space for sufficiently small \(\alpha\).
The second part would use superlogarithmic space if handled as before.
We thus replace the \(\Theta(r \log{\frac{n}{r^{i+1}}})\) bits of the
cyclic permutation by a well chosen approximation.

Let \(\beta = 2^{\Theta(\log^{\alpha} n)}\) and denote by \(n_i = n/r^i\)
an upper bound on the number of pseudolines intersecting a cell of level \(i\). For
each subcell of level \(i\), partition its cyclic permutation into
\(\beta_i \leq \min \{\, \beta , n_{i+1} \,\} \) blocks
of at most \(\lceil n_{i+1} / \beta_i \rceil\) intersections.
%
For each pseudoline intersecting a cell we store the block indices that
that pseudoline touches instead of storing the cyclic permutation indices.%
\footnote{For \(\beta\) a power of two, this can be implemented by truncating
the original cyclic permutation indices.}
Hence, the second part
of each signature only uses \(O(r \log \beta) = O(\log^{2\alpha})\) bits.

\paragraph*{\iftitlecase%
Intersection Oracle\else%
Intersection oracle\fi}
We construct a lookup table to compute in constant time,
for any given cell of any given level, the subcell in which \(q_i \cap q_j\)
lies.
%Computing it via scanning with so many subcells to check would waste any
%further savings.
For that we need a general observation on the precomputation of functions on
small universes.
%
\begin{observation}\label{obs:small-universe-functions}
  In the word-RAM model with word size \(w \geq \log n\), for any word-to-word
  function \(f:[2^w] \to [2^w]\), we can build a lookup table of total bitsize
  \(2^g h\) for all \(2^g\) inputs \(x \in [2^g]\) of bitsize \(g \leq w\),
  mapping to images \(y \in [2^h]\) of bitsize \(h \leq w\),
  in time \(2^g T(g)\) where \(T(g)\) is the complexity of computing \(f(x)\), 
  \(x \in [2^g]\). The image of bitsize \(h\) of any input of bitsize \(g\) can then be
  retrieved in \(O(1)\) time by a single lookup (since inputs and outputs fit in
  a single word). In particular, the preprocessing time \(2^g T(g)\) and the
  space \(2^g h\) are sublinear as long as \(T(g)=g^{O(1)}\) and \(g + \log h
  = o(\log n)\).
\end{observation}
%
In other words, any polynomial time computable word-to-word function can be
precomputed in sublinear time and space for all inputs and outputs of
sublogarithmic size.
%
\aurelien{Maybe move to preliminaries?}

Since each pseudoline signature fits in
\(O(r^2 + r \log \beta) = O(\log^{2\alpha}{n})\)
bits,
and since the number of subcells of each cell is \(O(\log^{2\alpha}{n})\),
we can choose an appropriate \(\alpha\) so as to satisfy the requirements given
above:
take \(\alpha < \frac 12\) so that two pseudoline signatures
have a combined bitsize of \(g = o(\log n)\).
The output size is the bitsize of a subcell identifier, which is \(h \leq 2
\alpha \log \log n = o(\log n)\).
%
In some cases, the block indices stored in the signatures will not contain
enough information to point to a unique output. In those cases we store a
special value that indicates ambiguity of the input.

We can thus precompute the function that sends two pseudoline
signatures to either the subcell containing their intersection or to some
special value in case of an ambiguous input. Since we compute the function for
all members of its universe, we can implement the lookup table using
direct addressing into an array.

Note that the output of this oracle is the same no matter what level or cell we
consider: for non-ambiguous inputs, all the information required to locate
\(q_i \cap q_j\) is included in the input.
We thus only need a single lookup table, and the space needed is proportional to
\begin{displaymath}
  2^g h
  =
  2^{\Theta(\log^{2\alpha} n)}.
\end{displaymath}

\paragraph*{Disambiguation}
An input for the intersection oracle is ambiguous if and only if at least one
boundary intersection of each input pseudoline appears in the same cyclic
permutation block of the cell that contains their intersection.
%
Thus, ambiguous inputs rarely occur:
%
less than the number of blocks times the number of pairs of boundary
intersections in a block,
%
that is, less than
\(
\beta \cdot {({n_{i+1}} / \beta)}^2
=
\frac{n^2}{r^{2(i+1)}} / \beta
\)
times per subcell of level \(i\) of the hierarchy.
%
When \(\beta \geq n_{i+1}\) all ambiguity is lifted so we can ignore
those cases.

The disambiguation table is a hash table storing the answer to all ambiguous
inputs for all cells of each level \(i \in \{\, 0,1, \ldots, \ell -1\,\}\).
%
This table maps a triple of a cell
\(\mathcal{C}_{0,j_1,j_2,\ldots,j_i}\),
a pseudoline \(p_a'\),
and a pseudoline \(p_b'\) to the index \( j_{i+1} \in \{\, 0,1, \ldots, O(r^2)\,\}\) of
the subcell \(\mathcal{C}_{0,j_1,j_2,\ldots,j_i,j_{i+1}}\) containing the
intersection \(p_a' \cap p_b'\).
%
Summing over all subcells of each level, we obtain that the
number of entries in this table is bounded above by
\begin{displaymath}
  \sum_{i=0}^{\ell - 1}  r^{2i} \cdot r^2 \cdot \frac{n^2}{r^{2(i+1)}} / \beta
  =
  \frac{n^2 \ell}{2^{\Theta(\log^{\alpha} n)}}.
\end{displaymath}
%
The number of bits of each entry is at most \(\ell \lceil \log cr^2 \rceil + 2 \log n =
O(\log n)\) for the key and
\(\lceil \log cr^2 \rceil\) for the value. Both get absorbed by the
\(2^{-\Theta(\log^{\alpha} n)}\) factor in the number of entries, so we can
keep the same expression for the number of bits used by the disambiguation table.

Since the keys and values fit in a constant number of words,
we can guarantee worst case constant query time using
cuckoo~hashing~\cite{PR04} or
perfect~hashing~\cite{FKS84}.
%
In both cases the construction of the table is randomized
and takes expected linear time in the number of entries.
Perfect hashing has the advantage that we can drop the entry keys since we
only query this table for existing entries.
Note that this is the only part of the construction that is randomized.

\paragraph*{\iftitlecase%
Augmented Traces\else%
Augmented traces\fi}

The augmented traces are simply the concatenation of the signature and the
first intersection indices as depicted in Figure~\ref{fig:new-trace}.
%
\begin{figure}
  \centering{}
  \includegraphics[scale=0.7]{figures/new-trace}
  \caption{%
	  An augmented trace \(\textsc{Tr}'(\mathcal{C}, p')\)
      for the same cell-pseudoline pair as in Figure~\ref{fig:trace}.
  }\label{fig:new-trace}
\end{figure}
%
This layout allows for constant time access to the signature.
Given a subcell index we can test its incidence bits in constant time.
Given an intersected subcell index we can access the first intersection index
of the pseudoline in that subcell in constant time.

As discussed earlier, the first part of the signature uses \(O(r^2)\) bits.
We already noted that the second part of the signature
uses \(O(r \log \beta) = O(\log^{2 \alpha} n)\) bits. However, a better bound
to use for the analysis of the total space is
\(O(r \log \beta_i) = O(r \log n_{i+1})\) bits, which is proportional to the
number of bits needed for the first intersection indices.

Summing over all pseudolines and all intersected cells of each level,
the space used for the augmented traces is proportional to
\begin{displaymath}
n \sum_{i=0}^{\ell-1} r^i \cdot \left( r^2 + r \log n_{i+1} \right)
=
O\left(\frac{n^2}{t} (\log t + r)\right),
%n \sum_{i=0}^{\ell-1} r^i \cdot \left( r^2 + r \log \beta \right)
%=
%n \cdot ( r + \log \beta ) \cdot \sum_{i=0}^{\ell - 1} r^{i+1}
%=
%O\left(\frac{n^2}{t} (r + \log \beta)\right)
%=
%O\left(\frac{n^2}{t} \log^{\alpha} n\right).
\end{displaymath}
%
as in Lemma~\ref{lem:space-2-hierarchy}.
Since \(r = \Theta(\log^{\alpha} n)\) the bound becomes
\(O\left(\frac{n^2}{t} (\log t + \log^{\alpha} n)\right)\).


\paragraph*{\iftitlecase%
Subcell Mapper\else%
Subcell mapper\fi}

We need a way to map a subcell index \(0 \leq j_i \leq c_h r^2 - 1\) to the
index \(0 \leq j_i' \leq c_z r - 1\) this subcell has in the list of subcells
intersected by a given pseudoline. If we can achieve this in constant time,
then we can also access the first intersection index this pseudoline has in
this subcell.

It is not hard to see that this operation boils down to computing
\(\operatorname{rank}_{\texttt{10}}(j_i)\) where the data array is composed of
the incidence bits of the given pseudoline. Hence this can be solved by adding
an auxiliary rank-select data structure to each trace~\cite{RRS07,BH17}.
Another solution is to reuse Observation~\ref{obs:small-universe-functions} to
construct a lookup table for all \(2^{\Theta(r^2)}\) possible incidence
vectors.
%
\aurelien{Maybe add rank-select data structure to preliminaries?}

With the first solution, the space used by the traces stays the same up to a
constant factor. With the second solution, the space used is dominated by the
space taken by the intersection oracle. Hence, we do not bother including it in
the space analysis.


All that is left to do is to properly solve the subproblems spawned by the last
level of the hierarchy. This is done exactly as in the previous section.
%
\paragraph*{\iftitlecase%
Leaves of the Hierarchy\else%
Leaves of the hierarchy\fi}
%
As before, we have \(O(\frac{n^2}{t^2})\) subproblems of size \(t\) to encode.
We follow the solution previously described to obtain:
\(O(\frac{n^2}{t^2})\) pointers of size \(\lceil \log{\nu(t)} \rceil\),
\(O(\frac{n^2}{t^2})\) canonical labelings of size \(\Theta(t \log t)\),
and \(\nu(t)\) lookup tables of size \(O(t^3)\).
%
This is sufficient to take care of each of those subproblems in constant time.
The total space usage for those leaves is unchanged and stays proportional to
\begin{displaymath}
  \frac{n^2}{t^2} \cdot (t \log t + \log{\nu(t)}) + t^3 \nu(t).
\end{displaymath}

\paragraph*{\iftitlecase%
Space Complexity\else%
Space complexity\fi}
Summing all terms together and adding the space taken by the
parameters of the hierarchy \(n\), \(r\) and \(t\), we obtain:
\begin{lemma}\label{lem:space-2-all-query}
  The space taken by the encoding described in Section~\ref{sec:query-time} is
  proportional to
    \begin{displaymath}
    \underbrace{\log ntr}_{\text{Parameters}}
    +
    \underbrace{\frac{n^2}{t} (\log t + \log^{\alpha} n)}_{\text{Traces}}
    +
    \underbrace{2^{\Theta(\log^{2\alpha} n)}}_{\text{Intersection Oracle}}
    +
    \underbrace{\frac{n^2 \ell}{2^{\Theta(\log^{\alpha} n)}}}_{\text{Disambiguation Table}}
    +
    \underbrace{\frac{n^2}{t^2} ( \log \nu(t) + t \log t) + t^3 \nu(t)}_{\text{Leaves}}.
    \end{displaymath}
\end{lemma}
As before,
we take \(t = \sqrt{\delta \log n}\) for abstract order types and \(t = \delta
\log n / \log\log n\) for realizable ones.
Taking \(\delta\) to be sufficiently small,
the space taken by the leaves of the hierarchy is thus \(\Theta(n^2)\) for
abstract order types and dominated by the term \(\frac{n^2}{t} \log t\)
in the case of realizable order types.
%
Setting \(\alpha < \frac 12\)
guarantees
that the space taken by the intersection oracle is subpolynomial,
and
that the space taken by the traces is subquadratic.
%
The space taken by the
disambiguation table is in \(O(\frac{n^2}{\log^c n})\) for all \(c\) and is
thus dominated by the other terms.

For abstract order types, all those terms are subquadratic, except for the
pointers at the leaves. The total space
usage for abstract order types is thus dominated by this term and
is quadratic.
%
For realizable order types, the total space is dominated by the term
\(\frac{n^2}{t} \log^{\alpha} n\).
Indeed, we can take \(\alpha\) as small as desired
to make the factor \(\log^{\alpha} n = O(\log^{\varepsilon} n)\).
%
This proves the space constraints in
Theorems~\ref{thm:abstract-loglog}~and~\ref{thm:realizable-loglog}.
%
Unfortunately, the present solution incurs a nonabsorbable
extra \(\log^{\varepsilon} n\) factor in the realizable case. Note that a
\(\log{\log{\log{n}}}\) factor can be squeezed from the query time without
increasing the space usage by choosing \(r = \Theta(\log \log n)\) instead.


\paragraph*{\iftitlecase%
Correctness and Query Complexity\else%
Correctness and query complexity\fi}
Given a query \(p_a', p_b', p_c'\) and a cell \(\mathcal{C}\),
%at each level,
the subcell \(\mathcal{C}'\) containing \(p_a' \cap p_b'\) is found in constant
time via the
intersection oracle and, if necessary, the disambiguation table.
The location of that subcell with respect to \(p_c'\) can then be retrieved by
a single lookup in the incidence bits of \(\textsc{Tr}'(\mathcal{C}, p_c')\).
In case of recursion, we can compute the address of the traces
\(\textsc{Tr}'(\mathcal{C}', p_a')\),
\(\textsc{Tr}'(\mathcal{C}', p_b')\), and
\(\textsc{Tr}'(\mathcal{C}', p_c')\) in constant time using the subcell mapper.
The base case is handled in constant time as before: using the pointers,
canonical labelings and order type lookup tables.

We now have a shallower decision tree of depth
\(\log_r{\frac{n}{t}} = O_\alpha(\frac{\log{n}}{\log{\log{n}}})\)
and the work at each level takes constant time.
This proves the query time constraints in
Theorems~\ref{thm:abstract-loglog}~and~\ref{thm:realizable-loglog}.

\paragraph*{\iftitlecase%
Preprocessing Time\else%
Preprocessing time\fi}
We prove Theorem~\ref{thm:preprocessing-loglog}.
\begin{proof}
  As before, the hierarchical cutting and all traces \(\textsc{Tr}'(C, p')\)
  can be computed in \(O(nr^\ell)\) time (with or without rank-select data
  structures). The lookup tables and leaf-table pointers can be computed in
  \(O(n^2)\) time. The intersection oracle, the disambiguation table, and the
  optional subcell mapper can be computed in subquadratic time.
\end{proof}


%\section{Higher-Dimensional Encodings}\label{sec:hyperplanes}
\section{Encoding Chirotopes of Hyperplane Arrangements}\label{sec:hyperplanes}

\section{Higher-Dimensional Encodings}\label{sec:hyperplanes}
We generalize our point configuration encoding to any dimension \(d\). The
chirotope of a point set in \(\mathbb{R}^d\) consists of all orientations of
simplices defined by \(d+1\) points of the set~\cite{RZ04}.
The orientation of the simplex with \(d+1\) ordered vertices \(p_i\) with
coordinates \((p_{i,1} , p_{i,2} , \ldots, p_{i,d} )\) is given by the sign of
the determinant
%
\begin{displaymath}
  \begin{vmatrix}
    1 & p_{1,1} & p_{1,2} & \hdots & p_{1,d} \\
    1 & p_{2,1} & p_{2,2} & \hdots & p_{2,d} \\
    \vdots & \vdots & \vdots & \ddots & \vdots \\
    1 & p_{d+1,1} & p_{d+1,2} & \hdots & p_{d+1,d}
  \end{vmatrix}.
\end{displaymath}

We obtain the following generalized result:
%
\TheoremGPTRealizableD*
\TheoremGPTPreprocessingD*

\paragraph*{Idea}

In the primal, the orientation of a simplex whose vertices are ordered can be
interpreted as the location of its last vertex with respect to the (oriented)
hyperplane spanned by its first \(d\) vertices. In the dual, this orientation
corresponds to the location of the intersection of the first \(d\) dual
hyperplanes with respect to the last (oriented) dual hyperplane.
%
\ifjournal%
  In the primal, degenerate simplices have orientation \(0\). In the dual, this
  corresponds to linearly dependent subsets of \(d+1\) hyperplanes.
\fi%

We gave an encoding for the two-dimensional case in Section~\ref{sec:lines-and-pseudolines}.
With this encoding, a query is answered by traversing the levels of
some hierarchical cutting, branching on the location of the intersection of two
of the three query lines. We generalize this idea to \(d\) dimensions. Now the
cell considered at the next level of the hierarchy depends on the location
of the intersection of \(d\) of the \(d+1\) query hyperplanes.
\ifjournal%
  We will also have to take care of degenerate cases.
\fi%

\paragraph*{\iftitlecase%
Intersection Location\else%
Intersection location\fi}

In Section~\ref{sec:lines-and-pseudolines}, we solved the following two-dimensional subproblem:

\begin{problem}
  Given a triangle and \(n\) lines in the plane, build a data structure that,
  given two of those lines, allows to decide whether their intersection
  lies in the interior of the triangle.
\end{problem}

In retrospective, we showed that there exists such a data structure using
\(O(n \log n)\) bits that allows for queries in \(O(1)\) time. We generalize
this result. Consider the following generalization of the problem in
\(d\) dimensions:

\begin{problem}
  Given a convex body and \(n\) hyperplanes in \(\mathbb{R}^d\), build a data
  structure that, given \(d\) of those hyperplanes, allows to decide whether their
  intersection is a vertex that lies in the interior of the convex body.
\end{problem}

Of course this problem can be solved using \(O(n^d)\) space by explicitly
storing the answers to all possible queries. If the input hyperplanes are
given in an arbitrary order, this is best possible for
\(d=1\). For \(d \geq 2\), we show how to reduce the space to \(O(n^{d-1} \log
n)\) by recursing on the dimension, taking \(d = 2\) as the base case.
\aurelien{Proof of a lower bound would be a nice addition.}

We encode the function \(\mathcal{I}_{C, H}\)
that maps a \(d\)-tuple of indices of input dual
hyperplanes \(H_i\) to \(1\) if their intersection is a vertex that lies in the
interior of a fixed convex body \(C\), and to \(0\) otherwise.
%
\begin{displaymath}
  \mathcal{I}_{C,H} \colon\, {[n]}^d \to \{\, 0,1\,\} \colon\,
  (i_1,i_2,\ldots,i_d) \mapsto
  (H_{i_1} \cap H_{i_2} \cap \cdots \cap H_{i_d})
  \in  C.
\end{displaymath}
%
We call this function the \emph{intersection function} of \((C,H)\).
%
We prove the following:
\begin{theorem}\label{thm:intersection-d}
  All intersection functions have an \((O(n^{d-1} \log n),O(d))\)-encoding.
\end{theorem}


\begin{proof}
  Consider a convex body \(C\) and \(n\) hyperplanes \(H_i\).
  At first, for simplicity, assume \(C\) is \(d\)-dimensional and
  assume that any \(d\) hyperplanes \(H_{i_j}\) meet in a single point.
  With those assumptions,
  we want a data structure that can answer any query of the type
  \begin{displaymath}
    (H_{i_1} \cap H_{i_2} \cap H_{i_3} \cap H_{i_4} \cap \cdots \cap
    H_{i_{d-1}} \cap H_{i_d})
    \cap C
    \neq \emptyset.
  \end{displaymath}

  Note that this is equivalent to deciding whether
  \begin{displaymath}
    (H_{i_1} \cap H_{i_2} \cap H_{i_3} \cap H_{i_4} \cap \cdots \cap H_{i_{d-1}})
    \cap (H_{i_d} \cap C)
    \neq \emptyset,
  \end{displaymath}
  where \(H_{i_d} \cap C\) is a convex body of dimension \(d-1\) (or empty), and the
  number of hyperplanes we want to intersect it with is \(d-1\).

  We unroll the recursion until the convex body is of dimension two (or empty),
  and only two hyperplanes are left to intersect. We then notice that the decision we
  are left with is equivalent to
  \begin{displaymath}
    (H_{i_1} \cap (H_{i_3} \cap H_{i_4} \cap \cdots \cap H_{i_d}))
    \cap
    (H_{i_2} \cap (H_{i_3} \cap H_{i_4} \cap \cdots \cap H_{i_d}))
    \cap (C \cap (H_{i_3} \cap H_{i_4} \cap \cdots \cap H_{i_d}))
    \neq \emptyset,
  \end{displaymath}
  which, if the three objects are non-empty, reads:
  %
  ``Given two lines and a convex body in some plane, do they intersect?''.
  %
  We can answer this query if we have the encoding for \(d = 2\) which is
  obtained by replacing \emph{triangle} by \emph{convex body} in the
  two-dimensional original problem. The total space taken is multiplied by
  \(n\) for each time we unroll the recursion times the space taken in two
  dimensions, which is proportional to \(n^{d-2} \cdot n \log{n} = O(n^{d-1} \log{n}) \).
  Queries can then be answered in \(O(d)\) time.

  Note that degenerate cases are likely to arise:
  %
  convex bodies that are empty because of nonintersecting hyperplanes, convex
  bodies that are higher dimensional because of linearly dependent hyperplanes,
  and convex bodies that are lower dimensional because \(C\) was not
  full-dimensional to start with.
  %
  However, all those cases can be dealt with appropriately:
  %
  If the query suffix
  \(H_{i_3} \cap H_{i_4} \cap \cdots \cap H_{i_d}\)
  leads to an empty convex body
  \(C \cap (H_{i_3} \cap H_{i_4} \cap \cdots \cap H_{i_d})\)
  then the query point is not in \(C\) and we encode
  \(0\) for all the queries in \(C\) ending in this suffix.
  This information can be encoded in a table of size \(O(n^{d-2})\).
  %
  If the query suffix
  \(H_{i_3} \cap H_{i_4} \cap \cdots \cap H_{i_d}\)
  leads to a convex body
  \(C \cap (H_{i_3} \cap H_{i_4} \cap \cdots \cap H_{i_d})\)
  of dimension \(\geq 3\) then the intersection of all objects
  is not a \(0\)-flat and we encode a \(0\) to follow the definition
  of \(\mathcal{I}_{C,H}\).
  Again, this information can be encoded in a table of size \(O(n^{d-2})\).
  %
  If the query suffix
  \(H_{i_3} \cap H_{i_4} \cap \cdots \cap H_{i_d}\)
  leads to a convex body
  \(C \cap (H_{i_3} \cap H_{i_4} \cap \cdots \cap H_{i_d})\)
  of dimension zero, one, or two,
  then we use the encoding of size \(O(n \log n)\) described in
  Section~\ref{sec:lines-and-pseudolines} as a base case.
\end{proof}

%Note that there is a linear lower bound for encoding this subproblem in
%dimension \(d=1\) and a \(\Omega(n \log n)\) lower bound for \(d=2\). We might be able
%to strengthen the lower bound for higher dimensions.

In the paragraphs that follow, we show how to plug this result in those of
the previous sections to obtain analogous results for the \(d\)-dimensional
version of the problem (Theorem~\ref{thm:realizable-d} and
Theorem~\ref{thm:preprocessing-d}).

\paragraph*{Encoding}
We build a hierarchical cutting as in
Section~\ref{sec:lines-and-pseudolines} (this time in dimension
\(d\)).
%
Given \(n\) hyperplanes in \(\mathbb{R}^d\) and some fixed
parameters \(r\) and \(\ell\), compute a \(\ell\)-levels hierarchical cutting
of parameter \(r\) for those
hyperplanes as in Lemma~\ref{lem:hierarchical-cutting-d}.
This hierarchical cutting consists of \( \ell \) levels
labeled \(0,1,\ldots,\ell-1\). Level \(i\) has \(O(r^{di})\)
cells. Each of those cells is further partitioned into
\(O(r^d)\) subcells. The \(O(r^{d(i+1)})\) subcells of level \(i\)
are the cells of level \(i+1\). Each cell of level \(i\) is intersected by at
most \(\frac{n}{r^i}\) hyperplanes, and hence each subcell is intersected by at
most \(\frac{n}{r^{i+1}}\) hyperplanes.

We compute and store a combinatorial representation of the hierarchical cutting
as follows: For each level of the hierarchy, for each cell in that level, for
each hyperplane intersecting that cell, for each subcell of that cell, we store
two bits to indicate the location of the hyperplane with respect to that
subcell\ifeurocg\else, that is, whether the hyperplane lies above
(\texttt{00}), lies below (\texttt{01}), intersects the interior of that
subcell (\texttt{10}), or contains the subcell (\texttt{11})\fi.
%
When a hyperplane \(H_{i}\) intersects the interior of a subcell \(\mathcal{C}'\),
we also store the two \(O(\log \frac{n}{r^{i+1}})\)-bits two-dimensional
intersections indices this hyperplane has in each of the
\(O\left({\left(\frac{n}{r^{i+1}}\right)}^{d-2}\right)\)
query suffixes \((H_{i_3} \cap H_{i_4} \cap \ldots \cap H_{i_d})\) with each of the
\(H_{i_j}\) properly intersecting \(\mathcal{C}'\).

This representation takes
\(O\left(
  \frac{n}{r^i}
  +
  {\left(\frac{n}{r^{i+1}}\right)}^{d-1} \log{\frac{n}{r^{i+1}}}
\right)\)
bits per subcell of level \(i\) by storing for each
hyperplane its location and, when needed, the bits they hold in
the encoding of the intersection function of the subcell.
At the last level of the hierarchy, let \(t =
\frac{n}{r^\ell}\) denote an upper bound on the number of hyperplanes
intersecting each subcell. For each of those \(O(r^{d \ell}) =
O(\frac{n^d}{t^d})\) subcells we store a pointer to a lookup table of size
\(O(t^{d+1})\) that allows to answer the query of the orientation of any
triple of hyperplanes intersecting that subcell.

Using the Zone Theorem in higher dimensions (Theorem~\ref{thm:zone-theorem-d}),
we can have all bits belonging to a single cell-hyperplane pair in a contiguous
block of memory with the same space bound.

\paragraph*{\iftitlecase%
Space Complexity\else%
Space complexity\fi}
As before, for the space taken by the lookup tables, their
associated pointers and canonical labelings at the leaves, and the parameters
of the hierarchy \(n\), \(r\) and \(t\), the analysis is immediate. If not
implicitly known, the dimension \(d\) can also trivially be added to the
encoding.

For the space taken by the hierarchy,
we generalize Lemma~\ref{lem:space-2-hierarchy} of
Section~\ref{sec:lines-and-pseudolines}. Let \(H_r^\ell(n,d) \in \mathbb{N}\) be the maximum
amount of space (bits), over all arrangements of \(n\) hyperplanes in \(\mathbb{R}^d\),
taken by the \(\ell \in \mathbb{N}\) levels of a hierarchy with parameter \(r \in
(1,+\infty)\).
%
\begin{lemma}\label{lem:space-d-hierarchy}
For \( r \geq 2 \) we have
\begin{displaymath}
H_r^\ell(n,d)
=
O\left(\frac{n^d}{t} \left(\log t + \frac{r}{t^{d-2}}\right)\right).
\end{displaymath}
\end{lemma}

\begin{proof}
By definition we have
\begin{displaymath}
H_r^\ell(n,d)
= O \left(
  \sum_{i=0}^{\ell-1} \left(
	r^{di} \cdot r^d \cdot \left(
	  \frac{n}{r^i} + {\left(\frac{n}{r^{i+1}}\right)}^{d-1} \cdot \log \frac{n}{r^{i+1}}
    \right)
  \right)
\right).
\end{displaymath}
%
Using \(t = \frac{n}{r^{\ell}}\), reversing the summation with
\(i \leftarrow \ell - i - 1\), and grouping the terms, we have
\begin{displaymath}
H_r^\ell(n,d)
=
O \left(
\frac{n^d}{t} \left(
	\frac{r}{t^{d-2}}
	\sum_{i=0}^{\ell-1} \frac{1}{{(r^{d-1})}^i}
	+
	\log t
	\sum_{i=0}^{\ell-1} \frac{1}{r^i}
	+
	\log r
	\sum_{i=0}^{\ell-1} \frac{i}{r^i}
\right)
\right).
\end{displaymath}
%
Using the geometric inequalities
(see the proof of Lemma~\ref{lem:space-2-hierarchy})
the statement follows from \(r \geq 2\).
%
\end{proof}


The final picture is almost the same as in Figure~\ref{fig:space-analysis}.
%
Summing all terms, we obtain
%
\begin{lemma}\label{lem:space-d-all}
    The space taken by the encoding described in Section~\ref{sec:hyperplanes}
    is proportional to
    \begin{displaymath}
      \underbrace{\log dntr}_{\text{Parameters}}
      +
      \underbrace{\frac{n^d}{t} \left( \log t + \frac{r}{t^{d-2}} \right)}_{\text{Traces}}
      +
      \underbrace{\frac{n^d}{t^d} ( \log \nu_d(t) + t \log t )}_{\text{Leaves}}
      +
      \underbrace{t^{d+1}\nu_d(t)}_{\text{Lookup Tables}},
    \end{displaymath}
    where \(\nu_d(t) = 2^{\Theta(d^2 t \log t)}\) denotes the number of
    realizable rank-\((d+1)\) chirotopes of size \(t\).
\end{lemma}

We pick \(r\) constant and choose \(t = \delta \log n / \log \log n\) for small
enough \(\delta\). The second term in Lemma~\ref{lem:space-d-all} dominates
with \(n^d {(\log \log n)}^2 / \log n\). This proves the space constraint in
Theorem~\ref{thm:realizable-d}.

\paragraph*{\iftitlecase%
Correctness and Query Complexity\else%
Correctness and query complexity\fi}
As before, a query is answered by traversing the hierarchy, which takes
\(O(\log n)\) time. The query time can be further improved using the method from
Section~\ref{sec:query-time} with \(r = \Theta(t^{d-2} \log t)\). This proves the
query time constraint in Theorem~\ref{thm:realizable-d}.

\paragraph*{\iftitlecase%
Preprocessing Time\else%
Preprocessing time\fi}
We prove Theorem~\ref{thm:preprocessing-d}.
\begin{proof}
  The hierarchical cuttings can be computed in \(O(n{(r^\ell)}^{d-1})\) time.
  The lookup table and leaf-table pointers can be computed in \(O(n^d)\) time
  using the canonical labeling and representation for rank-\((d+1)\) chirotopes
  given in~\cite{AILOW14}. The intersection oracle, the disambiguation table,
  and the subcell mapper can be computed in \(o(n^{d})\) time.
\end{proof}



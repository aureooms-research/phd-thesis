Given three sets of \(N\) real numbers
\(A = \{\, a_1 < a_2 < \cdots < a_N\,\} \),
\(B = \{\, b_1 < b_2 < \cdots < b_N\,\} \),
and \(C = \{\, c_1 < c_2 < \cdots < c_N\,\}\),
we wish to build a discrete data structure (using bits, words, and pointers) such that,
given any triple \((i,j,k) \in {[N]}^3\) it is possible to compute the sign of
\(a_i + b_j + c_k\) by only inspecting the data structure (we cannot consult
\(A\), \(B\), or \(C\)).
We refer to the map $\chi : {[N]}^3\to \{-,0,+\}, (i,j,k)\mapsto\mathrm{sgn}
(a_i+b_i+c_k)$ as the {\em 3SUM-type} of the instance $\langle A,B,C \rangle$.
Obviously, one can simply construct a lookup table of size \(O(N^3)\), such
that triple queries can be answered in \(O(1)\) time. We aim at improving on
this trivial solution.

In the 3SUM problem, we are given an array of numbers as input and are asked
whether any three of them sum to 0. In the mid-nineties, this problem was
identified as a bottleneck of many
important problems in geometry, such as detection of affine degeneracies or
motion planning~\cite{GO95}. Since then, it has become a central problem in
fine-grained complexity theory~\cite{PW10}. It has long been conjectured to
require $\Omega (N^2)$ time. In 2014, it was shown to be solvable in $o(N^2)$
time, but no algorithm with running time $O(N^{2-\delta})$ with constant
$\delta>0$ is known~\cite{GP18}.

Lower bounds exist in restricted models of computation. Most notably,
$\Omega(N^2)$ 3-linear queries are needed to solve 3SUM~\cite{Er99a},
and nontrivial lower bounds have also been proven for slightly more powerful linear
decision trees~\cite{AC05}. However, in a recent breakthrough contribution, Kane, Lovett,
and Moran showed that 3SUM could be solved using $O(N\log^2 N)$
6-linear queries~\cite{KLM18}, hence within a $O(\log N)$ factor of the
information-theoretic lower bound.

Linear decision trees are examples of {\em nonuniform algorithms}, in which we
are allowed to have different algorithms for different input sizes.
Algebraic decision trees generalize linear decision trees
by allowing decision based on the sign of constant-degree polynomials at each
node~\cite{SY82}.

Any decision tree identifying the 3SUM-type of a 3SUM instance yields a concise
encoding of this 3SUM-type:
just write down the outcome of the successive tests. Knowing the decision tree
by convention, this sequence of bits is
sufficient to recover the sign of any triple.

The question we consider here is how to make such a representation efficient,
in the sense that not only does it use merely a few bits, but the answer to any
triple query can be recovered efficiently. Understanding the interplay between
nonuniform algorithms and such data structures hopefully sheds light on the
intrinsic structure of the problem.

\paragraph{Contribution}

As there are only $O(N^3)$ queries, a table
of size $(\log_2 3) N^3 + O(1)$ bits suffices to give constant query time
\cite{DPT10}. This can be improved to $O(N^2\log N)$ bits of space by
storing for each pair $(i,j)$ the values
\(k_<(i,j) = \max \{ 0\}\cup \{k \colon\, a_i + b_j + c_k < 0\}\) and
\(k_>(i,j) = \min \{ N+1\}\cup \{k \colon\, a_i + b_j + c_k > 0\}\).
For a query \((i,j,k)\), we compare \(k\) against the values \(k_<(i,j)\) and \(k_>(i,j)\)
to recover \(\chi(i,j,k)\) in \(O(1)\) time. All \(k_<(i,j)\) and \(k_>(i,j)\)
can be computed in \(O(N^2)\) time via the classic quadratic time algorithm for
3SUM.
%: if \(k \leq k_<(i,j)\), then \(\chi(i,j,k)=-1\), if \( k_<(i,j)< k < k_>(i,j)\),
%then \(\chi(i,j,k)=0\), and if \( k_>(i,j)\leq k\), then \(\chi(i,j,k)=+1\).
%The representation takes \( O(N^2 \log N \) bits, and each query can be answered by comparing
%pairs of indices, which takes \(O(1)\) time.

One seemingly simple representation is to store the numbers in $A$, $B$ and
$C$; however these are reals and thus we need to make them representable using
a finite number of bits.
In Section~\ref{s:numbers} we show that a minimal integer representation of a
3SUM instance may require $\Theta(N)$ bits per value, which would give
rise to a $O(N)$ query time and $O(N^2)$ space, which is far from
impressive.
In \cite{CCILO18} the problem of given a set of $N$ lines, to create an
encoding of them so that the orientation of any triple (the \emph{order type})
can be determined was studied; our problem is a special case of this where the
lines only have three slopes.
Can we do better for the case of 3SUM? We answer this in the affirmative.
In Section~\ref{s:space} we show how to use an optimal $O(N \log N)$ bits of
space with a polynomial query time. Finally, in section~\ref{s:sscqt} we show
how to use $\tilde{O}(N^{3/2})$ space to achieve $O(1)$-time queries.

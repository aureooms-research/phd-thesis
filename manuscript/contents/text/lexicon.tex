\chapternonum{Lexicon}

space

time

an hyperplane \(H\).

a set of hyperplanes \(\mathcal{H}\).

a set of sets \(\mathcal{S}\)

an arrangement of hyperplanes \(\mathcal{A}(\mathcal{H})\).

a polygonal (convex) cell C (or P ?)

a simplicial cell \(S\).

a trapezoidal cell \(T\)

epsilon

\(\mathbb{R}^d\) euclidean space

\(k\)-subset

In what follows, we use the notation \([n] = \{\,1,2,\ldots ,n\,\}\).

Big-Oh notation

The notation $\tilde{O}$ ignores polylogarithmic.

Throughout this document, $\varepsilon$ denotes a positive real
number that can be made as small as desired.

\(O_{p_1,p_2, \ldots}(f(n))\) means that the constant of proportionality
depends on the parameters \(p_i\).

When we write \(O(f(n_1,n_2, \ldots))\) we assume that one of the variables
\(n_i\) is
the one going towards infinity in the big-oh notation, while the others are
monotone growing functions of \(n_i\).

Logarithms are in base two unless otherwise indicated.

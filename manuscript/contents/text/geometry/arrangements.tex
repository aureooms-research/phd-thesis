An \emph{hyperplane} in \(\mathbb{R}^d\) is the set of points \(q \in \mathbb{R}^d\)
satisfying a linear equation of the form \(\sum_{i=1}^{d} p_i q_i = 0\) for
some tuple of coefficients \(p \in \mathbb{R}^d\). An \emph{affine} hyperplane has an
additional independent term \(p_0\) to get the equation \(\sum_{i=1}^{d}
p_i q_i = p_0\). In general we will drop the adjective ``affine'' and talk about
hyperplanes without specifying whether \(p_0 = 0\).
%
In the plane an hyperplane is a line. In \(\mathbb{R}^3\) an hyperplane is
a plane. On the real line an hyperplane is a point.

An hyperplane \(H\) partitions the space \(\mathbb{R}^d\) into two halfspaces,
often denoted by \(H^-\) and \(H^+\).
Naturally, \(H^-\) is defined to be the set of points such that
\(\sum_{i=1}^{d} p_i q_i < p_0 \) and \(H^+\) is defined to be the set of
points such that \(\sum_{i=1}^{d} p_i q_i > p_0 \). When we want to be more
precise as to whether the inequality is strict we can write \(H^{<}\),
\(H^{\leq}\), \(H^{>}\), or \(H^{\geq}\).
We also define \(H^0\) and \(H^{=}\) as synonyms of \(H\).

A set of hyperplanes \(\mathcal{H} = \{\, H_1, H_2, \ldots, H_m\,\}\)
partitions \(\mathbb{R}^d\) into convex regions called \emph{cells}.
%
This partition is denoted by \(\mathcal{A}(\mathcal{H})\) and is called the
\emph{arrangement} of \(\mathcal{H}\).
%
Each sign vector \(\sigma \in {\{\,-,0,+\,\}}^{m}\) corresponds to a
(potentially empty) cell of this arrangement defined as the set of points \(q\)
such that \(q \in H_i^{\sigma_i}\) for all \(H_i \in \mathcal{H}\).

It is useful to understand the complexity of such an arrangement when
the number \(m\) of hyperplanes grows and the dimension \(d\) is fixed.
The goal is to bound the number of nonempty cells an arrangement
can have. When \(m \leq d\) each of the \(3^m\) cells is nonempty, assuming
general position. However, when we fix \(d\) and make \(m\) grow, we get a more
reasonable behavior.
%
\begin{theorem}[name={Buck~\cite{Bu43}, see also~\cite[Theorem~24.1.1 and Corollary~24.1.2]{Hal04}},label=thm:buck]
Consider the partition of space defined by an arrangement of $m$ hyperplanes in
$\R^d$.
The number of regions of dimension $k \le d$ is at most
\begin{displaymath}
	{m \choose d-k} \sum_{i=0}^{k} {m-d+k \choose i}.
\end{displaymath}
This bound is attained when the hyperplanes are in general position.
The number of regions of all dimensions is \(\Theta(m^d)\) in that case.
\end{theorem}




\paragraph*{\iftitlecase%
The Zone Theorem\else%
The zone theorem\fi}

The zone of a given pseudoline of an arrangement is the set of cells of the
arrangement supported by that pseudoline.
%
Figure~\ref{fig:a-zone-in-the-plane} illustrates a zone in a two-dimensional
arrangement of lines.
%
\begin{figure}
  \centering{}
  \includegraphics[scale=1]{figures/a-zone-in-the-plane}
  \caption{%
    The zone defined by the dashed line in the two-dimensional
    arrangement of the plain lines is emphasized in light grey.%
  }\label{fig:a-zone-in-the-plane}
\end{figure}

We define the complexity of each cell to be the number of its sides.
We define the complexity of a zone to be the sum of the complexities of its cells.
%
The Zone Theorem states that the complexity of any zone is linear.
%
\begin{theorem}[Zone Theorem in the plane~\cite{BEPY90}]\label{thm:zone-theorem-2}
Given an arrangement of \(n+1\) pseudolines,
%
%in \(\mathbb{R}^2\),
%
the sum of the numbers of sides
%
in all the cells supported by one of the pseudolines
%
is at most \(\lfloor 9.5 n \rfloor - 1\).%
\footnote{%
Note that an earlier weaker (worse constant factor) linear bound is implied by
a theorem in~\cite{CGL85}.%
}
\end{theorem}


\aurelien{Define constant \(c_z = 9.5\) to avoid Big-Oh notation used later?}

%
%This result is important because it allows optimal
%incremental construction of arrangements of line arrangements, a frequently
%used tool.

In~\cite{EOS86} and in the first edition of~\cite{Ed12}, there were claims of
generalization of this result to arrangements of hyperplanes in higher
dimension.
%
However, the proofs turned out to be flawed~\cite{ESS93}.
%
Edelsbrunner, Sturmfels, and Sharir were the first to provide a valid proof of
the generalized result~\cite{ESS93}.
%
\begin{theorem}[Zone Theorem~\cite{ESS93}]\label{thm:zone-theorem-d}
Given an arrangement of \(n\) hyperplanes
%
in \(\mathbb{R}^d\),
%
the sum of the numbers of \(0\)-faces, \(1\)-faces, \dots, and (\(d-1\))-faces
%
in all the cells supported by one of the hyperplanes
%
is \(O(n^{d-1})\).
\end{theorem}


%They also cite valid proofs by Houle and Matou{\v s}ek for a weaker version of the
%theorem:
%
%The Zone Theorem bounds the sum of complexities of the cells in a zone.
%
%The full generalization of the theorem defines the complexity of each cell to
%be the number of all \(0\)-faces, \(1\)-faces, \dots, and (\(d-1\))-faces
%supporting the cell.
%
%The weak generalization only counts (\(d-1\))-faces.
%
%However, no publication of those proofs is known to the authors.


% TODO \section{Bottom Vertex Triangulation}%
\label{sec:arrangements:triangulation}

\begin{figure}
  \centering{}
  \includegraphics[width=\linewidth]{figures/bottom-vertex-triangulation}
  \caption{%
    The bottom vertex triangulation of a cell in an arrangement of lines.%
  }%
  \label{fig:bvt}
\end{figure}

Given an arrangement of hyperplanes in \(\mathbb{R}^d\), its bottom vertex
triangulation is the partition of space obtained by triangulating each cell of
the arrangement recursively as follows: taking \(d=2\) as the base case
(because edges are already simplices),
let \(p\) be its bottom-most vertex,
for each facet of the cell not containing \(p\),
for each \(d-1\)-dimensional simplex \(S'\) of the bottom vertex triangulation
of the facet in \(\mathbb{R}^{d-1}\),
construct a \(d\)-dimensional simplex \(S\) that is the convex hull of
\(S'\) and \(p\).
%
The
bottom vertex triangulation of an arrangement of lines in the plane is illustrated
in Figure~\ref{fig:bvt}.

Weeelll known~\cite{GO04,Cla88}.

\todo{We use this in Meiser's algorithm and Order Type encoding paper. In
Meiser we only construct one simplex.}

% TODO \section{Vertical Decomposition}%
\label{sec:arrangements:vertical-decomposition}


\begin{problem}[Point Location (in an Arrangement of Hyperplanes)]\label{problem:point-location-hyperplane}
	Given a set \(\mathcal{H}\) of \(m\) hyperplanes and query point
	\(q\in\mathbb{R}^d\),
	decide whether \(q\) lies on any \(H \in \mathcal{H}\).
\end{problem}


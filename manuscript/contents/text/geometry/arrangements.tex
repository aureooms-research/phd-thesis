An \emph{hyperplane} in \(\mathbb{R}^d\) is the set of points \(q \in \mathbb{R}^d\)
satisfying a linear equation of the form \(\sum_{i=1}^{d} p_i q_i = 0\) for
some tuple of coefficients \(p \in \mathbb{R}^d\). An \emph{affine} hyperplane has an
additional independent term \(p_0\) to get the equation \(\sum_{i=1}^{d}
p_i q_i = p_0\). In general we will drop the adjective ``affine'' and talk about
hyperplanes without specifying whether \(p_0 = 0\).
%
In the plane an hyperplane is a line. In \(\mathbb{R}^3\) an hyperplane is
a plane. On the real line an hyperplane is a point.

An hyperplane \(H\) partitions the space \(\mathbb{R}^d\) into two halfspaces,
often denoted by \(H^-\) and \(H^+\).
Naturally, \(H^-\) is defined to be the set of points such that
\(\sum_{i=1}^{d} p_i q_i < p_0 \) and \(H^+\) is defined to be the set of
points such that \(\sum_{i=1}^{d} p_i q_i > p_0 \). When we want to be more
precise as to whether the inequality is strict we can write \(H^{<}\),
\(H^{\leq}\), \(H^{>}\), or \(H^{\geq}\).
We also define \(H^0\) and \(H^{=}\) as synonyms of \(H\).

A set of hyperplanes \(\mathcal{H} = \{\, H_1, H_2, \ldots, H_m\,\}\)
partitions \(\mathbb{R}^d\) into convex regions called \emph{cells}.
%
This partition is denoted by \(\mathcal{A}(\mathcal{H})\) and is called the
\emph{arrangement} of \(\mathcal{H}\).
%
Each sign vector \(\sigma \in {\{\,-,0,+\,\}}^{m}\) corresponds to a
(potentially empty) cell of this arrangement defined as the set of points \(q\)
such that \(q \in H_i^{\sigma_i}\) for all \(H_i \in \mathcal{H}\).

\section{\done Counting Cells}

It is useful to understand the complexity of such an arrangement when
the number \(m\) of hyperplanes grows and the dimension \(d\) is fixed.
The goal is to bound the number of nonempty cells an arrangement
can have. When \(m \leq d\) each of the \(3^m\) cells is nonempty, assuming
general position. However, when we fix \(d\) and make \(m\) grow, we get a more
reasonable behavior.
%
\begin{theorem}[name=Buck~\cite{Bu43},label=thm:buck]
Consider the partition of space defined by an arrangement of $m$ hyperplanes in
$\R^d$.
The number of regions of dimension $k \le d$ is at most
\begin{displaymath}
	{m \choose d-k}
	\left(
		{m-d+k \choose 0}
		+
		{m-d+k \choose 1}
		+
		\cdots
		+
		{m-d+k \choose k}
	\right)
\end{displaymath}
and the number of regions of all dimensions is \(O(m^d)\).
\end{theorem}


This bound allows us to derive precise lower and upper bounds for algorithms
manipulating those arrangements.
%
See~\cite{Hal04} for more details and generalizations.


\section{Pseudolines}

In the plane, line arrangements are generalized to pseudoline arrangements by
dropping the ``straight'' nature of lines.
%
A pseudoline in \(\mathbb{R}^2\) is a simple curve connecting points at
infinity.
An arrangement of pseudolines is a collection of pseudolines that pairwise
intersect exactly once.
%
This definition can be made more formal by considering pairwise-intersecting
simple closed curves in the projective plane instead.
%
Some important differences between line and pseudoline arrangements have
already been discussed in \S\ref{sec:history:pol:order-types}.

Putting those differences aside, pseudoline arrangements share sufficient
similarity with line arrangements as to allow a generic algorithmic
treatment of both kinds.
%
For instance,
%
their arrangement complexities
(\S\ref{sec:arrangements:arrangement-complexity})
are
asymptotically the same,
%
the Zone Theorem (\S\ref{sec:arrangements:zone-theorem}) holds in general,
%
canonical labelings (\S\ref{sec:point-configurations:canonical-labelings})
can be constructed by looking at the order type only
(see \S\ref{sec:history:pol:lb} for a definition of order type),
and generic cell decompositions
(\S\ref{sec:arrangements:cell-decomposition})
allow the use of the divide-and-conquer methods of
Chapter~\ref{chapter:divide-and-conquer} in both cases.

In Paper~\ref{paper:order-type-encoding} we consider the problem of encoding
order types of line or pseudoline arrangements.
%
Strictly speaking, an order type consists of pure combinatorial information and
does not require an embedding to be known.
%
In particular, given an order type, it is
\(\exists \mathbb{R}\)-complete to decide whether a straigth embedding
exists~\cite{Mne85,Mne88}.
%
For simplicity, the algorithms constructing the encodings in
Paper~\ref{paper:order-type-encoding} assume an embedding is given.

For more on pseudoline arrangements see Chapter
5 of the Handbook on the subject~\cite{Goo04}.
%
For more on equivalent representations of pseudoline arrangements see Chapter
6 of the Handbook on oriented matroids~\cite{RZ04}.

%For Basic properties~\cite[\S{}5.1]{Goo04}.
%Combinatorial results~\cite[\S{}5.4]{Goo04}.
%Topology~\cite[\S{}5.5]{Goo04}.
%Computation~\cite[\S{}5.6]{Goo04}.

%\begin{theorem}[Goodman~\cite{Go80}]
	%Every arrangement of pseudolines is isomorphic to a wiring diagram.
%\end{theorem}



The zone of a given pseudoline of an arrangement is the set of cells of the
arrangement supported by that pseudoline.
%
Figure~\ref{fig:a-zone-in-the-plane} illustrates a zone in a two-dimensional
arrangement of lines.
%
\begin{figure}
  \centering{}
  \includegraphics[width=\linewidth]{figures/a-zone-in-the-plane}
  \caption{%
    The zone defined by the dashed line in the two-dimensional
    arrangement of the plain lines is emphasized in light grey.%
  }\label{fig:a-zone-in-the-plane}
\end{figure}

We define the complexity of each cell to be the number of its sides.
We define the complexity of a zone to be the sum of the complexities of its cells.
%
The Zone Theorem states that the complexity of any zone is linear.
%
\begin{theorem}[Zone Theorem in the plane~\cite{BEPY90}]\label{thm:zone-theorem-2}
Given an arrangement of \(n+1\) pseudolines,
%
%in \(\mathbb{R}^2\),
%
the sum of the numbers of sides
%
in all the cells supported by one of the pseudolines
%
is at most \(\lfloor 9.5 n \rfloor - 1\).%
\footnote{%
Note that an earlier weaker (worse constant factor) linear bound is implied by
a theorem in~\cite{CGL85}.%
}
\end{theorem}



This result is important because it allows optimal
incremental construction of arrangements of line arrangements, a frequently
used tool.


\section{Bottom Vertex Triangulation}%
\label{sec:arrangements:triangulation}

\begin{figure}
  \centering{}
  \includegraphics[width=\linewidth]{figures/bottom-vertex-triangulation}
  \caption{%
    The bottom vertex triangulation of a cell in an arrangement of lines.%
  }%
  \label{fig:bvt}
\end{figure}

Given an arrangement of hyperplanes in \(\mathbb{R}^d\), its bottom vertex
triangulation is the partition of space obtained by triangulating each cell of
the arrangement recursively as follows: taking \(d=2\) as the base case
(because edges are already simplices),
let \(p\) be its bottom-most vertex,
for each facet of the cell not containing \(p\),
for each \(d-1\)-dimensional simplex \(S'\) of the bottom vertex triangulation
of the facet in \(\mathbb{R}^{d-1}\),
construct a \(d\)-dimensional simplex \(S\) that is the convex hull of
\(S'\) and \(p\).
%
The
bottom vertex triangulation of an arrangement of lines in the plane is illustrated
in Figure~\ref{fig:bvt}.

Weeelll known~\cite{GO04,Cla88}.

\todo{We use this in Meiser's algorithm and Order Type encoding paper. In
Meiser we only construct one simplex.}


\section{Vertical Decomposition}%
\label{sec:arrangements:vertical-decomposition}

\begin{figure}
	\centering{}
    \begin{subfigure}[t]{0.5\textwidth}
		\centering
		\includegraphics{figures/decomposition}
		\caption{Some curves in $\mathbb{R}^2$.}%
		\label{fig:some-curves-in-R2}
    \end{subfigure}%
    \begin{subfigure}[t]{0.5\textwidth}
		\centering
		\includegraphics{figures/recursion}
		\caption{Vertical decomposition of the curves in Figure~\ref{fig:some-curves-in-R2}.}%
    \end{subfigure}
	\caption{Vertical decomposition.}\label{fig:vd}
\end{figure}

Given an arrangement of curves in \(\mathbb{R}^2\), its vertical decomposition
is the partition of space obtained by shooting a vertical segment from each
vertex of the arrangement until it hits a curve of the arrangement. The
vertical decomposition of an arrangement of two circles is illustrated
in Figure~\ref{fig:vd}.

This decomposition can be generalized to work for hypersurfaces in
\(\mathbb{R}^d\). Unfortunatly, the behaviour of such decompositions quickly
degenerates with \(d\).
While bottom vertex triangulation gives a bound of \(O(n^d)\) cells in
\(\mathbb{R}^d\) and
vertical decomposition gives a bound of \(O(n^2)\) cells in \(\mathbb{R}^2\),
we only know of a few upper bounds in fixed dimensions and some of them are
worse than \(O(n^d)\). This is why in our application of Meiser's algorithm we
use the bottom vertex triangulation. However, as observed by Ezra and
Sharir~\cite{ES17}, the bad behaviour of vertical decompositions is not a
bottleneck of Meiser's algorithm since we only need to look at a single cell of
the decomposition. Moreover, the complexity of those cells is better than that
of simplicial ones: vertical decomposition yields prisms supported by at most
\(2d\) hyperplanes of the arrangement whereas simplices of
the bottom vertex triangulation are supported by \(d^2 + d\) hyperplanes in the
worst case. This has a direct impact on the VC-dimension of the range spaces
defined by those objects and makes vertical decomposition the winning strategy
for this particular algorithm.

For the construction of the vertical decomposition of an arrangment of
polynomial curves in \(\mathbb{R}^2\),
we refer the reader to Pach and Sharir~\cite{Alcala}, Chazelle et
al.~\cite{CEGS91}, and Edelsbrunner et al.~\cite{EGPPSS92}.

\todo{Add remark that we only care about vertical decomposition of a constant
number of curves in \(\mathbb{R}^2\)}.


The Point Location problem is a classic problem in Computational Geometry.
The Point Location problem is a classic problem in Computational Geometry.
The Point Location problem is a classic problem in Computational Geometry.
\input{text/definition/point-location}

Because the bounds in \Cref{thm:buck} are attained when \(\mathcal{H}\) is in
general position, there is a lower bound of \(\Omega(d \log m)\) on the depth
of decision trees solving this problem.

For our purpose, it is useful to see the \(k\)-SUM problem as a point location
problem in \(\mathbb{R}^n\) where the coordinates of \(q\) are the input
numbers and where \(\mathcal{H}\) is the set of all \(n \choose k\) hyperplanes
of equation \(\sum_{j=1}^{k} x_{i_j} = 0\).

Sorting can also be seen as a point location problem in \(\mathbb{R}^n\).
In this case \(\mathcal{H}\) is the set of all \(n \choose 2\) hyperplanes of
equation \(x_i = x_j\). The arrangement \(\mathcal{A}(\mathcal{H})\) has
exactly \(n!\) \(n\)-dimensional cell that correspond to the \(n!\)
permutations the input might have.


Because the bounds in \Cref{thm:buck} are attained when \(\mathcal{H}\) is in
general position, there is a lower bound of \(\Omega(d \log m)\) on the depth
of decision trees solving this problem.

For our purpose, it is useful to see the \(k\)-SUM problem as a point location
problem in \(\mathbb{R}^n\) where the coordinates of \(q\) are the input
numbers and where \(\mathcal{H}\) is the set of all \(n \choose k\) hyperplanes
of equation \(\sum_{j=1}^{k} x_{i_j} = 0\).

Sorting can also be seen as a point location problem in \(\mathbb{R}^n\).
In this case \(\mathcal{H}\) is the set of all \(n \choose 2\) hyperplanes of
equation \(x_i = x_j\). The arrangement \(\mathcal{A}(\mathcal{H})\) has
exactly \(n!\) \(n\)-dimensional cell that correspond to the \(n!\)
permutations the input might have.


Because the bounds in \Cref{thm:buck} are attained when \(\mathcal{H}\) is in
general position, there is a lower bound of \(\Omega(d \log m)\) on the depth
of decision trees solving this problem.

For our purpose, it is useful to see the \(k\)-SUM problem as a point location
problem in \(\mathbb{R}^n\) where the coordinates of \(q\) are the input
numbers and where \(\mathcal{H}\) is the set of all \(n \choose k\) hyperplanes
of equation \(\sum_{j=1}^{k} x_{i_j} = 0\).

Sorting can also be seen as a point location problem in \(\mathbb{R}^n\).
In this case \(\mathcal{H}\) is the set of all \(n \choose 2\) hyperplanes of
equation \(x_i = x_j\). The arrangement \(\mathcal{A}(\mathcal{H})\) has
exactly \(n!\) \(n\)-dimensional cell that correspond to the \(n!\)
permutations the input might have.


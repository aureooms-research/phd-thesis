\chapter{Arrangements}

An \emph{hyperplane} in \(\mathbb{R}^d\) is the set of points \(q \in \mathbb{R}^d\)
satisfying a linear equation of the form \(\sum_{i=1}^{d} p_i q_i = 0\) for
some tuple of coefficients \(p \in \mathbb{R}^d\). An \emph{affine} hyperplane has an
additional independent term \(p_0\) to get the equation \(\sum_{i=1}^{d}
p_i q_i = p_0\). In general we will drop the adjective ``affine'' and talk about
hyperplanes without specifying whether \(p_0 = 0\).
%
In the plane an hyperplane is a line. In \(\mathbb{R}^3\) an hyperplane is
a plane. On the real line an hyperplane is a point.

An hyperplane \(H\) partitions the space \(\mathbb{R}^d\) into two halfspaces,
often denoted by \(H^-\) and \(H^+\).
Naturally, \(H^-\) is defined to be the set of points such that
\(\sum_{i=1}^{d} p_i q_i < p_0 \) and \(H^+\) is defined to be the set of
points such that \(\sum_{i=1}^{d} p_i q_i > p_0 \). When we want to be more
precise as to whether the inequality is strict we can write \(H^{<}\),
\(H^{\leq}\), \(H^{>}\), or \(H^{\geq}\).
We also define \(H^0\) and \(H^{=}\) as synonyms of \(H\).

A set of hyperplanes \(\mathcal{H} = \{\, H_1, H_2, \ldots, H_m\,\}\)
partitions \(\mathbb{R}^d\) into convex regions called \emph{cells}.
%
This partition is denoted by \(\mathcal{A}(\mathcal{H})\) and is called the
\emph{arrangement} of \(\mathcal{H}\).
%
Each sign vector \(\sigma \in {\{\,-,0,+\,\}}^{m}\) corresponds to a
(potentially empty) cell of this arrangement defined as the set of points \(q\)
such that \(q \in H_i^{\sigma_i}\) for all \(H_i \in \mathcal{H}\).

\section{\done Counting Cells}

It is useful to understand the complexity of such an arrangement when
the number \(m\) of hyperplanes grows and the dimension \(d\) is fixed.
The goal is to bound the number of nonempty cells an arrangement
can have. When \(m \leq d\) each of the \(3^m\) cells is nonempty, assuming
general position. However, when we fix \(d\) and make \(m\) grow, we get a more
reasonable behavior.
%
\begin{theorem}[name=Buck~\cite{Bu43},label=thm:buck]
Consider the partition of space defined by an arrangement of $m$ hyperplanes in
$\R^d$.
The number of regions of dimension $k \le d$ is at most
\begin{displaymath}
	{m \choose d-k}
	\left(
		{m-d+k \choose 0}
		+
		{m-d+k \choose 1}
		+
		\cdots
		+
		{m-d+k \choose k}
	\right)
\end{displaymath}
and the number of regions of all dimensions is \(O(m^d)\).
\end{theorem}


This bound allows us to derive precise lower and upper bounds for algorithms
manipulating those arrangements.
%
See~\cite{Hal04} for more details and generalizations.



The zone of a given pseudoline of an arrangement is the set of cells of the
arrangement supported by that pseudoline.
%
Figure~\ref{fig:a-zone-in-the-plane} illustrates a zone in a two-dimensional
arrangement of lines.
%
\begin{figure}
  \centering{}
  \includegraphics[width=\linewidth]{figures/a-zone-in-the-plane}
  \caption{%
    The zone defined by the dashed line in the two-dimensional
    arrangement of the plain lines is emphasized in light grey.%
  }\label{fig:a-zone-in-the-plane}
\end{figure}

We define the complexity of each cell to be the number of its sides.
We define the complexity of a zone to be the sum of the complexities of its cells.
%
The Zone Theorem states that the complexity of any zone is linear.
%
\begin{theorem}[Zone Theorem in the plane~\cite{BEPY90}]\label{thm:zone-theorem-2}
Given an arrangement of \(n+1\) pseudolines,
%
%in \(\mathbb{R}^2\),
%
the sum of the numbers of sides
%
in all the cells supported by one of the pseudolines
%
is at most \(\lfloor 9.5 n \rfloor - 1\).%
\footnote{%
Note that an earlier weaker (worse constant factor) linear bound is implied by
a theorem in~\cite{CGL85}.%
}
\end{theorem}



This result is important because it allows optimal
incremental construction of arrangements of line arrangements, a frequently
used tool.


\section{\done Cell Decomposition}%
\label{sec:arrangements:cell-decomposition}

Cells of arrangements may not behave nicely. In particular, they may have
arbitray description complexity.
%
In this section we give two decomposition schemes that allow to partition cells
into low-complexity subcells.
%
Access to such decompositions is an essential ingredient of divide-and-conquer
methods described in \Cref{chapter:divide-and-conquer}.

Lookup the Handbook~\cite[\S{}24.3.2]{Hal04},
for a more general overview of cell decomposition techniques.

\subsection{Bottom Vertex Triangulation}%
\label{sec:arrangements:triangulation}

\begin{figure}
  \centering{}
  \includegraphics[width=.5\linewidth]{figures/bottom-vertex-triangulation}
  \caption{%
    The bottom vertex triangulation of a cell in an arrangement of lines.%
  }%
  \label{fig:bvt}
\end{figure}

Given an arrangement of hyperplanes in \(\mathbb{R}^d\), its bottom vertex
triangulation is the partition of space obtained by triangulating each cell of
the arrangement recursively as follows: taking \(d=2\) as the base case
(because edges are already simplices),
let \(p\) be the bottom-most vertex of the cell,
for each facet of the cell not containing \(p\),
for each (\(d-1\))-dimensional simplex \(S'\) of the bottom vertex triangulation
of the facet in \(\mathbb{R}^{d-1}\),
construct a \(d\)-dimensional simplex \(S\) that is the convex hull of
\(S'\) and \(p\).
%
The
bottom vertex triangulation of an arrangement of lines in the plane is illustrated
in Figure~\ref{fig:bvt}.

In Paper~\ref{paper:ksum-algorithm} we construct a simplex of the bottom vertex
triangulation of a \(\varepsilon\)-net as part of a point location procedure.
%
In Paper~\ref{paper:3pol-algorithm} (\S\ref{sec:algo:dominance})
and Paper~\ref{paper:order-type-encoding} (all sections)
we use hierarchical cuttings based on this decomposition in order to construct
an efficient point location data structure.

See also~\cite{Cla88} for a more thorough description and an application to
nearest neighbour data structures.

\subsection{Vertical Decomposition}%
\label{sec:arrangements:vertical-decomposition}

\begin{figure}
	\centering{}
    \begin{subfigure}[t]{0.5\textwidth}
		\centering
		\includegraphics{figures/decomposition}
		\caption{Some curves in $\mathbb{R}^2$.}%
		\label{fig:some-curves-in-R2}
    \end{subfigure}%
    \begin{subfigure}[t]{0.5\textwidth}
		\centering
		\includegraphics{figures/recursion}
		\caption{Vertical decomposition of the curves in Figure~\ref{fig:some-curves-in-R2}.}%
    \end{subfigure}
	\caption{Vertical decomposition.}\label{fig:vd}
\end{figure}

Given an arrangement of curves in \(\mathbb{R}^2\), its vertical decomposition
is the partition of space obtained by shooting a vertical segment from each
vertex of the arrangement until it hits a curve of the arrangement. The
vertical decomposition of an arrangement of two circles is illustrated
in Figure~\ref{fig:vd}.

We use this decomposition in Paper~\ref{paper:3pol-algorithm}
(\S\ref{sec:algo:point-curves-location}) in order to apply a divide-and-conquer
algorithm of Matou{\v s}ek~\cite{Ma93} to an arrangement of curves.
This algorithm was originally designed to work for an arrangement of
hyperplanes using bottom-vertex triangulation decomposition.
%
In Paper~\ref{paper:order-type-encoding} (\S\ref{sec:lines-and-pseudolines},
\S\ref{sec:query-time}), we use this decomposition in order to
encode the order type of an arrangement of pseudolines.

Note that in the first application we only use vertical decomposition on a
constant number of bounded-degree polynomial curves.
%
In the second application we only care about the existence of such a
decomposition.
%
For those reasons, we do not discuss efficient construction of this
decomposition.

For the construction of the vertical decomposition of an arrangment of
polynomial curves in \(\mathbb{R}^2\),
we refer the reader to Pach and Sharir~\cite{Alcala}, Chazelle et
al.~\cite{CEGS91}, and Edelsbrunner et al.~\cite{EGPPSS92}.

\paragraph{A Remark}
This decomposition can be generalized to work for hypersurfaces in
\(\mathbb{R}^d\). Unfortunatly, the behaviour of such decompositions quickly
degenerates with \(d\).
While bottom vertex triangulation gives a bound of \(O(n^d)\) cells in
\(\mathbb{R}^d\) and
vertical decomposition gives a bound of \(O(n^2)\) cells in \(\mathbb{R}^2\),
we only know of a few upper bounds in fixed dimensions and some of them are
worse than \(O(n^d)\). This is why in our application of Meiser's algorithm we
use the bottom vertex triangulation. However, as observed by Ezra and
Sharir~\cite{ES17}, the bad behaviour of vertical decompositions is not a
bottleneck of Meiser's algorithm since we only need to look at a single cell of
the decomposition. Moreover, the complexity of those cells is better than that
of simplicial ones: vertical decomposition yields prisms supported by at most
\(2d\) hyperplanes of the arrangement whereas simplices of
the bottom vertex triangulation are supported by \(d^2 + d\) hyperplanes in the
worst case. This has a direct impact on the VC-dimension of the range spaces
defined by those objects and makes vertical decomposition the winning strategy
for this particular algorithm.


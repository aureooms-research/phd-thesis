\section{Point Location}

The Point Location problem is a classic problem in Computational Geometry.
The Point Location problem is a classic problem in Computational Geometry.
The Point Location problem is a classic problem in Computational Geometry.
The Point Location problem is a classic problem in Computational Geometry.
\input{text/definition/point-location}

Because the bounds in \Cref{thm:buck} are attained when \(\mathcal{H}\) is in
general position, there is a lower bound of \(\Omega(d \log m)\) on the depth
of decision trees solving this problem.

For our purpose, it is useful to see the \(k\)-SUM problem as a point location
problem in \(\mathbb{R}^n\) where the coordinates of \(q\) are the input
numbers and where \(\mathcal{H}\) is the set of all \(n \choose k\) hyperplanes
of equation \(\sum_{j=1}^{k} x_{i_j} = 0\).

Sorting can also be seen as a point location problem in \(\mathbb{R}^n\).
In this case \(\mathcal{H}\) is the set of all \(n \choose 2\) hyperplanes of
equation \(x_i = x_j\). The arrangement \(\mathcal{A}(\mathcal{H})\) has
exactly \(n!\) \(n\)-dimensional cell that correspond to the \(n!\)
permutations the input might have.


Because the bounds in \Cref{thm:buck} are attained when \(\mathcal{H}\) is in
general position, there is a lower bound of \(\Omega(d \log m)\) on the depth
of decision trees solving this problem.

For our purpose, it is useful to see the \(k\)-SUM problem as a point location
problem in \(\mathbb{R}^n\) where the coordinates of \(q\) are the input
numbers and where \(\mathcal{H}\) is the set of all \(n \choose k\) hyperplanes
of equation \(\sum_{j=1}^{k} x_{i_j} = 0\).

Sorting can also be seen as a point location problem in \(\mathbb{R}^n\).
In this case \(\mathcal{H}\) is the set of all \(n \choose 2\) hyperplanes of
equation \(x_i = x_j\). The arrangement \(\mathcal{A}(\mathcal{H})\) has
exactly \(n!\) \(n\)-dimensional cell that correspond to the \(n!\)
permutations the input might have.


Because the bounds in \Cref{thm:buck} are attained when \(\mathcal{H}\) is in
general position, there is a lower bound of \(\Omega(d \log m)\) on the depth
of decision trees solving this problem.

For our purpose, it is useful to see the \(k\)-SUM problem as a point location
problem in \(\mathbb{R}^n\) where the coordinates of \(q\) are the input
numbers and where \(\mathcal{H}\) is the set of all \(n \choose k\) hyperplanes
of equation \(\sum_{j=1}^{k} x_{i_j} = 0\).

Sorting can also be seen as a point location problem in \(\mathbb{R}^n\).
In this case \(\mathcal{H}\) is the set of all \(n \choose 2\) hyperplanes of
equation \(x_i = x_j\). The arrangement \(\mathcal{A}(\mathcal{H})\) has
exactly \(n!\) \(n\)-dimensional cell that correspond to the \(n!\)
permutations the input might have.


Because the bounds in \Cref{thm:buck} are attained when \(\mathcal{H}\) is in
general position, there is a lower bound of \(\Omega(d \log m)\) on the depth
of decision trees solving this problem.

For our purpose, it is useful to see the \(k\)-SUM problem as a point location
problem in \(\mathbb{R}^n\) where the coordinates of \(q\) are the input
numbers and where \(\mathcal{H}\) is the set of all \(n \choose k\) hyperplanes
of equation \(\sum_{j=1}^{k} x_{i_j} = 0\).

Sorting can also be seen as a point location problem in \(\mathbb{R}^n\).
In this case \(\mathcal{H}\) is the set of all \(n \choose 2\) hyperplanes of
equation \(x_i = x_j\). The arrangement \(\mathcal{A}(\mathcal{H})\) has
exactly \(n!\) \(n\)-dimensional cell that correspond to the \(n!\)
permutations the input might have.

\section{Pseudolines}

In the plane, line arrangements are generalized to pseudoline arrangements by
dropping the ``straight'' nature of lines.
%
A pseudoline in \(\mathbb{R}^2\) is a simple curve connecting points at
infinity.
An arrangement of pseudolines is a collection of pseudolines that pairwise
intersect exactly once.
%
This definition can be made more formal by considering pairwise-intersecting
simple closed curves in the projective plane instead.
%
Some important differences between line and pseudoline arrangements have
already been discussed in \S\ref{sec:problem:pol:order-types}.

Putting those differences aside, pseudoline arrangements share sufficient
similarity with line arrangements as to allow a generic algorithmic
treatment of both kinds.
%
For instance,
%
their arrangement complexities
(\S\ref{sec:arrangements:arrangement-complexity})
are
asymptotically the same,
%
the Zone Theorem (\S\ref{sec:arrangements:zone-theorem}) holds in general,
%
canonical labelings (\S\ref{sec:point-configurations:canonical-labelings})
can be constructed by looking at the order type only
(see \S\ref{sec:problem:pol:lb} for a definition of order type),
and generic cell decompositions
(\S\ref{sec:arrangements:cell-decomposition})
allow the use of the divide-and-conquer methods of
Chapter~\ref{chapter:divide-and-conquer} in both cases.

In Paper~\ref{paper:order-type-encoding} we consider the problem of encoding
order types of line or pseudoline arrangements.
%
Strictly speaking, an order type consists of pure combinatorial information and
does not require an embedding to be known.
%
In particular, given an order type, it is
\(\exists \mathbb{R}\)-complete to decide whether a straigth embedding
exists~\cite{Mne85,Mne88}.
%
For simplicity, the algorithms constructing the encodings in
Paper~\ref{paper:order-type-encoding} assume an embedding is given.

For more on pseudoline arrangements see Chapter
5 of the Handbook on the subject~\cite{Goo04}.
%
For more on equivalent representations of pseudoline arrangements see Chapter
6 of the Handbook on oriented matroids~\cite{RZ04}.

%For Basic properties~\cite[\S{}5.1]{Goo04}.
%Combinatorial results~\cite[\S{}5.4]{Goo04}.
%Topology~\cite[\S{}5.5]{Goo04}.
%Computation~\cite[\S{}5.6]{Goo04}.

%\begin{theorem}[Goodman~\cite{Go80}]
	%Every arrangement of pseudolines is isomorphic to a wiring diagram.
%\end{theorem}

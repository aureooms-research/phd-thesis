\chapter{Chirotopes}%
\label{chapter:chirotopes}

Chirotopes are the generalization of order types introduced in
\S\ref{sec:history:pol:lb}.
%
We generalize Definitions~\ref{def:order-type}
and~\ref{def:realizable-order-type} to point sets in \(\mathbb{R}^d\).
%
\begin{definition}[%
	name={Chirotope of a Point Set in \(\mathbb{R}^d\)},
	label=definition:chirotope%
]
	Given a set of
	\(n\) points \(p_i = (p_{i,1},p_{i,2}, \ldots, p_{i,d}) \in \mathbb{R}^d\),
	its (rank-(\(d+1\))) chirotope is the function \(\chi \colon\,
	{[n]}^{d+1} \to \{\, - , 0 , +\,\}\) such that
	\begin{displaymath}
		\chi(i_1, i_2, \ldots, i_{d+1}) =
		\det
		\left(
		\begin{matrix}
			1 & p_{i_1,1} & p_{i_1,2} & \hdots & p_{i_1,d} \\
			1 & p_{i_2,1} & p_{i_2,2} & \hdots & p_{i_2,d} \\
			\vdots & \vdots & \vdots & \ddots & \vdots \\
			1 & p_{i_{d+1},1} & p_{i_{d+1},2} & \hdots & p_{i_{d+1},d}
		\end{matrix}
		\right),
	\end{displaymath}
	up to isomorphism.
\end{definition}
%
\begin{definition}[name={Realizable Chirotope},label={def:realizable-chirotope}]
When considering an arbitrary function \(f \colon\, {[n]}^{d+1} \to \{\, -, 0,
+\,\}\) we say that \(f\) is a realizable chirotope if there is a
\(n\)-set \(P \subset \mathbb{R}^2\)
such that its chirotope is \(f\).
\end{definition}


In this chapter,
we give details on two essential ingredients of our data structures in
Paper~\ref{paper:order-type-encoding}:
the classic point-hyperplane duality
and
the canonical labeling of order types and chirotopes.
%
The first one is necessary to be able to apply the encodings we obtain
for line arrangements
and hyperplane arrangements to point configurations.
%
The second is used to get the preprocessing time of those encodings down to
\(O(n^d)\).

\section{Duality}%
\label{sec:point-configurations:duality}

Technically speaking, the encoding we describe for realizable chirotopes
in Paper~\ref{paper:order-type-encoding}
encodes the chirotope of a given arrangement of lines or hyperplanes.
Moreover, for ease of presentation, we make the assumption that the vertices of
this arrangement have finite coordinates. In the two-dimensional case, this
is equivalent to having no two lines parallel. In these paragraphs, we give the
details necessary to rigorously handle all realizable chirotopes, including
degenerate ones. This is especially important in higher dimension, where the
situation is a bit more complicated than in two dimensions.

In two dimensions, we wish to encode order types of point configurations.
Since our encoding construction algorithm works with an arrangement of lines as
input, we need a mapping from those primal points to their dual lines. This
mapping should preserve the order type of the point configuration, hence it
needs to be \emph{order-preserving}. One such order-preserving duality is
the mapping \((a,b) \leftrightarrow y = ax - b\) (see Figure~\ref{fig:duality}).
\aurelien{This is not order-preserving. It is orientation-reversing.
Combinatorially equivalent.}

\begin{figure}
  \centering{}
  \includegraphics[scale=1]{figures/duality}
  \caption{Order preserving duality: ``\(p\) is above \(l\)'' if and only if
  ``\(l'\) is above \(p'\)''.}\label{fig:duality}
\end{figure}

To avoid parallel lines in the dual, it suffices to avoid intersection points
at infinity. In the primal, this translates to avoiding two points of the
configuration defining a vertical line, that is, with the same \(x\) coordinate.
This is easily done by performing a tiny rotation in the primal.
This (proper) rotation does not change the order type of the point set.

In higher dimension, the order-preserving point-line duality generalizes
to the following order-preserving point-hyperplane duality: We map each
\(d\)-dimensional point \((x_1, x_2, \ldots, x_d) \in \mathbb{R}^d\) to the hyperplane \(y_d =
\sum_{i=1}^{d-1} x_i y_i - x_d \) and the hyperplane \(x_d = \sum_{i=1}^{d-1}
y_i x_i - y_d \) to the \(d\)-dimensional point \(( y_1, y_2, \ldots, y_d) \in
\mathbb{R}^d\).

As before, we want hyperplanes to be non-parallel. In fact, we need an even
stronger assumption: We want all linearly independent subsets of \(d\)
hyperplanes to intersect in a point with finite coordinates.
%
Having no intersection points at infinity in the dual
means having no \(d\) points spanning a hyperplane parallel to the
\(x_d\) axis in the primal. This is easy to avoid by applying tiny rotations
in the primal. Again, those (proper) rotations do not change the chirotope of the
point set.

In dimension three and higher, one would think degenerate arrangements lead to
annoying nongeneral situations. However, those situations are easy to handle
with our technique: Degenerate subsets of hyperplanes are linearly dependent.
The determinant corresponding to a query asking about a
degenerate \(d+1\) subset is therefore zero. Our technique will identify those
degenerate queries and map them to the correct answer in a space-efficient way.

\aurelien{Shouldn't we simply work with the orientation predicate to construct
a nice realizing arrangement in \(O(n^d)\) time and be done with it?}

\aurelien{Maybe add a remark that most of the examples we give ignore
degenerate cases, even though our text does not.}



Given a point set, the composition of its order type \(\chi\) with a
permutation \(\rho\) produces a new order type \(\chi' = \chi \circ \rho\).
This composition corresponds to a relabeling of the point set.
%
Aloupis et al.~\cite{AILOW14} defined the canonical labeling \(\rho^*(\chi)\)
of an order type \(\chi\) to be a permutation such that for all permutations
\(\pi\) we have \(\rho^*(\chi \circ \pi) = \pi^{-1} \circ \rho^*(\chi)\).
In other words, given two isomorphic order types \(\chi\) and \(\chi'\), we
have \(\chi \circ \rho^*(\chi) = \chi' \circ \rho^*(\chi')\), and
\({\rho^*(\chi')}^{-1} \circ \rho^*(\chi)\) is the isomorphism that sends
\(\chi\) to \(\chi'\).%
\footnote{Sometimes, two order types \(\chi\) and \(- \chi\) are also considered
to be isomorphic. See~\cite{AILOW14} for more details.}
They proved that the function \(\rho^*\) is
computable in \(O(n^2)\) time.
%
This first tool is useful to identify isomorphic order types.

They also showed that given any order type \(\chi\), a string \(E(\chi)\) of
\(O(n^2)\) bits, called the representation of \(\chi\), can be computed in
\(O(n^2)\) time, such that, if \(\chi\) and \(\chi'\) are two isomorphic order
types, then \(E(\chi) = E(\chi')\).
%
This second tool is useful to quickly compare two order types (a naive solution
would take \(\Theta(n^3)\) time by first computing a canonical labeling, and
then comparing all triples).

\begin{lemma}[Aloupis et al.~\cite{AILOW14}]\label{lem:canonical-labeling}
  Given an order type presented as an oracle,
  its canonical labeling of \(O(n \log n)\) bits
  and
  its canonical representation of \(O(n^2)\) bits
  can be computed in \(O(n^2)\) time
  in the word-RAM model.
\end{lemma}

Both tools generalize to chirotopes of point configurations in any dimension
\(d\) and, more generally, to chirotopes of rank \(d+1\).

\begin{lemma}[Aloupis et al.~\cite{AILOW14}]\label{lem:canonical-labeling-d}
  For all \(d \geq 2\),
  given a rank-(\(d+1\)) chirotope presented as an oracle,
  its canonical labeling of \(O(n \log n)\) bits
  and
  its canonical representation of \(O(n^d)\) bits
  can be computed in \(O(n^d)\) time
  in the word-RAM model.
\end{lemma}


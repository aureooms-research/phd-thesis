\chapter{\done Divide and Conquer}

A general divide-and-conquer paradigm in algorithm design is,
given an instance of size \(n\), to 1) divide it into at most
\(a\) smaller instances of size at most \(\frac{n}{b}\) in \(p(n)\)
preprocessing time, 2) solve those instances recursively,
solving constant size instances by brute force, and 3) postprocess the
solutions of those smaller instances to obtain the solution to the original
one in \(p(n)\) time.
%
An upper bound \(T(n)\) on the time complexity of such an algorithm is the
reccurence
\begin{displaymath}
	T(n) = a \,\, T\mleft(\frac nb\mright) + p(n).
\end{displaymath}
%
The asymptotic behavior of such a recurrence can then be derived from the
Master Theorem~\cite{BHS80,CLRS09}.

In this chapter we expose standard divide-an-conquer techniques in
Computational Geometry that allow the implementation of this paradigm (or
variants of it).
%
Papers~\ref{paper:ksum-algorithm},~\ref{paper:3pol-algorithm},
and~\ref{paper:order-type-encoding} explicitly rely on those techniques.
%
The construction in
Paper~\ref{paper:3sum-encoding} can also be interpreted as an ad-hoc
implementation of those concepts.

\section{Epsilon Nets and Cuttings}%
\label{sec:divide-and-conquer:epsilon-nets-and-cuttings}

The efficient construction of \(\varepsilon\)-nets and cuttings
is a necessary condition for many divide-and-conquer schemes
in Computational Geometry.
%
Those constructions are based on the concept
of Vapnik-Chervonenkis dimension
(VC-dimension) of a range space.

\subsection{Range Spaces}

A \emph{range space} (or set system) consists of a \emph{ground set} (or universe) and a family of
subsets of this ground set called \emph{ranges}.
%
A simple example of a range space is to take a finite set of points on the real
line as the ground set and to take the subsets of points induced by intervals
as the ranges.
%
For simplicity, the examples, definitions, and lemmas
we state here consider the case of finite
ground sets but can be generalized to infinite universes.
%
In general we have the following definition.
%
\begin{definition}[%
	name={Range Space}%
]
A pair \((X, \mathcal{R})\) such that \(X\) is a set (finite or infinite) and
\(\mathcal{R} \subseteq X\) is a family of subsets of \(2^X\).
\end{definition}


In general \(\mathcal{R}\) could be equal to \(2^{X}\).
%
If \(X\) is taken to be points in \(\mathbb{R}^d\) and the ranges are taken to
be subsets induced by simple geometric objects, as in most geometric
applications, this is not possible.

Take the example of points and intervals on the real line for instance.
For any set of points only \(O({| X |}^2)\) subsets can be induced by
intervals.
%
Another example is that of points in the
plane and subsets induced by halfplanes. Because any halfplane can be moved to
have exactly two points on its boundary, there are \(O({| X |}^2)\) such subsets.
%
Those observations can be generalized. For that we need a few more definitions.

\subsection{VC-dimension}

A set is \emph{shattered} by a family of ranges if each of its subsets can be induced by a
range of the family.
\begin{definition}[Shattered Set]
	Let \((X, \mathcal{R})\) be a range space. A subset of \(T \subseteq X\) is
	said to be shattered by \(\mathcal{R}\) if every subset \(T' \subseteq
	T \) is such that \(T' = T \cap R\) for some \(R \in \mathcal{R}\).
\end{definition}

%
In our points and intervals example it is easy to see that a set of two
distinct points
can be shattered but a set of three points cannot.
%
For the points and halfplanes example sets of three noncollinear points are
shattered but not sets of four points.

The VC-dimension of a range space is defined to be the size of the largest
shattered subset.
\begin{definition}[VC-dimension]
	Let \((X, \mathcal{R})\) be a range space.
	Let \(v\) be the size of the largest \(T \subseteq X\) such that
	\(T\) is shattered by \(\mathcal{R}\).
	The VC-dimension of the range space is defined to be \(v\).
\end{definition}


The Perles-Sauer-Shelah-Vapnik-Chervonenkis lemma gives a direct connection
between this dimension and the number of ranges.
\begin{lemma}[%
	Vapnik and Chervonenkis~\cite{VC71},
	Sauer~\cite{Sau72},
	Shelah~\cite{She72}%
]
Let \((X, \mathcal{R})\) be a range space.
If \(| \mathcal{R} | > \sum_{i=0}^{k-1} {| X | \choose i} \)
then the VC-dimension of the range space is at least \(k\).
Conversely, if the VC-dimension of the range space is \(d\) then
\(| \mathcal{R} | \leq \sum_{i=0}^{d-1} {| X | \choose i} = O({| X |}^d)\).
\end{lemma}

In our examples, notice the discrepancy between the ``identical''
\(O({|X|}^2)\) bounds on the number of ranges and the different VC-dimensions
(two and three respectively).

\subsection{Epsilon Nets}
\label{sec:divide-and-conquer:epsilon-nets}

A net is a subset of the ground set such that it hits every large range.
%
\begin{definition}[\(\varepsilon\)-net]
	\dots
\end{definition}

%
In our algorithms and data structures,
%
we consider nets of range spaces in \(\mathbb{R}^d\).
They are used as follows: construct an \(\varepsilon\)-net,
partition the space into a small number of ranges so that no range of the
partition is hit by the net. Then we have the guarantee that none of those
ranges is large.
%
For many problems,
such partitions of space with a small number of
small ranges usually lead to practical divide-and-conquer schemes.
These partitions are called \emph{cuttings} (see
\S\ref{sec:divide-and-conquer:cuttings}).

A very neat result is that, for constant VC-dimension, a reasonably small
\(\varepsilon\)-net can be constructed efficiently by random sampling.
\begin{definition}[\(\varepsilon\)-net]
	\dots
\end{definition}


See~\cite[Section~40.4]{CMR04} for more on range spaces and \(\varepsilon\)-nets.

\subsection{Hyperplanes in Linear Dimension}

Consider the range space where the ground set is a finite set of hyperplanes
\(\mathcal{H}\) in \(\mathbb{R}^n\)
and where the ranges are the subsets of hyperplanes that can be obtained by
intersecting the ground set with any simplex.

By combining a theorem due to Blumer et
al.~\cite{BEHW89} with the results of Meiser~\cite{M93}\footnote{Note that
Meiser used an older result due to Haussler and Welzl~\cite{H87} and got an
extra $\log n$ factor in the size of the $\varepsilon$-net. We thank Hervé
Fournier for pointing this out.}, it is possible to
obtain good bounds on \(\varepsilon\)-nets constructed by
random sampling.
%
In Paper~\ref{paper:ksum-algorithm}
we are interested in the dependency of those bounds on the ambiant dimension
\(n\).
This is not explicitly tackled in \Cref{thm:enet-general}
since the VC-dimension of this range space depends on \(n\).
%
\begin{theorem}\label{thm:enet}
	For all real numbers $\varepsilon > 0, c \ge 1$, if we choose at least \(c
	n^2 \log n \varepsilon^{-1} \log \varepsilon^{-1} \) hyperplanes of \(\Hy\)
	uniformly at random and denote this selection \(\net\) then for any simplex
	intersected by more than \(\varepsilon \card{\Hy}\) hyperplanes of \(\Hy\),
	with probability $1 - 2^{-\Omega(c)}$, at least one of the intersecting
	hyperplanes is contained in \(\net\).
\end{theorem}

%
The contrapositive states that if no hyperplane in \(\net\) intersects
a given simplex, then with high probability the number of hyperplanes of
\(\Hy\) intersecting the simplex is at most \(\frac{| \Hy |}{r}\).

\subsection{Cuttings}%
\label{sec:divide-and-conquer:cuttings}

A cutting in \(\mathbb{R}^d\)
is a set of (possibly unbounded and/or non-full dimensional)
bounded-complexity cells that together partition \(\mathbb{R}^{d}\).
%
For our purposes, a cell is of bounded complexity if its boundary is defined by
a number of lines, pseudolines, or hyperplanes of some arrangement
that depends only on the dimension, and not on the size of the arrangement.
%
A \(\frac{1}{c}\)-cutting of a set of \(n\) hyperplanes is a cutting with the
constraint that each of its cells is intersected by at most \(\frac{n}{c}\)
hyperplanes.

In constant dimension,
a straightforward way to construct a \(\frac 1c\)-cutting is to first contruct
a \(\frac 1c\)-net for the range space consisting of hyperplanes as elements
and simplices as ranges and then triangulate its arrangement as in
\S\ref{sec:arrangements:triangulation}. Because none of the simplices of the
triangulation intersect the net, each simplex is intersected by at most a
\(\frac 1c\)-fraction of the ground set.
%
Through this construction we obtain \(O(c^d \log^d c)\) simplicial cells each
intersected by at most \(\frac{n}{c}\) hyperplanes.

In \S\ref{sec:algo:point-curves-location} we use a similar technique to obtain
a partition of \(\mathbb{R}^2\) into \(O(c^2 \log^2 c)\) pseudotrapezoidal
cells so that each cell is intersected by at most a \(\frac 1c\)-fraction of
some input polynomial curves.

For hyperplanes in constant dimension,
there exist various ways of constructing \(\frac{1}{c}\)-cuttings of size
\(O(c^d)\), thereby removing the polylogarithmic factor overhead (see for
instance \S\ref{sec:divide-and-conquer:hierarchical-cuttings}).


\subsection{Derandomization}

The
\(\varepsilon\)-net
construction method based on sampling
described above can be made deterministic.
%
Derandomization is achieved through
the method of conditional probabilities of Raghavan~\cite{Rag88}
and Spencer~\cite{Spe94}.

% TODO rephrase this
For cuttings, $\varepsilon$-nets and derandomization, we
refer the reader to Matou\v{s}ek~\cite{M95,M96}, Chazelle and
Matou\v{s}ek~\cite{CM96} and Brönnimann et al.~\cite{BCM99}.

\todo{Construction in linear time}.

See~\cite[Section~40.1]{CMR04} for more on randomized divide-and-conquer.
See~\cite[Section~40.6]{CMR04} for more on derandomization.
See~\cite[Section~40.7]{CMR04} for more on the deterministic construction of
\(\varepsilon\)-nets.

\subsection{More Complex Ranges}

Linearization~\cite{YY85,AM94}.



The hierarchical cuttings of Chazelle
have the additional property that they can be composed without multiplying the
hidden constant factors in the big-O notation. In particular, they allow
for \(O(n^d)\)-space \(O(\log n)\)-query \(d\)-dimensional point location data
structures (for constant \(d\)).
\ifjournal
  \begin{definition}[Hierarchical Cutting]
    Given \(n\) hyperplanes in \(\mathbb{R}^d\),
    a \(\ell\)-levels hierarchical cutting of parameter \(r > 1\)
    for those hyperplanes
    is a sequence of \(\ell\) levels labeled \(0,1, \ldots, \ell - 1\)
    such that%
    \footnote{In~\cite{C93}, Chazelle refers to this parameter as
    \(r_0\) and uses \(r\) to mean \(r_0^\ell\). Here we drop the subscript for
    ease of presentation.}
    \begin{itemize}
      \item Level \(i\) has \(O(r^{2i})\) cells,
      \item Each of those cells is further partitioned into \(O(r^2)\)
        subcells,
      \item The collection of subcells is a \(\frac{1}{r^{i+1}}\)-cutting for
        the set of hyperplanes,
      \item The \(O(r^{2(i+1)})\) subcells of level \(i\) are the cells of level \(i+1\).
    \end{itemize}
  \end{definition}
  \aurelien{Use constant \(c_k\) or \(c_d\) instead of Big-Oh?}
  It is clear from reading through the various references that those
  hierarchical cuttings can be constructed for arrangements of pseudolines with
  the same properties:
  In~\cite{C93},
  Chazelle first proves a vertex-count estimation lemma
  that only relies on incidence properties of line
  arrangements~\cite[Lemma~2.1]{C93}. Then he summons a lemma from~\cite{Ma93}
  that relies on the finite VC-dimension of the range space at
  hand~\cite[Lemma 3.1]{C93}.
  Here the ground set is the set of pseudolines and the ranges are the
  subsets induced by intersections with pseudosegments.
  The VC-dimension of this range space
  is easily shown to be finite and is known to be at most
  \(8\): every arrangement of \(9\) pseudolines contains a subset of
  \(6\) pseudolines in hexagonal formation~\cite{HM94}, which cannot be
  shattered.%
  \footnote{This a quote from~\cite{BMP05}. We could not access
  the original paper.}
  %
  \aurelien{The VC-dimension is not even needed. Only enumerating ranges is
  necessary.}
  Finally, he proves a lemma for the efficient construction of
  sparse \(\varepsilon\)-nets whose correctness again only relies on incidence
  properties of line arrangements~\cite[Lemma 3.2]{C93}.
  Using those three lemmas together with bottom vertex triangulation he is
  able to prove his main result:
\else%
In the plane, hierarchical cuttings can be
constructed for arrangement of pseudolines with the same properties.
\fi

\ifjournal
\begin{lemma}[{Chazelle~\cite[Theorem 3.3]{C93}}]\label{lem:hierarchical-cutting-d}
  Given \(n\) hyperplanes in \(\mathbb{R}^d\), for any real parameter \(r >
  1\), we can construct a \(\ell\)-levels hierarchical cutting of parameter
  \(r\) for those hyperplanes in time \(O(nr^{\ell(d-1)})\).
\end{lemma}

For pseudoline arrangements, bottom vertex triangulation can be traded for
vertical decomposition.
\begin{lemma}\label{lem:hierarchical-cutting-2}
  Given \(n\) pseudolines, for any real parameter \(r > 1\), we can construct
  a \(\ell\)-levels hierarchical cutting of parameter
  \(r\) for those pseudolines in time \(O(nr^\ell)\).
\end{lemma}

In particular, we will use those lemmas with
\(\ell = \lceil \log_r \frac nt \rceil\),
for some parameter \(t\),
so that the last level of the hierarchy defines a \(\frac
tn\)-cutting whose cells are each intersected by at most \(t\) pseudolines (or
hyperplanes).

Note that in~\cite{C93} the construction is described for constant
parameter \(r\).
This restriction on the parameter is easily lifted:
We can construct a hierarchical
cutting with superconstant parameter \(r\) by constructing a hierarchical
cutting with some appropriate constant parameter \(r'\), and then skip levels that we do
not need. This is used in Section~\ref{sec:query-time} to reduce the query time
from \(O(\log n)\) to \(O(\frac{\log n}{\log \log n})\).
\fi


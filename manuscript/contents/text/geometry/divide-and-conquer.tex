\chapter{\done Divide and Conquer}

A general divide-and-conquer paradigm in algorithms is, given an instance of
size \(n\), to divide it into at most
\(a\) smaller instances of size at most \(\frac{n}{b}\) solved recursively
with an additive preprocessing and postprocessing cost of \(f(n)\), solving
constant size instances by brute force.
%
The time complexity \(T(n)\) of such an algorithm can then be expressed as the
reccurence
\begin{displaymath}
	T(n) = a \cdot T\left(\frac nb\right) + f(n).
\end{displaymath}
%
The asymptotic behavior of such a recurrence can then be derived from the
Master Theorem~\cite{BHS80,CLRS09}.

In this chapter we expose standard divide-an-conquer techniques in
Computational Geometry that allow the implementation of this paradigm (or
variants of it).
%
Papers~\ref{paper:ksum-algorithm},~\ref{paper:3pol-algorithm},
and~\ref{paper:order-type-encoding} explicitly rely on those techniques.
%
The construction in
Paper~\ref{paper:3sum-encoding} can also be interpreted as an ad-hoc
implementation of those concepts.

\section{Epsilon Nets}%
\label{sec:divide-and-conquer:epsilon-nets}

The concept of \enets{} is due to ??~\cite{???}.
%
\begin{definition}[\(\varepsilon\)-net]
	Let \((X, \mathcal{R})\) be a range space
	%
	and
	%
	let \(\varepsilon\) be a real number in the \([0,1)\) interval.
	%
	A subset \(N \subseteq X\) is an \(\varepsilon\)-net for the range space if
	%
	for every \(R \in \mathcal{R}\)
	%
	such that \(| R | > \varepsilon | X |\)
	%
	we have that \(N \cap R \neq \emptyset\).
\end{definition}


By combining a theorem due to Blumer et
al.~\cite{BEHW89} with the results of Meiser~\cite{M93}\footnote{Note that
Meiser used an older result due to Haussler and Welzl~\cite{H87} and got an
extra $\log n$ factor in the size of the $\varepsilon$-net. We thank Hervé
Fournier for pointing this out.}, it is possible to
construct an \enet{} \(\NH\) for the range space
defined by hyperplanes and simplices using a random uniform sampling on \(\Hy\).
%
\begin{lemma}\label{thm:enet}
	For all real numbers $r > 1$ and $c \ge 1$, if we choose at least \(c
	n^2 \log n r \log r \) hyperplanes of \(\Hy\)
	uniformly at random and denote this selection \(\net\) then for any simplex
	intersected by more than \(\frac{| \Hy |}{r}\) hyperplanes of \(\Hy\),
	with probability $1 - 2^{-\Omega(c)}$, at least one of the intersecting
	hyperplanes is contained in \(\net\).
\end{lemma}

%
The contrapositive states that if no hyperplane in \(\net\) intersects
a given simplex, then with high probability the number of hyperplanes of
\(\Hy\) intersecting the simplex is at most \(\varepsilon \card{\Hy}\).

\paragraph{Derandomization}

\todo{Make hyperplane nets and polynomial nets cohabitate somehow}

% TODO rephrase this
For cuttings, $\varepsilon$-nets and derandomization, we
refer the reader to Matou\v{s}ek~\cite{M95,M96}, Chazelle and
Matou\v{s}ek~\cite{CM96} and Brönnimann et al.~\cite{BCM99}.

\todo{Construction in linear time}.

\todo{We use this in Meiser's algo and we use it in the 3POL algo for the batch range
searching crap. Check polynomial dominance reporting though.}


\section{Cuttings}%
\label{sec:divide-and-conquer:cuttings}

A cutting in \(\mathbb{R}^d\)
is a set of (possibly unbounded and/or non-full dimensional)
bounded-complexity cells that together partition \(\mathbb{R}^{d}\).
%
For our purposes, a cell is of bounded complexity if its boundary is defined by
a number of lines or pseudolines (and later, hyperplanes) of the arrangement
that depends only on the dimension, and not on the size of the arrangement.
%
A \(\frac{1}{c}\)-cutting of a set of \(n\) hyperplanes is a cutting with the
constraint that each of its cells is intersected by at most \(\frac{n}{c}\)
hyperplanes. There exist various ways of constructing \(\frac{1}{c}\)-cuttings of
size \(O(c^d)\).
Those cuttings allow for efficient divide-and-conquer
solutions to many geometric problems.

\todo{We use this in 3POL algo.}


\section{Hierarchical Cuttings}%
\label{sec:divide-and-conquer:hierarchical-cuttings}

Compared to standard cuttings, hierarchical cuttings
have the additional property that they can be composed without multiplying the
hidden constant factors in the big-O notation~\cite{C93}.
%
In particular, they allow for \(O(n^d)\)-space \(O(\log n)\)-query
\(d\)-dimensional point location data structures (for constant \(d\)).
%
This is exploited in Paper~\ref{paper:3pol-algorithm}
(\S\ref{sec:algo:dominance}) and Paper~\ref{paper:order-type-encoding} (all
sections).

\begin{definition}[Hierarchical Cutting]
  Given \(n\) hyperplanes in \(\mathbb{R}^d\),
  a \(\ell\)-levels hierarchical cutting of parameter \(r > 1\)
  for those hyperplanes
  is a sequence of \(\ell\) levels labeled \(0,1, \ldots, \ell - 1\)
  such that%
  \footnote{In~\cite{C93}, Chazelle refers to this parameter as
  \(r_0\) and uses \(r\) to mean \(r_0^\ell\). Here we drop the subscript for
  ease of presentation.}
  \begin{itemize}
    \item Level \(i\) has \(O(r^{2i})\) cells,
    \item Each of those cells is further partitioned into \(O(r^2)\)
      subcells,
    \item The collection of subcells is a \(\frac{1}{r^{i+1}}\)-cutting for
      the set of hyperplanes,
    \item The \(O(r^{2(i+1)})\) subcells of level \(i\) are the cells of level \(i+1\).
  \end{itemize}
\end{definition}
%
It is clear from reading through the various references that those
hierarchical cuttings can be constructed for arrangements of pseudolines with
the same properties:
In~\cite{C93},
Chazelle first proves a vertex-count estimation lemma
that only relies on incidence properties of line
arrangements~\cite[Lemma~2.1]{C93}. Then he summons a lemma from~\cite{Ma93}
that relies on the finite VC-dimension of the range space at
hand~\cite[Lemma~3.1]{C93}.
Here the ground set is the set of pseudolines and the ranges are the
subsets induced by intersections with pseudosegments.
The VC-dimension of this range space
is easily shown to be finite and is known to be at most
\(8\): every arrangement of \(9\) pseudolines contains a subset of
\(6\) pseudolines in hexagonal formation~\cite{HM94}, which cannot be
shattered.%
\footnote{This a quote from~\cite{BMP05}. We could not access
the original paper.}
%
\aurelien{The VC-dimension is not even needed. Only enumerating ranges is
necessary.}
Finally, he proves a lemma for the efficient construction of
sparse \(\varepsilon\)-nets whose correctness again only relies on incidence
properties of line arrangements~\cite[Lemma 3.2]{C93}.
Using those three lemmas together with bottom vertex triangulation
(\S\ref{sec:arrangements:triangulation})
he is
able to prove his main result:

\begin{lemma}[{Chazelle~\cite[Theorem 3.3]{C93}}]\label{lem:hierarchical-cutting-d}
Given \(n\) hyperplanes in \(\mathbb{R}^d\), for any real parameter \(r >
1\), we can construct a \(\ell\)-levels hierarchical cutting of parameter
\(r\) for those hyperplanes in time \(O(nr^{\ell(d-1)})\).
\end{lemma}

For pseudoline arrangements, bottom vertex triangulation can be traded for
vertical decomposition
(\S\ref{sec:arrangements:vertical-decomposition}).
\begin{lemma}\label{lem:hierarchical-cutting-2}
Given \(n\) pseudolines, for any real parameter \(r > 1\), we can construct
a \(\ell\)-levels hierarchical cutting of parameter
\(r\) for those pseudolines in time \(O(nr^\ell)\).
\end{lemma}

In particular, we will use those lemmas with
\(\ell = \lceil \log_r \frac nt \rceil\),
for some parameter \(t\),
so that the last level of the hierarchy defines a \(\frac
tn\)-cutting whose cells are each intersected by at most \(t\) pseudolines (or
hyperplanes).

Note that in~\cite{C93} the construction is described for constant
parameter \(r\).
This restriction on the parameter is easily lifted:
We can construct a hierarchical
cutting with superconstant parameter \(r\) by constructing a hierarchical
cutting with some appropriate constant parameter \(r'\), and then skip levels that we do
not need. This is used in \S\ref{sec:query-time} to reduce the query time
from \(O(\log n)\) to \(O(\frac{\log n}{\log \log n})\).


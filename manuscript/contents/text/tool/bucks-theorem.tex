\section{\done Counting Cells}%
\label{sec:arrangements:arrangement-complexity}

It is useful to understand the complexity of such an arrangement when
the number \(m\) of hyperplanes grows and the dimension \(d\) is fixed.
The goal is to bound the number of nonempty cells an arrangement
can have. When \(m \leq d\) each of the \(3^m\) cells is nonempty, assuming
general position. However, when we fix \(d\) and make \(m\) grow, we get a more
reasonable behavior.
%
\begin{theorem}[name=Buck~\cite{Bu43},label=thm:buck]
Consider the partition of space defined by an arrangement of $m$ hyperplanes in
$\R^d$.
The number of regions of dimension $k \le d$ is at most
\begin{displaymath}
	{m \choose d-k}
	\left(
		{m-d+k \choose 0}
		+
		{m-d+k \choose 1}
		+
		\cdots
		+
		{m-d+k \choose k}
	\right)
\end{displaymath}
and the number of regions of all dimensions is \(O(m^d)\).
\end{theorem}


This bound allows us to derive precise lower and upper bounds for algorithms
manipulating those arrangements.
%
See~\cite{Hal04} for more details and generalizations.

\section{\done Counting Cells}%
\label{sec:arrangements:arrangement-complexity}

It is useful to understand the complexity of such an arrangement when
the number \(m\) of hyperplanes grows and the dimension \(d\) is fixed.
The goal is to bound the number of nonempty cells an arrangement
can have. When \(m \leq d\) each of the \(3^m\) cells is nonempty, assuming
general position. However, when we fix \(d\) and make \(m\) grow, we get a more
reasonable behavior.
%
\begin{theorem}[name={Buck~\cite{Bu43}, see also~\cite[Theorem~24.1.1 and Corollary~24.1.2]{Hal04}},label=thm:buck]
Consider the partition of space defined by an arrangement of $m$ hyperplanes in
$\R^d$.
The number of regions of dimension $k \le d$ is at most
\begin{displaymath}
	{m \choose d-k} \sum_{i=0}^{k} {m-d+k \choose i}.
\end{displaymath}
This bound is attained when the hyperplanes are in general position.
The number of regions of all dimensions is \(\Theta(m^d)\) in that case.
\end{theorem}


This bound allows us to derive precise lower and upper bounds for algorithms
manipulating those arrangements.
%
See~\cite{Hal04} for more details and generalizations.

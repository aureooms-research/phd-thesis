We encode the order type of an arrangement via
hierarchical cuttings as defined in~\cite{C93}. A cutting in \(\mathbb{R}^d\)
is a set of (possibly unbounded and/or non-full dimensional)
bounded-complexity cells that together partition \(\mathbb{R}^{d}\).
%
For our purposes, a cell is of bounded complexity if its boundary is defined by
a number of lines or pseudolines (and later, hyperplanes) of the arrangement
that depends only on the dimension, and not on the size of the arrangement.
%
A \(\frac{1}{c}\)-cutting of a set of \(n\) hyperplanes is a cutting with the
constraint that each of its cells is intersected by at most \(\frac{n}{c}\)
hyperplanes. There exist various ways of constructing \(\frac{1}{c}\)-cuttings of
size \(O(c^d)\).
Those cuttings allow for efficient divide-and-conquer
solutions to many geometric problems.

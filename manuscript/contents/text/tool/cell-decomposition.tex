\section{\done Cell Decomposition}%
\label{sec:arrangements:cell-decomposition}

Cells of arrangements may not behave nicely. In particular, they may have
arbitray description complexity.
%
In this section we give two decomposition schemes that allow to partition cells
into low-complexity subcells.
%
Access to such decompositions is an essential ingredient of divide-and-conquer
methods described in \Cref{chapter:divide-and-conquer}.

Lookup the Handbook~\cite[\S{}24.3.2]{Hal04},
for a more general overview of cell decomposition techniques.

\subsection{Bottom Vertex Triangulation}%
\label{sec:arrangements:triangulation}

\begin{figure}
  \centering{}
  \includegraphics[width=.5\linewidth]{figures/bottom-vertex-triangulation}
  \caption{%
    The bottom vertex triangulation of a cell in an arrangement of lines.%
  }%
  \label{fig:bvt}
\end{figure}

Given an arrangement of hyperplanes in \(\mathbb{R}^d\), its bottom vertex
triangulation is the partition of space obtained by triangulating each cell of
the arrangement recursively as follows: taking \(d=2\) as the base case
(because edges are already simplices),
let \(p\) be the bottom-most vertex of the cell,
for each facet of the cell not containing \(p\),
for each (\(d-1\))-dimensional simplex \(S'\) of the bottom vertex triangulation
of the facet in \(\mathbb{R}^{d-1}\),
construct a \(d\)-dimensional simplex \(S\) that is the convex hull of
\(S'\) and \(p\).
%
The
bottom vertex triangulation of an arrangement of lines in the plane is illustrated
in Figure~\ref{fig:bvt}.

In Paper~\ref{paper:ksum-algorithm} we construct a simplex of the bottom vertex
triangulation of a \(\varepsilon\)-net as part of a point location procedure.
%
In Paper~\ref{paper:3pol-algorithm} (\S\ref{sec:algo:dominance})
and Paper~\ref{paper:order-type-encoding} (all sections)
we use hierarchical cuttings based on this decomposition in order to construct
an efficient point location data structure.

See also~\cite{Cla88} for a more thorough description and an application to
nearest neighbour data structures.

\subsection{Vertical Decomposition}%
\label{sec:arrangements:vertical-decomposition}

\begin{figure}
	\centering{}
    \begin{subfigure}[t]{0.5\textwidth}
		\centering
		\includegraphics{figures/decomposition}
		\caption{Some curves in $\mathbb{R}^2$.}%
		\label{fig:some-curves-in-R2}
    \end{subfigure}%
    \begin{subfigure}[t]{0.5\textwidth}
		\centering
		\includegraphics{figures/recursion}
		\caption{Vertical decomposition of the curves in Figure~\ref{fig:some-curves-in-R2}.}%
    \end{subfigure}
	\caption{Vertical decomposition.}\label{fig:vd}
\end{figure}

Given an arrangement of curves in \(\mathbb{R}^2\), its vertical decomposition
is the partition of space obtained by shooting a vertical segment from each
vertex of the arrangement until it hits a curve of the arrangement. The
vertical decomposition of an arrangement of two circles is illustrated
in Figure~\ref{fig:vd}.

We use this decomposition in Paper~\ref{paper:3pol-algorithm}
(\S\ref{sec:algo:point-curves-location}) in order to apply a divide-and-conquer
algorithm of Matou{\v s}ek~\cite{Ma93} to an arrangement of curves.
This algorithm was originally designed to work for an arrangement of
hyperplanes using bottom-vertex triangulation decomposition.
%
In Paper~\ref{paper:order-type-encoding} (\S\ref{sec:lines-and-pseudolines},
\S\ref{sec:query-time}), we use this decomposition in order to
encode the order type of an arrangement of pseudolines.

Note that in the first application we only use vertical decomposition on a
constant number of bounded-degree polynomial curves.
%
In the second application we only care about the existence of such a
decomposition.
%
For those reasons, we do not discuss efficient construction of this
decomposition.

For the construction of the vertical decomposition of an arrangment of
polynomial curves in \(\mathbb{R}^2\),
we refer the reader to Pach and Sharir~\cite{Alcala}, Chazelle et
al.~\cite{CEGS91}, and Edelsbrunner et al.~\cite{EGPPSS92}.

\paragraph{A Remark}
This decomposition can be generalized to work for hypersurfaces in
\(\mathbb{R}^d\). Unfortunatly, the behaviour of such decompositions quickly
degenerates with \(d\).
While bottom vertex triangulation gives a bound of \(O(n^d)\) cells in
\(\mathbb{R}^d\) and
vertical decomposition gives a bound of \(O(n^2)\) cells in \(\mathbb{R}^2\),
we only know of a few upper bounds in fixed dimensions and some of them are
worse than \(O(n^d)\). This is why in our application of Meiser's algorithm we
use the bottom vertex triangulation. However, as observed by Ezra and
Sharir~\cite{ES17}, the bad behaviour of vertical decompositions is not a
bottleneck of Meiser's algorithm since we only need to look at a single cell of
the decomposition. Moreover, the complexity of those cells is better than that
of simplicial ones: vertical decomposition yields prisms supported by at most
\(2d\) hyperplanes of the arrangement whereas simplices of
the bottom vertex triangulation are supported by \(d^2 + d\) hyperplanes in the
worst case. This has a direct impact on the VC-dimension of the range spaces
defined by those objects and makes vertical decomposition the winning strategy
for this particular algorithm.

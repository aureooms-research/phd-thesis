\section{Duality}%
\label{sec:point-configurations:duality}

Technically speaking, the encoding we describe for realizable chirotopes
in Paper~\ref{paper:order-type-encoding}
encodes the chirotope of a given arrangement of lines or hyperplanes.
Moreover, for ease of presentation, we make the assumption that the vertices of
this arrangement have finite coordinates. In the two-dimensional case, this
is equivalent to having no two lines parallel. In these paragraphs, we give the
details necessary to rigorously handle all realizable chirotopes, including
degenerate ones. This is especially important in higher dimension, where the
situation is a bit more complicated than in two dimensions.

In two dimensions, we wish to encode order types of point configurations.
Since our encoding construction algorithm works with an arrangement of lines as
input, we need a mapping from those primal points to their dual lines. This
mapping should preserve the order type of the point configuration, hence it
needs to be \emph{order-preserving}. One such order-preserving duality is
the mapping \((a,b) \leftrightarrow y = ax - b\) (see Figure~\ref{fig:duality}).
\aurelien{This is not order-preserving. It is orientation-reversing.
Combinatorially equivalent.}

\begin{figure}
  \centering{}
  \includegraphics[scale=1]{figures/duality}
  \caption{Order preserving duality: ``\(p\) is above \(l\)'' if and only if
  ``\(l'\) is above \(p'\)''.}\label{fig:duality}
\end{figure}

To avoid parallel lines in the dual, it suffices to avoid intersection points
at infinity. In the primal, this translates to avoiding two points of the
configuration defining a vertical line, that is, with the same \(x\) coordinate.
This is easily done by performing a tiny rotation in the primal.
This (proper) rotation does not change the order type of the point set.

In higher dimension, the order-preserving point-line duality generalizes
to the following order-preserving point-hyperplane duality: We map each
\(d\)-dimensional point \((x_1, x_2, \ldots, x_d) \in \mathbb{R}^d\) to the hyperplane \(y_d =
\sum_{i=1}^{d-1} x_i y_i - x_d \) and the hyperplane \(x_d = \sum_{i=1}^{d-1}
y_i x_i - y_d \) to the \(d\)-dimensional point \(( y_1, y_2, \ldots, y_d) \in
\mathbb{R}^d\).

As before, we want hyperplanes to be non-parallel. In fact, we need an even
stronger assumption: We want all linearly independent subsets of \(d\)
hyperplanes to intersect in a point with finite coordinates.
%
Having no intersection points at infinity in the dual
means having no \(d\) points spanning a hyperplane parallel to the
\(x_d\) axis in the primal. This is easy to avoid by applying tiny rotations
in the primal. Again, those (proper) rotations do not change the chirotope of the
point set.

In dimension three and higher, one would think degenerate arrangements lead to
annoying nongeneral situations. However, those situations are easy to handle
with our technique: Degenerate subsets of hyperplanes are linearly dependent.
The determinant corresponding to a query asking about a
degenerate \(d+1\) subset is therefore zero. Our technique will identify those
degenerate queries and map them to the correct answer in a space-efficient way.

\aurelien{Shouldn't we simply work with the orientation predicate to construct
a nice realizing arrangement in \(O(n^d)\) time and be done with it?}

\aurelien{Maybe add a remark that most of the examples we give ignore
degenerate cases, even though our text does not.}

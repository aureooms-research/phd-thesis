
The zone of a given pseudoline of an arrangement is the set of cells of the
arrangement supported by that pseudoline.
%
Figure~\ref{fig:a-zone-in-the-plane} illustrates a zone in a two-dimensional
arrangement of lines.
%
\begin{figure}
  \centering{}
  \includegraphics[width=\linewidth]{figures/a-zone-in-the-plane}
  \caption{%
    The zone defined by the dashed line in the two-dimensional
    arrangement of the plain lines is emphasized in light grey.%
  }\label{fig:a-zone-in-the-plane}
\end{figure}

We define the complexity of each cell to be the number of its sides.
We define the complexity of a zone to be the sum of the complexities of its cells.
%
The Zone Theorem states that the complexity of any zone is linear.
%
\begin{theorem}[Zone Theorem in the plane~\cite{BEPY90}]\label{thm:zone-theorem-2}
Given an arrangement of \(n+1\) pseudolines,
%
%in \(\mathbb{R}^2\),
%
the sum of the numbers of sides
%
in all the cells supported by one of the pseudolines
%
is at most \(\lfloor 9.5 n \rfloor - 1\).%
\footnote{%
Note that an earlier weaker (worse constant factor) linear bound is implied by
a theorem in~\cite{CGL85}.%
}
\end{theorem}



This result is important because it allows optimal
incremental construction of arrangements of line arrangements, a frequently
used tool.

\section{Hierarchical Cuttings}%
\label{sec:divide-and-conquer:hierarchical-cuttings}

Compared to standard cuttings, hierarchical cuttings
have the additional property that they can be composed without multiplying the
hidden constant factors in the big-O notation~\cite{C93}.
%
In particular, they allow for \(O(n^d)\)-space \(O(\log n)\)-query
\(d\)-dimensional point location data structures (for constant \(d\)).
%
This is exploited in Paper~\ref{paper:3pol-algorithm}
(\S\ref{sec:algo:dominance}) and Paper~\ref{paper:order-type-encoding} (all
sections).

\begin{definition}[Hierarchical Cutting]
  Given \(n\) hyperplanes in \(\mathbb{R}^d\),
  a \(\ell\)-levels hierarchical cutting of parameter \(r > 1\)
  for those hyperplanes
  is a sequence of \(\ell\) levels labeled \(0,1, \ldots, \ell - 1\)
  such that%
  \footnote{In~\cite{C93}, Chazelle refers to this parameter as
  \(r_0\) and uses \(r\) to mean \(r_0^\ell\). Here we drop the subscript for
  ease of presentation.}
  \begin{itemize}
    \item Level \(i\) has \(O(r^{2i})\) cells,
    \item Each of those cells is further partitioned into \(O(r^2)\)
      subcells,
    \item The collection of subcells is a \(\frac{1}{r^{i+1}}\)-cutting for
      the set of hyperplanes,
    \item The \(O(r^{2(i+1)})\) subcells of level \(i\) are the cells of level \(i+1\).
  \end{itemize}
\end{definition}
%
It is clear from reading through the various references that those
hierarchical cuttings can be constructed for arrangements of pseudolines with
the same properties:
In~\cite{C93},
Chazelle first proves a vertex-count estimation lemma
that only relies on incidence properties of line
arrangements~\cite[Lemma~2.1]{C93}. Then he summons a lemma from~\cite{Ma93}
that relies on the finite VC-dimension of the range space at
hand~\cite[Lemma~3.1]{C93}.
Here the ground set is the set of pseudolines and the ranges are the
subsets induced by intersections with pseudosegments.
The VC-dimension of this range space
is easily shown to be finite and is known to be at most
\(8\): every arrangement of \(9\) pseudolines contains a subset of
\(6\) pseudolines in hexagonal formation~\cite{HM94}, which cannot be
shattered.%
\footnote{This a quote from~\cite{BMP05}. We could not access
the original paper.}
%
\aurelien{The VC-dimension is not even needed. Only enumerating ranges is
necessary.}
Finally, he proves a lemma for the efficient construction of
sparse \(\varepsilon\)-nets whose correctness again only relies on incidence
properties of line arrangements~\cite[Lemma 3.2]{C93}.
Using those three lemmas together with bottom vertex triangulation
(\S\ref{sec:arrangements:triangulation})
he is
able to prove his main result:

\begin{lemma}[{Chazelle~\cite[Theorem 3.3]{C93}}]\label{lem:hierarchical-cutting-d}
Given \(n\) hyperplanes in \(\mathbb{R}^d\), for any real parameter \(r >
1\), we can construct a \(\ell\)-levels hierarchical cutting of parameter
\(r\) for those hyperplanes in time \(O(nr^{\ell(d-1)})\).
\end{lemma}

For pseudoline arrangements, bottom vertex triangulation can be traded for
vertical decomposition
(\S\ref{sec:arrangements:vertical-decomposition}).
\begin{lemma}\label{lem:hierarchical-cutting-2}
Given \(n\) pseudolines, for any real parameter \(r > 1\), we can construct
a \(\ell\)-levels hierarchical cutting of parameter
\(r\) for those pseudolines in time \(O(nr^\ell)\).
\end{lemma}

In particular, we will use those lemmas with
\(\ell = \lceil \log_r \frac nt \rceil\),
for some parameter \(t\),
so that the last level of the hierarchy defines a \(\frac
tn\)-cutting whose cells are each intersected by at most \(t\) pseudolines (or
hyperplanes).

Note that in~\cite{C93} the construction is described for constant
parameter \(r\).
This restriction on the parameter is easily lifted:
We can construct a hierarchical
cutting with superconstant parameter \(r\) by constructing a hierarchical
cutting with some appropriate constant parameter \(r'\), and then skip levels that we do
not need. This is used in \S\ref{sec:query-time} to reduce the query time
from \(O(\log n)\) to \(O(\frac{\log n}{\log \log n})\).

\todo{We use this in the order type encoding paper}

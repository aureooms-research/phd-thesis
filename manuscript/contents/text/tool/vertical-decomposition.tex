\section{Vertical Decomposition}%
\label{sec:arrangements:vertical-decomposition}

\begin{figure}
	\centering{}
    \begin{subfigure}[t]{0.5\textwidth}
		\centering
		\includegraphics{figures/decomposition}
		\caption{Some curves in $\mathbb{R}^2$.}%
		\label{fig:some-curves-in-R2}
    \end{subfigure}%
    \begin{subfigure}[t]{0.5\textwidth}
		\centering
		\includegraphics{figures/recursion}
		\caption{Vertical decomposition of the curves in Figure~\ref{fig:some-curves-in-R2}.}%
    \end{subfigure}
	\caption{Vertical decomposition.}\label{fig:vd}
\end{figure}

Given an arrangement of curves in \(\mathbb{R}^2\), its vertical decomposition
is the partition of space obtained by shooting a vertical segment from each
vertex of the arrangement until it hits a curve of the arrangement. The
vertical decomposition of an arrangement of two circles is illustrated
in Figure~\ref{fig:vd}.

This decomposition can be generalized to work for hypersurfaces in
\(\mathbb{R}^d\). Unfortunatly, the behaviour of such decompositions quickly
degenerates with \(d\).
While bottom vertex triangulation gives a bound of \(O(n^d)\) cells in
\(\mathbb{R}^d\) and
vertical decomposition gives a bound of \(O(n^2)\) cells in \(\mathbb{R}^2\),
we only know of a few upper bounds in fixed dimensions and some of them are
worse than \(O(n^d)\). This is why in our application of Meiser's algorithm we
use the bottom vertex triangulation. However, as observed by Ezra and
Sharir~\cite{ES17}, the bad behaviour of vertical decompositions is not a
bottleneck of Meiser's algorithm since we only need to look at a single cell of
the decomposition. Moreover, the complexity of those cells is better than that
of simplicial ones: vertical decomposition yields prisms supported by at most
\(2d\) hyperplanes of the arrangement whereas simplices of
the bottom vertex triangulation are supported by \(d^2 + d\) hyperplanes in the
worst case. This has a direct impact on the VC-dimension of the range spaces
defined by those objects and makes vertical decomposition the winning strategy
for this particular algorithm.

For the construction of the vertical decomposition of an arrangment of
polynomial curves in \(\mathbb{R}^2\),
we refer the reader to Pach and Sharir~\cite{Alcala}, Chazelle et
al.~\cite{CEGS91}, and Edelsbrunner et al.~\cite{EGPPSS92}.

\todo{Add remark that we only care about vertical decomposition of a constant
number of curves in \(\mathbb{R}^2\)}.

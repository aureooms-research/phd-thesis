\section{Bottom Vertex Triangulation}%
\label{sec:arrangements:triangulation}

\begin{figure}
  \centering{}
  \includegraphics[width=\linewidth]{figures/bottom-vertex-triangulation}
  \caption{%
    The bottom vertex triangulation of a cell in an arrangement of lines.%
  }%
  \label{fig:bvt}
\end{figure}

Given an arrangement of hyperplanes in \(\mathbb{R}^d\), its bottom vertex
triangulation is the partition of space obtained by triangulating each cell of
the arrangement recursively as follows: taking \(d=2\) as the base case
(because edges are already simplices),
let \(p\) be its bottom-most vertex,
for each facet of the cell not containing \(p\),
for each \(d-1\)-dimensional simplex \(S'\) of the bottom vertex triangulation
of the facet in \(\mathbb{R}^{d-1}\),
construct a \(d\)-dimensional simplex \(S\) that is the convex hull of
\(S'\) and \(p\).
%
The
bottom vertex triangulation of an arrangement of lines in the plane is illustrated
in Figure~\ref{fig:bvt}.

Weeelll known~\cite{GO04,Cla88}.

\todo{We use this in Meiser's algorithm and Order Type encoding paper. In
Meiser we only construct one simplex.}

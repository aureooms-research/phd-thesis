\section{Epsilon Nets and Cuttings}%
\label{sec:divide-and-conquer:epsilon-nets-and-cuttings}

The efficient construction of \(\varepsilon\)-nets and cuttings
is an existential condition for many divide-and-conquer schemes
in Computational Geometry.
%
Those constructions are based on the concept Vapnik-Chervonenkis dimension
(VC-dimension) of range spaces.

A range space consists of a ground set (or universe) and a family of
subsets of this ground set called ranges.
%
A simple example of a range space is to take a finite set of points on the real
line as the ground set and to take the subsets of points induced by intervals
as the ranges.
%
For simplicity, the examples, definitions, and lemmas
we state here consider only the case of finite
ground sets but can be generalized to infinite universes.
%
In general we have the following definition.
%
\begin{definition}[%
	name={Range Space}%
]
A pair \((X, \mathcal{R})\) such that \(X\) is a set (finite or infinite) and
\(\mathcal{R} \subseteq X\) is a family of subsets of \(2^X\).
\end{definition}


In general \(\mathcal{R}\) could be equal to \(2^{X}\).
If \(X\) is taken to be points in \(\mathbb{R}^d\) and the ranges are taken to be subsets
induced by simple geometric objects, as in most applications, this is not possible.

Take the example of points and intervals on the real line for instance.
For any set of points only \(O({| X |}^2)\) subsets can be induced by
intervals.
%
Another example is that of points in the
plane and subsets induced by halfplanes. Because any halfplane can be moved to
have exactly two points on its boundary, there are \(O({| X |}^2)\) such subsets.
%
Those observations can be generalized. For that we need a few more definitions.

A set is \emph{shattered} by a family of ranges if each of its subsets can be induced by a
range of the family.
\begin{definition}[Shattered Set]
	Let \((X, \mathcal{R})\) be a range space. A subset of \(T \subseteq X\) is
	said to be shattered by \(\mathcal{R}\) if every subset \(T' \subseteq
	T \) is such that \(T' = T \cap R\) for some \(R \in \mathcal{R}\).
\end{definition}

%
In our points and intervals example it is easy to see that a set of two
distinct points
can be shattered but a set of three points cannot.
%
For the points and halfplanes example sets of three noncollinear points are
shattered but not sets of four points.

The VC-dimension of a range space is defined to be the size of the largest
shattered subset.
\begin{definition}[VC-dimension]
	Let \((X, \mathcal{R})\) be a range space.
	Let \(v\) be the size of the largest \(T \subseteq X\) such that
	\(T\) is shattered by \(\mathcal{R}\).
	The VC-dimension of the range space is defined to be \(v\).
\end{definition}


The Perles-Sauer-Shelah-Vapnik-Chervonenkis lemma gives a direct connection
between this dimension and the number of ranges.
\begin{lemma}[%
	Vapnik and Chervonenkis~\cite{VC71},
	Sauer~\cite{Sau72},
	Shelah~\cite{She72}%
]
Let \((X, \mathcal{R})\) be a range space.
If \(| \mathcal{R} | > \sum_{i=0}^{k-1} {| X | \choose i} \)
then the VC-dimension of the range space is at least \(k\).
Conversely, if the VC-dimension of the range space is \(d\) then
\(| \mathcal{R} | \leq \sum_{i=0}^{d-1} {| X | \choose i} = O({| X |}^d)\).
\end{lemma}

In our examples, notice the discrepancy between the ``identical''
\(O({|X|}^2)\) bounds on the number of ranges and the different VC-dimensions
(two and three respectively).

A net is a subset of the ground set such that it hits every large range.
%
\begin{definition}[\(\varepsilon\)-net]
	\dots
\end{definition}

%
In our algorithms and data structures,
%
we consider nets of range spaces in \(\mathbb{R}^d\).
They are used as follows: construct an \(\varepsilon\)-net,
partition the space into a small number of ranges so that no range of the
partition is hit by the net. Then we have the guarantee that none of those
ranges is large.
%
For many problems,
such partitions of space with a small number of
small ranges usually leads to practical divide-and-conquer schemes.
These partitions are called \emph{cuttings}.

\subsection{Cuttings}%
\label{sec:divide-and-conquer:cuttings}

A cutting in \(\mathbb{R}^d\)
is a set of (possibly unbounded and/or non-full dimensional)
bounded-complexity cells that together partition \(\mathbb{R}^{d}\).
%
For our purposes, a cell is of bounded complexity if its boundary is defined by
a number of lines or pseudolines (and later, hyperplanes) of the arrangement
that depends only on the dimension, and not on the size of the arrangement.
%
A \(\frac{1}{c}\)-cutting of a set of \(n\) hyperplanes is a cutting with the
constraint that each of its cells is intersected by at most \(\frac{n}{c}\)
hyperplanes. There exist various ways of constructing \(\frac{1}{c}\)-cuttings of
size \(O(c^d)\).
Those cuttings allow for efficient divide-and-conquer
solutions to many geometric problems.

\todo{We use this in 3POL algo.}

A very neat result is that, for constant VC-dimension, a reasonably small
\(\varepsilon\)-net can be constructed efficiently by random sampling.
\begin{definition}[\(\varepsilon\)-net]
	\dots
\end{definition}


See~\cite[Section~40.4]{CMR04} for more on
range spaces and \(\varepsilon\)-nets.

\paragraph{Hyperplanes and Simplices}

Define a range space where the ground set is a finite set of hyperplanes and
where the ranges be the subsets of hyperplanes that can be obtained by
intersecting the ground set with any simplex.

By combining a theorem due to Blumer et
al.~\cite{BEHW89} with the results of Meiser~\cite{M93}\footnote{Note that
Meiser used an older result due to Haussler and Welzl~\cite{H87} and got an
extra $\log n$ factor in the size of the $\varepsilon$-net. We thank Hervé
Fournier for pointing this out.}, it is possible to
construct an \enet{} \(\NH\) for this range space
using a random uniform sampling on \(\Hy\).
%
\begin{theorem}\label{thm:enet}
	For all real numbers $\varepsilon > 0, c \ge 1$, if we choose at least \(c
	n^2 \log n \varepsilon^{-1} \log \varepsilon^{-1} \) hyperplanes of \(\Hy\)
	uniformly at random and denote this selection \(\net\) then for any simplex
	intersected by more than \(\varepsilon \card{\Hy}\) hyperplanes of \(\Hy\),
	with probability $1 - 2^{-\Omega(c)}$, at least one of the intersecting
	hyperplanes is contained in \(\net\).
\end{theorem}

%
The contrapositive states that if no hyperplane in \(\net\) intersects
a given simplex, then with high probability the number of hyperplanes of
\(\Hy\) intersecting the simplex is at most \(\varepsilon \card{\Hy}\).

\paragraph{Derandomization}

The
\(\varepsilon\)-net
construction method based on sampling
described above can be made deterministic when a deterministic algorithm is
sought.
%
Derandomization is achieved through
the method of conditional probabilities of Raghavan~\cite{Rag88}
and Spencer~\cite{Spe94}.

\todo{Make hyperplane nets and polynomial nets cohabitate somehow}

% TODO rephrase this
For cuttings, $\varepsilon$-nets and derandomization, we
refer the reader to Matou\v{s}ek~\cite{M95,M96}, Chazelle and
Matou\v{s}ek~\cite{CM96} and Brönnimann et al.~\cite{BCM99}.

\todo{Construction in linear time}.

\todo{We use this in Meiser's algo and we use it in the 3POL algo for the batch range
searching crap. Check polynomial dominance reporting though.}

See~\cite[Section~40.1]{CMR04} for more on randomized divide-and-conquer.
See~\cite[Section~40.6]{CMR04} for more on derandomization.
See~\cite[Section~40.7]{CMR04} for more on the deterministic construction of
\(\varepsilon\)-nets.

\paragraph{More Complex Ranges}

Linearization~\cite{YY85,AM94}.


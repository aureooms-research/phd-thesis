\section{Epsilon Nets and Cuttings}%
\label{sec:divide-and-conquer:epsilon-nets-and-cuttings}

The efficient construction of \(\varepsilon\)-nets and cuttings
is a necessary condition for many divide-and-conquer schemes
in Computational Geometry.
%
Those constructions are based on the concept
of Vapnik-Chervonenkis dimension
(VC-dimension) of a range space.

\subsection{Range Spaces}

A \emph{range space} (or set system) consists of a \emph{ground set} (or universe) and a family of
subsets of this ground set called \emph{ranges}.
%
A simple example of a range space is to take a finite set of points on the real
line as the ground set and to take the subsets of points induced by intervals
as the ranges.
%
In general, we have the following definition.
%
\begin{definition}[%
	name={Range Space}%
]
A pair \((X, \mathcal{R})\) such that \(X\) is a set (finite or infinite) and
\(\mathcal{R} \subseteq X\) is a family of subsets of \(2^X\).
\end{definition}

%
Note that, for simplicity, the examples, definitions, and lemmas
we state here consider the case of finite
ground sets. Those can be generalized to infinite universes.

In the worst case, \(\mathcal{R}\) could be equal to \(2^{X}\).
%
If \(X\) is taken to be points in \(\mathbb{R}^d\) and the ranges are taken to
be subsets induced by simple geometric objects, as in most geometric
applications, this is not possible.

For instance, consider points and intervals on the real line.
For any set of points, only \(O({| X |}^2)\) subsets can be induced by
intervals.
%
Another example is that of points in the
plane and subsets induced by halfplanes. Because any halfplane can be moved to
have exactly two points on its boundary, there are \(O({| X |}^2)\) such subsets.
%
Those observations can be generalized. For that we need a few more definitions.

\subsection{VC-dimension}

A set is \emph{shattered} by a family of ranges if each of its subsets can be induced by a
range of the family.
\begin{definition}[Shattered Set]
	Let \((X, \mathcal{R})\) be a range space. A subset of \(T \subseteq X\) is
	said to be shattered by \(\mathcal{R}\) if every subset \(T' \subseteq
	T \) is such that \(T' = T \cap R\) for some \(R \in \mathcal{R}\).
\end{definition}

%
In our points and intervals example it is easy to see that a set of two
distinct points
can be shattered but a set of three points cannot.
%
For the points and halfplanes example sets of three noncollinear points are
shattered but not sets of four points.

The VC-dimension of a range space is defined to be the size of the largest
shattered subset.
\begin{definition}[VC-dimension]
	Let \((X, \mathcal{R})\) be a range space.
	Let \(v\) be the size of the largest \(T \subseteq X\) such that
	\(T\) is shattered by \(\mathcal{R}\).
	The VC-dimension of the range space is defined to be \(v\).
\end{definition}


The Perles-Sauer-Shelah-Vapnik-Chervonenkis lemma gives a direct connection
between this parameter and the number of ranges.
%
\begin{lemma}[%
	Vapnik and Chervonenkis~\cite{VC71},
	Sauer~\cite{Sau72},
	Shelah~\cite{She72}%
]
Let \((X, \mathcal{R})\) be a range space.
If \(| \mathcal{R} | > \sum_{i=0}^{k-1} {| X | \choose i} \)
then the VC-dimension of the range space is at least \(k\).
Conversely, if the VC-dimension of the range space is \(d\) then
\(| \mathcal{R} | \leq \sum_{i=0}^{d-1} {| X | \choose i} = O({| X |}^d)\).
\end{lemma}

%
In our examples, note the discrepancy between the ``identical''
\(O({|X|}^2)\) bounds on the number of ranges and the different VC-dimensions
(two and three respectively).

\subsection{Epsilon Nets}
\label{sec:divide-and-conquer:epsilon-nets}

A net is a subset of the ground set such that it hits every large range.
%
\begin{definition}[\(\varepsilon\)-net]
	\dots
\end{definition}

%
In our algorithms and data structures,
%
we consider nets of range spaces in \(\mathbb{R}^d\).
They are used as follows: construct an \(\varepsilon\)-net,
partition the space into a small number of ranges so that no range of the
partition is hit by the net. Then we have the guarantee that none of those
ranges is large.
%
For many problems,
such partitions of space with a small number of
small ranges usually lead to practical divide-and-conquer schemes.
These partitions are called \emph{cuttings} (see
\S\ref{sec:divide-and-conquer:cuttings}).

A very neat result is that, for constant VC-dimension, a reasonably small
\(\varepsilon\)-net can be constructed efficiently by random sampling.
\begin{definition}[\(\varepsilon\)-net]
	\dots
\end{definition}


See~\cite[Section~40.4]{CMR04} for more on range spaces and \(\varepsilon\)-nets.

\subsection{Hyperplanes in Linear Dimension}

Consider the range space where the ground set is a finite set of hyperplanes
\(\mathcal{H}\) in \(\mathbb{R}^n\)
and where the ranges are the subsets of hyperplanes that can be obtained by
intersecting the ground set with a simplex.

By combining a theorem due to
Blumer,
Ehrenfeucht,
Haussler, and
Warmuth~\cite{BEHW89}
with the results of Meiser~\cite{Mei93}\footnote{Note that
Meiser used an older result due to Haussler and Welzl~\cite{H87} and got an
extra $\log n$ factor in the size of the $\varepsilon$-net. We thank Hervé
Fournier for pointing this out.}, it is possible to
obtain good bounds on \(\varepsilon\)-nets constructed by
random sampling.
%
In Paper~\ref{paper:ksum-algorithm}
we are interested in the dependency of those bounds on the ambiant dimension
\(n\).
This is not explicitly tackled in \Cref{thm:enet-general}
since here the VC-dimension of this range space depends on \(n\).
%
\begin{theorem}\label{thm:enet}
	For all real numbers $\varepsilon > 0, c \ge 1$, if we choose at least \(c
	n^2 \log n \varepsilon^{-1} \log \varepsilon^{-1} \) hyperplanes of \(\Hy\)
	uniformly at random and denote this selection \(\net\) then for any simplex
	intersected by more than \(\varepsilon \card{\Hy}\) hyperplanes of \(\Hy\),
	with probability $1 - 2^{-\Omega(c)}$, at least one of the intersecting
	hyperplanes is contained in \(\net\).
\end{theorem}

%
The contrapositive states that if no hyperplane in \(\net\) intersects
a given simplex, then with high probability the number of hyperplanes of
\(\Hy\) intersecting the simplex is at most \(\frac{| \Hy |}{r}\).

\subsection{Cuttings}%
\label{sec:divide-and-conquer:cuttings}

A cutting in \(\mathbb{R}^d\)
is a set of (possibly unbounded and/or non-full dimensional)
bounded-complexity cells that together partition \(\mathbb{R}^{d}\).
%
For our purposes, a cell is of bounded complexity if its boundary is defined by
a number of lines, pseudolines, or hyperplanes of some arrangement
that solely depends on the dimension, and not on the size of the arrangement.
%
A \(\frac{1}{c}\)-cutting of a set of \(n\) elements is a cutting with the
constraint that each of its cells is intersected by at most \(\frac{n}{c}\)
elements.

In constant dimension,
a straightforward way to construct a \(\frac 1c\)-cutting for a set of
hyperplanes is to first contruct
a \(\frac 1c\)-net for the range space consisting of hyperplanes as elements
and simplices as ranges and then triangulate its arrangement as in
\S\ref{sec:arrangements:triangulation}. Because none of the simplices of the
triangulation intersect the net, each simplex is intersected by at most a
\(\frac 1c\)-fraction of the hyperplanes.
%
Through this construction we obtain \(O(c^d \log^d c)\) simplicial cells each
intersected by at most \(\frac{n}{c}\) hyperplanes.

For hyperplanes in constant dimension,
there exist various ways of constructing \(\frac{1}{c}\)-cuttings of size
\(O(c^d)\), thereby removing the polylogarithmic factor overhead (see for
instance \S\ref{sec:divide-and-conquer:hierarchical-cuttings}).

\subsection{Algebraic Range Spaces}

In Paper~\ref{paper:3pol-algorithm} (\S\ref{sec:algo:point-curves-location}) we use a
a similar technique to obtain
a partition of \(\mathbb{R}^2\) into \(O(c^2 \log^2 c)\) pseudotrapezoidal
cells so that each cell is intersected by at most a \(\frac 1c\)-fraction of
some input polynomial curves. Because those curves have bounded degree, we can
design a range space with finite VC-dimension and exploit it.

In the same paper (\S\ref{sec:algo:dominance}) we use the technique of
linearization~\cite{YY85,AM94} in order to tackle a more general problem
in any constant dimension: given a system of polynomial inequalities, we can
define a single variable for each monomial so that each inequality becomes
linear. This has the effect of greatly increasing the ambient dimension but
allows to use the techniques that normally only apply to linear ranges.

\subsection{Derandomization}

The
\(\varepsilon\)-net
construction based on sampling
described above can be made deterministic.
%
Derandomization is usually achieved through
the method of conditional probabilities of Raghavan~\cite{Rag88}
and Spencer~\cite{Spe94} (as in~\cite{C93}).
%
This method yields a construction time that is linear in the size of
the ground set for \(\varepsilon\)-nets and \(\varepsilon\)-cuttings with
constant \(\varepsilon\).
%
\begin{lemma}[%
	name={Matou{\v s}ek~\cite[Corollary~4.5]{M95}}%
]
Let \((X, \mathcal{R})\) be a range space of finite VC-dimension and
let \(r > 1\) be a constant.
Under some reasonnable assumptions,
we can compute a \(\frac 1r\)-net for \((X, \mathcal{R})\)
of size \(O(r \log r)\) in \(O(| X |)\) time.
\end{lemma}


For other concrete examples of derandomization of nets and cuttings, we
refer the reader to Matou\v{s}ek~\cite{M96}, Chazelle and
Matou\v{s}ek~\cite{CM96} and Brönnimann, Chazelle, and Matou\v{s}ek~\cite{BCM99}.
%
The Handbook has a nice overview of the
subject~\cite[Section~40.7]{CMR04}.
See also~\cite[Section~40.1]{CMR04} for more on randomized divide-and-conquer
and~\cite[Section~40.6]{CMR04} for more on derandomization.
%


Given a point set, the composition of its order type \(\chi\) with a
permutation \(\rho\) produces a new order type \(\chi' = \chi \circ \rho\).
This composition corresponds to a relabeling of the point set.
%
Aloupis et al.~\cite{AILOW14} defined the canonical labeling \(\rho^*(\chi)\)
of an order type \(\chi\) to be a permutation such that for all permutations
\(\pi\) we have \(\rho^*(\chi \circ \pi) = \pi^{-1} \circ \rho^*(\chi)\).
In other words, given two isomorphic order types \(\chi\) and \(\chi'\), we
have \(\chi \circ \rho^*(\chi) = \chi' \circ \rho^*(\chi')\), and
\({\rho^*(\chi')}^{-1} \circ \rho^*(\chi)\) is the isomorphism that sends
\(\chi\) to \(\chi'\).%
\footnote{Sometimes, two order types \(\chi\) and \(- \chi\) are also considered
to be isomorphic. See~\cite{AILOW14} for more details.}
They proved that the function \(\rho^*\) is
computable in \(O(n^2)\) time.
%
This first tool is useful to identify isomorphic order types.

They also showed that given any order type \(\chi\), a string \(E(\chi)\) of
\(O(n^2)\) bits, called the representation of \(\chi\), can be computed in
\(O(n^2)\) time, such that, if \(\chi\) and \(\chi'\) are two isomorphic order
types, then \(E(\chi) = E(\chi')\).
%
This second tool is useful to quickly compare two order types (a naive solution
would take \(\Theta(n^3)\) time by first computing a canonical labeling, and
then comparing all triples).

\begin{lemma}[Aloupis et al.~\cite{AILOW14}]\label{lem:canonical-labeling}
  Given an order type presented as an oracle,
  its canonical labeling of \(O(n \log n)\) bits
  and
  its canonical representation of \(O(n^2)\) bits
  can be computed in \(O(n^2)\) time
  in the word-RAM model.
\end{lemma}

Both tools generalize to chirotopes of point configurations in any dimension
\(d\) and, more generally, to chirotopes of rank \(d+1\).

\begin{lemma}[Aloupis et al.~\cite{AILOW14}]\label{lem:canonical-labeling-d}
  For all \(d \geq 2\),
  given a rank-(\(d+1\)) chirotope presented as an oracle,
  its canonical labeling of \(O(n \log n)\) bits
  and
  its canonical representation of \(O(n^d)\) bits
  can be computed in \(O(n^d)\) time
  in the word-RAM model.
\end{lemma}

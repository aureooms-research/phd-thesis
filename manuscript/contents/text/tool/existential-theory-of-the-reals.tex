\chapter{[DONE] Existential Theory of the Reals}
%As is the custom in the computational geometry
%literature~\cite{PS85,EGPPSS92}, one could assume the roots of a constant-degree
%polynomial can be computed in constant time.
%However, to strenghten our results
%we will not make this assumption and \ldots

The problems we consider
in Paper~\ref{paper:3pol-algorithm}
require our algorithms to manipulate polynomial
expressions and, potentially, their real roots. For that purpose, we will rely
on Collins's cylindrical algebraic decomposition (CAD)~\cite{C75}.
%
To understand the power of this method, and why it is useful for us, we give some
background on the related concept of first-order theory of the reals.

\begin{definition}
	A Tarski formula $\phi \in \mathbb{T}$ is a grammatically correct formula
	consisting of real variables ($x \in \mathbb{R}$), universal and
	existential quantifiers on those real variables
	($\forall,\exists\colon\,\mathbb{R}\times\mathbb{T}\to\mathbb{T}$), the
	boolean operators of conjunction and disjunction
	($\land,\lor\colon\,\mathbb{T}^2\to\mathbb{T}$), the six comparison
	operators ($<,\le,=,\ge,>,\ne\colon\,\mathbb{R}^2\to\mathbb{T}$), the four
	arithmetic operators ($+,-,*,/\colon\,\mathbb{R}^2\to\mathbb{R}$), the
	usual parentheses that modify the priority of operators, and constant real
	numbers (\(\mathbb{R}\)).
	%
	A Tarski sentence is a fully quantified Tarski formula.
	%
	%The first-order language of the reals is the set of all (true \emph{and}
	%false) Tarski sentences.
	%
	The first-order theory of the reals (\FOTR{}) is
	the set of true Tarski sentences.
\end{definition}

Tarski~\cite{T51} and Seidenberg~\cite{Sei74} proved that \FOTR{} is decidable.
However, the algorithm resulting from their proof has nonelementary complexity.
%
This proof, as well as other known algorithms, are based on quantifier
elimination, that is, the translation of the input formula to a much longer
quantifier-free formula, whose validity can be checked.
There exists a family of formulas for which any method of quantifier elimination
produces a doubly exponential size quantifier-free formula~\cite{DH88}.

\section{Cylindrical Algebraic Decomposition}
%
Collins's CAD matches this doubly exponential complexity.
\begin{theorem}[Collins~\cite{C75}]
	\FOTR{} can be solved in $2^{2^{O(n)}}$ time in the real-RAM model, where
	\(n\) is the size of the input Tarski sentence.
\end{theorem}

See
%Collins~\cite{C75} for the original description of the algorithm,
Basu, Pollack, and Roy~\cite{BPR06} for additional details,
Basu, Pollack, and Roy~\cite{BPR96b} for a singly exponential algorithm when all
quantifiers are existential
(existential theory of the reals, \ETR{}),
Caviness and Johnson~\cite{CJ12} for an anthology of key papers on the subject,
and Mishra~\cite{M04} for a review of techniques to compute with roots of
polynomials.

Collins's CAD solves any \emph{geometric} decision problem that does not involve
quantification over the integers in time doubly exponential in the problem
size. This does not harm our results as we exclusively use this algorithm to
solve constant size subproblems. Geometric is to be understood in the sense of Descartes and Fermat, that
is, the geometry of objects that can be expressed with polynomial equations. In
particular, it allows us to make the following computations in the real-RAM and
bounded-degree ADT models:
%, with only the four arithmetic operators:
\begin{enumerate}
\setlength{\itemsep}{0pt}
\setlength{\parskip}{0pt}
\setlength{\parsep}{0pt}
\item Given a constant-degree univariate polynomial, count its real roots
	in $O(1)$ operations,
\item Sort $O(1)$ real numbers given implicitly as roots of some
	constant-degree univariate polynomials in $O(1)$ operations,
\item Given a point in the plane and an arrangement of a constant number of
constant-degree polynomial planar curves, locate the point in the
arrangement in $O(1)$ operations.
%\item Given a constant-degree polynomial planar curve and an $N \times N$ grid
%defined by $N$ vertical lines and $N$ horizontal lines compute the $O(N)$
%cells of the grid intersected by the curve in $O(N)$ operations.
%\item Given a set of $N$ constant-degree polynomial planar curves compute a
%$\frac 1r$-net of size $O(r \log r)$ for the range space defined by those curves and $y$-axis
%aligned trapezoidal patches whose top and bottom sides are pieces of the
%curves in $O_r(N)$ operations.
%\item Given a set $\Gamma$ of $M$ constant-degree polynomial planar curves, a
%set $\Pi$ of $N$ points in the plane and a subset $\Gamma_r
%\subseteq \Gamma$ of size $O(r \log r)$, for all cells $C$ of the vertical decomposition of the
%arrangement of $\Gamma_r$, find the curves in $\Gamma \setminus
%\Gamma_r$ and the points in $\Pi$ that intersect $C$ in
%$O_r(M+N)$ operations total.
\end{enumerate}

%It has been proved that adding the function $\sin(x)$ to the set of allowed
%operators makes the extended theory undecidable, since it allows the encoding
%of sentences in the first-order theory of the integers, which is
%undecidable~\cite{R68}.
%Tarski's exponential function problem asks
%whether the theory can be extended with the exponential function $2^x$
%without rendering it undecidable. This question is to this day still open.
%Macintyre and Wilkie~\cite{MW96} showed that the decidability of this theory
%follows from Schanuel's conjecture.
%If proven true, it would allow us to apply our results to more general curves,
%for example, curves defined by logarithmic and exponential functions.

Instead of bounded-degree algebraic decision trees as the nonuniform model
we could consider decision trees in which
each decision involves a constant-size instance of the decision problem in the
first-order theory of the reals. The depth of a bounded-degree algebraic
decision tree simulating such a tree would only be blown up by a constant factor.

\section{Epsilon Nets}%
\label{sec:divide-and-conquer:epsilon-nets}

The concept of \enets{} is due to ??~\cite{???}.
%
\begin{definition}[\(\varepsilon\)-net]
	Let \((X, \mathcal{R})\) be a range space
	%
	and
	%
	let \(\varepsilon\) be a real number in the \([0,1)\) interval.
	%
	A subset \(N \subseteq X\) is an \(\varepsilon\)-net for the range space if
	%
	for every \(R \in \mathcal{R}\)
	%
	such that \(| R | > \varepsilon | X |\)
	%
	we have that \(N \cap R \neq \emptyset\).
\end{definition}


By combining a theorem due to Blumer et
al.~\cite{BEHW89} with the results of Meiser~\cite{M93}\footnote{Note that
Meiser used an older result due to Haussler and Welzl~\cite{H87} and got an
extra $\log n$ factor in the size of the $\varepsilon$-net. We thank Hervé
Fournier for pointing this out.}, it is possible to
construct an \enet{} \(\NH\) for the range space
defined by hyperplanes and simplices using a random uniform sampling on \(\Hy\).
%
\begin{lemma}\label{thm:enet}
	For all real numbers $r > 1$ and $c \ge 1$, if we choose at least \(c
	n^2 \log n r \log r \) hyperplanes of \(\Hy\)
	uniformly at random and denote this selection \(\net\) then for any simplex
	intersected by more than \(\frac{| \Hy |}{r}\) hyperplanes of \(\Hy\),
	with probability $1 - 2^{-\Omega(c)}$, at least one of the intersecting
	hyperplanes is contained in \(\net\).
\end{lemma}

%
The contrapositive states that if no hyperplane in \(\net\) intersects
a given simplex, then with high probability the number of hyperplanes of
\(\Hy\) intersecting the simplex is at most \(\varepsilon \card{\Hy}\).

\paragraph{Derandomization}

\todo{Make hyperplane nets and polynomial nets cohabitate somehow}

% TODO rephrase this
For cuttings, $\varepsilon$-nets and derandomization, we
refer the reader to Matou\v{s}ek~\cite{M95,M96}, Chazelle and
Matou\v{s}ek~\cite{CM96} and Brönnimann et al.~\cite{BCM99}.

\todo{Construction in linear time}.

\todo{We use this in Meiser's algo and we use it in the 3POL algo for the batch range
searching crap. Check polynomial dominance reporting though.}

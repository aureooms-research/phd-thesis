\section{Epsilon Nets}%
\label{sec:divide-and-conquer:epsilon-nets}

The construction of \(\epsilon\)-nets and cuttings
is an essential tool for efficient divide and conquer
in Computational Geometry.
%
Those constructions are based on the concept Vapnik-Chervonenkis dimension
(VC-dimension) of range spaces.

A range space consists of a ground set (or universe) and a family of
subsets of this ground set called ranges.
%
An simple example of a range space is to take the set of points on the real
line as the ground set and to take the intervals on this real line as the
ranges.
%
In general we have the following definition.
%
\begin{definition}[%
	name={Range Space}%
]
A pair \((X, \mathcal{R})\) such that \(X\) is a set and
\(\mathcal{R} \subseteq 2^X\) is a family of subsets of \(X\).
\end{definition}


In general \(\mathcal{R}\) could be as big as \(2^{| X |}\).
In \(X\) is taken to be \(\mathbb{R}^d\) and the ranges are taken to be subsets
induced by simple geometric objects, as in most applications, this is not possible.

Take the example of points and intervals on the real line for instance. Only
\(O(n^2)\) subsets can be induced. Another example is that of points in the
plane and subsets induced by halfplanes. Because any halfplane can be moved to
have exactly two points on its boundary, there are \(O(n^2)\) such subsets.
%
Those observations can be generalized. For that we need a few more definitions.

\begin{definition}[Shattered Set]
	Let \((X, \mathcal{R})\) be a range space. A subset of \(T \subset X\) is
	said to be shattered by \(\mathcal{R}\) if for every subset \(T' \subseteq
	T \) is such that \(T' = T \cap R\) for some \(R \in \mathcal{R}\).
\end{definition}


Example points-halfplanes.

\begin{definition}[VC-dimension]
	Let \((X, \mathcal{R})\) be a range space.
	Let \(d\) be the size of the largest \(T \subset X\) such that
	\(T\) is shattered by \(\mathcal{R}\).
	The VC-dimension of the range space is defined to be \(d\).
\end{definition}


\input{text/theorem/sauer-lemma}


The concept of \enets{} is due to ??~\cite{???}.
%
\begin{definition}[\(\varepsilon\)-net]
	Let \((X, \mathcal{R})\) be a range space
	%
	and
	%
	let \(\varepsilon\) be a real number in the \([0,1)\) interval.
	%
	A subset \(N \subseteq X\) is an \(\varepsilon\)-net for the range space if
	%
	for every \(R \in \mathcal{R}\)
	%
	such that \(| R | > \varepsilon | X |\)
	%
	we have that \(N \cap R \neq \emptyset\).
\end{definition}


Possible to construct an epsilon by random sampling.

\begin{definition}[\(\varepsilon\)-net]
	Let \((X, \mathcal{R})\) be a range space
	%
	and
	%
	let \(\varepsilon\) be a real number in the \([0,1)\) interval.
	%
	A subset \(N \subseteq X\) is an \(\varepsilon\)-net for the range space if
	%
	for every \(R \in \mathcal{R}\)
	%
	such that \(| R | > \varepsilon | X |\)
	%
	we have that \(N \cap R \neq \emptyset\).
\end{definition}


See~\cite[Section~40.4]{CMR04} for more on
range spaces and \(\varepsilon\)-nets.

\paragraph{Hyperplanes and Simplices}

Define a range space where the ground set is a finite set of hyperplanes and
where the ranges be the subsets of hyperplanes that can be obtained by
intersecting the ground set with any simplex.

By combining a theorem due to Blumer et
al.~\cite{BEHW89} with the results of Meiser~\cite{M93}\footnote{Note that
Meiser used an older result due to Haussler and Welzl~\cite{H87} and got an
extra $\log n$ factor in the size of the $\varepsilon$-net. We thank Hervé
Fournier for pointing this out.}, it is possible to
construct an \enet{} \(\NH\) for this range space
using a random uniform sampling on \(\Hy\).
%
\begin{lemma}\label{thm:enet}
	For all real numbers $r > 1$ and $c \ge 1$, if we choose at least \(c
	n^2 \log n r \log r \) hyperplanes of \(\Hy\)
	uniformly at random and denote this selection \(\net\) then for any simplex
	intersected by more than \(\frac{| \Hy |}{r}\) hyperplanes of \(\Hy\),
	with probability $1 - 2^{-\Omega(c)}$, at least one of the intersecting
	hyperplanes is contained in \(\net\).
\end{lemma}

%
The contrapositive states that if no hyperplane in \(\net\) intersects
a given simplex, then with high probability the number of hyperplanes of
\(\Hy\) intersecting the simplex is at most \(\varepsilon \card{\Hy}\).

\paragraph{Derandomization}

The
\(\epsilon\)-net
construction method based on sampling
described above can be made deterministic when a deterministic algorithm is
sought.
%
Derandomization is achieved through
the method of conditional probabilities of Raghavan~\cite{Rag88}
and Spencer~\cite{Spe94}.

\todo{Make hyperplane nets and polynomial nets cohabitate somehow}

% TODO rephrase this
For cuttings, $\varepsilon$-nets and derandomization, we
refer the reader to Matou\v{s}ek~\cite{M95,M96}, Chazelle and
Matou\v{s}ek~\cite{CM96} and Brönnimann et al.~\cite{BCM99}.

\todo{Construction in linear time}.

\todo{We use this in Meiser's algo and we use it in the 3POL algo for the batch range
searching crap. Check polynomial dominance reporting though.}

See~\cite[Section~40.1]{CMR04} for more on randomized divide-and-conquer.
See~\cite[Section~40.6]{CMR04} for more on derandomization.
See~\cite[Section~40.7]{CMR04} for more on the deterministic construction of
\(\varepsilon\)-nets.

\paragraph{More Complex Ranges}

Linearization~\cite{YY85,AM94}.

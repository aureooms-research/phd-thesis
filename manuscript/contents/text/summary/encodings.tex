Naive algorithms for Element Uniqueness, 3SUM, and \(k\)-SUM would search
all possible combinations of \(2\), \(3\), or \(k\) input numbers for a match
and reach horrible running times: respectively \(O(n^2)\), \(O(n^3)\), and \(O(n^k)\).

The way better uniform algorithms for those problems work
is by a combination of sorting and searching. The sorting part
constructs a data structure and the searching part uses this data structure to
answer the question at hand. The achieved running time is a balance between the
cost of sorting and the cost of searching. This results in better
running times than the naive ``search only'' solutions.

For instance, an efficient algorithm for the
Element Uniqueness problem would first sort the input, then scan it for
duplicates among adjacent numbers in this sorted order. The data structure that
is constructed is the permutation we talked about earlier. This structure
achieves a good compressing ratio by encoding the answer to all \(O(n^2)\)
pairwise comparisons of two input numbers in \(O(n \log n)\) bits.

For this reason Element Uniqueness is comparatively simpler than the 3SUM and
\(k\)-SUM problems.
For a nonuniform algorithm, constructing this structure is sufficient to solve
the problem: Once the construction is over, we can discard the input because
all the information we need is encoded in the data structure. Since the nonuniform
models of computation we consider do not care about computations not involving
the input, the searching part does not cost anything.

The case of 3SUM and \(k\)-SUM is more complex~\cite{GP18}:
The sorting phase only encodes
a fraction of all possible queries. This leaves a significant amount of work
for the searching phase. Therefore, both the sorting phase and the searching
phase contribute to the nonuniform cost of those algorithms.

The case of GPT is also interesting. The naive algorithm queries
\(O(n^{d+1})\) input tuples while the better algorithm constructs the dual
arrangement in \(O(n^d)\) time~\cite[Theorem 24.4.1]{Hal04}.
As in the case of Element Uniqueness,
this dual arrangement is the data structure that encodes the answer to all
\(O(n^{d+1})\) queries while achieving a significant space gain.

While the goal sought in the design of algorithms is to find the best possible
balance between those sorting and searching phases, a different question becomes
evident: ``What is the most resource efficient data structure encoding all the
combinatorial information carried by the input?''. For this question, resource
efficiency can be measured in three ways: space requirements, construction
time, and query time. If one only cares about space, a trivial answer
points its head out: if there are only \(X\) combinatorial types then each type
can be encoded with \(\lceil \log X \rceil\) bits. However, this solution is
unlikely to yield good construction time or good query time.

In the case of Element Uniqueness, this question has been answered long ago.
Sorting the input in \(O(n \log n)\) time will construct a permutation using
\(O(n \log n)\) bits that can be queried for any pairwise comparison in
\(O(1)\) time. However, the question is still widely open for
3SUM, \(k\)-SUM, and GPT.

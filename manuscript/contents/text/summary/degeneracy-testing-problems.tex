%In this thesis, we
We
study \emph{Degeneracy Testing Problems}:
an instance of size \(n\) of such a problem is a single point in
high-dimensional euclidean space \(q \in \mathbb{R}^{O(n)}\). Such an instance
is called \emph{general} if and only if it passes a series of algebraic tests
(usually \(n^{O(1)}\) of them). If it fails one of the tests, it is
called \emph{degenerate}.
%
Our goal is to determine how fast an instance can be classified as general or
degenerate.

The terminology is justified because most instances
are general: the set of degenerate instances is a zero-measure subset of the
input space. It also makes sense to visualize the input space as the euclidean
space: the algebraic tests naturally induce a partition of
the input space into semialgebraic sets. Solving the problem therefore amounts
to locate the input point \(q\) in this partition of space. Our goal
is thus to determine how fast this input point can be located.

Degeneracy testing problems are easy decision problems because there are only
a finite number of candidate tests to try. The ones we study can all be solved
by brute-force in polynomial time because the number of tests is polynomial.
We show how, in some cases, this naive approach is definitely subsumed by
divide and conquer techniques exploiting the geometry of the setting.

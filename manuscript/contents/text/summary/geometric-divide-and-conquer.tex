What makes the success of our methods are now-standard geometric
divide-and-conquer tools: \(\varepsilon\)-nets and cuttings.

The idea of an \(\varepsilon\)-net is simple, yet powerful. Given a sample of
\(m\) hyperplanes in \(\mathbb{R}^d\), construct a sample of those hyperplanes
of size \(f(d, \varepsilon)\) uniformly at random. Then, with high probability,
any simplex that is not intersected by an hyperplane of the sample is intersected
by at most \(\varepsilon m\) hyperplanes of the original set. The sample is
called a \emph{net} because it does not let the fat simplices through: a
simplex intersected by more than \(\varepsilon m\) hyperplanes must be intersected
by one of the sample.
%
This yields an efficient point location tool: construct a sample, find the cell
of its arrangement that contains the query point, find the simplex of its
triangulation that contains the query point, and recurse on the fraction of
hyperplanes that intersect it.

Triangulate the whole arrangement of the sample and you get a
\(\varepsilon\)-cutting: a partition of space into simplices such
that each simplex is intersected by at most \(\varepsilon m\) hyperplanes.
We use cuttings in applications where multiple queries are made on the same set
of hyperplanes.

The paper of Gr\o nlund and Pettie also discusses the following generalization of
the 3SUM problem: ``For a fixed \(k\), given a set of \(n\) real numbers,
decide whether there exists a \(k\)-subset whose elements sum to zero.''
This problem is called the \(k\)-SUM problem.

Obviously, the 3SUM problem is the \(k\)-SUM problem where \(k=3\).
Moreover, there is a simple reduction from \(k\)-SUM to 2SUM when \(k\) is even
and to 3SUM when \(k\) is odd. Those reductions yield a
\(O(n^{\frac{k}{2}} \log n)\) time real-RAM algorithm for \(k\) even and a
\(O(n^{\frac{k+1}{2}})\) time real-RAM algorithm for \(k\) odd.

In their paper, in addition to the slightly subquadratic uniform algorithm for
3SUM, Gr\o nlund and Pettie give a strongly subquadratic nonuniform
algorithm for 3SUM. The algorithms runs in time \(\tilde{O}(n^{3/2})\), and,
together with the aforementioned reduction, immediately yields an improved
\(\tilde{O}(n^{\frac{k}{2}})\) nonuniform time complexity for \(k\)-SUM when
\(k\) is odd.%
%
\footnote{%
Gold and Sharir~\cite{GS15} give an improvement on
the polylogarithmic factors hidden by the \( \tilde{O}(\cdot) \)
notation.%
}

As for uniform time complexity we do not know whether this nonuniform
improvement can be transferred to the real-RAM model: we do not know of any
real-RAM \(o(n^{\frac{k+1}{2}})\) time algorithm for \(k\)-SUM when
\(k\) is odd.

Shallower linear decision trees exist for \(k\)-SUM.
The \(k\)-SUM problem reduces to the following point location problem: ``Given
a input point \(q \in \mathbb{R}^n\), locate \(q\) in the arrangement of
\(n \choose k\) hyperplanes of equation \(x_{i_1} + x_{i_2} + \cdots +
x_{i_k} = 0\).'' Applying the best nonuniform algorithms for point location in
arrangements of hyperplanes by Meyer auf der Heide~\cite{Mey84} and
Meiser~\cite{Mei93} yields linear decision trees of depth \(n^{O(1)}\) for
\(k\)-SUM, where the constant of proportionality in the big-oh does not depend
on \(k\).

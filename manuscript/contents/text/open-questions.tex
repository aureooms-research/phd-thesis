\chapter{Open Questions}

There are a few interesting questions that are left unanswered:

The best linear decision tree for \(k\)-SUM has depth \(O_k(n \log^2 n)\) while
the lower bound is \(\Omega_k(n \log n)\).
Can we match the information theoretic lower bound for \(k\)-SUM with linear
decision trees, that is,
\begin{openquestion}
Is there a \(O_k(n \log n)\)-depth LDT for \(k\)-SUM?
\end{openquestion}

The same question can be asked about the point location problem. The best
linear decision tree for arbitrary hyperplanes has depth \(O(d^2 \log m)\) and
the lower bound is \(\Omega(d \log m)\).
\begin{openquestion}
Is there a \(O(d \log m)\)-depth LDT for point location?
\end{openquestion}

We showed how to efficiently implement Meiser's algorithm in the word-RAM model
when applied to \(k\)-SUM. However, we did not manage to get the query
complexity down to the current state of the art.
\begin{openquestion}
Is there a word-RAM algorithm for \(k\)-SUM using \(n^{3 - \Omega(1)}\)
linear queries and running in time \(n^{\frac{k}{2}+O(1)}\)?
\end{openquestion}

We also have not managed to match the running time of the best known uniform
algorithm.
\begin{openquestion}
Is there a word-RAM algorithm for \(k\)-SUM using \(n^{o(k)}\) linear queries and
running in time \( \tilde{O}(n^{\lceil \frac k2 \rceil})\)?
\end{openquestion}

Gr\o nlund and Pettie~\cite{GP18}
showed how to solve 3SUM in slightly subquadratic time in the real-RAM model.
However, so far, their technique and its subsequent
improvements~\cite{Fr15,GS15,Ch18}, have failed to generalize to uniform algorithms
for \(k\)-SUM.
\begin{openquestion}
	Is there a \(o(n^{\lceil \frac k2 \rceil})\)-time real-RAM algorithm for
	\(k\)-SUM with \(k \geq 4\)?
\end{openquestion}

The best algorithms for 3SUM and 3POL are only \emph{slightly} subquadratic,
shaving a polylogarithmic factor from quadratic runtime. The best lower bound
we have for those problems is \(\Omega(n \log n)\). The current conjecture is that
no significant improvement can be made with respect to known algorithms.
\ConjectureSUM*

We have to ask if this is indeed true.
\begin{openquestion}
	Is there a \(n^{2-\Omega(1)}\)-time real-RAM algorithm for 3SUM?
\end{openquestion}

The question can also be asked more generally for 3POL.
\begin{openquestion}
	Is there a \(n^{2-\Omega(1)}\)-time real-RAM algorithm for 3POL?
\end{openquestion}

The best known nonuniform algorithm for 3SUM is a \(6\)-linear decision tree of
depth \(O(n \log^2 n)\), the nonuniform algorithms in~\cite{GP18,Fr15,GS15} are
\(4\)-linear decision trees of depth \( \tilde{O}(n^{3/2}) \), and we know that
there is no \(3\)-linear decision tree of depth \(o(n^2)\) for
3SUM~\cite{Er99a}.
We want to know whether better \(4\)-linear decision trees exists for 3SUM.
\begin{openquestion}
	Is there a \(o(n^{3/2})\)-depth \(4\)-linear decision tree for 3SUM?
\end{openquestion}

If this is the case, then we would also like to know if this model of
computation suffers from more severe limitations than the \(6\)-linear decision tree
model.
\begin{openquestion}
	Is there a \(O(n \polylog n)\)-depth \(4\)-linear decision tree for 3SUM?
\end{openquestion}

The inference dimension method introduced in~\cite{KLM18} is essentially a
pigeon hole argument applied to linear combinations of linear equations with small coefficients.
At first sight, this kind of argument does not seem to apply to polynomial
equations, even when the polynomials are of degree two. One can of course apply
a Veronese embedding~\cite{Har77,Har13} to linearize the equations but this has
the effect of blowing up the dimension which directly impacts the efficiency of
the method. Could some generalization of inference dimension, or any other
method, yield better nonuniform algorithms for 3POL?
\begin{openquestion}
	Is there a \(o(n^{12/7})\)-depth bounded-degree algebraic decision tree for
	3POL?
\end{openquestion}

Despite all our efforts, we still lack a subquadratic algorithm for GPT, even a
nonuniform one. Like 3SUM and 3POL, the best known lower bound is \(\Omega(n
\log n)\)~\cite{???}.
\begin{openquestion}
	Is there a \(o(n^2)\)-depth bounded-degree algebraic decision tree for
	GPT?
\end{openquestion}

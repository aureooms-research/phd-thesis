\section{Strongly Subquadratic Encodings for 3SUM}

In the 3SUM problem, we are given an array of numbers as input and are asked
whether any three of them sum to \(0\). In the mid-nineties, this problem was
identified as a bottleneck of many important problems in geometry, such as
detection of affine degeneracies or motion planning~\cite{GO95}. Since then, it
has become a central problem in fine-grained complexity theory~\cite{PW10}. It
has long been conjectured to require $\Omega(n^2)$ time. In 2014, it was shown
to be solvable in $o(n^2)$ time, but no algorithm with running time
$O(n^{2-\delta})$ with constant $\delta>0$ is known~\cite{GP18}.

Lower bounds exist in restricted models of computation. Most notably,
$\Omega(n^2)$ \(3\)-linear queries are needed to solve 3SUM~\cite{Er99a}, and
nontrivial lower bounds have also been proven for slightly more powerful linear
decision trees~\cite{AC05}. However, in a recent breakthrough contribution,
Kane, Lovett, and Moran showed that 3SUM could be solved using $O(n\log^2 n)$
6-linear queries~\cite{KLM18}, hence within a $O(\log n)$ factor of the
information-theoretic lower bound.

Linear decision trees are examples of {\em nonuniform algorithms}, in which we
are allowed to have different algorithms for different input sizes.
Algebraic decision trees generalize linear decision trees
by allowing decision based on the sign of constant-degree polynomials at each
node~\cite{SY82}.

Any decision tree identifying the 3SUM type of a 3SUM instance yields a concise
encoding of this 3SUM type:
just write down the outcome of the successive tests. Knowing the decision tree
by convention, this sequence of bits is sufficient to recover the sign of any
triple.

The question we consider here is how to make such a representation efficient,
in the sense that not only does it use merely a few bits, but the answer to any
triple query can be recovered efficiently. Understanding the interplay between
nonuniform algorithms and such data structures hopefully sheds light on the
intrinsic structure of the problem.

We consider the following problem: given three sets of real numbers, output a
word-RAM data structure from which we can efficiently recover the sign of the
sum of any triple of numbers, one in each set.

This problem is similar to the problem of encoding the order
type of a finite set of points studied in Paper~\ref{paper:order-type-encoding}.
While this previous work showed that it was
possible to achieve slightly subquadratic space and logarithmic query time, we
show here that we can encode the \emph{3SUM type} of \(n\) numbers using
\(\tilde{O}(n^{\frac 32})\) bits to allow for constant time queries in the
word-RAM.

We also study lower and upper bounds of other natural 3SUM type encodings.

As there are only $O(n^3)$ queries, a table
of size $(\log_2 3) n^3 + O(1)$ bits suffices to give constant query time
\cite{DPT10}. This can be improved to $O(n^2\log n)$ bits of space by 
storing for each pair $(i,j)$ the values
\(k_<(i,j) = \max \{ 0\}\cup \{k \colon\, a_i + b_j + c_k < 0\}\) and
\(k_>(i,j) = \min \{ n+1\}\cup \{k \colon\, a_i + b_j + c_k > 0\}\).
For a query \((i,j,k)\), we compare \(k\) against the values \(k_<(i,j)\) and \(k_>(i,j)\)
to recover \(\chi(i,j,k)\) in \(O(1)\) time. All \(k_<(i,j)\) and \(k_>(i,j)\)
can be computed in \(O(n^2)\) time via the classic quadratic time algorithm for
3SUM.
%: if \(k \leq k_<(i,j)\), then \(\chi(i,j,k)=-1\), if \( k_<(i,j)< k < k_>(i,j)\),
%then \(\chi(i,j,k)=0\), and if \( k_>(i,j)\leq k\), then \(\chi(i,j,k)=+1\).
%The representation takes \( O(n^2 \log n \) bits, and each query can be answered by comparing
%pairs of indices, which takes \(O(1)\) time.

One seemingly simple representation is to store the numbers in $A$, $B$ and
$C$; however these are reals and thus we need to make them representable using
a finite number of bits.
In Section~\ref{s:numbers} we show that a minimal integer representation of a
3SUM instance may require $\Theta(n)$ bits per value, which would give
rise to a $O(n)$ query time and $O(n^2)$ space, which is far from
impressive.
%
\begin{contribution}[label=lem:bitsize,restate=TheoremSUMEncodingBitsize]
Every 3SUM instance has an equivalent integer instance
where all values have absolute value at most $2^{O(n)}$. Furthermore, there
exists an instance of 3SUM where all equivalent integer instances
require numbers at least as large as the $n$th Fibonacci number and where the
standard binary representation of the instance requires $\Omega(n^2)$ bits.
\end{contribution}


In \cite{CCILO18} the problem of given a set of $n$ lines, to create an
encoding of them so that the orientation of any triple (the \emph{order type})
can be determined was studied; our problem is a special case of this where the
lines only have three slopes.
Can we do better for the case of 3SUM? We answer this in the affirmative.
In Section~\ref{s:space} we show how to use an optimal $O(n \log n)$ bits of
space with a polynomial query time. Finally, in section~\ref{s:sscqt} we show
how to use $\tilde{O}(n^{1.5})$ space to achieve $O(1)$-time queries.
\begin{contribution}[label=theorem:3sum-encoding:space-optimal,restate=TheoremSUMEncodingSpaceOptimal]
	All 3SUM types have a \(O(n \log n)\)-bits \(n^{O(1)}\)-querytime encoding.
\end{contribution}

\begin{contribution}[label=theorem:3sum-encoding:gp,restate=TheoremSUMEncodingGP]
	All 3SUM types have a \(\tilde{O}(n^{3/2})\)-bits \(O(1)\)-querytime encoding.
\end{contribution}


Table~\ref{tor} gives a summary of our results.

\begin{table}
\centering
\caption{Table of results}\label{tor}
\begin{tabular}{cccc}
Encoding & Query time & Space (in bits) & Preprocessing time \\ \hline
Trivial & $O(1)$ & $O(n^3)$ & $O(n^3)$ \\
Almost trivial & $O(1)$ & $O(n^2 \log n)$ & $O(n^2)$ \\
Order type \cite{CCILO18} & $O(\log n)$ & $O(\frac{n^2 \log^2 \log n}{\log n})$ & $O(n^2) $\\
Order type \cite{CCILO18} & $O(\frac{\log n}{\log \log n})$ & $O(\frac{n^2 }{\log^{1-\epsilon} n})$ & $O(n^2)$ \\
Numerical (\S\ref{s:numbers}) & $O(n)$ & $O(n^2)$ & $n^{O(1)}$\\
Space-optimal (\S\ref{s:space}) & $n^{O(1)}$ & $O(n \log n)$ & $n^{O(1)}$\\
Query-optimal (\S\ref{s:sscqt}) &  $O(1)$ & $\tilde{O}(n^{1.5})$ & $O(n^{2})$ \\
\end{tabular}
\end{table}

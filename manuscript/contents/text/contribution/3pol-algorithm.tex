\section{\done Gr\o nlund and Pettie Applied to 3POL}


%Similarly to Gr\o nlund and Pettie~\cite{GP18}, we consider both nonuniform
%and uniform models of computation.
%
%For the uniform model we consider the real-RAM model with only the four
%arithmetic operators.

%For the nonuniform model, Gr\o nlund and Pettie consider linear
%decision trees, where one is only allowed to manipulate the input numbers
%through linear queries to an oracle. Each linear query has constant cost and
%all other operations are free but cannot inspect the input.
%

In Paper~\ref{paper:3pol-algorithm}
we focus on the computational complexity of 3POL\@. Since 3POL contains 3SUM,
an interesting question is whether a generalization of Gr\o nlund and Pettie's
3SUM algorithm exists for 3POL\@. If this is true, then we might wonder whether
we can ``beat'' the $O(n^{11/6}) = O(n^{1.833\ldots})$ combinatorial bound of Raz,
Sharir and de Zeeuw~\cite{RSZ15} with nonuniform algorithms
(see \S\ref{sec:problem:pol:combinatorics}).
%
We give a positive
answer to both questions: we design
a uniform
$O(n^2 {(\log \log n)}^{3/2} / {(\log n)}^{1/2})$-time
real-RAM algorithm
and
a nonuniform
$O(n^{12/7+\varepsilon}) = O(n^{1.7143})$-depth
bounded-degree algebraic decision tree
for 3POL\@.
%
To prove our uniform result, we present a fast algorithm for the Polynomial
Dominance Reporting (PDR) problem, a far reaching generalization of the
Dominance Reporting problem. As the algorithm for Dominance Reporting and its
analysis by Chan~\cite{Cha08} is used in fast algorithms for all-pairs shortest
paths, (min,+)-convolutions, and 3SUM, we expect this new algorithm will have
more applications.

Our results can be applied to many algebraic degeneracy testing problems, such
as GPT, a well known 3SUM-hard problem for which no subquadratic algorithm is
known (see \S\ref{sec:problem:pol}).
%
Raz, Sharir
and de Zeeuw results on the 3POL problem~\cite{RSZ15} can be applied to obtain
a combinatorial bound of $O(n^{11/6})$ on the
number of collinear triples when the input points are known to be lying on
a constant number of polynomial curves, provided those curves are neither
lines nor cubic curves.
%
A corollary of our first result is that
GPT where the input points are constrained to lie on
$o({(\log n)}^{1/6}/{(\log \log n)}^{1/2})$
constant-degree polynomial curves (including lines and cubic curves)
admits a subquadratic real-RAM algorithm and
a strongly subquadratic bounded-degree algebraic decision tree.
Interestingly, both reductions from 3SUM to GPT on 3 lines (map $a$ to $(a,0)$,
$b$ to $(b,2)$, and $c$ to $(-\frac c2, 1)$) and from 3SUM to GPT on a
cubic curve (map $a$ to $(a^3,a)$, $b$ to $(b^3,b)$, and $c$ to $(c^3,c)$)
construct such special instances of GPT (see
\S\ref{sec:problem:pol:reductions-from-ksum} for details on those reductions).
This constitutes the first step towards closing the major open question of
whether GPT can be solved in subquadratic time.
%
To further convince the reader of the expressive power of the 3POL problem,
we also give reductions
from the problem of counting triples of points spanning
unit circles (\S\ref{sec:paper:3pol-algorithm:application:circles}),
from the problem of counting triples of points spanning unit area
triangles (\S\ref{sec:paper:3pol-algorithm:application:triangles}),
and
from the problem of counting collinear triples in any dimension
(\S\ref{sec:paper:3pol-algorithm:application:gpt}).

To make the exposition of our results simpler,
we study two different variants of the 3POL problem.
The first variant is the 3POL problem as defined in the previous chapter
(\S\ref{sec:problem:pol:3pol}).
We recall its definition here.
\ProblemPOLImplicit*
The second variant is a special case of the 3POL problem where we
restrict the trivariate polynomial $F$ to have the form $F(a,b,c) = f(a,b) -
c$. We call it the explicit 3POL problem because the dependency on the third
variable is explicitly given:
\begin{problem}[name=explicit 3POL,restate=ProblemPOLExplicit]%
\label{problem:3pol:explicit}
Let $f \in \mathbb{R}[x,y]$ be a bivariate polynomial of constant degree,
given three sets $A$, $B$, and $C$, each containing $n$ real numbers, decide
whether there exist $a \in A$, $b \in B$, and $c \in C$ such that $c=f(a,b)$.
\end{problem}

We first design uniform and nonuniform algorithms for this easier problem.

%\begin{contribution}[{label=thm:explicit:nonuniform,restate=[name=from page \pageref{thm:explicit:nonuniform}]TheoremPOLNonuniformExplicit}]%
\begin{contribution}[%
	name={
		$O(n^{12/7+\varepsilon})$-depth algebraic decision tree for Explicit 3POL
	},%
	label=thm:explicit:nonuniform,%
	restate=TheoremPOLNonuniformExplicit%
]%
	Explicit 3POL can be solved in
	$O(n^{12/7+\varepsilon})$ time
	in the bounded-degree algebraic decision tree model.
\end{contribution}

\begin{contribution}[label=thm:explicit:uniform,restate=TheoremPOLUniformExplicit]
Explicit 3POL can be solved in
$O(\frac{n^2 {(\log \log n)}^{3/2}}{{(\log n)}^{1/2}})$
time in the real-RAM model.
\end{contribution}


Those algorithms can then be adapted to work for the more general 3POL
problem.

\begin{theorem}[label=thm:implicit:3POL,restate=TheoremPOLNonuniformImplicit]%
	3POL can be solved in
	$O(n^{12/7+\varepsilon})$ time
	in the bounded-degree algebraic decision tree model.
\end{theorem}

\begin{contribution}[label=thm:implicit:uniform,restate=TheoremPOLUniformImplicit]%
3POL can be solved in
%$O(n^2 {(\log \log n)}^{3/2} / {(\log n)}^{1/2})$
$O(\frac{n^2 {(\log \log n)}^{3/2}}{{(\log n)}^{1/2}})$
time in the real-RAM model.
\end{contribution}


Table~\ref{tor:3pol-algorithm} gives a summary of our results.

\begin{table}
\centering
\caption{Algorithmic results for 3POL and its explicit variant.}
\label{tor:3pol-algorithm}
\begin{tabular}{|c|c|c|}
	\hline

	Reference & Model & Complexity \\
	\hline
	\hline
	\S\ref{sec:algo:explicit:nonuniform} (\ref{thm:explicit:nonuniform}),
	\S\ref{sec:algo:implicit:nonuniform} (\ref{thm:implicit:nonuniform}) &
	Nonuniform &
	\(O(n^{12/7 + \epsilon})\)
	\\

	\hline

	\S\ref{sec:algo:explicit:uniform}(\ref{thm:explicit:uniform}),
	\S\ref{sec:algo:implicit:uniform}(\ref{thm:implicit:uniform}) &
	Uniform &
	$O(n^2 {(\log \log n)}^{3/2} / {(\log n)}^{1/2})$
	\\

	\hline

\end{tabular}
\end{table}


\paragraph{A Remark}

Similarly to Gr\o nlund and Pettie~\cite{GP18}, we consider both nonuniform
and uniform models of computation.
%
For the nonuniform model, Gr\o nlund and Pettie consider linear
decision trees, where one is only allowed to manipulate the input numbers
through linear queries to an oracle. Each linear query has constant cost and
all other operations are free but cannot inspect the input.
%
In this paper, we consider
\emph{bounded-degree algebraic decision trees (ADT)}~\cite{R72,Y81,SY82},
an algebraic generalization of linear decision trees,
as the nonuniform model. In a bounded-degree algebraic decision tree, one
performs constant cost branching operations that amount to test the sign of
a constant-degree polynomial of the input numbers. Again,
operations not involving the input are free.
See \S\ref{sec:models-of-computation:algorithms:trees} for details.
%
For the uniform model we consider the real-RAM model with only the four
arithmetic operators (see \S\ref{sec:models-of-computation:algorithms:ram}).

The algorithms we present manipulate polynomial expressions.
%
In computational geometry, it is customary to assume the real-RAM model can be
extended to allow the computation of roots of constant degree polynomials.
We distance ourselves from this practice and take particular care
of using the real-RAM model and the bounded-degree algebraic decision tree
model with only the four arithmetic operators (see Chapter~\ref{chapter:etr} on
the Existential Theory of the Reals).


\section{Subquadratic Encodings for GPT}
At SoCG'86, Bernard Chazelle asked~\cite{GP93}:
\begin{quotation}
``How many bits does it take to know an order type?''
\end{quotation}

This question is of importance in Computational Geometry for the following two
reasons:
%
First, in many algorithms dealing with sets of points in the plane,
the only relevant information carried by the input is the combinatorial
configuration of the points given by the orientation of each triple of points in the
set (clockwise, counterclockwise, or collinear)~\cite{Knu92,Ed12,Epp18}.
%
Second, computers as we know them can only handle numbers with
finite description and we cannot assume that they can handle arbitrary
real numbers without some sort of encoding. The study of \emph{robust}
algorithms is focused on ensuring the correct solution of problems on finite
precision machines. Chapter 41 of The Handbook of Discrete and Computational
Geometry is dedicated to this issue~\cite{Ya04}.

The (counterclockwise) orientation \(\nabla(p,q,r) \in \{\, -, 0, +\,\}\) of a triple of points
\(p\), \(q\), and \(r\) with coordinates \((x_p, y_p)\), \((x_q, y_q)\), and
\((x_r, y_r)\) is the sign of the determinant
\begin{displaymath}
    \begin{vmatrix}
	1 & x_p & y_p \\
	1 & x_q & y_q \\
	1 & x_r & y_r
    \end{vmatrix}.
\end{displaymath}

Given a set of \(n\) labeled points \(P = \{\, p_1, p_2, \ldots, p_n\,\}\), we
define the \emph{order type} of \(P\) to be the function \(\chi \colon\,
{[n]}^3 \to \{\, -, 0, +\,\} \colon\, (a,b,c) \mapsto \nabla(p_a, p_b, p_c)\)
that maps each triple of point labels to the orientation of the corresponding
points, up to isomorphism.
%
\ifeurocg\else%
A great deal of the literature in computational geometry deals with this
notion~\cite{%
AAK02a,
AAK02b,
ACKLV16,
AKPV14,
AK01,
AK05,
AKMPW15,
AMP13,
Al86,
AILOW14,
BLSWZ93,
BMS01,
BRS92,
Epp18,
EHN99,
Fe96,
FV11,
FL78,
Go80,
GP83,
GP84,
GP86,
GP91,
GP93,
GPS89,
HM94,
HMMS11,
Knu92,
Le26,
MMIB12,
NV98,
Ri89,
RZ04,
Ri56,
St97%
}.
\fi%
The order type of a point set has been further abstracted into combinatorial
objects known as (rank-three) \emph{oriented matroids}~\cite{FL78}. The
\emph{chirotope axioms} define consistent systems of signs of
triples~\cite{BLSWZ93}.
%
From the topological representation theorem~\cite{BMS01}, all such
\emph{abstract} order types correspond to pseudoline arrangements, while, from
the standard projective duality, order types of point sets correspond to
straight line arrangements. See Chapter 6 of The Handbook for more
details~\cite{RZ04}.

When the order type of a pseudoline arrangement can be realized by an
arrangement of straight lines, we call the pseudoline arrangement
\emph{stretchable}.
%
As an example of a nonstretchable arrangement, Levi gives Pappus's
configuration where eight triples of concurrent straight lines force a ninth,
whereas the ninth triple cannot be enforced by pseudolines~\cite{Le26} (see
Figure~\ref{fig:pappus}).
%
Ringel shows how to convert the so-called ``non-Pappus'' arrangement of
Figure~\ref{fig:pappus}~(b) to a simple arrangement while preserving
nonstretchability~\cite{Ri56}.
%
All arrangements of eight or fewer pseudolines are stretchable~\cite{GP80}, and
the only nonstretchable simple arrangement of nine pseudolines is the one given
by Ringel~\cite{Ri89}.
%
More information on pseudoline arrangements is available in Chapter 5 of The
Handbook~\cite{Go04}.

\begin{figure}
	\centering{}
    \begin{subfigure}[t]{0.5\textwidth}
		\centering{}
		\includegraphics[scale=.7]{figures/pappus-realizable}
		\caption{Realizable order type.}
    \end{subfigure}%
    \begin{subfigure}[t]{0.5\textwidth}
		\centering{}
		\includegraphics[scale=.7]{figures/pappus-abstract}
		\caption{Abstract order type which is not realizable.}
    \end{subfigure}
	\caption{Pappus's configuration.}\label{fig:pappus}
\end{figure}

Figure~\ref{fig:pappus} shows that not all pseudoline arrangements are
stretchable. Indeed, most are not: there are \(2^{\Theta(n^2)}\)
abstract order types~\cite{Fe96} and only \(2^{\Theta(n \log n)}\) realizable
order types~\cite{Al86,GP86}.
%
\ifeurocg\else%
This discrepancy stems from the algebraic nature of realizable order types, as
illustrated by the main tool used in the upper bound proofs (the
Petrovski\u{\i}-Ole\u{\i}nik-Thom-Milnor Theorem~\cite{Mi64,Th65,BPR06}).
\fi%
%
Therefore, information theory implies that \(\Theta(n^2)\) bits are necessary
and sufficient for abstract order types whereas \(\Theta(n \log n)\) bits are
necessary and sufficient for realizable order types.
Optimal encodings matching those bounds can be produced
by a simple enumeration algorithm.
%
However, it is unclear how the original information can be
efficiently reconstructed from those encodings.
%
On the other hand, storing all \( \binom{n}{3} \) orientations in a lookup
table to render this information accessible seems wasteful.

Another obvious idea for storing the order type of a point set is to store
the coordinates of the points, and answer orientation queries
by computing the corresponding determinant. While this should work in many practical
settings, it cannot work for all point sets. Perles's configuration shows that
some configuration of points, containing collinear triples, forces at least one
coordinate to be irrational~\cite{Gr05}\ifeurocg\else%
(see Figure~\ref{fig:perles})\fi.
%
It is easy to see that
order types of points in general position can always be represented by rational
(or integer) coordinates.
%
However, it is well known that some configurations require doubly
exponential coordinates, hence coordinates with exponential bitsizes if
represented in the binary numeral system~\cite{GPS89}.

\ifeurocg\else%
\begin{figure}
	\centering{}
	\includegraphics[scale=.7]{figures/perles}
	\caption{Perles's configuration.}\label{fig:perles}
\end{figure}
\fi%

Goodman and Pollack defined \(\lambda\)-matrices which can encode abstract order
types using \( O(n^2 \log{n}) \) bits~\cite{GP83} and can be constructed in
\(O(n^2)\) time~\cite{EOS86}. They asked if the space
requirements could be moved closer to the information-theoretic lower bounds.
Everett, Hurtado, and Noy complained that this encoding does not
allow a fast decoding for individual triples~\cite{EHN99}.
Knuth and Streinu independently gave new encodings of size \(O(n^2 \log n)\)
that allow orientation queries in constant time~\cite{Knu92,St97}.\footnote{%
We attract the attention of the reader on the fact that we express size in
bits.
Other authors,~\cite{EHN99} and~\cite{St97} in particular,
express size in number of words, which is off by at least a logarithmic factor.}
Felsner and Valtr showed how to encode abstract order types optimally in
\(O(n^2)\) bits via the wiring diagram of their corresponding allowable
sequence~\cite{Fe96, FV11} (as defined in~\cite{Go80}). Aloupis et al.\ gave
an encoding of size \(O(n^2)\) that can be computed in \(O(n^2)\) time and that
can be used to test for the isomorphism of two distinct point sets in the same
amount of time~\cite{AILOW14}. However, it is not known how to decode the
orientation of one triple from any of those optimal encodings in, say, sublinear time.
Moreover, since the information-theoretic lower bound for realizable order
types is only \(\Omega(n \log{n})\), we must ask if this space bound is
approachable for those order types while keeping orientation queries reasonably
efficient.
\subsection*{\iftitlecase%
Our Results\else%
Our results\fi}\label{sec:results}

In this contribution, we are interested in \emph{compact} encodings for
order types: we wish to design data structures using as few bits as possible
that can be used to quickly answer orientation queries of a given abstract or
realizable order type.
\ifeurocg%
\DefinitionEncoding*
\fi%
%
\ifjournal%
	In Section~\ref{sec:preliminaries} we give the tools necessary to
	attain our main result. This section can be skipped entirely if deemed
	unnecessary.
\fi%
In Section~\ref{sec:lines-and-pseudolines}, we
give the first optimal encoding for abstract
order types that allows efficient query of the orientation of any triple: the
encoding is a data structure that uses \( O(n^2) \) bits of space with queries
taking \(O(\log n)\) time in the word-RAM model with word size \(w \geq \log
n\).
\begin{theorem}\label{thm:abstract}
  All abstract order types have an \((O(n^2), O(\log n))\)-encoding.
\end{theorem}

%
Our encoding is far from being space-optimal for realizable order types.
We show that its construction can be easily tuned to only require \(O(n^2
{(\log{\log{n}})}^2 / \log{n})\) bits in this case.
\begin{contribution}[label=thm:realizable,restate=TheoremGPTRealizable]
  All realizable order types have
  a
  \(O(\frac{n^2 {(\log \log n)}^2}{\log n})\)-bits
  encoding
  allowing for queries in \(O(\log n)\) time.
  %All realizable order types have a
  %\(O(\frac{n^2 {(\log \log n)}^2}{\log n})\)-bits
  %\(O(\log n)\)-querytime encoding.
\end{contribution}

%
\ifeurocg%
We \else%
In Section~\ref{sec:query-time}, we \fi%
further refine our encoding to
reduce the query time to \(O(\log{n}/\log{\log{n}})\).
\begin{contribution}[label=thm:abstract-loglog,restate=TheoremGPTAbstractLogLog]
  All abstract order types have an encoding
  using \(O(n^2)\) bits of space
  and allowing for queries in \(O(\frac{\log n}{\log \log n})\) time.
  %(\(O(n^2)\), \(O(\frac{\log n}{\log \log n})\))-encoding.
  %\(O(n^2)\)-bits
  %\(O(\frac{\log{n}}{\log{\log{n}}}))\)-querytime
  %encoding.
\end{contribution}

\begin{contribution}[
  name={
    \(o(n^2)\)-bits
    \(o(\log n)\)-querytime
    encodings for realizable OT
  },%
  label=thm:realizable-loglog,%
  restate=TheoremGPTRealizableLogLog%
]
  All realizable order types have
  a \(O(\frac{n^2 \log^\epsilon n}{\log n})\)-bits encoding
  allowing for queries in \(O(\frac{\log n}{\log \log n})\) time.
  %All realizable order types have a
  %\(O(\frac{n^2\log^\epsilon n}{\log n})\)-bits
  %\(O(\frac{\log{n}}{\log{\log{n}}}))\)-querytime encoding.
\end{contribution}

%
In the realizable case, we give quadratic upper bounds on the
preprocessing time required to compute an encoding in the real-RAM model.
\begin{theorem}\label{thm:preprocessing}
  In the real-RAM model and the constant-degree algebraic decision tree model,
  given \(n\) real-coordinate input points in \(\mathbb{R}^2\) we can compute
  the encoding of their order type as in
  Theorems~\ref{thm:abstract}~and~\ref{thm:realizable} in \(O(n^2) \) time.
\end{theorem}

\begin{theorem}\label{thm:preprocessing-loglog}
  In the real-RAM model and the constant-degree algebraic decision tree model,
  given \(n\) real-coordinate input points in \(\mathbb{R}^2\) we can compute
  the encoding of their order type as in
  Theorems~\ref{thm:abstract-loglog}~and~\ref{thm:realizable-loglog} in
  \(O(n^2)\) time.
\end{theorem}

%
In \S\ref{sec:hyperplanes} we
generalize our encodings to chirotopes of
point sets in higher dimension.
\begin{contribution}[
  name={
    \(o(n^{k-1})\)-bits
    \(o(\log n)\)-querytime
    encodings for realizable chirotopes of rank \(k\)
  },%
  label=thm:realizable-d,%
  restate=TheoremGPTRealizableD%
]
  All realizable chirotopes of rank \(k \geq 4\) have
  an encoding using
  \(O(\frac{n^{k-1}{(\log{\log{n}})}^2}{\log n})\) bits of space
  and allowing for queries in
  \(O(\frac{\log{n}}{\log{\log{n}}})\) time.
\end{contribution}

\begin{contribution}[label=thm:preprocessing-d,restate=TheoremGPTPreprocessingD]
  In the real-RAM model and the constant-degree algebraic decision tree model,
  given \(n\) real-coordinate input points in \(\mathbb{R}^d\) we can compute
  the encoding of their chirotope as in Theorem~\ref{thm:realizable-d} in \(
  O(n^{d}) \) time.
\end{contribution}


\subsection*{\iftitlecase%
A Remark\else%
A remark\fi}\label{sec:a-remark}

Our data structure is the first subquadratic encoding for realizable order
types that allows efficient query of the orientation of any triple. It is not
known whether a subquadratic algebraic computation tree exists for
identifying the order type of a given point set.
Any such computation tree would yield another subquadratic encoding for
realizable order types, although the obtained encoding
might not be query-efficient. We see the design of compact encodings for realizable
order types as a subgoal towards subquadratic nonuniform algorithms for this
related problem, a major open problem in Computational Geometry. Note that
pushing the preprocessing time below quadratic would yield such an algorithm.

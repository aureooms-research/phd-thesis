\section{Subquadratic Encodings for GPT}
At SoCG'86, Bernard Chazelle asked~\cite{GP93}:
\begin{quotation}
``How many bits does it take to know an order type?''
\end{quotation}

This question is of importance in Computational Geometry for the following two
reasons:
%
First, in many algorithms dealing with sets of points in the plane,
the only relevant information carried by the input is the combinatorial
configuration of the points given by the orientation of each triple of points in the
set (clockwise, counterclockwise, or collinear)~\cite{Knu92,Ed12,Epp18}.
%
Second, computers as we know them can only handle numbers with
finite description and we cannot assume that they can handle arbitrary
real numbers without some sort of encoding. The study of \emph{robust}
algorithms is focused on ensuring the correct solution of problems on finite
precision machines. Chapter 41 of The Handbook of Discrete and Computational
Geometry is dedicated to this issue~\cite{Ya04}.

In this contribution, we are interested in \emph{compact} encodings for
order types: we wish to design data structures using as few bits as possible
that can be used to quickly answer orientation queries of a given abstract or
realizable order type.
%
There exist various ways to encode abstract order types optimally (see
\S\ref{sec:problem:pol:order-types}).
%
However, it is not known how to decode the orientation of one triple from any
of those optimal encodings in, say, sublinear time.
%
Moreover, since the information-theoretic lower bound for realizable order
types is only \(\Omega(n \log{n})\), we must ask if this space bound is
approachable for those order types while keeping orientation queries reasonably
efficient.

In Section~\ref{sec:lines-and-pseudolines}, we
give the first optimal encoding for abstract
order types that allows efficient query of the orientation of any triple: the
encoding is a data structure that uses \( O(n^2) \) bits of space with queries
taking \(O(\log n)\) time in the word-RAM model with word size \(w \geq \log
n\).
\begin{theorem}\label{thm:abstract}
  All abstract order types have an \((O(n^2), O(\log n))\)-encoding.
\end{theorem}

%
Our encoding is far from being space-optimal for realizable order types.
We show that its construction can be easily tuned to only require \(O(n^2
{(\log{\log{n}})}^2 / \log{n})\) bits in this case.
\begin{contribution}[label=thm:realizable,restate=TheoremGPTRealizable]
  All realizable order types have
  a
  \(O(\frac{n^2 {(\log \log n)}^2}{\log n})\)-bits
  encoding
  allowing for queries in \(O(\log n)\) time.
  %All realizable order types have a
  %\(O(\frac{n^2 {(\log \log n)}^2}{\log n})\)-bits
  %\(O(\log n)\)-querytime encoding.
\end{contribution}

%
\ifeurocg%
We \else%
In Section~\ref{sec:query-time}, we \fi%
further refine our encoding to
reduce the query time to \(O(\log{n}/\log{\log{n}})\).
\begin{contribution}[label=thm:abstract-loglog,restate=TheoremGPTAbstractLogLog]
  All abstract order types have an encoding
  using \(O(n^2)\) bits of space
  and allowing for queries in \(O(\frac{\log n}{\log \log n})\) time.
  %(\(O(n^2)\), \(O(\frac{\log n}{\log \log n})\))-encoding.
  %\(O(n^2)\)-bits
  %\(O(\frac{\log{n}}{\log{\log{n}}}))\)-querytime
  %encoding.
\end{contribution}

\begin{contribution}[
  name={
    \(o(n^2)\)-bits
    \(o(\log n)\)-querytime
    encodings for realizable OT
  },%
  label=thm:realizable-loglog,%
  restate=TheoremGPTRealizableLogLog%
]
  All realizable order types have
  a \(O(\frac{n^2 \log^\epsilon n}{\log n})\)-bits encoding
  allowing for queries in \(O(\frac{\log n}{\log \log n})\) time.
  %All realizable order types have a
  %\(O(\frac{n^2\log^\epsilon n}{\log n})\)-bits
  %\(O(\frac{\log{n}}{\log{\log{n}}}))\)-querytime encoding.
\end{contribution}

%
In the realizable case, we give quadratic upper bounds on the
preprocessing time required to compute an encoding in the real-RAM model.
\begin{theorem}\label{thm:preprocessing}
  In the real-RAM model and the constant-degree algebraic decision tree model,
  given \(n\) real-coordinate input points in \(\mathbb{R}^2\) we can compute
  the encoding of their order type as in
  Theorems~\ref{thm:abstract}~and~\ref{thm:realizable} in \(O(n^2) \) time.
\end{theorem}

\begin{theorem}\label{thm:preprocessing-loglog}
  In the real-RAM model and the constant-degree algebraic decision tree model,
  given \(n\) real-coordinate input points in \(\mathbb{R}^2\) we can compute
  the encoding of their order type as in
  Theorems~\ref{thm:abstract-loglog}~and~\ref{thm:realizable-loglog} in
  \(O(n^2)\) time.
\end{theorem}

%
In \S\ref{sec:hyperplanes} we
generalize our encodings to chirotopes of
point sets in higher dimension.
\begin{contribution}[
  name={
    \(o(n^{k-1})\)-bits
    \(o(\log n)\)-querytime
    encodings for realizable chirotopes of rank \(k\)
  },%
  label=thm:realizable-d,%
  restate=TheoremGPTRealizableD%
]
  All realizable chirotopes of rank \(k \geq 4\) have
  an encoding using
  \(O(\frac{n^{k-1}{(\log{\log{n}})}^2}{\log n})\) bits of space
  and allowing for queries in
  \(O(\frac{\log{n}}{\log{\log{n}}})\) time.
\end{contribution}

\begin{contribution}[label=thm:preprocessing-d,restate=TheoremGPTPreprocessingD]
  In the real-RAM model and the constant-degree algebraic decision tree model,
  given \(n\) real-coordinate input points in \(\mathbb{R}^d\) we can compute
  the encoding of their chirotope as in Theorem~\ref{thm:realizable-d} in \(
  O(n^{d}) \) time.
\end{contribution}


Table~\ref{tor:order-type-encoding} gives a summary of our results.

\begin{table}
\centering
\caption{Table of results}\label{tor:order-type-encoding}
\begin{tabular}{|c|c|c|c|}
\hline
Encoding & Query time & Space (bits) & Preprocessing \\ \hline \hline
Trivial & $O(1)$ & $O(n^3)$ & $O(n^3)$ \\ \hline
Enumeration & \(2^{\Omega(n \log n)}\) & \(O(n \log n)\) & \(2^{\Omega(n \log n)}\) \\ \hline
\(\lambda\)-matrices~\cite{GP83} & \(O(n^2)\) & \(O(n^2 \log n)\) & \(O(n^2)\) \\ \hline
Permutations~\cite{Knu92,St97} & $O(1)$ & $O(n^2 \log n)$ & $O(n^2)$ \\ \hline
Wiring diagrams~\cite{Fe96,FV11} & \(O(n^2)\) & \(O(n^2)\) & \(O(n^2)\) \\ \hline
Canonical Labeling~\cite{AILOW14} & ? & \(O(n^2)\) & \(O(n^2)\) \\ \hline
Numerical~\cite{GPS89} & $O(n)$ & \(2^{\Theta(n)}\) & ? \\ \hline
\hline
Abstract (Contrib.~\ref{thm:abstract-loglog}) & $O(\frac{\log n}{\log \log n})$ & $O(n^2)$ & $O(n^2)$ \\ \hline
Realizable (Contrib.~\ref{thm:realizable}) & $O(\log n)$ & $O(\frac{n^2 \log^2 \log n}{\log n})$ & $O(n^2) $\\ \hline
Realizable (Contrib.~\ref{thm:realizable-loglog}) & $O(\frac{\log n}{\log \log n})$ & $O(\frac{n^2}{\log^{1-\epsilon} n})$ & $O(n^2)$ \\ \hline
Realizable in \(\mathbb{R}^d\) (Contrib.~\ref{thm:realizable-d}) &
$O(\frac{\log n}{\log \log n})$ & $O(\frac{n^d \log^2 \log n}{\log n})$ & $O(n^d)$ \\ \hline
\end{tabular}
\end{table}

\subsection*{\iftitlecase%
A Remark\else%
A remark\fi}\label{sec:a-remark}

Our data structure is the first subquadratic encoding for realizable order
types that allows efficient query of the orientation of any triple. It is not
known whether a subquadratic algebraic computation tree exists for
identifying the order type of a given point set.
Any such computation tree would yield another subquadratic encoding for
realizable order types, although the obtained encoding
might not be query-efficient. We see the design of compact encodings for realizable
order types as a subgoal towards subquadratic nonuniform algorithms for this
related problem, a major open problem in Computational Geometry. Note that
pushing the preprocessing time below quadratic would yield such an algorithm.

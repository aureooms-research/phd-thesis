\chapternonum{Summary}

QUOTE Given a set of \(n\) points in the plane \dots

FIGURE Three points on a line.

This thesis is about problems. A problem asks to map a given data to some output.
In Computer Science, we solve problems with computers: we organize the data
with data structures, and process the data with algorithms.
%
Problems are considered harder if they take more time to solve.

Geometry is about points, lines, bodies, metric, topology.
%
This thesis is about geometric problems. The given data is geometric, the
question about the data is geometric. The proposed algorithms and data
structures therefore use geometry.

This thesis is about silly geometric problems. Yet, nobody has managed to
solve them. One of them asks to decide, given \(n\) points in the plane,
whether three of them lie on a common line. Of course, we can solve this
problem. What is of interest here is how fast this can be decided.

It is easy to solve this problem by brute force in cubic time and by some
more involved but standard construction in quadratic time. However, we do not
know of any reason why this problem would be any harder than sorting, which can
be solved in near-linear time.

Does this matter at all? Well, it turns out those silly problems appear to
be bottlenecks of modern day algorithms: unless we manage to improve our
understanding of those problems, for many other practical problems,
we are stuck with solutions from the 90's, 80's, sometimes even the 70's, and
we do not know why.

In this thesis, we expose two novel algorithms and two novel data
structures related to this problem.

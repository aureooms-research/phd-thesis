\chapternonum{To the Profane}

TODO avoid all technical terms here

QUOTE Given a set of \(n\) points in the plane \dots

FIGURE Three points on a line.

This thesis is about problems.
Solving a problem consists in mapping a given data to some output in some
automated way.
In Computer Science, we solve problems with computers: we organize the data
with data structures, and process the data with algorithms.
%
Solving problems consumes resources, for example, time.
Problems are considered harder if they take more resources to solve.

Geometry is about points, lines, bodies, metric, topology.
%
This thesis is about geometric problems. The given data is geometric, the
question about the data is geometric. The proposed algorithms and data
structures therefore use geometry.

This thesis studies questions about silly geometric problems. Yet, nobody
has managed to answer them. One of the problems asks to decide, given \(n\) points in the plane,
whether three of them lie on a common line. Of course, we can solve this
problem. The question of interest here is how fast this can be decided.

It is easy to solve this problem by testing all possible candidates but that is
quite inneficient. There is a
more involved but standard construction that takes significantly less time to
execute but that is still considered inneficient. As of today, we do not know of
any reason why this problem would be any harder than sorting, which can be
solved almost as fast as the time it takes to read the data from begin to end.

Does this matter at all? Well, it turns out those silly problems appear to
be bottlenecks of modern day algorithms: unless we manage to improve our
understanding of those problems, for many other practical problems,
we are stuck with solutions from the 90's, 80's, sometimes even the 70's, and
we do not know why.

In this thesis, we expose two novel algorithms and two novel data
structures related to this problem.

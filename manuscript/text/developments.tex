\section{Better Nonuniform Algorithms for \(k\)-SUM}

Meiser's algorithm solves a broader class of problems than the \(k\)-SUM
problems. In its general formulation, this algorithm solves
the point location problem (Problem~\ref{problem:point-location-hyperplane}):
to locate an input point in a fixed arrangement of $m$ hyperplanes in $R^d$.

Our analysis of Meiser's algorithm yields a nonuniform upper bound of \(O(d^3 \log
d \log m)\) for this problem. However, this is far from optimal.
The information theoretic lower bound obtained through
Buck's theorem (Theorem~\ref{thm:buck}) is only \(\Omega(d \log m)\).

Since the publication of our results, two new developments have surfaced.
%
The first reduces the nonuniform complexity of Meiser's algorithm for the point
location problem from
\(O(d^3 \log d \log m)\) down to \(O(d^2 \log m)\).
%
The second invents a completely novel technique to reduce the nonuniform
complexity of the same problem down to \(O(d \log d \log m)\), provided the
coefficients of the hyperplanes are constants.

We briefly sketch those two developments in the next subsections.

\subsection{Using Vertical Decomposition}

In 2016, Ezra and Sharir~\cite{ES17} improve the decision tree depth of
Meiser's algorithm by using vertical decomposition as the cell refinement
subroutine.

A big player in the complexity of our algorithm is the size of the sample
needed for the \(\varepsilon\)-net. It turns out that with
vertical decomposition the sample size can be significantly reduced: $O(d)$
instead of $O(d^2 \log d)$ for simplices.
%
A sample size of $O(d \log d)$ can easily be attained using existing results on
the VC-dimension of certain range spaces. One of the key results in~\cite{ES17}
is to get the sample size down to $O(d)$.

For instance,
when applying this improved algorithm to the \(k\)-SUM problem or the
subset-sum problem,
the change in the sample size causes Meiser's decision tree to be
much shallower than our implementation:
the depth of the tree is reduced by a \(n \log n\) factor in both cases leading
to overall complexities of \(O(n^2 \log n)\) and \(O(n^3)\) respectively.

\subsection{Using Inference Dimension}

In 2017, Kane, Lovett, and Moran~\cite{KLM18} proved that the point location
problem can be solved in \(O(d \log d \log m)\) linear queries.

Their decision tree is also a prune and search algorithm following
the generic approach of Meiser: sample, locate,
prune, recurse. The key ingredient is a new tool dubbed \emph{inference
dimension} that takes the role of the VC-dimension in Meiser's algorithm.

The major catch with this new decision tree is that it will only work for
sparse hyperplanes: hyperplanes with a small hamming weight. This is precisely
the case for the application of point location to the \(k\)-SUM and subset-sum
problems. For those applications, the decision trees in~\cite{KLM18} are shallower than the
decision trees in~\cite{ES17} by a factor of $n / \log n$:
$O(n \log^2 n)$ for \(k\)-SUM and \(O(n^2 \log n)\) for subset-sum.

Another feature of this algorithm is that it only uses \emph{comparison
queries}. When applied to \(k\)-SUM, our algorithm and the one in~\cite{ES17} are \(n\)-linear
decision trees, while the one in~\cite{KLM18} is a \(2k\)-linear decision tree.

In 2018, the authors of~\cite{KLM18} published another paper generalizing their
decision tree to work for arbitrary hyperplanes~\cite{KLM18b}. However,
the depth they obtain is much worse in that case: $O(d^4 \log d \log m)$ for
arbitrary hyperplanes. Moreover, it requires the use of \emph{generalized
comparison queries} which are more complex than the comparison queries
in~\cite{KLM18}.

In 2019,  Hopkins, Kane, and Lovett~\cite{HKL19} provide another decision
tree for point location of depth \(O(d \log^2 m)\) when the set of
hyperplanes is drawn from a distribution with weak restrictions. This new
algorithm only uses comparison queries.

\section{Timothy Chan Strikes Again}

In 2017, Timothy Chan presented a
\(O((n^2 / \log^2 n)(\log \log n)^{O(1)})\)-time real-RAM algorithm for
3SUM~\cite{Ch18}, shaving a logarithmic factor from previous solutions~\cite{Fr15,GS15}.

His technique relies on cuttings in near-logarithmic dimension on an augmented
real-RAM model with constant time nonstandard operations on $\Theta(\log n)$
bits words. Cuttings essentially replace dominance reporting in Gr\o nlund and
Pettie's algorithm~\cite{GP18}.
This is inspired by another one of his papers on APSP~\cite{Cha10}.

This new technique for shaving off logarithmic factors
can be applied to other problems, such as (\(\text{median}, +\))-convolution,
(\(\text{median}, +\))-matrix multiplication, and 3SUM-hard problems in
computational geometry including explicit 3POL.

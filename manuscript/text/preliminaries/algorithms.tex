\chapter{Algorithms}

In general, when we think of algorithms 



To make clear what will be considered an algorithm in this thesis, we explain
the different computation model that will be

\section{RAM Alorithms}

Real-RAM vs Word-RAM

\section{Nonuniform Algorithms}

%\improve{In general, I think there can hardly be too many warnings and remarks on the
%notion of nonuniformity. I expect reviewers to ask questions on, and
%misunderstand, everything nonuniform. Maybe it would be worth having an whole
%paragraph on this in the introduction, pointing to where nonuniformity is
%exploited.}


Allow a different algorithm for each input size.
Compare to circuits: time to construct circuit does not count.

\paragraph{Computation Trees and Decision Trees}
We consider \emph{algebraic computation trees}, whose internal nodes
are labeled with arithmetic (\(r \gets o_1 \op o_2, \op \in
\{\,+,-,\times,\div\,\}\)) and branching (\(z : 0\)) operations. We say that an
algebraic computation tree $T$ \emph{realizes} an algorithm $A$ if the paths
from the root to the leaves of $T$ correspond to the execution paths of \(A\)
on all possible inputs \(q \in \mathbb{R}^n\), where \(n\) is fixed. A leaf is
labeled with the output of the corresponding execution path of \(A\). Such a
tree is \emph{well-defined} if any internal node labeled \(r \gets o_1 \op
o_2\) has outdegree \(1\) and is such that either \(o_k = q_i\) for some \(i\)
or there exists an ancestor \(o_k \gets x \op y\) of this node, and any
internal node labeled \(z : 0\) has outdegree \(3\) and is such that either \(z
= q_i\) for some \(i\) or there exists an ancestor \(z \gets x \op y\) of this
node. In the algebraic computation tree model, we define the complexity
\(f(n)\) of an algorithm $A$ to be the minimum depth of a well-defined
computation tree that realizes $A$ for inputs of size $n$.

In the algebraic computation tree model, we only count the operations that
involve the input, that is, members of the input or results of previous
operations involving the input.

Algebraic Computation Trees,
Decision Trees,
Algebraic Decision Trees,
Linear Decision Trees.



\part{Preliminaries}\label{sec:preliminaries}

\chapter{Algorithms}

\section{RAM Alorithms}

Real-RAM vs Word-RAM

\section{Nonuniform Algorithms}

Allow a different algorithm for each input size.
Compare to circuits: time to construct circuit does not count.

\section{Computation Trees and Decision Trees}

Algebraic Computation Trees,
Decision Trees,
Algebraic Decision Trees,
Linear Decision Trees.


\chapter{Data Structures}

\section{Construction}

\section{Queries}

\section{Encodings}


\chapter{Geometry}

\section{Arrangements of Hyperplanes}

\subsection{Counting Cells}

\subsection{Zone Theorem}

The zone of a given pseudoline of an arrangement is the set of cells of the
arrangement supported by that pseudoline.
%
Figure~\ref{fig:a-zone-in-the-plane} illustrates a zone in a two-dimensional
arrangement of lines.
%
\begin{figure}
  \centering{}
  \includegraphics[width=\linewidth]{figures/a-zone-in-the-plane}
  \caption{%
    The zone defined by the dashed line in the two-dimensional
    arrangement of the plain lines is emphasized in light grey.%
  }\label{fig:a-zone-in-the-plane}
\end{figure}

We define the complexity of each cell to be the number of its sides.
We define the complexity of a zone to be the sum of the complexities of its cells.
%
The Zone Theorem states that the complexity of any zone is linear.
%
\begin{theorem}[Zone Theorem in the plane~\cite{BEPY90}]\label{thm:zone-theorem-2}
Given an arrangement of \(n+1\) pseudolines,
%
%in \(\mathbb{R}^2\),
%
the sum of the numbers of sides
%
in all the cells supported by one of the pseudolines
%
is at most \(\lfloor 9.5 n \rfloor - 1\).%
\footnote{%
Note that an earlier weaker (worse constant factor) linear bound is implied by
a theorem in~\cite{CGL85}.%
}
\end{theorem}



This result is important because it allows optimal
incremental construction of arrangements of line arrangements, a frequently
used tool.


\subsection{Point Location}

\section{Point Configurations}

\subsection{Wire Diagrams, \(\lambda\)-matrices, Stars, and Canonical Labelings}

Given a point set, the composition of its order type \(\chi\) with a
permutation \(\rho\) produces a new order type \(\chi' = \chi \circ \rho\).
This composition corresponds to a relabeling of the point set.
%
Aloupis et al.~\cite{AILOW14} defined the canonical labeling \(\rho^*(\chi)\)
of an order type \(\chi\) to be a permutation such that for all permutations
\(\pi\) we have \(\rho^*(\chi \circ \pi) = \pi^{-1} \circ \rho^*(\chi)\).
In other words, given two isomorphic order types \(\chi\) and \(\chi'\), we
have \(\chi \circ \rho^*(\chi) = \chi' \circ \rho^*(\chi')\), and
\({\rho^*(\chi')}^{-1} \circ \rho^*(\chi)\) is the isomorphism that sends
\(\chi\) to \(\chi'\).%
\footnote{Sometimes, two order types \(\chi\) and \(- \chi\) are also considered
to be isomorphic. See~\cite{AILOW14} for more details.}
They proved that the function \(\rho^*\) is
computable in \(O(n^2)\) time.
%
This first tool is useful to identify isomorphic order types.

They also showed that given any order type \(\chi\), a string \(E(\chi)\) of
\(O(n^2)\) bits, called the representation of \(\chi\), can be computed in
\(O(n^2)\) time, such that, if \(\chi\) and \(\chi'\) are two isomorphic order
types, then \(E(\chi) = E(\chi')\).
%
This second tool is useful to quickly compare two order types (a naive solution
would take \(\Theta(n^3)\) time by first computing a canonical labeling, and
then comparing all triples).

\begin{lemma}[Aloupis et al.~\cite{AILOW14}]\label{lem:canonical-labeling}
  Given an order type presented as an oracle,
  its canonical labeling of \(O(n \log n)\) bits
  and
  its canonical representation of \(O(n^2)\) bits
  can be computed in \(O(n^2)\) time
  in the word-RAM model.
\end{lemma}

Both tools generalize to chirotopes of point configurations in any dimension
\(d\) and, more generally, to chirotopes of rank \(d+1\).

\begin{lemma}[Aloupis et al.~\cite{AILOW14}]\label{lem:canonical-labeling-d}
  For all \(d \geq 2\),
  given a rank-(\(d+1\)) chirotope presented as an oracle,
  its canonical labeling of \(O(n \log n)\) bits
  and
  its canonical representation of \(O(n^d)\) bits
  can be computed in \(O(n^d)\) time
  in the word-RAM model.
\end{lemma}


\section{Duality}

\section{Duality}%
\label{sec:point-configurations:duality}

Technically speaking, the encoding we describe for realizable chirotopes
in Paper~\ref{paper:order-type-encoding}
encodes the chirotope of a given arrangement of lines or hyperplanes.
Moreover, for ease of presentation, we make the assumption that the vertices of
this arrangement have finite coordinates. In the two-dimensional case, this
is equivalent to having no two lines parallel. In these paragraphs, we give the
details necessary to rigorously handle all realizable chirotopes, including
degenerate ones. This is especially important in higher dimension, where the
situation is a bit more complicated than in two dimensions.

In two dimensions, we wish to encode order types of point configurations.
Since our encoding construction algorithm works with an arrangement of lines as
input, we need a mapping from those primal points to their dual lines. This
mapping should preserve the order type of the point configuration, hence it
needs to be \emph{order-preserving}. One such order-preserving duality is
the mapping \((a,b) \leftrightarrow y = ax - b\) (see Figure~\ref{fig:duality}).
\aurelien{This is not order-preserving. It is orientation-reversing.
Combinatorially equivalent.}

\begin{figure}
  \centering{}
  \includegraphics[scale=1]{figures/duality}
  \caption{Order preserving duality: ``\(p\) is above \(l\)'' if and only if
  ``\(l'\) is above \(p'\)''.}\label{fig:duality}
\end{figure}

To avoid parallel lines in the dual, it suffices to avoid intersection points
at infinity. In the primal, this translates to avoiding two points of the
configuration defining a vertical line, that is, with the same \(x\) coordinate.
This is easily done by performing a tiny rotation in the primal.
This (proper) rotation does not change the order type of the point set.

In higher dimension, the order-preserving point-line duality generalizes
to the following order-preserving point-hyperplane duality: We map each
\(d\)-dimensional point \((x_1, x_2, \ldots, x_d) \in \mathbb{R}^d\) to the hyperplane \(y_d =
\sum_{i=1}^{d-1} x_i y_i - x_d \) and the hyperplane \(x_d = \sum_{i=1}^{d-1}
y_i x_i - y_d \) to the \(d\)-dimensional point \(( y_1, y_2, \ldots, y_d) \in
\mathbb{R}^d\).

As before, we want hyperplanes to be non-parallel. In fact, we need an even
stronger assumption: We want all linearly independent subsets of \(d\)
hyperplanes to intersect in a point with finite coordinates.
%
Having no intersection points at infinity in the dual
means having no \(d\) points spanning a hyperplane parallel to the
\(x_d\) axis in the primal. This is easy to avoid by applying tiny rotations
in the primal. Again, those (proper) rotations do not change the chirotope of the
point set.

In dimension three and higher, one would think degenerate arrangements lead to
annoying nongeneral situations. However, those situations are easy to handle
with our technique: Degenerate subsets of hyperplanes are linearly dependent.
The determinant corresponding to a query asking about a
degenerate \(d+1\) subset is therefore zero. Our technique will identify those
degenerate queries and map them to the correct answer in a space-efficient way.

\aurelien{Shouldn't we simply work with the orientation predicate to construct
a nice realizing arrangement in \(O(n^d)\) time and be done with it?}

\aurelien{Maybe add a remark that most of the examples we give ignore
degenerate cases, even though our text does not.}


\section{A Geometric View of Algebraic Decision Problems}

Explain how any decision about polynomial predicates applied to tuples of input
real numbers is a geometry problem. Define point location.

Point location: finding the prism of the VD that contains the input point is
sufficient to answer all queries. Such a prism is defined by a linear number of
supporting queries.

\section{Cell Paritioning}

\subsection{Bottom Vertex Triangulation}

\subsection{Vertical Decomposition}


\section{Divide and Conquer}

This generalizes the idea of divide-and-conquer in sorting.

\subsection{Nets}

\subsection{Cuttings}

We encode the order type of an arrangement via
hierarchical cuttings as defined in~\cite{C93}. A cutting in \(\mathbb{R}^d\)
is a set of (possibly unbounded and/or non-full dimensional)
bounded-complexity cells that together partition \(\mathbb{R}^{d}\).
%
For our purposes, a cell is of bounded complexity if its boundary is defined by
a number of lines or pseudolines (and later, hyperplanes) of the arrangement
that depends only on the dimension, and not on the size of the arrangement.
%
A \(\frac{1}{c}\)-cutting of a set of \(n\) hyperplanes is a cutting with the
constraint that each of its cells is intersected by at most \(\frac{n}{c}\)
hyperplanes. There exist various ways of constructing \(\frac{1}{c}\)-cuttings of
size \(O(c^d)\).
Those cuttings allow for efficient divide-and-conquer
solutions to many geometric problems.


\subsection{Hierarchical Cuttings}

The hierarchical cuttings of Chazelle
have the additional property that they can be composed without multiplying the
hidden constant factors in the big-O notation. In particular, they allow
for \(O(n^d)\)-space \(O(\log n)\)-query \(d\)-dimensional point location data
structures (for constant \(d\)).
\ifjournal
  \begin{definition}[Hierarchical Cutting]
    Given \(n\) hyperplanes in \(\mathbb{R}^d\),
    a \(\ell\)-levels hierarchical cutting of parameter \(r > 1\)
    for those hyperplanes
    is a sequence of \(\ell\) levels labeled \(0,1, \ldots, \ell - 1\)
    such that%
    \footnote{In~\cite{C93}, Chazelle refers to this parameter as
    \(r_0\) and uses \(r\) to mean \(r_0^\ell\). Here we drop the subscript for
    ease of presentation.}
    \begin{itemize}
      \item Level \(i\) has \(O(r^{2i})\) cells,
      \item Each of those cells is further partitioned into \(O(r^2)\)
        subcells,
      \item The collection of subcells is a \(\frac{1}{r^{i+1}}\)-cutting for
        the set of hyperplanes,
      \item The \(O(r^{2(i+1)})\) subcells of level \(i\) are the cells of level \(i+1\).
    \end{itemize}
  \end{definition}
  \aurelien{Use constant \(c_k\) or \(c_d\) instead of Big-Oh?}
  It is clear from reading through the various references that those
  hierarchical cuttings can be constructed for arrangements of pseudolines with
  the same properties:
  In~\cite{C93},
  Chazelle first proves a vertex-count estimation lemma
  that only relies on incidence properties of line
  arrangements~\cite[Lemma~2.1]{C93}. Then he summons a lemma from~\cite{Ma93}
  that relies on the finite VC-dimension of the range space at
  hand~\cite[Lemma 3.1]{C93}.
  Here the ground set is the set of pseudolines and the ranges are the
  subsets induced by intersections with pseudosegments.
  The VC-dimension of this range space
  is easily shown to be finite and is known to be at most
  \(8\): every arrangement of \(9\) pseudolines contains a subset of
  \(6\) pseudolines in hexagonal formation~\cite{HM94}, which cannot be
  shattered.%
  \footnote{This a quote from~\cite{BMP05}. We could not access
  the original paper.}
  %
  \aurelien{The VC-dimension is not even needed. Only enumerating ranges is
  necessary.}
  Finally, he proves a lemma for the efficient construction of
  sparse \(\varepsilon\)-nets whose correctness again only relies on incidence
  properties of line arrangements~\cite[Lemma 3.2]{C93}.
  Using those three lemmas together with bottom vertex triangulation he is
  able to prove his main result:
\else%
In the plane, hierarchical cuttings can be
constructed for arrangement of pseudolines with the same properties.
\fi

\ifjournal
\begin{lemma}[{Chazelle~\cite[Theorem 3.3]{C93}}]\label{lem:hierarchical-cutting-d}
  Given \(n\) hyperplanes in \(\mathbb{R}^d\), for any real parameter \(r >
  1\), we can construct a \(\ell\)-levels hierarchical cutting of parameter
  \(r\) for those hyperplanes in time \(O(nr^{\ell(d-1)})\).
\end{lemma}

For pseudoline arrangements, bottom vertex triangulation can be traded for
vertical decomposition.
\begin{lemma}\label{lem:hierarchical-cutting-2}
  Given \(n\) pseudolines, for any real parameter \(r > 1\), we can construct
  a \(\ell\)-levels hierarchical cutting of parameter
  \(r\) for those pseudolines in time \(O(nr^\ell)\).
\end{lemma}

In particular, we will use those lemmas with
\(\ell = \lceil \log_r \frac nt \rceil\),
for some parameter \(t\),
so that the last level of the hierarchy defines a \(\frac
tn\)-cutting whose cells are each intersected by at most \(t\) pseudolines (or
hyperplanes).

Note that in~\cite{C93} the construction is described for constant
parameter \(r\).
This restriction on the parameter is easily lifted:
We can construct a hierarchical
cutting with superconstant parameter \(r\) by constructing a hierarchical
cutting with some appropriate constant parameter \(r'\), and then skip levels that we do
not need. This is used in Section~\ref{sec:query-time} to reduce the query time
from \(O(\log n)\) to \(O(\frac{\log n}{\log \log n})\).
\fi



\section{Existential Theory of the Reals}

%As is the custom in the computational geometry
%literature~\cite{PS85,EGPPSS92}, one could assume the roots of a constant-degree
%polynomial can be computed in constant time.
%However, to strenghten our results
%we will not make this assumption and \ldots

The problems we consider require our algorithms to manipulate polynomial
expressions and, potentially, their real roots. For that purpose, we will rely
on Collins's cylindrical algebraic decomposition (CAD)~\cite{C75}.
%
To understand the power of this method, and why it is useful for us, we give some
background on the related concept of first-order theory of the reals.

\begin{definition}
	A Tarski formula $\phi \in \mathbb{T}$ is a grammatically correct formula
	consisting of real variables ($x \in \mathbb{R}$), universal and
	existential quantifiers on those real variables
	($\forall,\exists\colon\,\mathbb{R}\times\mathbb{T}\to\mathbb{T}$), the
	boolean operators of conjunction and disjunction
	($\land,\lor\colon\,\mathbb{T}^2\to\mathbb{T}$), the six comparison
	operators ($<,\le,=,\ge,>,\ne\colon\,\mathbb{R}^2\to\mathbb{T}$), the four
	arithmetic operators ($+,-,*,/\colon\,\mathbb{R}^2\to\mathbb{R}$), the
	usual parentheses that modify the priority of operators, and constant real
	numbers (\(\mathbb{R}\)).
	%
	A Tarski sentence is a fully quantified Tarski formula.
	%
	%The first-order language of the reals is the set of all (true \emph{and}
	%false) Tarski sentences.
	%
	The first-order theory of the reals (\FOTR{}) is
	the set of true Tarski sentences.
\end{definition}

Tarski~\cite{T51} and Seidenberg~\cite{Sei74} proved that \FOTR{} is decidable.
However, the algorithm resulting from their proof has nonelementary complexity.
%
This proof, as well as other known algorithms, are based on quantifier
elimination, that is, the translation of the input formula to a much longer
quantifier-free formula, whose validity can be checked.
There exists a family of formulas for which any method of quantifier elimination
produces a doubly exponential size quantifier-free formula~\cite{DH88}.
%
Collins's CAD matches this doubly exponential complexity.
\begin{theorem}[Collins~\cite{C75}]
	\FOTR{} can be solved in $2^{2^{O(n)}}$ time in the real-RAM model, where
	\(n\) is the size of the input Tarski sentence.
\end{theorem}

See
%Collins~\cite{C75} for the original description of the algorithm,
Basu, Pollack, and Roy~\cite{BPR06} for additional details,
Basu, Pollack, and Roy~\cite{BPR96b} for a singly exponential algorithm when all
quantifiers are existential
(existential theory of the reals, \ETR{}),
Caviness and Johnson~\cite{CJ12} for an anthology of key papers on the subject,
and Mishra~\cite{M04} for a review of techniques to compute with roots of
polynomials.

Collins's CAD solves any \emph{geometric} decision problem that does not involve
quantification over the integers in time doubly exponential in the problem
size. This does not harm our results as we exclusively use this algorithm to
solve constant size subproblems. Geometric is to be understood in the sense of Descartes and Fermat, that
is, the geometry of objects that can be expressed with polynomial equations. In
particular, it allows us to make the following computations in the real-RAM and
bounded-degree ADT models:
%, with only the four arithmetic operators:
\begin{enumerate}
\setlength{\itemsep}{0pt}
\setlength{\parskip}{0pt}
\setlength{\parsep}{0pt}
\item Given a constant-degree univariate polynomial, count its real roots
	in $O(1)$ operations,
\item Sort $O(1)$ real numbers given implicitly as roots of some
	constant-degree univariate polynomials in $O(1)$ operations,
\item Given a point in the plane and an arrangement of a constant number of
constant-degree polynomial planar curves, locate the point in the
arrangement in $O(1)$ operations.
%\item Given a constant-degree polynomial planar curve and an $N \times N$ grid
%defined by $N$ vertical lines and $N$ horizontal lines compute the $O(N)$
%cells of the grid intersected by the curve in $O(N)$ operations.
%\item Given a set of $N$ constant-degree polynomial planar curves compute a
%$\frac 1r$-net of size $O(r \log r)$ for the range space defined by those curves and $y$-axis
%aligned trapezoidal patches whose top and bottom sides are pieces of the
%curves in $O_r(N)$ operations.
%\item Given a set $\Gamma$ of $M$ constant-degree polynomial planar curves, a
%set $\Pi$ of $N$ points in the plane and a subset $\Gamma_r
%\subseteq \Gamma$ of size $O(r \log r)$, for all cells $C$ of the vertical decomposition of the
%arrangement of $\Gamma_r$, find the curves in $\Gamma \setminus
%\Gamma_r$ and the points in $\Pi$ that intersect $C$ in
%$O_r(M+N)$ operations total.
\end{enumerate}

%It has been proved that adding the function $\sin(x)$ to the set of allowed
%operators makes the extended theory undecidable, since it allows the encoding
%of sentences in the first-order theory of the integers, which is
%undecidable~\cite{R68}.
%Tarski's exponential function problem asks
%whether the theory can be extended with the exponential function $2^x$
%without rendering it undecidable. This question is to this day still open.
%Macintyre and Wilkie~\cite{MW96} showed that the decidability of this theory
%follows from Schanuel's conjecture.
%If proven true, it would allow us to apply our results to more general curves,
%for example, curves defined by logarithmic and exponential functions.

Instead of bounded-degree algebraic decision trees as the nonuniform model
we could consider decision trees in which
each decision involves a constant-size instance of the decision problem in the
first-order theory of the reals. The depth of a bounded-degree algebraic
decision tree simulating such a tree would only be blown up by a constant factor.


\chapter{Problems}

\section{Sorting}

\section{\(k\)-SUM and Linear Degeneracy Testing}

\section{3SUM}

\section{Sorting \(X+Y\)}

\section{SUBSET-SUM}

\section{Hopcroft's Problem}

\begin{problem}[Hopcroft's problem]
	Given a set of $n$ points and $m$ lines in $\mathbb{R}^2$,
	does any point lie on any line?
\end{problem}
There are combinatorial upper bounds~\cite{ST83} on the number of point-line
incidences an instance can have and there are algorithmic lower
bounds~\cite{E96} essentially matching the complexity of Matou\v{s}ek's algorithm.

\paragraph{Szemeredi-Trotter theorem}
Szemeredi~and~Trotter~\cite{ST83} give an upper bound on the number of
incidences between points and lines in $\mathbb{R}^2$.
\begin{theorem}[Szemeredi and Trotter~\cite{ST83}]
	The number of incidences between $n$ points and $m$ lines in the plane, is
	$O(m^{2/3}n^{2/3}+n+m)$.
\end{theorem}


\begin{problem}[Hopcroft's problem (any dimension)]
	Given a set of $n$ points and $m$ hyperplanes in $\mathbb{R}^d$,
	is any point contained in any hyperplane?
\end{problem}

Combinatorial upper bounds~\cite{AA92,CEGSW90}.
Algorithmic lower bounds~\cite{BK03}.

\section{Dominance Reporting}

\section{General Position Testing}

\section{Point Location}

\section{3POL}

\section{Polynomial Dominance Reporting}

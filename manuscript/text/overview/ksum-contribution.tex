Our first contribution with paper A is a finer analysis of Meiser's algorithm
that shows that the depth of his decision tree when applied to \(k\)-SUM is
actually \(O(n^3 \log^2 n)\).  On top of that, we show how to implement a
variant of this decision tree in the real-RAM model so that its uniform running
time is \(n^{\frac{k}{2} + O(1)}\) while keeping the nonuniform running time
unchanged. Note that a naive implementation of Meiser's algorithm has a uniform
running time of \(n^{k + O(1)}\). Those contributions are described in detail
in Chapter~\ref{chapter:ksum-algorithm}.

The approach taken by Meiser's algorithm follows the prune and search method:
Take a sample of the hyperplanes for which we do not yet know the relative
location of the input point. Locate the input point with respect to this sample
by brute-force. This amounts to identifying the cell of the sample's arrangement
which contains the input point. Refine the location of the input point in the
sample by partitioning the containing cell into low complexity subcells. Whenever
this low complexity subcell is located completely on one side of an hyperplane,
the input point is located on the same side, and so we can discard this
hyperplane without a single query to the input point. Since some hyperplanes
may intersect the low complexity subcell, rince and repeat.

The location refinement is necessary because this is the only way we can
guarantee a bound on the number of hyperplanes we have to recurse on.
Using the theory of \(\varepsilon\)-nets, we can show that, for a sample of
reasonable size, any simplex that is not intersected by a sample hyperplane
is not intersected by more than a constant fraction of the set of
hyperplanes from which the sample is drawn (with high probability). We
define the refined subcell of the algorithm presented above to be the simplex
of the bottom-vertex triangulation of the sample's arrangement that contains
the input point.

\todo{Explain cutting tools and how they allow for fast algorithm.}

What makes the success of our methods are geometric divide-and-conquer
tools: \(\varepsilon\)-nets and cuttings.

To illustrate, we make the parallel between those tools and
a very simple divide-and-conquer sorting algorithm: quicksort.

Remember that the sorting problem consists in uncovering the answer to the
\(n \choose 2\) pairwise comparisons of input numbers.
The quicksort algorithm progresses as follows: it picks one element of the
list, the pivot \(x_p\), and compares all the others to it. Let us say the algorithm is
lucky, and at least \(\Omega(n)\) of the remaining input numbers are
smaller than
\(x_p\)
and at least \(\Omega(n)\) of the remaining input numbers are larger than
\(x_p\). Then, by transitivity, we know the answer to at least
\(\Omega(n^2)\) comparisons
even though we only explicitly asked for \(O(n)\) of them. It is easy to repeat
this process by noticing that those \(\Omega(n^2)\) answers can be efficiently encoded by
constructing two lists, one with the numbers smaller than \(x_p\) and one with
the larger ones. Then the comparisons that are still unanswered are exactly the
ones between two elements from the same list. Hence, we can solve the problem
recursively by sorting the two lists individually.
Too see that this takes only \(O(\log n)\) steps

\todo{Random pivot vs e-net}

\todo{Deterministic pivot vs cuttings. Also mention mergesort.}

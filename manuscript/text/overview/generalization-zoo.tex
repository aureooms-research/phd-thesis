Obviously, the \(k\)-SUM problem is not the only possible way to generalize
Sorting.

Hopcroft's problem asks whether given \(n\) points and \(n\) hyperplanes in
\(\mathbb{R}^d\), one of the points lies on one of the hyperplanes. When
\(d=1\), this problem is Element Uniqueness. Finding the location (``above /
below'') of each point
with respect to each hyperplane generalizes Sorting.

The dominance reporting problem asks, given \(n\) points in \(\mathbb{R}^d\),
to report all pairs of points such that the first dominates the other in all
dimensions. Once again, for \(d=1\), this problem is Sorting because it asks
for the answer to all comparisons of the type \(p_i \leq q_i\).

Sorting \(X+Y\) is also a canonical problem in \(P\): given two sets
\(X\) and \( Y \) of \( n \) numbers each, sort the set \( \{\, x + y \colon\,
x \in X, y \in Y\,\} \). Sorting \(X+Y\) reduces linearly to the sorting
version of 4SUM because it asks for the sign of all comparisons of the type
\(x+y \leq x'+y'\).

We already saw that GPT and Sorting belong to the same family of
high-dimensional point location problems. There is a good reason for that: when
\(d=1\), GPT is Element Uniqueness. In one-dimensional space, GPT
asks whether any two points are the same. Picturing this space as
the (horizontal) real line, we see that the ``sorting version'' of GPT asks to
compute for each pair of points which one is on the ``left'' of the other which
simply amounts to Sorting the one-dimensional input points.

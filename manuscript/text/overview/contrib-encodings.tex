In paper C we design the first subquadratic space data structure for encoding
the combinatorial type of a two-dimensional GPT instance. This data structure
can be constructed in quadratic time and queries are answered in sublogarithmic
time. Those results can be adapted to work for higher-dimensional GPT to yield
sub-\(O(n^d)\) space data structures with good construction and query times.

Since 3SUM reduces to GPT, the results of paper C can be applied to encode the
combinatorial type of 3SUM instances. However, since 3SUM is much better
understood than GPT we should aim for better encodings. This is exactly what we
do in paper D. By filling the gaps in the partial data structure
used in Gr\o nlund and Pettie's algorithm~\cite{GP18}, we design an encoding
that uses \(O(n^{3/2} \log n)\) bits, can be constructed in \(O(n^2)\) time and
answers queries in constant time.

\todo{Conclude with remark on encodings derived from decision trees.}

Since publication of those papers, a few developments have surfaced.

Ezra and Sharir~\cite{ES17} show how trading simplices of the bottom-vertex
triangulation for prisms of the vertical decomposition in Meiser's algorithm
yields a shallower decision tree of depth \(O(n^2 \log n)\). Essentially, the
improvement over our result in paper A lies in the fact that,
for vertical decomposition, the sample size can be taken to be an order of
magnitude smaller.

%All nonuniform algorithm for \(k\)-SUM we have mentioned query the input with
%\(s\)-linear queries: a \(s\)-linear query asks for the sign of weighted sums of
%\(s\) input numbers. A decision tree that queries the input exclusively with
%\(s\)-linear queries is called a \(s\)-linear decision tree.

%The nonuniform algorithm of Gr\o nlund and Pettie for
%\(k\)-SUM with \(k\) odd uses many more queries than the methods of Meyer auf der
%Heide and Meiser. However, its merit lies in that it only asks very simple
%questions about the input: this algorithm probes the input
%with (\(2k-2\))-linear queries while the others use \(n\)-linear queries.

In a breakthrough paper, Kane, Lovett, and Moran~\cite{KLM18}, give a
\(O(n \log^2 n)\) nonuniform linear decision tree for \(k\)-SUM, almost
matching the \(\Omega(n \log n)\) lower bound. This improves both on paper A
and Ezra and Sharir~\cite{ES17}.

In~\cite{Ch18}, Chan shaves more logarithmic factors from the time complexity
of uniform algorithms for 3SUM and 3POL. While we focused on applications that solve
3SUM-hard geometric problems with one-dimensional input in paper B, he shows how
the ideas that work for 3POL also work for some 3SUM-hard geometric problems
with two-dimensional data.

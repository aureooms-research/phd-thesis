\chapter{\(k\)-SUM}

The \(k\)-SUM problem is a straightforward generalization of the 3SUM problem:
given a list of \(n\) real numbers, decide whether \(k\) of them sum to zero.

The paper of Gr\o nlund and Pettie also discusses the following generalization of
the 3SUM problem: ``For a fixed \(k\), given a set of \(n\) real numbers,
decide whether there exists a \(k\)-subset whose elements sum to zero.''
This problem is called the \(k\)-SUM problem.

Obviously, the 3SUM problem is the \(k\)-SUM problem where \(k=3\).
Moreover, there is a simple reduction from \(k\)-SUM to 2SUM when \(k\) is even
and to 3SUM when \(k\) is odd. Those reductions yield a
\(O(n^{\frac{k}{2}} \log n)\) time real-RAM algorithm for \(k\) even and a
\(O(n^{\frac{k+1}{2}})\) time real-RAM algorithm for \(k\) odd.

In their paper, in addition to the slightly subquadratic uniform algorithm for
3SUM, Gr\o nlund and Pettie give a strongly subquadratic nonuniform
algorithm for 3SUM. The algorithms runs in time \(\tilde{O}(n^{3/2})\), and,
because of the aforementioned reduction, immediately yields an improved
\(\tilde{O}(n^{\frac{k}{2}})\) nonuniform time complexity for \(k\)-SUM when
\(k\) is odd.

As for uniform time complexity we do not know whether this nonuniform
improvement can be transferred to the real-RAM model: we do not know of any
real-RAM \(o(n^{\frac{k+1}{2}})\) time algorithm for \(k\)-SUM when
\(k\) is odd.

The \(k\)-SUM problem reduces to the following point location problem: ``Given
a input point \(q \in \mathbb{R}^n\), locate \(q\) in the arrangement of
\(n \choose k\) hyperplanes of equation \(x_{i_1} + x_{i_2} + \cdots +
x_{i_k} = 0\).'' Applying the best nonuniform algorithms for point location in
arrangements of hyperplanes by Meyer auf der Heide~\cite{M84} and
Meiser~\cite{M93} yields linear decision trees of depth \(n^{O(1)}\) for
\(k\)-SUM, where the constant of proportionality in the big-oh does not depend
on \(k\).


We consider the \(k\)-SUM problem for \(k=O(1)\).
For fixed \(k\), the \(k\)-SUM problem can be solved in polynomial time
\(O(n^k)\) by testing all possible candidate solutions.
The interesting question is whether it is possible to improve on
this brute-force solution.

The \(k\)-SUM problem is a fixed-parameter version of the subset-sum problem, a
standard \textit{NP}-complete problem. The \(k\)-SUM problem, and in particular
the special case of 3SUM, has proved to be a cornerstone of the fine-grained
complexity program aiming at the construction of a complexity theory for
problems in $P$. In particular, there are deep connections between the
complexity of \(k\)-SUM, the Strong Exponential Time
Hypothesis~\cite{PW10,CGIMPS15}, and the complexity of many other major
problems in
$P$~\cite{GO95,BH99,MO01,P10,ACLL14,AVW14,GP18,KPP14,ALW14,AWY15,CL15}.
It has been known for long that \(k\)-SUM is $W[1]$-hard. Recently, it was shown
to be $W[1]$-complete by Abboud et al.~\cite{ALW14}.

%In what follows, we use the notation \([n] = \{\,1,2,\ldots ,n\,\}\).
The problem amounts to deciding in $n$-dimensional space, for each hyperplane
\(H\) of equation \(x_{i_1} + x_{i_2} + \cdots +x_{i_k} = 0\), whether \(q\)
lies on, above, or below \(H\). Hence this indeed amounts to locating the point
$q$ in the arrangement formed by those hyperplanes. We emphasize that the set
of hyperplanes depends only on $k$ and $n$ and not on the actual input vector
$q$.

Linear degeneracy testing (\(k\)-LDT) is a generalization of \(k\)-SUM where we
have arbitrary rational coefficients\footnote{The usual definition of \(k\)-LDT
allows arbitrary \emph{real} coefficients. However, the algorithm we provide
for Lemma~\ref{lem:multiple} needs the vertices of the arrangement of
hyperplanes to have rational coordinates.}
and an independent term in the equations
of the hyperplanes.
\begin{problem}[\(k\)-LDT]
 Given an input vectors \(q\in\mathbb{R}^n\) and
 $\alpha \in \mathbb{Q}^n$ and constant $c \in \mathbb{Q}$
 decide whether there exists a
 $k$-tuple \((i_1, i_2,\ldots ,i_k) \in {[n]}^k\) such that
 \(c + \sum_{j=1}^k \alpha_j q_{i_j} = 0\).
 \end{problem}
Our algorithms apply to this more general problem with only minor changes.

The \emph{\(s\)-linear decision tree model} is a standard model of computation
in which several lower bounds for \(k\)-SUM\ have been proven. In the decision tree
model, one may ask well-defined questions to an oracle that are answered
``yes'' or ``no.'' For $s$-linear decision trees, a well-defined question consists
of testing the sign of a linear function on at most \(s\) numbers \(q_{i_1},\ldots,q_{i_s}\) of the
input \(q_1,\ldots,q_n\) and can be written as
$$
	c + \alpha_1 q_{i_1} + \cdots + \alpha_s q_{i_s} \ask{\le} 0
$$
Each question is defined to cost a single unit. All other operations can be
carried out for free but may not examine the input vector $q$. We refer to
$n$-linear decision trees simply as linear decision trees.

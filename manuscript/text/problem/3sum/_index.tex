\chapter{3SUM}

The 3SUM problem is defined as follows: given $n$ distinct real numbers, decide
whether any three of them sum to zero.
%
A popular conjecture is that no $O(n^{2-\delta})$-time algorithm for 3SUM
exists, for any $\delta > 0$. This conjecture has been used to show conditional
lower bounds for problems in P, notably in computational geometry with problems
such as
GeomBase, general position~\cite{GO95}
and
Polygonal Containment~\cite{BH01},
and more recently for string problems such as
Local Alignment~\cite{AVW14}
and
Jumbled Indexing~\cite{ACLL14},
as well as
dynamic versions of graph problems~\cite{P10,AV14},
triangle enumeration and Set Disjointness~\cite{KPP16}.
%
For this reason, 3SUM is considered one of the key subjects of an
emerging theory of complexity-within-P, along with other problems such as
all-pairs shortest paths,
orthogonal vectors,
boolean matrix multiplication,
and conjectures such as
the Strong Exponential Time Hypothesis~\cite{AVY15,HKNS15,CGIMPS16}.

Because fixing two of the numbers $a$ and $b$ in a triple only allows for one
solution to the equation $a + b + x = 0$, an instance of 3SUM has at most
$n^2$ degenerate triples. An instance giving a matching lower bound is for
example the set $\{\,\frac{1-n}{2},\ldots,\frac{n-1}{2}\,\}$ (for odd $n$)
with $\frac{3}{4} n^2 + \frac 14$ degenerate triples.
%
One might be tempted to think that the number of ``solutions'' to the problem
would lower bound the complexity of algorithms for the decision version of the
problem, as it is the case for this problem, and other problems, in restricted
models of computation~\cite{Er96,Er99a}.
%
%This is a common misconception.
This intuition is incorrect.
%
Indeed, Gr\o nlund and Pettie~\cite{GP18} proved that there exist
$\tilde{O}(n^{3/2})$-depth linear decision trees and $o(n^2)$-time real-RAM
algorithms for 3SUM\@.

As for upper bounds, Baran et al.~\cite{BDP08} gave subquadratic Las Vegas
algorithms for 3SUM on integer and
rational numbers in the circuit RAM, word RAM, external memory, and
cache-oblivious models of computation. The idea of their approach is to exploit
the parallelism of the models, using linear and universal hashing.


More recently, Gr{\o}nlund and Pettie~\cite{GP18} proved the existence of a linear decision tree
solving the 3SUM problem using a strongly subquadratic number of linear queries.
The classical quadratic algorithm for 3SUM uses \(3\)-linear queries
while the decision tree of Gr{\o}nlund and Pettie uses \(4\)-linear queries and
requires $O(n^{\sfrac{3}{2}} \sqrt{\log n})$ of them.

They also provide two subquadratic 3SUM
algorithms. A deterministic one running in
$O(n^2/{(\log n/\log \log n)}^{\sfrac{2}{3}})$
time and a randomized one running in
$O(n^2 {(\log \log n)}^2 / \log n)$ time with high probability.
These results refuted the long-lived conjecture that
3SUM cannot be solved in subquadratic time in the RAM model.

Freund~\cite{Fr15} and Gold and Sharir~\cite{GS15} later gave improvements on the
results of Gr{\o}nlund and Pettie~\cite{GP18}. Freund~\cite{Fr15} gave a deterministic algorithm for
3SUM running in \(O( {n^2\log \log n}/{\log n})\) time.
Gold and Sharir~\cite{GS15} gave another deterministic algorithm for 3SUM
with the same running time and shaved off the $\sqrt{\log n}$ factor in the
decision tree complexities of 3SUM and \(k\)-SUM given by Gr{\o}nlund and Pettie.


The seminal paper by Gajentaan and Overmars~\cite{GO95} showed the crucial role
of 3SUM in understanding the complexity of several problems in
computational geometry.
Since then, there has been an enormous amount of work focusing on the complexity of
3SUM and this problem is now considered a key tool of
complexity-within-P~\cite{GO95,BH99,MO01,BDP08,P10,ACLL14,AVW14,GP18,KPP14,ALW14,AWY15,CL15}.
The current conjecture is that no $O(n^{2-\delta})$-time algorithm exists for 3SUM.

In Erickson~\cite{Er99a}, it is shown that we cannot solve 3SUM in
subquadratic time in the \(3\)-linear decision tree model:
\begin{theorem}[Erickson~\cite{Er99a}]
The optimal depth of a \(k\)-linear decision tree that solves
the \(k\)-LDT problem is $\Theta(n^{\lceil\frac{k}{2}}\rceil)$.
\end{theorem}
The proof uses an adversary argument which can be explained geometrically. As
we already observed, we can solve \(k\)-LDT problems by modeling them as point
location problems in an arrangement of hyperplanes. Solving one such problem
amounts to determining which cell of the arrangement contains the input point.
The adversary argument of Erickson~\cite{Er99a} is that there exists a cell having
$\Omega(n^{\lceil\frac{k}{2}}\rceil)$ boundary facets and in this model point
location in such a cell requires testing each facet.

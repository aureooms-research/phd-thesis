\section{3POL}

\todo{Give at least one paragraph per citation on 3POL.}

We consider an algebraic generalization of the 3SUM problem: we replace the sum
function by a constant-degree polynomial in three variables $F \in
\mathbb{R}[x,y,z]$ and ask to determine whether there exists a
\emph{degenerate} triple $(a,b,c)$ of input numbers such that $F(a,b,c)=0$. We
call this new problem the \emph{3POL problem}.

Some combinatorics aspects of the 3POL problem have already been studied.
%
For the particular case $F(x,y,z) = f(x,y) - z$ where $f \in \mathbb{R}[x,y]$
is a constant-degree bivariate polynomial, Elekes and Rónyai~\cite{ER00} show
that the number of degenerate triples is $o(n^2)$ unless $f$ is
\emph{special}. Special for $f$ means that $f$ has one of the two special forms
\begin{displaymath}
f(u,v)=h(\varphi(u)+\psi(v))
\qquad
\text{or}
\qquad
f(u,v)=h(\varphi(u)\cdot\psi(v)),
\end{displaymath}
where $h,\varphi,\psi$ are univariate polynomials of constant degree.
It must be noted that the 3SUM problem falls in the special category since, in
that case, \( f \) is the sum function.
%
Elekes and Szabó~\cite{ES12} later generalized this result to a broader range
of functions $F$ using a wider definition of specialness.
%
Raz, Sharir and Solymosi~\cite{RSS14} and Raz, Sharir and de Zeeuw~\cite{RSZ15}
improved both bounds to $O(n^{11/6})$.
%
They translated the problem into an incidence problem between points and
constant-degree algebraic curves. Then, they showed that unless $f$ (or $F$) is
special, these curves have low multiplicities. Finally, they applied a theorem
due to Pach and Sharir~\cite{PS98} bounding the number of incidences between
the points and the curves. Some of these ideas appear in our approach.


\subsection{Combinatorics Results on 3POL and GPT}
In a series of results spanning fifteen years,
Elekes and Rónyai~\cite{ER00},
Elekes and Szabó~\cite{ES12},
Raz, Sharir and Solymosi~\cite{RSS14}, and
Raz, Sharir and de Zeeuw~\cite{RSZ15}
give upper bounds on the number of degenerate triples for the 3POL problem.
The last and strongest result is the following:
\begin{theorem}[Raz, Sharir and de Zeeuw~\cite{RSZ15}]
	Let $A$, $B$, $C$ be $n$-sets of real numbers and $F \in \mathbb{R}[x,y,z]$
	be a polynomial of constant degree, then
	\begin{displaymath}
		%| \{\, (a,b,c) \in ( A \times B \times C ) \st F(a,b,c) = 0 \,\} | = O(n^{11/6}),
		| Z(F) \cap ( A \times B \times C ) | = O(n^{11/6}),
	\end{displaymath}
	unless $F$ has some group related form.\footnote{Because our results do not
	depend on the meaning of \emph{group related form}, we do not bother
	defining it here. We refer the reader to Raz, Sharir and de Zeeuw~\cite{RSZ15}
	for the exact definition.}
\end{theorem}

Raz, Sharir and de Zeeuw~\cite{RSZ15} also look at the number of degenerate
triples for the General Position Testing problem when the input is restricted
to points lying on a constant number of constant-degree algebraic curves.
\begin{theorem}[Raz, Sharir and de Zeeuw~\cite{RSZ15}]\label{thm:rsz15:col}
Let $C_1, C_2, C_3$ be three (not necessarily distinct) irreducible algebraic curves
of degree at most $d$ in $\mathbb{C}^2$, and let $S_1 \subset C_1, S_2 \subset C_2, S_3
\subset C_3$ be finite subsets. Then the
number of proper collinear triples in $S_1 \times S_2 \times S_3$ is
\begin{displaymath}
	O_d( |S_1|^{1/2} |S_2|^{2/3} |S_3|^{2/3} + |S_1|^{1/2} (|S_1|^{1/2} + |S_2| +
|S_3| ) ),
\end{displaymath}
unless $C_1 \cup C_2 \cup C_3$ is a line or a cubic curve.
\end{theorem}

Nassajian Mojarrad, Pham, Valculescu and de Zeeuw~\cite{MPVd16} and
Raz, Sharir and de Zeeuw~\cite{RSZ16} proved bounds for versions of the
problem where $F$ is a $4$-variate polynomial.

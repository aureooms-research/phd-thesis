\section{First Subquadratic Algorithms for 3POL}
We focus on the computational complexity of 3POL\@. Since 3POL contains 3SUM,
an interesting question is whether a generalization of Gr\o nlund and Pettie's
3SUM algorithm exists for 3POL\@. If this is true, then we might wonder whether
we can ``beat'' the $O(n^{11/6}) = O(n^{1.833\ldots})$ combinatorial bound of Raz,
Sharir and de Zeeuw~\cite{RSZ15} with nonuniform algorithms. We give a positive
answer to both questions: we design
a uniform
$O(n^2 {(\log \log n)}^{3/2} / {(\log n)}^{1/2})$-time
real-RAM algorithm
and
a nonuniform
$O(n^{12/7+\varepsilon}) = O(n^{1.7143})$-depth
bounded-degree algebraic decision tree
for 3POL\@.%
\footnote{Throughout this document, $\varepsilon$ denotes a positive real
number that can be made as small as desired.}
To prove our uniform result, we present a fast algorithm for the Polynomial
Dominance Reporting (PDR) problem, a far reaching generalization of the
Dominance Reporting problem. As the algorithm for Dominance Reporting and its
analysis by Chan~\cite{Cha08} is used in fast algorithms for all-pairs shortest
paths, (min,+)-convolutions, and 3SUM, we expect this new algorithm will have
more applications.

Our results can be applied to many algebraic degeneracy testing problems, such
as the General Position Testing~(GPT) problem: ``Given $n$ points in the plane, do
three of them lie on a line?'' It is well known that GPT is 3SUM-hard,
and it is open whether GPT admits a subquadratic algorithm. Raz, Sharir
and de Zeeuw results on the 3POL problem~\cite{RSZ15} can be applied to obtain
a combinatorial bound of $O(n^{11/6})$ on the
number of collinear triples when the input points are known to be lying on
a constant number of polynomial curves, provided those curves are neither
lines nor cubic curves. A corollary of our first result is that
GPT where the input points are constrained to lie on
$o({(\log n)}^{1/6}/{(\log \log n)}^{1/2})$
constant-degree polynomial curves (including lines and cubic curves)
admits a subquadratic real-RAM algorithm and
a strongly subquadratic bounded-degree algebraic decision tree.
Interestingly, both reductions from 3SUM to GPT on 3 lines (map $a$ to $(a,0)$,
$b$ to $(b,2)$, and $c$ to $(\frac c2, 1)$) and from 3SUM to GPT on a
cubic curve (map $a$ to $(a^3,a)$, $b$ to $(b^3,b)$, and $c$ to $(c^3,c)$)
construct such special instances of GPT\@.
This constitutes the first step towards closing the major open question of
whether GPT can be solved in subquadratic time.
%
To further convince the reader of the expressive power of the 3POL problem,
we also give reductions from the problem of counting triples of points spanning
unit circles, from the problem of counting triples of points spanning unit area
triangles, and from the problem of counting collinear triples in any dimension.

The algorithms we present manipulate polynomial expressions.
%
In computational geometry, it is customary to assume the real-RAM model can be
extended to allow the computation of roots of constant degree polynomials.
We distance ourselves from this practice and take particular care
of using the real-RAM model and the bounded-degree algebraic decision tree
model with only the four arithmetic operators.

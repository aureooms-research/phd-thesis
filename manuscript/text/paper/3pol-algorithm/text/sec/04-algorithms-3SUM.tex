\section{Algorithmic Results on the 3SUM Problem}
For the sake of simplicity, we consider the following definition of 3SUM\@:
\begin{problem}[3SUM]
Given 3 sets $A$, $B$, and $C$, each containing $n$ real numbers, decide
whether there exist $a \in A$, $b \in B$, and $c \in C$ such that $c=a+b$.
\end{problem}

Gajentaan and Overmars~\cite{GO95} were the first to take serious interest in
the 3SUM problem. They introduced the concept of \(n^2\)-hard (or
3SUM-hard) problems:
they revealed a connection between seemingly unrelated geometric
problems by showing that each of them is at least as hard as 3SUM.

A quadratic lower bound for solving 3SUM holds in a restricted model of
computation: the $3$-linear decision tree model. Erickson~\cite{E99}
and Ailon and Chazelle~\cite{AC05} showed
that, in this model, where one is only allowed to test the sign of a linear
expression of up to three input numbers, no matter which decision tree you
use, there always exists an instance for which a quadratic number of
critical tuples must be tested.
%\improve{Noam asks if there is some lower bound analog for 3POL.}
\begin{theorem}[Erickson~\cite{E99}]
The depth of a \(3\)-linear decision tree for 3SUM is $\Omega(n^2)$.
\end{theorem}


While no evidence suggested that this lower bound could be extended to other
models of computation, it was eventually conjectured that 3SUM requires
$\Omega(n^2)$ time.

Baran et al.~\cite{BDP08} were the first to give concrete evidence
for doubting the conjecture.
They gave subquadratic Las Vegas algorithms for 3SUM, where input
numbers are restricted to be integer or rational, in the circuit RAM,
word RAM, external memory, and cache-oblivious models of computation. Their idea
is to exploit the parallelism of the models, using linear and
universal hashing.

Gr\o nlund~and~Pettie~\cite{GP14}, using a trick due to Fredman~\cite{F76},
showed that there exist subquadratic decision trees for 3SUM when the queries
are allowed to be $4$-linear.
\begin{theorem}[Gr\o nlund and Pettie~\cite{GP14}]
There is a $4$-linear decision tree of depth
$O(n^{3/2} \sqrt{\log n})$ for 3SUM\@.
\end{theorem}

They also gave deterministic and randomized
subquadratic real-RAM algorithms for 3SUM, refuting the conjecture.
Similarly to the subquadratic $4$-linear decision trees, these new results
use the power of $4$-linear queries.
%
These algorithms were later improved by Freund~\cite{F15} and
Gold~and~Sharir~\cite{GS15}.
The currently best published bound for real-RAM 3SUM is
\begin{theorem}[Freund~\cite{F15}, Gold~and~Sharir~\cite{GS15}]
There is a $O(n^2 \log \log n / \log n)$-time real-RAM algorithm
for 3SUM\@.%
\footnote{Chan~\cite{Ch18} shows that an additional logarithmic factor can be shaved
by augmenting the real-RAM model with constant time nonstandard operations on
$\Theta(\log n)$ bits words. His improvements extend to 3POL.}
\end{theorem}

Since then, the conjecture was eventually updated. This new conjecture is
considered an essential part of the theory of complexity-within-P.
\begin{conjecture}[label=conj:3sum,restate=ConjectureSUM]
	There is no \(n^{2-\Omega(1)}\)-time real-RAM algorithm for 3SUM.
\end{conjecture}


The
$k$-SUM problem is a generalization of 3SUM where we are given an $n$-set of
real numbers and are asked to decide whether it contains a $k$-tuple that sums
to zero. The lower bound of Erickson generalizes to \( \Omega (n^{\lceil k/2
\rceil }) \) for \( k \)-linear decision trees that solve \( k \)-SUM\@.
All the aforementioned subquadratic algorithms for 3SUM inspect the input
through $4$-linear queries.  Gr\o nlund and Pettie's decision tree can be
turned into a \( O(n^{k/2} \sqrt{\log{n}}) \)-depth $(2k-2)$-linear decision
tree for $k$-SUM, for all odd $k \geq 3$.\footnote{The main result in Gold and Sharir's paper~\cite{GS15}
is a \emph{randomized} \(O(n^{k/2})\)-depth \((2k-2)\)-linear decision tree for
\(k\)-SUM, for all odd \(k \geq 3\).}

Early results of Meyer auf der Heide~\cite{M84} and Meiser~\cite{M93} show that if one is allowed to
use $n$-linear queries, the complexity drops to a polynomial in the input size
whose exponent does not depend on $k$. Cardinal, Iacono, and
Ooms~\cite{CIO16}
carefully analyzed the complexity of Meiser's algorithm to show that $k$-SUM
can be solved in \( O(n^3 \log^2 n) \) $n$-linear queries. They also showed how
to efficiently implement this decision tree in the real-RAM model. This decision
tree is a prune and search algorithm that relies on the simplicial
decomposition of an arrangement of hyperplanes. Ezra and
Sharir~\cite{ES17}
later improved the decision tree depth to \( O(n^2 \log n) \) by using
vertical decomposition instead.
%

In a breakthrough result, Kane, Lovett, and Moran~\cite{KLM17} proved
that $k$-SUM can be solved using $O(n \log^2 n)$ $2k$-linear queries. This is
close to the information theoretic lower bound of $\Omega(n \log n)$. Their
decision tree is also a prune and search algorithm. The techniques used rely
heavily on the linearity of the sum function in the 3SUM problem. We do not see
how to apply their techniques to obtain subquadratic-depth decision trees for
the 3POL problem.

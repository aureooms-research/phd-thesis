\section{Introduction}
%\improve{In general, I think there can hardly be too many warnings and remarks on the
%notion of nonuniformity. I expect reviewers to ask questions on, and
%misunderstand, everything nonuniform. Maybe it would be worth having an whole
%paragraph on this in the introduction, pointing to where nonuniformity is
%exploited.}

The 3SUM problem is defined as follows: given $n$ distinct real numbers, decide
whether any three of them sum to zero.
%
A popular conjecture is that no $O(n^{2-\delta})$-time algorithm for 3SUM
exists, for any $\delta > 0$. This conjecture has been used to show conditional
lower bounds for problems in P, notably in computational geometry with problems
such as
GeomBase, general position~\cite{GO95}
and
Polygonal Containment~\cite{BH01},
and more recently for string problems such as
Local Alignment~\cite{AVW14}
and
Jumbled Indexing~\cite{ACLL14},
as well as
dynamic versions of graph problems~\cite{P10,AV14},
triangle enumeration and Set Disjointness~\cite{KPP16}.
%
For this reason, 3SUM is considered one of the key subjects of an
emerging theory of complexity-within-P, along with other problems such as
all-pairs shortest paths,
orthogonal vectors,
boolean matrix multiplication,
and conjectures such as
the Strong Exponential Time Hypothesis~\cite{AVY15,HKNS15,CGIMPS16}.

Because fixing two of the numbers $a$ and $b$ in a triple only allows for one
solution to the equation $a + b + x = 0$, an instance of 3SUM has at most
$n^2$ degenerate triples. An instance giving a matching lower bound is for
example the set $\{\,\frac{1-n}{2},\ldots,\frac{n-1}{2}\,\}$ (for odd $n$)
with $\frac{3}{4} n^2 + \frac 14$ degenerate triples.
%
One might be tempted to think that the number of ``solutions'' to the problem
would lower bound the complexity of algorithms for the decision version of the
problem, as it is the case for this problem, and other problems, in restricted
models of computation~\cite{E96,E99}.
%
%This is a common misconception.
This intuition is incorrect.
%
Indeed, Gr\o nlund and Pettie~\cite{GP14} proved that there exist
$\tilde{O}(n^{3/2})$-depth linear decision trees and $o(n^2)$-time real-RAM
algorithms for 3SUM\@.

We consider an algebraic generalization of the 3SUM problem: we replace the sum
function by a constant-degree polynomial in three variables $F \in
\mathbb{R}[x,y,z]$ and ask to determine whether there exists a
\emph{degenerate} triple $(a,b,c)$ of input numbers such that $F(a,b,c)=0$. We
call this new problem the \emph{3POL problem}.

Some combinatorics aspects of the 3POL problem have already been studied.
%
For the particular case $F(x,y,z) = f(x,y) - z$ where $f \in \mathbb{R}[x,y]$
is a constant-degree bivariate polynomial, Elekes and Rónyai~\cite{ER00} show
that the number of degenerate triples is $o(n^2)$ unless $f$ is
\emph{special}. Special for $f$ means that $f$ has one of the two special forms
\begin{displaymath}
f(u,v)=h(\varphi(u)+\psi(v))
\qquad
\text{or}
\qquad
f(u,v)=h(\varphi(u)\cdot\psi(v)),
\end{displaymath}
where $h,\varphi,\psi$ are univariate polynomials of constant degree.
It must be noted that the 3SUM problem falls in the special category since, in
that case, \( f \) is the sum function.
%
Elekes and Szabó~\cite{ES12} later generalized this result to a broader range
of functions $F$ using a wider definition of specialness.
%
Raz, Sharir and Solymosi~\cite{RSS14} and Raz, Sharir and de Zeeuw~\cite{RSZ15}
improved both bounds to $O(n^{11/6})$.
%
They translated the problem into an incidence problem between points and
constant-degree algebraic curves. Then, they showed that unless $f$ (or $F$) is
special, these curves have low multiplicities. Finally, they applied a theorem
due to Pach and Sharir~\cite{PS98} bounding the number of incidences between
the points and the curves. Some of these ideas appear in our approach.

%\subsection{Our results}
We focus on the computational complexity of 3POL\@. Since 3POL contains 3SUM,
an interesting question is whether a generalization of Gr\o nlund and Pettie's
3SUM algorithm exists for 3POL\@. If this is true, then we might wonder whether
we can ``beat'' the $O(n^{11/6}) = O(n^{1.833\ldots})$ combinatorial bound of Raz,
Sharir and de Zeeuw~\cite{RSZ15} with nonuniform algorithms. We give a positive
answer to both questions: we design
a uniform
$O(n^2 {(\log \log n)}^{3/2} / {(\log n)}^{1/2})$-time
real-RAM algorithm
and
a nonuniform
$O(n^{12/7+\varepsilon}) = O(n^{1.7143})$-depth
bounded-degree algebraic decision tree
for 3POL\@.%
\footnote{Throughout this document, $\varepsilon$ denotes a positive real
number that can be made as small as desired.}
To prove our uniform result, we present a fast algorithm for the Polynomial
Dominance Reporting (PDR) problem, a far reaching generalization of the
Dominance Reporting problem. As the algorithm for Dominance Reporting and its
analysis by Chan~\cite{Cha08} is used in fast algorithms for all-pairs shortest
paths, (min,+)-convolutions, and 3SUM, we expect this new algorithm will have
more applications.

Our results can be applied to many algebraic degeneracy testing problems, such
as the General Position Testing~(GPT) problem: ``Given $n$ points in the plane, do
three of them lie on a line?'' It is well known that GPT is 3SUM-hard,
and it is open whether GPT admits a subquadratic algorithm. Raz, Sharir
and de Zeeuw results on the 3POL problem~\cite{RSZ15} can be applied to obtain
a combinatorial bound of $O(n^{11/6})$ on the
number of collinear triples when the input points are known to be lying on
a constant number of polynomial curves, provided those curves are neither
lines nor cubic curves. A corollary of our first result is that
GPT where the input points are constrained to lie on
$o({(\log n)}^{1/6}/{(\log \log n)}^{1/2})$
constant-degree polynomial curves (including lines and cubic curves)
admits a subquadratic real-RAM algorithm and
a strongly subquadratic bounded-degree algebraic decision tree.
Interestingly, both reductions from 3SUM to GPT on 3 lines (map $a$ to $(a,0)$,
$b$ to $(b,2)$, and $c$ to $(\frac c2, 1)$) and from 3SUM to GPT on a
cubic curve (map $a$ to $(a^3,a)$, $b$ to $(b^3,b)$, and $c$ to $(c^3,c)$)
construct such special instances of GPT\@.
This constitutes the first step towards closing the major open question of
whether GPT can be solved in subquadratic time.
%
To further convince the reader of the expressive power of the 3POL problem,
we also give reductions from the problem of counting triples of points spanning
unit circles, from the problem of counting triples of points spanning unit area
triangles, and from the problem of counting collinear triples in any dimension.

The algorithms we present manipulate polynomial expressions.
%
In computational geometry, it is customary to assume the real-RAM model can be
extended to allow the computation of roots of constant degree polynomials.
We distance ourselves from this practice and take particular care
of using the real-RAM model and the bounded-degree algebraic decision tree
model with only the four arithmetic operators.

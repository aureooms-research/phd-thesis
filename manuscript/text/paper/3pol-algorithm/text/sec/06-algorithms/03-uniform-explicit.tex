\section{Uniform algorithm for explicit 3POL}%
\label{sec:algo:explicit:uniform}

We now build on the first algorithm and prove the following:
\begin{theorem}\label{thm:explicit:uniform}
	Explicit 3POL can be solved in
	$O(n^2 {(\log \log n)}^{3/2} / {(\log n)}^{1/2})$ time.
\end{theorem}
We generalize again Gr\o nlund and Pettie~\cite{GP14}. The
algorithm we present is derived from the first subquadratic algorithm in their
paper.

\paragraph{Idea}
We want the implementation of step~\textbf{2} in Algorithm~\ref{algo:ne} to be
uniform, because then, the whole algorithm is.
We use the same partitioning scheme as before except we choose $g$ to be much
smaller.
This allows to
store all permutations on $g^2$ items in a lookup table, where
$g$ is chosen small enough to make the size of the lookup table $\Theta(n^\varepsilon)$.
The preprocessing part of the previous algorithm is replaced by $ g^2! $ calls to
an algorithm that determines for which cells a given permutation gives the
correct sorted
order. This preprocessing step stores a constant-size%
\footnote{In the real-RAM and word-RAM models.}
pointer from each cell to
the corresponding permutation in the lookup table.
Search can now be done efficiently: when searching a value $c$ in $f(A_i
\times B_j)$, retrieve the corresponding permutation on $g^2$ items from the
lookup table,
then perform binary search on the sorted order defined by that permutation.
The sketch of the algorithm is exactly Algorithm~\ref{algo:ne}. The only
differences with respect to \S\ref{sec:algo:explicit:nonuniform} are the choice
of $g$ and the implementation of step~\textbf{2}.

\paragraph{$A \times B$ grid partitioning}
We use the same partitioning scheme as before, hence
Lemma~\ref{lem:intersections} and Lemma~\ref{lem:preprocessing} hold.
We just need to find a replacement for Lemma~\ref{lem:batch}.
%to identify the
%cells that correspond to a specific sorted order.

\paragraph{Preprocessing}
For their simple subquadratic 3SUM algorithm, Gr\o nlund and Pettie~\cite{GP14}
explain that for a permutation to give the correct sorted order for a cell, that
permutation defines a \emph{certificate} --- a set of inequalities --- that
the cell must verify. They cleverly note --- using Fredman's Trick~\cite{F76}
as in Chan~\cite{Cha08} and Bremner et al.~\cite{BCDEHILPT14}
--- that the verification of a single certificate by all cells amounts to
solving a red/blue point dominance reporting problem.
We generalize their method.
For each permutation $\pi\colon\,[g^2]\to{[g]}^2$, where $\pi =
(\pi_r,\pi_c)$ is decomposed into row and column functions
$\pi_r,\pi_c\colon\,[g^2]\to[g]$, we enumerate all cells $A_i^* \times B_j^*$
%,with
%\begin{displaymath}
%A_i\times B_j=\{\,
    %(a_{i,1},b_{j,1}),
    %(a_{i,1},b_{j,2}),
    %\ldots,
    %(a_{i,2},b_{j,1}),
    %(a_{i,2},b_{j,2}),
    %\ldots,
    %(a_{i,g},b_{j,g-1}),
    %(a_{i,g},b_{j,g})
%\,\},
%\end{displaymath}
for which the following \emph{certificate} holds:
\begin{displaymath}
	f\Big(a_{i,\pi_r(1)},b_{j,\pi_c(1)}\Big)
	\le
	f\Big(a_{i,\pi_r(2)},b_{j,\pi_c(2)}\Big)
	\le
	\cdots
	\le
	f\Big(a_{i,\pi_r(g^2)},b_{j,\pi_c(g^2)}\Big).
\end{displaymath}

\paragraph{Remark}
Since some entries may be equal, to make sure each cell corresponds to exactly
one certificate, we replace $\le$ symbols by choices of $g^2-1$ symbols in
$\{\,=,<\,\}$. Each permutation $\pi$ gets a certificate for each of
those choices.
This adds a $2^{g^2-1}$ factor to the number of certificates
to test, which will eventually be negligible.
Note that some of those $2^{g^2-1}$ certificates are equivalent. We
need to skip some of them, as otherwise we might output some cells more than
once, and then there will be no guarantee with respect to the output size. For
example, the certificate $f(a_{i,9},b_{j,5}) = f(a_{i,6},b_{j,7}) < \cdots <
f(a_{i,4},b_{j,4})$ is equivalent to the certificate $f(a_{i,6},b_{j,7}) =
f(a_{i,9},b_{j,5}) < \cdots < f(a_{i,4},b_{j,4})$. Among equivalent
certificates, we only consider the certificate whose permutation $\pi$ precedes
the others lexicographically. In the previous example,
$((6,7),(9,5),\ldots,(4,4)) \prec ((9,5),(6,7),\ldots,(4,4))$ hence we would
only process the second certificate. For the sake of simplicity,
we will write inequality when we mean either strict inequality or equation, and
``$\le$'' when we mean either ``$<$'' or ``$=$''.

\paragraph{Fredman's Trick}
This is where Fredman's Trick comes into play.
By Lemma~\ref{lem:fredman}, each inequality
$f(a_{i,\pi_r(t)},b_{j,\pi_c(t)}) \le f(a_{i,\pi_r(t+1)},b_{j,\pi_c(t+1)})$
of a certificate can be checked
by computing the relative position of $(a_{i,\pi_r(t)},a_{i,\pi_r(t+1)})$ with
respect to $\gamma_{b_{j,\pi_c(t)},b_{j,\pi_c(t+1)}}$.
For a given certificate, for each $A_i$ and each $B_j$, define
%
\begin{align*}
p_i &= \mleft(
    \Big(a_{i,\pi_r(1)},a_{i,\pi_r(2)}\Big),
    %\Big(a_{i,\pi_r(2)},a_{i,\pi_r(3)}\Big),
    \ldots,
    \Big(a_{i,\pi_r(g^2-1)},a_{i,\pi_r(g^2)}\Big)
\mright),\\
q_j &= \mleft(
    f\Big(x,b_{j,\pi_c(1)}\Big) \le f\Big(y,b_{j,\pi_c(2)}\Big),
    %f(x,b_{j,\pi_c(2)}) \le f(y,b_{j,\pi_c(3)}),
    \ldots,
    f\Big(x,b_{j,\pi_c(g^2-1)}\Big) \le f\Big(y,b_{j,\pi_c(g^2)}\Big)
\mright).
\end{align*}
%
A certificate is verified by a cell $A_i \times B_j$ if and only if, for all
$t \in [g^2-1]$, the point $p_{i,t}$ verifies the inequality $q_{j,t}$.
Enumerating all cells $A_i\times B_j$ for which the
certificate holds therefore amounts to solving the following problem:
%
\begin{problem}[Polynomial Dominance Reporting (PDR)]
Given $N$ $k$-tuples $p_i$ of points in $\mathbb{R}^2$ and $N$ $k$-tuples $q_j$
of bivariate polynomial inequalities of degree at most $\Delta$,
output all pairs $(p_i,q_j)$ where, for all $t \in [k]$,
the point $p_{i,t}$ verifies the inequality $q_{j,t}$.
\end{problem}
%
In the next section, we explain how to solve PDR efficiently
and prove the following:
\begin{lemma}\label{lem:dominance}
    We can output
    all $\ell$ such pairs in time
    $2^{O(k)} N^{2-\frac{4}{\Delta^2+3\Delta+2}+\varepsilon} + O(\ell)$.
\end{lemma}

We can now give a uniform implementation of step~\textbf{2} in
Algorithm~\ref{algo:ne}:
\begin{algorithm}[Sorting the $f(A_i \times B_j)$ with a uniform algorithm]\label{algo:sfaixbj-uniform}
\item[input] $A = \{\,a_1<a_2<\cdots<a_n\,\},B = \{\,b_1<b_2<\cdots<b_n\,\}
    \subset \mathbb{R}$
\item[output] The sets $f(A_i \times B_j)$, sorted.
\item[2.1.] Initialize a table that will contain all $ g^2! $ permutations on $g^2$ elements.
\item[2.2.] For each permutation $\pi\colon\,[g^2]\to{[g]}^2$,
\item[2.2.1.] Append permutation $\pi$ to the table,
\item[2.2.2.] For each choice of $g^2-1$ symbols in $\{\,=,<\,\}$,
\item[2.2.2.1.] If there is any ``$=$'' symbol that corresponds to a
    lexicographically decreasing
    pair of tuples of indices in $\pi$, skip this choice of symbols.
\item[2.2.2.2.] Solve the PDR instance associated
    with $A,B,\pi$ and the choice of symbols.
%\footnote{See Algorithm~\ref{algo:pdr} for the implementation.}
\item[2.2.2.3.] For each output pair $(i,j)$, store a pointer
    to the last entry in the table.
    %from $A_i^* \times B_j^*$ to the last entry in the lookup table.
\end{algorithm}


\paragraph{Analysis} Plugging in $k=g^2-1$, $N=\frac ng$, iterating over all
permutations ($\sum_{\pi} \ell = {(n/g)}^2$), and adding the binary search step we
get that
%, in the real-RAM model,
explicit 3POL can be solved in time
\begin{displaymath}
    (g^2!)2^{g^2}2^{O(g^2)}
    {(n/g)}^{2-\frac{4}{{\deg(f)}^2+3\deg(f)+2}+\varepsilon} + O({(n/g)}^2) +
    O(n^2 \log g / g).
\end{displaymath}
The first two terms correspond to the complexity of step~\textbf{2} in
Algorithm~\ref{algo:ne}, and the last term corresponds to the complexity of
step~\textbf{3} in Algorithm~\ref{algo:ne}.
To get subquadratic time we can set
$g = c_{\deg(f)}\sqrt{\log n/\log \log n}$,
because then
for some appropriate choice of the constant factor $c_{\deg(f)}$,
$(g^2)!2^{g^2}2^{O(g^2)} = n^{\delta}$ where $\delta<
4/({\deg(f)}^2+3\deg(f)+2) - \varepsilon$, making the first term negligible.
The complexity of the algorithm is dominated by $O(n^2 \log g / g) = O(n^2
{(\log \log n)}^{3/2} / {(\log n)}^{1/2} )$.
This proves Theorem~\ref{thm:explicit:uniform}.

%\remark{}
%It is plausible that using the more sophisticated machinery of Gr\o nlund and
%Pettie~\cite{GP14}, Freund~\cite{F15}, and Gold and Sharir~\cite{GS15} it would
%be possible to decrease the complexity to $O(n^2/\log n)$.

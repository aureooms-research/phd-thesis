%\subsection{Polynomial Batch Range Searching}
\section{Polynomial Batch Range Searching}%
\label{sec:algo:point-curves-location}

In this section we present a uniform algorithm that computes the relative
position of \(M\) points with respect to \(N\) $\gamma_{b,b'}$ curves. We call
such a problem an \((M,N)\)-problem. When \(M=N\) the complexity of the
algorithm is \(O(N^{4/3+\varepsilon})\).
The algorithm gives the output in ``concise form'':
it outputs a set of $(\Pi_\alpha, \Gamma_\beta, \sigma)$ triples where
$\Pi_\alpha$ is a subset of input points, $\Gamma_\beta$ is a subset of input curves, and $\sigma \in
\{\,-,0,+\,\}$ indicates the relative position of all points in $\Pi_\alpha$ with
respect to all curves in $\Gamma_\beta$.
Note that if one is only interested in incident point-curve pairs, the
algorithm can explicitly report all of them in
$O(N^{4/3+\varepsilon})$ time, because there are at most $O(N^{4/3})$
such pairs and because they can easily be filtered from the output.

\paragraph{Tools} The proof of Lemma~\ref{lem:batch} involves stantard
computational geometry tools: vertical decomposition of an
arrangement of polynomial curves (see Figure~\ref{fig:vd}),
$\varepsilon$-nets, cuttings and derandomization.
For the construction of the vertical
decomposition of an arrangment of polynomial curves, we refer the reader to
Pach and Sharir~\cite{Alcala}, Chazelle et al.~\cite{CEGS91}, and Edelsbrunner
et al.~\cite{EGPPSS92}. For cuttings, $\varepsilon$-nets and derandomization, we
refer the reader to Matou\v{s}ek~\cite{M95,M96}, Chazelle and
Matou\v{s}ek~\cite{CM96} and Brönnimann et al.~\cite{BCM99}.
\begin{figure*}
    \centering
    \includegraphics{figures/decomposition}
    \caption{Some curves in $\mathbb{R}^2$.}
    \label{fig:some-curves-in-R2}
\end{figure*}
\begin{figure*}
    \centering
    \includegraphics{figures/recursion}
    \caption{Vertical decomposition of the curves in Figure~\ref{fig:some-curves-in-R2}.}%
    \label{fig:vd}
\end{figure*}

\begin{proof}[Proof of Lemma~\ref{lem:batch}]
Fix some constant $r \geq 2$.
If $M < r^2$ or $N < r$, solve by brute-force in $O(M+N)$ time. Otherwise,
consider the range space defined by $\gamma_{b,b'}$ curves and $y$-axis aligned
trapezoidal patches whose top and bottom sides are pieces of $\gamma_{b,b'}$
curves. This range space has constant VC-dimension.
Compute an $\frac 1r$-net of size $O(r \log r)$ for the input curves with respect to
this range space. Compute the vertical decomposition \(\Xi\) of the arrangement
of this $\frac 1r$-net. This decomposition is a $\frac 1r$-cutting: it partitions
$\mathbb{R}^2$ into $O(r^2 \log^2 r)$ cells of constant complexity each of
which intersects at most \(\frac{N}{r}\) input curves.
Note that some of those cells are degenerate trapezoidal
patches. The decomposition contains vertices, line segments, and curve
segments as cells, each of which could contain input points and be intersected
or contained by an input curve.
Denote by \(\Pi_C\) the set of points contained in the cell \(C \in \Xi\).
Partition each \(\Pi_C\) into \(\left\lceil \frac{\lvert \Pi_C \rvert}{M
r^{-2}} \right\rceil\) disjoint subsets of size at most \(\frac{M}{r^2}\). All
of this can be done in \(O(M+N)\) time.
The last step consists of solving \(O(r^2 \log^2 r)\)
\((\frac{M}{r^2},\frac{N}{r})\)-problems, that is, solving the problem
recursively for the points and curves intersecting each cell.
Each recursive call is done by swapping the roles of the points and curves
using a form of duality to be described below.
%\improve{Stefan: Why is a randomized algorithm sufficient? To show the
%existence of a (deterministic) computation tree?}
%\improve{Stefan: No, that is not enough!~\cite{MP15} uses this canonical
%decomposition to apply [de Berg \& Shwarzkopf'95] to show the existence of cuttings.}
The whole algorithm can be formally described as follows,
\begin{algorithm}[Polynomial Batch Range Searching]\label{algo:pbrs}
\item[input] A set $\Pi$ of $M$ points $(a,a')$, A set $\Gamma$ of $N$ curves $\gamma_{b,b'}$.
\item[output] A set of triples $(\Pi_\alpha,\Gamma_\beta,\sigma)$ covering $\Pi \times \Gamma$
    such that, for any triple\\
    $(\Pi_\alpha,\Gamma_\beta,\sigma)$,
    for all points $(a,a')$ in $\Pi_\alpha$ and all curves $\gamma$ in
    $\Gamma_\beta$,
    $(a,a') \in \gamma^\sigma$.
\item[0.] If $M < r^2$ or $N < r$, solve the problem by brute force in $O(M+N)$ time.
\item[1.] Compute an $\frac 1r$-net of size $O(r \log r)$ for the input curves.
\item[2.] Compute the vertical decomposition \(\Xi\) of the arrangement of this
    $\frac 1r$-net.
\item[3.] Denote by \(\Pi_C\) the set of points contained in the cell \(C \in \Xi\).
    Partition each \(\Pi_C\) into \(\left\lceil \frac{\lvert \Pi_C \rvert}{M
    r^{-2}} \right\rceil\) disjoint subsets
    $\Pi_{C,i}$ of size at most \(\frac{M}{r^2}\).
\item[4.] For each cell $C$ of the vertical decomposition,
\item[4.1.] For each subset $\Pi_{C,i}$ of points contained in that cell,
\item[4.1.1.] Solve an $(\frac{N}{r},\frac{M}{r^2})$-problem on the curves
    intersecting that cell and the points in $\Pi_{C,i}$, swapping the roles of
    lines and curves via duality.
\item[4.2.] For each curve $\gamma$ not intersecting $C$,
\item[4.2.1.] Compute the location $\sigma_{C,\gamma}$ of any point in $C$ with
    respect to $\gamma$.
\item[4.3.] Output $(\{\,\gamma \st \sigma_{C,\gamma} = -\,\},\Pi_C, -)$.
\item[4.4.] Output $(\{\,\gamma \st \sigma_{C,\gamma} = +\,\},\Pi_C, +)$.
\item[4.5.] Output $(\{\,\gamma \st \sigma_{C,\gamma} = 0\,\},\Pi_C, 0)$.
\end{algorithm}

\paragraph{Correctness} We want to locate each point with respect to each
curve. When considering a curve-cell pair, there are two cases: either the
curve intersects the cell, or it does not. For the first case we locate
each point in the cell with respect to the curve in one of the recursive steps.
For the second case, the relative position of all points in the cell with
respect to the curve is the same, it suffices thus to locate one of those points
with respect to the curve to get the location of all the points in
$O(1)$ time.
Each recursive call divides $M$ and
$N$ by some constant strictly greater than one, hence, the base case is reached
in each of the paths of the recursion tree and the algorithm always terminates.

\paragraph{Analysis} For $c_1$ some constant and bounding $c_1 r^{2} \log^2 r$
above by $c_2 r^{2+\varepsilon}$ for some large enough constant $c_2$, the
complexity $T(M,N)$ of an $(M,N)$-problem is thus
\begin{align*}
    T(M,N) &\le c_2 r^{2+\varepsilon}\,T\mleft(\frac{M}{r^{2}},\frac{N}{r}\mright) + O(M+N).
\end{align*}
The complexity $T(N,M)$ of a \((N,M)\)-problem is the same as the
complexity $T(M,N)$ of an \((M,N)\)-problem by the following point-curve
duality result whose proof is straightforward
\begin{lemma}\label{lem:dual}
Define
\begin{displaymath}
    \hat{\gamma}_{a,a'} = \{\, (x,y) \st f(a,x)=f(a',y)\,\},
\end{displaymath}
then, locating $(a,a')$ with respect to $\gamma_{b,b'}$ amounts to
locating $(b,b')$ with respect to $\hat{\gamma}_{a,a'}$.
\end{lemma}

By doing alternately one step in the primal with the points $(a,a')$ and the
curves $\gamma_{b,b'}$, then a second step with the dual points $(b,b')$ and the
dual curves $\hat{\gamma}_{a,a'}$, we get the following recurrence
\begin{align*}
    T(M,N) &\le c_2^2 r^{4+\varepsilon}\,T\mleft(\frac{M}{r^{3}},\frac{N}{r^{3}}\mright) +
	c_2 r^{2+\varepsilon}\,O\mleft(\frac{M}{r^2} + \frac{N}{r}\mright) +
	O(M+N)\\
	&\le c_2^2 r^{4+\varepsilon}\,T\mleft(\frac{M}{r^{3}},\frac{N}{r^{3}}\mright) +
	O(M+N)
\end{align*}
Hence, for some large enough constant $c_3$,
\begin{align*}
T(N,N) = T(N) &\le c_3 r^{4+\varepsilon}\,T\mleft(\frac{N}{r^{3}}\mright) + O(N)\\
	      &\le O\mleft(N^{\log_{r^{3}} (c_3  r^{4+\varepsilon})}\mright)\\
	&\le O(N^{\frac 43 + \varepsilon}).
\end{align*}
\end{proof}


%\remark{
	%Here we should (maybe?) cite the recent work of Agarwal, Matou\v{s}ek and
	%Sharir~\cite{AMS13} who show that there exists efficient data structures to
	%solve our problem for arbitrary $d$. However, their solution would need
	%some tweaks to be:
	%(1) as efficient as ours
	%(2) really working

	%Explanation of point (1): their data structure uses linear space. There
	%exist data structures with space-time tradeoff for the point-hyperplane
	%range searching problem that run in time $O(m + \frac{n^2}{m^{\frac 1d}} +
	%\cdots)$, where $m$ is the space used. If you do not care about space and
	%want to optimize time then you get a $O(n^{\frac{2d}{d+1}})$-time
	%algorithm.
	%Back in October, Stefan suggested that by using some old tricks of
	%Matou\v{s}ek we might be able to get the space-time tradeoff for this
	%data-structure.

	%Explanation of point (2): their data structure requires a nondegeneracy
	%assumption to work (i.e. no too many points lying on any surface of some
	%bounded degree). If you want to drop this assumption then you get a
	%\emph{boundary-fuzzy} data-structure: it might behave incorrectly for point
	%that are exactly on the surfaces.
	%They see this feature as a minor bug since they consider queries of the
	%type ``How many points are in this cell of the arrangment of surfaces?''.
	%However, in our case, the fact that a point is exactly on a surface instead
	%of $\varepsilon$ away is actually the only thing that matters.
%}

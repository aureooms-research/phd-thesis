\section{Emails}

\subsection{Stefan Langerman Oct 1}

So far we proved that we can find all solutions of
\(A\times B\times C \st c = f(a,b)\) for any $f$ of constant degree in
\(O^*(n^\frac{12}{7})\).
It seems that (as Noam predicted), using the functions
\(\gamma_{b,b'} = \{(x,y)\st F(x,b,z)=F(y,b',z)=0\) for some \(z\}\)
we can extend the result to the implicit case $F(a,b,c)=0$.
That curve decomposes the (a,a') plane in regions where the roots of
\(F(a,b,z)=0\) and \(F(a',b',z)=0\) have the same relative order.

Noam: Is it easy to show that, for \(b\) fixed, the number of values of $a$
for which \(F(a,b,z)=0\) has two  or more equal roots, is constant? I
think that might help.

Here are a few directions:
\begin{itemize}
	\item Looking at the implicit case, maybe we could use a better
		partitioning scheme in the divide and conquer approach, for splitting
		$a$, $b$, and $c$.
	\item Can we use the algorithmic approach above to prove a \(O(n^\frac{12}{7})\) bound
		on the number of solutions directly? If $f$ is not special, then the
		curves \(\gamma_{b,b'}\) will should have low multiplicity (Noam: is this
		true?), and then presumably not too many $(a,b)$ can share a same $c$
		within a same $ab$ cell.
	\item Why are the exponent of the algorithm ($\frac{12}{7}$) and the combinatorial
		bound ($\frac{11}{6}$) different?
	\item How about functions of 4 parameters? Can we always solve this in
		\(O(n^2)\)? How many solutions can there be? Are there special functions?
	\item How about functions of $k$ parameters? Jean says if we can reach 6
		parameters, we get a subquadratic algorithm for general position.
	\item Can we use something similar to the second part of the Pettie paper
		to get truly subquadratic algorithms for all that?
\end{itemize}

\subsection{Noam Solomon Oct 1 (1)}
About your question:
For fixed \(b\), \(F(a,b,z)\) is a polynomial in two variables. For the $a$ value,
\(F(a,b,z)\) has two or more equal roots if and only if \(F(a,b,z)\) and
\(F_z(a,b,z)\)
(the derivative with respect to \(z\)) has a common root (in \(z\)). This implies
that their \emph{resultant} with respect to \(z\) vanishes at \(a\), i.e., if
\(\res(F(a,b,z),F_z(a,b,z);z)=0\) (at $a$).

When $b$ is fixed, this resultant \(\res(F(a,b,z), F_z(a,b,z);z)\) is a polynomial
in $a$ of constant degree (if \(F\) has constant degree) and the conclusion is
that the number of \(a\)'s such that this vanishes is constant.

The short answer to your question, assuming I got it correctly is YES the
number of such $a$'s is constant as long as $F$ has constant degree

\subsection{Noam Solomon Oct 1 (2)}
About using this divide and
conquer technique.
The idea is intriguing but remember that you use Agarwal-Matou\v{s}ek-Sharir
strategy, which does not upper bounds the number of solutions, so you lose
some ``incidences'' in this case, but it would be great if we can recover
the $\frac{11}{6}$ using this algorithmic approach (as their bound is not
constructive).
I think the fact that the number of incidences between these curves and the
$a,a'$ could be bigger than the time complexity of computing the incidences
you get the different exponents.

You asked a second question, and the answer is again yes, if the function
is not special then there is low multiplicity.

\subsection{Stefan Langerman Oct 2}
Here is the daily progress report:

\begin{itemize}
\item Functions of 4 parameters:
$\{(a,b,c,d) \in A\times B\times C\times D \st F(a,b,c,d)=0\}$
Define curves $\gamma_{c,d}(x,y)$ as
$\{(x,y) \st F(x,y,c,d)=0\}$. Now we just need to find all incidences
between all the $n^2$ gamma curves and all $n^2$ points $(a,b)$.

This can be solved in $O(n^{\frac{8}{3}+\varepsilon})$ using the standard Matou\v{s}ek tools.
Note also that if $F$ is just $a+b+c+d$, then there could be $n^3$ solutions,
however if $F$ is not special for some definition of special, then
Noam could prove to us that the gamma functions have constant
multiplicity, and then Szemeredi-Trotter gives us a $n^{\frac{8}{3}}$ combinatorial
bound as well. Is this tight? I suspect yes! (we should prove all this)

\item Same can be done for functions of k parameters, k even, and we get
a bound of
(NDLR; $O({(n^\frac{k}{2})}^\frac{2d}{d+1})$ with $d=\frac{k}{2}$)
$O(n^{\frac{k^2}{k + 2}}) = O(n^{k - \frac{2k}{k+2}})$. Not so glorious.

\item Actually much better (but non-uniform) can probably be done by using
the approach of Meiser for k-SUM: Using epsilon-nets, it should be
possible to do point location of a point in $\mathbb{R}^n$ in an arrangement of
$\binom{n}{k}$ algebraic surfaces $F(x_{i_1},\ldots,x_{i_k})=0$ for all choices of
indices $i_1 \ldots i_k$, in polynomial time in $n$. To do this, the idea would be
to do a trapezoidal decomposition rather than the simplicial
decomposition of Meiser and the rest should go through. This should
give us a polynomial time non-uniform algorithm to test general
position in any fixed dimension.
\end{itemize}

\subsection{Noam Solomon Oct 3}

\begin{displayquote}
This can be solved in $O(n^{\frac{8}{3}+\varepsilon})$ using the standard Matou\v{s}ek tools.
Note also that if $F$ is just $a+b+c+d$, then there could be $n^3$ solutions,
however if $F$ is not special for some definition of special, then
Noam could prove to us that the gamma functions have constant
multiplicity, and then Szemeredi-Trotter gives us a $n^{\frac{8}{3}}$ combinatorial
bound as well. Is this tight? I suspect yes! (we should prove all this)
\end{displayquote}

Stefan -- the question you ask in this email is probably harder. As far as I
know, generalizing Elekes-Ronyai ($z=f(x,y)$) or Elekes-Szabo ($F(x,y,z)=0$) to
four variables is an open problem. It is unclear to me (without thinking
about it more seriously) where the difficulty lies. But what you write
seems plausible (I will think about it).
Another question: in Pettie's original paper, he uses the
$O(n^{\frac{3}{2}}\sqrt{\log n})$ decision tree complexity to derive a subquadratic
algorithm. The more general approach (for the $z=f(x,y)$ as you suggested)
would not yield such a sub-quadratic algorithm, right?

\subsection{Stefan Langerman Oct 4 (1)}
The main difficulty seems to reside in the part of
Elekes-Szabo/Ronyai that I don't understand: Can we impose some
condition on $F(a,b,c,d)$ such that $F(a,b,x,y)=0$ for all $a \in A, b \in B$,
have low multiplicity?
This would be enough to get the combinatorial bound.
The algorithmic bound works and does not depend on this.

For your second question, this was the last question in my first email:
- Can we use something similar to the second part of the Pettie paper
to get truly subquadratic algorithms for all that?
The idea would be to take a bloc size K much smaller than $n^\alpha$,
probably $\sqrt{\frac{\log n}{\log \log n}}$ or something like that. Now the number
of possible orderings in each cell \(K\times K\) is $O(K \log K)$, therefore we
could build a large lookup table of size $O(K^K)$ that contains all
possible orderings. Now we just need the preprocessing algorithm to
identify the ordering entry for each of the cells.
The trick is to transform this into a dominating pairs problem in
higher dimension, and use Chan's algorithm. It shouldn't be too hard
to get at least some subquadratic algorithm.

\subsection{Noam Solomon Oct 4 (1)}
The intuition is that there is something similar to Elekes-Szabo with four
variables, i.e., that these curves should have low multiplicity unless $F$ is
special. I will try to sort this out (or at least understand why and if
this doesn't work). My concern is that this problem is known (Elekes-Szabo
in four variables) and I don't know of any work on this.
Your suggestion below is interesting!
I wonder if this can help in Jean's original question: finding if a set of
points in the plane is in general position (or find all collinear triples).
You can define a polynomial in the six variables (the corresponding
determinant) and then apply this idea (instead of four variables do it with
6).

\subsection{Stefan Langerman Oct 4 (2)}
The problem with this is that for $6$ parameters, we would need to solve
a Hopcroft problem for $n^3$ points and surfaces in $\mathbb{R}^3$, which would
take much more than quadratic time.

\subsection{Noam Solomon Oct 4 (2)}
Sure but the question is how does this compare to the best known algorithm
for finding if a planar point set is in general position (this is not a
rhetorical question, I actually don't know what is the best known algorithm
for this problem).

\subsection{Noam Solomon Oct 4 (3)}
There is a more restricted scenario, when one wants to count all collinear
triples where the points lie on an algebraic curve (actually three curves),
Elekes-Szabo gives you some non-trivial bound, provided the curve is not a
line or a cubic curve.
You can apply your approach to find an algorithm that finds these collinear
triples.
This is less a natural question, but if you wanna have a look -- Theorem 6.1
in the link below
\url{http://arxiv.org/pdf/1504.05012v1.pdf}.

\subsection{Stefan Langerman Oct 4 (3)}
For the general problem of $n$ points in the plane, you can find all
collinear triples in $O(n^2)$ by building the dual arrangement.

The restricted problem where the 3 points are on 3 curves can indeed
be solved in subquadratic time using our result!

\subsection{Noam Solomon Oct 5 (1)}
Here are some first impressions about the multiplicity business for the
four-dimensional case.
We have $F(x_1,x_2,x_3,x_4)=0$ for a polynomial $F$ of constant degree, and we
are looking for solutions $x_i \in A_i$ for $i=1,2,3,4$ and $A_i$ are sets of
real points with $|A_i|=n$.

For $a\in A_1, b \in A_2$ we define
$\gamma_{a,b} = \{(x_3,x_4) | F(a,b,x_3,x_4)=0\}$.
this is a planar curve
we also have the dual curves $\gamma^*_{c,d}$ which are defined similarly (for
the third and fourth coordinates.

Notice that if two curves $\gamma_{a,b}$ and $\gamma_{a',b'}$ do not have a
common component then they intersect in at most $\deg(F)^2=$ CONSTANT points
by B\'ezout.
For the Szemeredi-Trotter type theorem we need that there are very few
curves which are common to many $\gamma_{a,b}$'s.
Assume (to the contrary) that there is some (perhaps many) irreducible
curve $\gamma$ which is a component of
$\gamma_{a_j,b_j}$ for $j=1,\ldots, k$ and $k$ is some large number (not a
constant).
What does this mean? it means that for any $x_3,x_4$ on $\gamma$
$F(a_j,b_j, x_3,x_4)=0$.

In fact, we can also do the same trick for the dual curves, and deduce that
there is some $\gamma^*$ which is common to many $\gamma^*_{c,d}$'s
(otherwise we're in good shape) and I think that with some work we can
(hope to) prove that this implies that
$F(x_1,x_2,x_3,x_4)=0$ when $(x_1,x_2)\in \gamma^*$ and $(x_3,x_4)\in \gamma$
or in other words, the product of $\gamma^*$ and $\gamma$ is contained in $Z(F)$.
Since $Z(F)$ is 3-dimensional and $\gamma^*\times \gamma$ is 2-dimensional this
is not so surprising. (THIS DOES NOT HAPPEN for the three variables analog,
there it is surprising that the entire two-dimensional thing is the product
of two curves or something like that).

My partial conclusion is that the way we define the curves in the 4-dim
case might lead to curves with high multiplicity. This is not a proof, just
first impressions.

There is another point why I am quite doubtful that we can get
COMBINATORIAL bounds on the number of solutions using these curves (that is
to say, these are not the natural generalizations of the Elekes-Szabo
curves in 3-dim. I can elaborate further on this.

\subsection{Noam Solomon Oct 5 (2)}
Just to elaborate about the last point.
in the original Elekes-Ronyai, when you count the number of triples
$z=f(x,y)$ where $x\in A, y\in B, z\in C$
one observation is that if there are $\ll n^2$ solutions, then there are some
$z$'s which appear many times, i.e., there are many $a,b$'s such that $f(a,b)=z$.

This is why it makes sense to study the size of $\{(a,b,a',b'): f(a,b)=f(a',b')\}$
Then one define the curves $\gamma_{a,a'} = \{(b,b'): f(a,b)=f(a',b')\}$
Assuming we are now generalizing this to four dimensions
$w=f(x,y,z)$
then we will examine 6-tuples $\{(a,b,c,a',b',c'): f(a,b,c)=f(a',b',c')\}$
it is already unclear how to define the ``curves'' in this case
perhaps $\gamma_{a,a',b,b'} = \{(c,c'): f(a,b,c)=f(a',b',c')\}$
There are other options (more symmetric) but I don't quite see what the
right one is (if any).
The main (and important) difference between the two problems is that the
original Elekes-Szabo tells you that if there are ``few'' solutions then
the set of quadruples is going to be large and so there will be many
incidences. In what we deal with we are looking for a
characterization/division of space into parts where we don't need to do any
more ``comparisons''. In this aspect it is likely (but still open) that the
Elekes-Szabo thing would work over finite fields (not just $\mathbb{R}$ or
$\mathbb{C}$), but I
think that there is no hope to generalize your idea of using Matou\v{s}ek to
finite fields.. (you need the basic bigger than 0 or smaller than 0 that
exists only for real fields).

\subsection{Noam Solomon Oct 6}
I thought more about the semi-proof (it is still vague) I wrote about how
to use multiplicities to deduce that there are curves $\gamma$ and $\gamma^*$
which are common to many of the $\gamma_{a,b}$ and $\gamma^*_{c,d}$
(respectively), and then $Z(F)$ contains a surface which is product of the
two 1-dim curves (which are separated). This seemed to me (in a closer
look) to be more rigid then I initially thought (we can also do it for any
pairing $(a,c)$ and $(b,d)$ and $(a,d)$ and $(b,c)$).
So today I asked Micha if there is a hope to prove the 4-dimensional
Elekes-Szabo.
In fact, as micha informed me, it is part of a work in progress that he is
doing with his co-authors (aha Jean :)) and the bound they get is
$n^{\frac{8}{3}}$
unless $F$ is special. So in the 4-dimensional case the combinatorial and
computational bounds coincide.
I did not ask him for the details on how they prove it, but I imagine that
they have more or less the same curves as we do.
The question is how to use the proposed computational algorithm to deduce a
truly $n^{\frac{8}{3}}$ computational algorithm (not just these many comparisons). I
guess that the complexity of the point-curve computation we do in each cell
should be better.
Anyway, it would be great if we can prove this in the 4-dim case and also
find some interesting computational geometric questions we can solve using
it.

\subsection{Stefan Langerman Oct 7}
The bound of $n^{\frac{8}{3}}$ is for a uniform algorithm (if that's what you
mean by "truly algorithmic") and this is quite easy:
we just have $n^2$ curves and $n^2$ points and just need to find one or
all incidences. This is a standard application of the old algorithm
which does not rely on low multiplicities. I suspect our bound of
$O(n^{k - \frac{2k}{k+2}})$ for $k$ even can be very slightly improved for $k$ odd
(like we did for $k=3$) but that would give a non-uniform algorithm.

So some notion of ``specialness'' can be extended to $k=4$! I wish I
understood the definition for $k=3$.
And now we're in a race against Micha! I hope he's not looking at
computational aspects too much

If we want to have an idea of what ``specialness'' could mean:
Take $F(a,b,c,d) = d-ac-b$
Note that $F(a,b,x,y)=0$ is the line $y=ax+b$, and it is a distinct line
for every choice of $a$ and $b$, therefore the multiplicity of a function
is 1. Just applying Szemeredi-Trotter on the $n^2$ points and $n^2$ lines
gives the $n^{\frac{8}{3}}$ bound but is it achievable for such a case? Well the
bound for Szemeredi Trotter is achieved for points and lines on a
grid, in fact exactly the case when
$|A|=k, |B|=k^2, |C|=k^2, |D|=k$, so here the number of points and lines
is $k^3$, and the number of incidences is $k^4$.
But this lower bound seems to rely on the fact that the 4 sets are
unbalanced. Can we find some function $F$ (or even the same) for which a
$\Omega(n^2)$ bound can be achieved if all 4 sets are of size $n$?

\subsection{Noam Solomon Oct 7}

\begin{displayquote}
The bound of $n^{\frac{8}{3}}$ is for a uniform algorithm (if that's what you
mean by ``truly algorithmic'') and this is quite easy:
we just have $n^2$ curves and $n^2$ points and just need to find one or
all incidences. This is a standard application of the old algorithm
which does not rely on low multiplicities.
\end{displayquote}

Right, of course!

\begin{displayquote}
I suspect our bound of
$O(n^{k - \frac{2k}{k+2}})$ for $k$ even can be very slightly improved for $k$ odd
(like we did for $k=3$) but that would give a non-uniform algorithm.
\end{displayquote}

\begin{displayquote}
So some notion of ``specialness'' can be extended to $k=4$! I wish I
understood the definition for $k=3$.
And now we're in a race against Micha! I hope he's not looking at
computational aspects too much
\end{displayquote}

We are not in a race against Micha. He will not try to compete with us on
the computational thing I am convinced about it.
Their paper is almost
ready and the notion of specialness is very close to the 3-dim case. What
would you like to understand better about the 3-dim case? (it is much less
natural I agree!) The interesting thing is that I asked Micha if they have
a nice form of specialness for the case $w=f(x,y,z)$ (which is the analog of
Elekes-Ronyai) and he said that they don't.

\begin{displayquote}
If we want to have an idea of what ``specialness'' could mean:
Take $F(a,b,c,d) = d-ac-b$
Note that $F(a,b,x,y)=0$ is the line $y=ax+b$, and it is a distinct line
for every choice of a and b, therefore the multiplicity of a function
is 1. Just applying Szemeredi-Trotter on the $n^2$ points and $n^2$ lines
gives the $n^{\frac{8}{3}}$ bound but is it achievable for such a case? Well the
bound for Szemeredi Trotter is achieved for points and lines on a
grid, in fact exactly the case when
$|A|=k, |B|=k^2, |C|=k^2, |D|=k$, so here the number of points and lines
is $k^3$, and the number of incidences is $k^4$.
But this lower bound seems to rely on the fact that the $4$ sets are
unbalanced. Can we find some function $F$ (or even the same) for which a
Omega($n^2$) bound can be achieved if all $4$ sets are of size $n$?
\end{displayquote}

I am confused here by what you write: notice that if
$F(a,b,c,d)=d-a-b-c$,
which is equivalent to $d=a+b+c$ (i.e. the 4SUM) (this is a special
function) you can just take $A=B=C=\{0,...,n-1\}$ and $D=\{0,...,n-1\}$ and you
will get $n^3$ solutions. SPECIAL FUNCTIONS give you $n^3$ solutions.
Non-special give you $O(n^{\frac{8}{3}})$ you don't beat the quadratic bound,
you beat the cubic bound. or did I miss your point?

\subsection{Stefan Langerman Oct 8}
Yes, $F(a,b,c,d)=a+b+c+d$ is special, since $F(a,b,x,y)=0$ has high
multiplicity, for example if $A = B = \{1,...,n\}$. So you have $\Theta(n^3)$
solutions and that is tight.
On the other hand for the function I gave, $F(a,b,c,d) = d-ac-b$, it is
not special since every choice of $a,b$ gives a distinct line. In this
case, you have an upper bound of $O(n^{\frac{8}{3}})$, in fact
$O((|A||B||C||D|)^{\frac{2}{3}})$. My question was whether this is tight. I
noted that the $n^{\frac{4}{3}}$ lower bound for point-line incidences is exactly
this function $F$, but with $|A|=|D|=k$, $|B|=|C|=k^2$. So for this case,
the bound is tight. Can we generalize this lower bound example to show
this is tight for any sizes of the 4 sets? In particular when
$|A|=|B|=|C|=|D|$?

You said there is no known lower bound for the 3-set case $F(a,b,c)=0$?
Maybe we could try a similar operation?
We want a function $F(a,b,c)$ such that the $\gamma_{b,b'}$ would be
expressive enough to describe any line in the plane (and with low
multiplicity)?
Recall $\gamma_{b,b'} = \{(x,y)|F(x,b,z)=F(y,b',z)=0$ for some $z\}$

For example $F(a,b,c)= \frac{ab-b^2}{c-1}$, then if $F(x,b,z)=0$, then $z=xb-b^2$
and if $F(y,b',z)=0$ then $z = yb'-b'^2$ so $\gamma_{b,b'}$ is the line
$xb-yb'=b^2-b'^2$. Are those lines are all distinct for different
choices of $b$ and $b'$? Can we then find some set $A$ for which there are
many incidences?

\subsection{Noam Solomon Oct 8}
About the first question, I prefer to view it as the following
$d=ac+b$, without using incidences between points and lines.
Indeed, you can get $k^4$ solutions with the parameters that you wrote, but I
am almost convinced that you cannot extend it to the balanced case. It is
like using Elekes-Ronyai twice
$d = X + b$ and $X = ac$
these two functions are special and this is a mix of the two things. You
can similarly add some noise like
$d = f_1(X) + f_2(b)$ and $X = f_3(a)f_4(c)$
for some polynomials $f_i$
this example shows that perhaps in the four-dimensional case the theorem is
more involved.
There are the special functions which give you $n^3$ solutions,
then there are the semi-special functions which are mixes of 3-dim special
functions, and in the unbalanced case they give you more solutions then
what you get ``generally''.
Proving that in the ``balanced'' case you get LESS $\ll n^{\frac{8}{3}}$ seems to me
like a hard question. might be as difficult as doing what Micha is doing
now in his paper. I hope my reservations are clear.

About the 3-dim case, I think the conjecture is $n^{\frac{3}{2}}$ (though the proof
is only $n^{\frac{11}{6}}$).
There might even be an example obtaining this lower bound (and is balanced,
i.e., $A,B,C$ of size $n$). I will think about it later, but this does not
really have a lot to do with incidences.. (only for the upper bound proof I
think).

\subsection{Aurélien Ooms Oct 8 (1)}
In order to bootstrap the redaction of the results
here is a draft I wrote (following the indications of Stefan and Jean) for
the case $c=f(a,b)$.

Some questions:

\begin{enumerate}
	\item Why can we discard non-intersected cells without queries?
	\item Why are all curves simple (non-self-intersecting)?
\end{enumerate}

\subsection{Aurélien Ooms Oct 8 (2)}
ERRATUM:
\begin{displayquote}
\dots\\
The complexity of the algorithm is
$$T(n) = O(n^{2-a} \myworries{+} n^{(1+a)\frac{4}{3}})$$
\dots
\end{displayquote}

\subsection{Jean Cardinal Oct 13 (1)}
That is looking good.

You may want to give more details on how you obtain the number of
intersected cells (lemma?) and Matou\v{s}ek's algorithm (complete statement +
reference).

Also, the actual computation model is not clear. I guess algebraic decision
trees are fine. Can we say something about the size of the queries? Also,
the sorting part is actually uniform, right?

The next steps would as follows (see previous emails):
\begin{itemize}
	\item Extend to $f(a,b,c)$
	\item See how we can get an actual (uniform) subquadratic algorithm by adapting
the domination business.
\item for arbitrary $k$, it seemed like we could adopt a Meiser-like approach but
using trapezoidal decompositions, so that we are not limited by linear
queries.
\end{itemize}

Again, the question of the exact computation model in which it
could be implemented is a bit tricky. But as long as it involves
constant-size polynomial systems, it is still meaningful.

\subsection{Noam Solomon Oct 13}
Here are two ideas/suggestions:
\begin{itemize}
	\item[(*)] I suggest to add the parameter $\deg(f)$ as part of the input and have an
expression for the complexity as a function also of $\deg(f)$.

	\item[(**)] Here is another very natural generalization: suppose you are in
		$\mathbb{R}^4$
and you want if there exists a solution to the two polynomial equations
$F(x,y,z,w)=G(x,y,z,w)=0$, where $x \in A, y\in B, z\in C, w \in D$ for $A,B,C,D$
sets of real numbers (of the same size $n$, or of different size).
(You might decide, as we had before that $\deg(F), \deg(G)$ are constants,
or, alternatively, study this as a function of these degrees).
One way to solve this question is find ALL solutions to $F(x,y,z,w)$ and then
check for each solution if it satisfies $G(x,y,z,w)=0$.

Another alternative is to compute $\res(F,G; x)$, i.e., the resultant of F and
G with respect to some variables (say $x$). This polynomial has degree at
most $\deg(F)\deg(G)$ and now you can use our algorithm to find all solutions
$(y,z,w) \in B\times C \times D$ to $\res(F,G;x)=0$, and for each solution $(y_0,
z_0, w_0)$ decompose the polynomial $F(x,y_0,z_0,w_0)$ into at most $\deg(F)$
factors $(x_{\alpha_i})$ and for each $\alpha_i$  check if it lies in A (it
should take $O(\deg(\res(F,G,x))$ to compute it).

In the case of four variables probably both algorithms have the same
complexity. But in five variables I think using resultants reduce the
problem to the four-variable case and then we get an improvement.

This motivates the following question: Assume you are in $\mathbb{R}^d$, and you have
a variety V defined as $V=Z(F_1,\ldots, F_s)$. You want to know if there
exists a solution $(x_1,...,x_d)\in A_1 \times \ldots \times
A_d$, where $A_i$
are sets of real numbers of size $n$ each.

A first approach is to find all solutions to $F_1(x_1,,\ldots, x_d)=0$ and
check for each of them if it satisfies $F_2,\ldots, F_s$.
A second option is to compute the resultants and project on $\mathbb{R}^{\dim V + 1}$.
I believe that this approach should lead to better running times. If this
seems interesting I can try to make this more precise (it is still very
sketchy..).
\end{itemize}

\subsection{Jean Cardinal Oct 13 (2)}
\begin{displayquote}
	Here is another very natural generalization: suppose you are in $\mathbb{R}^4$
and you want if there exists a solution to the two polynomial equations
$F(x,y,z,w)=G(x,y,z,w)=0$, where $x \in A, y\in B, z\in C, w \in D$ for $A,B,C,D$
sets of real numbers (of the same size $n$, or of different size).
(You might decide, as we had before that $\deg(F), \deg(G)$ are constants,
or, alternatively, study this as a function of these degrees).
One way to solve this question is find ALL solutions to $F(x,y,z,w)$ and then
check for each solution if it satisfies $G(x,y,z,w)=0$.

Another alternative is to compute $\res(F,G; x)$, i.e., the resultant of F and
G with respect to some variables (say $x$). This polynomial has degree at
most $\deg(F)\deg(G)$ and now you can use our algorithm to find all solutions
$(y,z,w) \in B\times C \times D$ to $\res(F,G;x)=0$, and for each solution $(y_0,
z_0, w_0)$ decompose the polynomial $F(x,y_0,z_0,w_0)$ into at most $\deg(F)$
factors $(x_{\alpha_i})$ and for each $\alpha_i$  check if it lies in A (it
should take $O(\deg(\res(F,G,x))$ to compute it).

In the case of four variables probably both algorithms have the same
complexity. But in five variables I think using resultants reduce the
problem to the four-variable case and then we get an improvement.
\end{displayquote}

Sounds cool!
But for 4 variables, the algorithm is actually a straightforward
application of the Matou\v{s}ek range searching.
It would be nice to work out the general case.

For the rest of us who might be wondering about the applications of those
problems, here is the original paper :
\url{http://arxiv.org/abs/1401.7419}
(it was digged somewhere in the conversation)

\subsection{noam solomon oct 20}
i thought of some idea that could be nice to add. (jean -- it is related to
our work with michael).
assume that we are in the 3-dim setting, we have a constant degree
$f(x,y,z)=0$ and we have some set $a$ of size $n$, and we are looking for the
size of a maximal subset $b$ of $a$ such that for any $x,y,z \in b, f(x,y,z) \ne
0$.
by the following theorem of spencer (which generalizes turan)
\begin{theorem}[Spencer~\cite{S72}]\label{thm:spencer}
Let $h$ be a $t$-uniform hypergraph with $n$ vertices and $m$ edges. If
$m<\frac{n}{t}$ then $\alpha(h) > \frac{n}{2}$. Otherwise

$$
\alpha (h) \ge \frac{t-1}{t^{t/(t-1)}} \frac n{(m/n)^{1/(t-1)}}.
$$
\end{theorem}

we can deduce that the size of $b$ is at least $\sqrt n$, and if $f$ is not
special, then $b$ is of size at least $n^{\frac{7}{12}}$. notice that if
$f(x,y,z)=z-x-y$ then intuitively you cannot improve $\sqrt n$ because if you
take $a=\{0,1,2,\ldots,n\}$ and you take a subset of size $3\sqrt n$ such that
$x_1+y_1 \ne x_2+y_2$ for any $(x_1,y_1) \ne (x_2,y_2)$ and $\ne (y_2,x_2)$
you get more than $n$ elements in $a$ which is impossible.

but i am cheating because if we just take
$b=\{\frac{n}{2}+1,\ldots,n\}$ then it is
impossible to find $x,y,z \in b$ such that $z=x+y$. the reason why the previous
argument did not work is that $x+y$ is not necessarily in $a$.

perhaps we can take sets $a_1, a_2, a_3$ of size $n$ and look for solutions
$(x,y,z)$ such that $f(x,y,z)=0$ and
$x\in a_1, y\in a_2$ and $z\in a_3$. now we look for subsets $b_1,b_2, b_3$ such
that
$f(x,y,z) \ne 0$ to any $x\in b_1, y\in b_2$ and $z\in b_3$. we need to use some
3-partite version of spencer's theorem (which i am sure exists) and then we
can deduce that the bound $\sqrt n$ on the $b_i$ is tight, but this is somewhat
less elegant.


in any case, i think finding an efficient algortihm (perhaps a variant of
what we already have) to compute these maximal independent sets will be
nice.

\subsection{Jean Cardinal Dec 3}

Aurélien is preparing a draft for the first part of the results
($c = f(a,b)$).

In passing, we realized that \(\frac{12}{7} < \frac{11}{6}\). Hence the
complexity of the nonuniform algorithm is \emph{smaller} than the Elekes-Ronyai
bound!

Finally, I stumbled upon a paper that seems relevant to the last part of
our plan:

\begin{displayquote}
Actually much better (but non-uniform) can probably be done by using
the approach of Meiser for $k$-SUM: Using epsilon-nets, it should be
possible to do point location of a point in $\mathbb{R}^n$ in an arrangement of n
choose $k$ algebraic surfaces $F(x_{i_1},...,x_{i_k})=0$ for all choices of
indices $i_1 \ldots i_k$, in polynomial time in $n$. (...)
\end{displayquote}

Something very similar seems to have been done already:
\url{http://link.springer.com/chapter/10.1007/978-3-642-13411-1\_5}.
We should improve on this using trapezoids.

\subsection{Noam Dec 20}

There is a lecture today by Matya Katz, and so I looked into his
result with Boris Aronov. \url{http://arxiv.org/pdf/1412.0962v2.pdf}
I wonder if our approach (partially written by Aurelien) can improve
Theorem 3.2. This might be a nice application. As far as I remember,
our approach works also for triples $z>f(x,y)$, so I think this should
give a subquadratic algorithm to Theorem 3.2
Does Aurelien intend to continue his writings for the general case?
The implicit function case for three and higher variables.

\subsection{Stefan Dec 30}

Noam has noticed this cool potential application of our result. If
this works, it should definitely be added to the writeup.
(NDLR) looks like it does not work.

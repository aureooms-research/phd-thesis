
\subsection{Algebraic computation trees}
\label{app:act}

The following theorem follows immediately from
the analysis of the linearity of queries
\begin{theorem}\label{thm:act}
	The algebraic computation tree complexity of \(k\)-LDT is
	\(\tilde{O}(n^3)\).
\end{theorem}

\begin{proof}
We go through each step of Algorithm~\ref{algo:meiser}.
Indeed, each \(k\)-linear query of step \step{1} can be implemented as
\(O(k)\) arithmetic operations, so step \step{1} has complexity
\(O(\card{\net})\).
The construction of the simplex in step \step{2} must be handled carefully.
What we need to show is that each \(n\)-linear query we use can be implemented
using $O(k)$ arithmetic operations. It is not difficult to see from the
expressions given in~Appendix~\ref{app:keeplinear} that a constant number of arithmetic
operations and dot products suffice to
compute the queries. A dot product in this case involves a constant number
of arithmetic operations because the \(d_i\) are such that they each have
exactly \(k\) non-zero components. The only expression that involves a
non-constant number of operations is the product \(\prod_{k=0}^{s}
d_{\theta_{k}} \cdot \vec{\nu q}^{(k)}\), but this is equivalent to
\((\prod_{k=0}^{s-1}
	d_{\theta_{k}} \cdot \vec{\nu q}^{(k)})(d_{\theta_{s}} \cdot
	\vec{\nu q}^{(s)})\)
where the first factor has already been computed during a previous step and
the second factor is of constant complexity. Since each query costs a constant
number of arithmetic operations and branching operations, step \step{2}
has complexity \(O(n\card{\net})\).
Finally, steps \step{3} and \step{4} are free since they do not involve the
input. The complexity of Algorithm~\ref{algo:meiser} in this model is thus also \(O(n^3
\log n \log \card{\Hy})\).
\end{proof}


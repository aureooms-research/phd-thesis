The $k$-SUM problem is given $n$ input real numbers to determine whether any
$k$ of them sum to zero. The problem is of tremendous importance in the
emerging field of complexity theory within $P$, and it is in particular open
whether it admits an algorithm of complexity $O(n^c)$ with $c<\lceil
\frac{k}{2} \rceil$. Inspired by an algorithm due to Meiser (1993), we show
that there exist linear decision trees and algebraic computation trees of depth
$O(n^3\log^2 n)$ solving $k$-SUM. Furthermore, we show that there exists a
randomized algorithm that runs in $\tilde{O}(n^{\lceil\frac{k}{2}\rceil+8})$ time,
and performs $O(n^3\log^2 n)$ linear queries on the input. Thus, we show that
it is possible to have an algorithm with a runtime almost identical (up to the
$+8$) to the best known algorithm but for the first time also with the number
of queries on the input a polynomial that is independent of $k$. The
$O(n^3\log^2 n)$ bound on the number of linear queries is also a tighter bound
than any known algorithm solving $k$-SUM, even allowing unlimited total time
outside of the queries.
By simultaneously achieving few queries to the input without significantly
sacrificing runtime vis-\`{a}-vis known algorithms, we deepen the understanding
of this canonical problem which is a cornerstone of complexity-within-$P$.

We also consider a range of tradeoffs between the number of terms involved in
the queries and the depth of the decision tree. In particular, we prove that
there exist $o(n)$-linear decision trees of depth $\tilde{O}(n^3)$ for the
$k$-SUM problem.

\section{Sorting}

QUOTE ``Please, pretty please, order those data from smallest to largest''.

The sorting problem is one of the oldest and most practical data management
problems.
%
Usually, sorting is about permutations in \emph{arrays}, but we do not like
that. We use a different abstraction:

Before continuing, let us go back to the origins of those different problems.
The link between them will be made even clearer.

Sorting is one of the oldest and most relevant data management problems.
It is the archetypal computation tree problem.
%
Usually presented, sorting is about permutations in \emph{arrays}, but we do
not like that. We use a different abstraction:
%
Before continuing, let us go back to the origins of those different problems.
The link between them will be made even clearer.

Sorting is one of the oldest and most relevant data management problems.
It is the archetypal computation tree problem.
%
Usually presented, sorting is about permutations in \emph{arrays}, but we do
not like that. We use a different abstraction:
%
Before continuing, let us go back to the origins of those different problems.
The link between them will be made even clearer.

Sorting is one of the oldest and most relevant data management problems.
It is the archetypal computation tree problem.
%
Usually presented, sorting is about permutations in \emph{arrays}, but we do
not like that. We use a different abstraction:
%
\input{text/definition/sorting}

In other words, we see the sorting problem as an information retrieval problem:
how many comparisons do we need to make in order to \emph{know} the answer to
all comparisons.
%
The usual sorting algorithms rearrange an array of input numbers into sorted
order. Another way to think about it is that they compute the permutation that
sends the input array to a sorted version of itself. This permutation is a data
structure such that given any index in the input array, we can query for the
corresponding index in the sorted array called the \emph{rank} of the element).
To retrieve the result of a comparison of two elements of the input array we can
compare their ranks in this permutation.
The usual way of defining the sorting problem restricts the data structure that
should be used to encode the \(n \choose 2\) comparisons.
The way we model the sorting problem lifts this restriction because it does not
ask to structure this information in a nice way.

The sorting problem also has its decision problem variant: Element Uniqueness.
%
\input{text/definition/element-uniqueness}

It is easy to see
that Sorting amounts to locating the point \(q\) in
the arrangement of hyperplanes of equations \(x_i - x_j = 0\)%
, and
that Element Uniqueness reduces both to and from the 2SUM
problem in linear time%
.
%
Actually, under reasonable assumptions on the computation model, sorting and
element uniqueness are the same problems: if all questions we ask about the
input are linear in the input numbers, then proving that the input
does not contain any duplicate entries requires to sort the input.
%
The same statement carries over to the \(k\)-SUM with respect to its ``sorting
version'': computing the sign of all \(n \choose k\) sums of \(k\) input
numbers.
%
Therefore, in those models of computation, the sorting problem is equivalent to
2SUM.
%
Because of this relationship between Sorting and the \(k\)-SUM problem,
we see why the better understanding of \(k\)-SUM is a natural next move.

Among all the problems this thesis touches, Sorting and Element Uniqueness
are the best understood. We know \(\Omega(n \log n)\) lower bounds in many
nonuniform models of computation and we also know a long list of simple
real-RAM algorithms for those problems whose running times match those lower
bounds. For all that matters here, those problems are solved.


In other words, we see the sorting problem as an information retrieval problem:
how many comparisons do we need to make in order to \emph{know} the answer to
all comparisons.
%
The usual sorting algorithms rearrange an array of input numbers into sorted
order. Another way to think about it is that they compute the permutation that
sends the input array to a sorted version of itself. This permutation is a data
structure such that given any index in the input array, we can query for the
corresponding index in the sorted array called the \emph{rank} of the element).
To retrieve the result of a comparison of two elements of the input array we can
compare their ranks in this permutation.
The usual way of defining the sorting problem restricts the data structure that
should be used to encode the \(n \choose 2\) comparisons.
The way we model the sorting problem lifts this restriction because it does not
ask to structure this information in a nice way.

The sorting problem also has its decision problem variant: Element Uniqueness.
%
\begin{problem}[Element Uniqueness]
	Given \(n\) numbers \(q_1, q_2, \ldots, q_n \in \mathbb{R}\),
	decide whether there exists \(1 \leq i < j \leq n\),
	such that \(q_i = q_j\).
\end{problem}


It is easy to see
that Sorting amounts to locating the point \(q\) in
the arrangement of hyperplanes of equations \(x_i - x_j = 0\)%
, and
that Element Uniqueness reduces both to and from the 2SUM
problem in linear time%
.
%
Actually, under reasonable assumptions on the computation model, sorting and
element uniqueness are the same problems: if all questions we ask about the
input are linear in the input numbers, then proving that the input
does not contain any duplicate entries requires to sort the input.
%
The same statement carries over to the \(k\)-SUM with respect to its ``sorting
version'': computing the sign of all \(n \choose k\) sums of \(k\) input
numbers.
%
Therefore, in those models of computation, the sorting problem is equivalent to
2SUM.
%
Because of this relationship between Sorting and the \(k\)-SUM problem,
we see why the better understanding of \(k\)-SUM is a natural next move.

Among all the problems this thesis touches, Sorting and Element Uniqueness
are the best understood. We know \(\Omega(n \log n)\) lower bounds in many
nonuniform models of computation and we also know a long list of simple
real-RAM algorithms for those problems whose running times match those lower
bounds. For all that matters here, those problems are solved.


In other words, we see the sorting problem as an information retrieval problem:
how many comparisons do we need to make in order to \emph{know} the answer to
all comparisons.
%
The usual sorting algorithms rearrange an array of input numbers into sorted
order. Another way to think about it is that they compute the permutation that
sends the input array to a sorted version of itself. This permutation is a data
structure such that given any index in the input array, we can query for the
corresponding index in the sorted array called the \emph{rank} of the element).
To retrieve the result of a comparison of two elements of the input array we can
compare their ranks in this permutation.
The usual way of defining the sorting problem restricts the data structure that
should be used to encode the \(n \choose 2\) comparisons.
The way we model the sorting problem lifts this restriction because it does not
ask to structure this information in a nice way.

The sorting problem also has its decision problem variant: Element Uniqueness.
%
\begin{problem}[Element Uniqueness]
	Given \(n\) numbers \(q_1, q_2, \ldots, q_n \in \mathbb{R}\),
	decide whether there exists \(1 \leq i < j \leq n\),
	such that \(q_i = q_j\).
\end{problem}


It is easy to see
that Sorting amounts to locating the point \(q\) in
the arrangement of hyperplanes of equations \(x_i - x_j = 0\)%
, and
that Element Uniqueness reduces both to and from the 2SUM
problem in linear time%
.
%
Actually, under reasonable assumptions on the computation model, sorting and
element uniqueness are the same problems: if all questions we ask about the
input are linear in the input numbers, then proving that the input
does not contain any duplicate entries requires to sort the input.
%
The same statement carries over to the \(k\)-SUM with respect to its ``sorting
version'': computing the sign of all \(n \choose k\) sums of \(k\) input
numbers.
%
Therefore, in those models of computation, the sorting problem is equivalent to
2SUM.
%
Because of this relationship between Sorting and the \(k\)-SUM problem,
we see why the better understanding of \(k\)-SUM is a natural next move.

Among all the problems this thesis touches, Sorting and Element Uniqueness
are the best understood. We know \(\Omega(n \log n)\) lower bounds in many
nonuniform models of computation and we also know a long list of simple
real-RAM algorithms for those problems whose running times match those lower
bounds. For all that matters here, those problems are solved.


Without any

\subsection{Algorithms}
Explain models of computation: uniform nonuniform etc

\subsection{Data Structures}

Most (if not all) algorithms for sorting receive their input data as a list and
output their answer as a permutation of this data. This permutation is a
compact data structure: the structure takes space proportional to the minimum
amount of space possible (about \(\log_2 n!\)), and the answer to each
comparison can be retrieved instantly from the structure.


\section{The \(k\)-SUM Problem}

The \(k\)-SUM problem is simple enough: given a list of \(n\) real numbers,
decide whether \(k\) of them sum to zero.

The paper of Gr\o nlund and Pettie also discusses the following generalization of
the 3SUM problem: ``For a fixed \(k\), given a set of \(n\) real numbers,
decide whether there exists a \(k\)-subset whose elements sum to zero.''
This problem is called the \(k\)-SUM problem.

Obviously, the 3SUM problem is the \(k\)-SUM problem where \(k=3\).
Moreover, there is a simple reduction from \(k\)-SUM to 2SUM when \(k\) is even
and to 3SUM when \(k\) is odd. Those reductions yield a
\(O(n^{\frac{k}{2}} \log n)\) time real-RAM algorithm for \(k\) even and a
\(O(n^{\frac{k+1}{2}})\) time real-RAM algorithm for \(k\) odd.

In their paper, in addition to the slightly subquadratic uniform algorithm for
3SUM, Gr\o nlund and Pettie give a strongly subquadratic nonuniform
algorithm for 3SUM. The algorithms runs in time \(\tilde{O}(n^{3/2})\), and,
because of the aforementioned reduction, immediately yields an improved
\(\tilde{O}(n^{\frac{k}{2}})\) nonuniform time complexity for \(k\)-SUM when
\(k\) is odd.

As for uniform time complexity we do not know whether this nonuniform
improvement can be transferred to the real-RAM model: we do not know of any
real-RAM \(o(n^{\frac{k+1}{2}})\) time algorithm for \(k\)-SUM when
\(k\) is odd.

The \(k\)-SUM problem reduces to the following point location problem: ``Given
a input point \(q \in \mathbb{R}^n\), locate \(q\) in the arrangement of
\(n \choose k\) hyperplanes of equation \(x_{i_1} + x_{i_2} + \cdots +
x_{i_k} = 0\).'' Applying the best nonuniform algorithms for point location in
arrangements of hyperplanes by Meyer auf der Heide~\cite{M84} and
Meiser~\cite{M93} yields linear decision trees of depth \(n^{O(1)}\) for
\(k\)-SUM, where the constant of proportionality in the big-oh does not depend
on \(k\).


For fixed \(k\), the \(k\)-SUM problem can be solved in polynomial time
\(O(n^k)\) by testing all possible candidate solutions.
The interesting question is whether it is possible to improve on
this brute-force solution.

The \(k\)-SUM problem is defined as follows: given a collection of $n$ real
numbers decide whether any $k$ of them sum to zero, where $k$ is a constant.
It is a fixed-parameter version of the subset-sum problem, a standard \textit{NP}-complete
problem. The \(k\)-SUM problem, and in particular the special case of 3SUM,
has proved to be a cornerstone of the fine-grained complexity program aiming at
the construction of a complexity theory for problems in $P$. In particular,
there are deep connections between the complexity of \(k\)-SUM, the Strong
Exponential Time Hypothesis~\cite{PW10,CGIMPS15}, and the complexity of many
other major problems in
$P$~\cite{GO95,BH99,MO01,P10,ACLL14,AVW14,GP14,KPP14,ALW14,AWY15,CL15}.

It has been long known that the \(k\)-SUM problem can be solved in time
$O(n^{\frac{k}{2}}\log n)$ for even $k$, and $O(n^{\frac{k+1}{2}})$ for odd
$k$. Erickson~\cite{E99} proved a near-matching lower bound in the $k$-linear
decision tree model. In this model, the complexity is measured by the depth of
a decision tree, every node of which corresponds to a query of the form
$q_{i_1} + q_{i_2} + \cdots + q_{i_k} \ask{\le} 0$, where $q_1, q_2, \ldots, q_n$ are the
input numbers. In a recent breakthrough paper, Gr\o nlund and
Pettie~\cite{GP14} showed that in the $(2k-2)$-linear decision tree model,
where queries test the sign of weighted sums of up to $2k-2$ input numbers, only
$O(n^\frac{k}{2}\sqrt{\log n})$ queries are required for odd values of $k$. In
particular, there exists a $4$-linear decision tree for 3SUM of depth
$\tilde{O}(n^\frac{3}{2})$ (here the notation $\tilde{O}$ ignores polylogarithmic
factors), while every 3-linear decision tree has depth $\Omega
(n^2)$~\cite{E99}. This indicates that increasing the size of the queries,
defined as the maximum number of input numbers involved in a query, can yield
significant improvements on the depth of the minimal-height decision tree. Ailon and
Chazelle~\cite{AC05} slightly extended the range of query sizes for which a
nontrivial lower bound could be established, elaborating on Erickson's
technique.

It has been well established that there exist nonuniform
polynomial-time algorithms for the subset-sum problem. One of them was
described by Meiser~\cite{M93}, and is derived from a data structure for point
location in arrangements of hyperplanes using the bottom vertex decomposition.
This algorithm can be cast as the construction of a linear decision tree in which
the queries have non-constant size.

\subsection{Our results}
In Section~\ref{sec:query-complexity}, we show the existence of an $n$-linear
decision tree of depth $\tilde{O}(n^3)$ for \(k\)-SUM using a careful
implementation of Meiser's algorithm~\cite{M93}.
Although the high-level algorithm itself is not new, we refine the
implementation and analysis for the \(k\)-SUM problem.%
\footnote{After submitting this manuscript, we learned from a personal communication
with Herv\'e Fournier that a similar analysis for arbitrary hyperplanes appears in his PhD
thesis~\cite{F01} (in French).}
Meiser presented his algorithm as a general method of point location in $m$
given $n$-dimensional hyperplanes that yielded a $\tilde{O}(n^4 \log
m)$-depth algebraic computation tree; when viewing the \(k\)-SUM problem as a point
location problem, $m$ is $O(n^k)$ and thus Meiser's algorithm can be viewed
as giving a $\tilde{O}(n^4)$-depth algebraic computation tree.
We show that while the
original algorithm was cast as a nonuniform polynomial-time algorithm,
it can be implemented in the linear decision tree model with an
$\tilde{O}(n^3)$ upper bound.
Moreover, this result implies the same improved upper bound on the depth of
algebraic computation trees for the $k$-SUM problem,
as shown in Appendix~\ref{app:act}.

There are two subtleties to this result. The first is inherent to the chosen
complexity model: even if the number of queries to the input is small (in
particular, the degree of the polynomial complexity is invariant on $k$), the
time required to \emph{determine which queries should be performed} may be
arbitrary. In a na\"ive analysis, we show it can be trivially bounded by
$\tilde{O}(n^{k+2})$. In Section~\ref{sec:time-complexity} we present an algorithm to
choose
which decisions to perform whereby the running time can be reduced to
$\tilde{O}(n^{\frac{k}{2}+8})$. Hence, we obtain an
$\tilde{O}(n^{\frac{k}{2}+8})$ time randomized algorithm in the RAM model
expected to perform $\tilde{O}(n^3)$ linear queries on the input\footnote{%
	Gr{\o}nlund and
	Pettie~\cite{GP14} mention the algorithms of Meyer auf der Heyde~\cite{M84} and
	Meiser~\cite{M93}, and state ``(\ldots) it was known that all \(k\)-LDT problems
	can be solved by $n$-linear decision trees with depth $O(n^5\log
	n)$~\cite{M93}, or with depth $O(n^4\log (nK))$ if the coefficients of the
	linear function are integers with absolute value at most $K$~\cite{M84}.
	Unfortunately these decision trees are not efficiently constructible. The
	time required to determine \emph{which} comparisons to make is
	exponential.'' We prove that the trees can have depth $\tilde{O}(n^3)$ and
	that the whole algorithm can run in randomized polynomial-time.}.

The second issue we address is that the linear queries in the above algorithm may
have size $n$, that is, they may use all the components of the input.
The lower bound of Erickson shows that if the queries are of minimal size, the number
of queries cannot be a polynomial independent of $k$ such as what we obtain, so
non-minimal query size is clearly essential to a drastic reduction in the number of queries needed.
This gives rise to the natural question as to what is the relation between query size and number of queries.
In particular, one natural question is whether queries of size less than $n$
would still allow the problem to be solved using a number of queries that is a
polynomial independent of $k$. We show that this is possible; in Section~\ref{sec:query-size},
we introduce a range of
algorithms exhibiting an explicit tradeoff between the number of queries and
their size. Using a blocking scheme, we show that we can restrict to
$o(n)$-linear decision trees. We also give a
range of tradeoffs for $O(n^{1-\alpha})$-linear decision trees. Although the
proposed algorithms still involve nonconstant-size queries, this is
the first time such tradeoffs are explicitly tackled. Table~\ref{tab:results} summarizes our results.

\begin{table}
\centering
\caption{Complexities of our new algorithms for the \(k\)-SUM problem. The query
size is the maximum number of elements of the input that can be involved in a
single linear query. The number of blocks is a parameter that allows us to
change the query size (see Section~\ref{sec:query-size}).
The origin of the constant in the exponent of the time complexity is due to
Lemma~\ref{lem:bound}. We conjecture it can be reduced, though substantial
changes in the analysis will likely be needed to do so.}
\label{tab:results}
\begin{tabular}{|c|c|c|c|c|}
	\hline

	& \# blocks & query size & \# queries & time \\
	\hline
	Theorem~\ref{thm:cube} &
	$1$ &
	$n$ &
	$\tilde{O}(n^3)$ & $\tilde{O}(n^{\lceil\frac{k}{2}}+8\rceil)$
	\\

	\hline

	Theorem~\ref{thm:query-size} &
	$b$ &
	$k\lceil\frac{n}{b}\rceil$ &
	$\tilde{O}(b^{k-4}n^3)$ &
	$\tilde{O}(b^{\lfloor\frac{k}{2}}-9\rfloor n^{\lceil\frac{k}{2}}+8\rceil)$
	\\

	\hline

	Corollary~\ref{cor:logn} &
	$b=\Theta(\polylog(n))$ &
	$o(n)$ &
	$\tilde{O}(n^3)$ &
	$\tilde{O}(n^{\lceil\frac{k}{2}}+8\rceil)$
	\\

	\hline

	Corollary~\ref{cor:ne} &
	$b=\Theta(n^\alpha)$ &
	$O(n^{1-\alpha})$ &
	$\tilde{O}(n^{3+(k-4)\alpha})$ &
	$\tilde{O}(n^{(1+\alpha)\frac{k}{2} +8.5})$
	\\

	\hline
\end{tabular}
\end{table}

\section{The Geometric View}

\chapter{Uniform and Nonuniform Algorithms}

\section{Definitions and previous work}

\subsection{Definitions}
We consider the \(k\)-SUM problem for \(k=O(1)\). In what follows, we use the
notation \([n] = \{\,1,2,\ldots ,n\,\}\).
\begin{problem}[\(k\)-SUM]
 Given an input vector \(q\in\mathbb{R}^n\), decide whether there exists a
 $k$-tuple \((i_1, i_2,\ldots ,i_k) \in {[n]}^k\) such that \(\sum_{j=1}^k
 q_{i_j} = 0\).
\end{problem}
The problem amounts to deciding in $n$-dimensional space, for each hyperplane
\(H\) of equation \(x_{i_1} + x_{i_2} + \cdots +x_{i_k} = 0\), whether \(q\)
lies on, above, or below \(H\). Hence this indeed amounts to locating the point
$q$ in the arrangement formed by those hyperplanes. We emphasize that the set
of hyperplanes depends only on $k$ and $n$ and not on the actual input vector
$q$.

Linear degeneracy testing (\(k\)-LDT) is a generalization of \(k\)-SUM where we
have arbitrary rational coefficients\footnote{The usual definition of \(k\)-LDT
allows arbitrary \emph{real} coefficients. However, the algorithm we provide
for Lemma~\ref{lem:multiple} needs the vertices of the arrangement of
hyperplanes to have rational coordinates.}
and an independent term in the equations
of the hyperplanes.
\begin{problem}[\(k\)-LDT]
 Given an input vectors \(q\in\mathbb{R}^n\) and
 $\alpha \in \mathbb{Q}^n$ and constant $c \in \mathbb{Q}$
 decide whether there exists a
 $k$-tuple \((i_1, i_2,\ldots ,i_k) \in {[n]}^k\) such that
 \(c + \sum_{j=1}^k \alpha_j q_{i_j} = 0\).
 \end{problem}
Our algorithms apply to this more general problem with only minor changes.

The \emph{\(s\)-linear decision tree model} is a standard model of computation
in which several lower bounds for \(k\)-SUM\ have been proven. In the decision tree
model, one may ask well-defined questions to an oracle that are answered
``yes'' or ``no.'' For $s$-linear decision trees, a well-defined question consists
of testing the sign of a linear function on at most \(s\) numbers \(q_{i_1},\ldots,q_{i_s}\) of the
input \(q_1,\ldots,q_n\) and can be written as
$$
	c + \alpha_1 q_{i_1} + \cdots + \alpha_s q_{i_s} \ask{\le} 0
$$
Each question is defined to cost a single unit. All other operations can be
carried out for free but may not examine the input vector $q$. We refer to
$n$-linear decision trees simply as linear decision trees.

In this paper, we consider algorithms in the standard integer RAM model with
$\Theta(\log n)$-size words, but in which the input $q\in\mathbb{R}^n$ is
accessible \emph{only} via a linear query oracle. Hence we are not allowed to
manipulate the input numbers directly. The complexity is measured in two ways:
by counting the total number of queries, just as in the linear decision tree
model, and by measuring the overall running time, taking into account the time
required to determine the sequence of linear queries. This two-track
computation model, in which the running time is distinguished from the query
complexity, is commonly used in results on comparison-based sorting problems
where analyses of both runtime and comparisons are of interest (see for
instance~\cite{SS95,CFJJM10,CFJJM13}).

\subsection{Previous Results}
The seminal paper by Gajentaan and Overmars~\cite{GO95} showed the crucial role
of 3SUM in understanding the complexity of several problems in
computational geometry.
Since then, there has been an enormous amount of work focusing on the complexity of
3SUM and this problem is now considered a key tool of
complexity-within-P~\cite{GO95,BH99,MO01,BDP08,P10,ACLL14,AVW14,GP14,KPP14,ALW14,AWY15,CL15}.
The current conjecture is that no $O(n^{2-\delta})$-time algorithm exists for 3SUM.
It has been known for long that \(k\)-SUM is $W[1]$-hard. Recently, it was shown
to be $W[1]$-complete by Abboud et al.~\cite{ALW14}.

In Erickson~\cite{E99}, it is shown that we cannot solve 3SUM in
subquadratic time in the \(3\)-linear decision tree model:
\begin{theorem}[Erickson~\cite{E99}]
The optimal depth of a \(k\)-linear decision tree that solves
the \(k\)-LDT problem is $\Theta(n^{\lceil\frac{k}{2}}\rceil)$.
\end{theorem}
The proof uses an adversary argument which can be explained geometrically. As
we already observed, we can solve \(k\)-LDT problems by modeling them as point
location problems in an arrangement of hyperplanes. Solving one such problem
amounts to determining which cell of the arrangement contains the input point.
The adversary argument of Erickson~\cite{E99} is that there exists a cell having
$\Omega(n^{\lceil\frac{k}{2}}\rceil)$ boundary facets and in this model point
location in such a cell requires testing each facet.

Ailon and Chazelle~\cite{AC05} study \(s\)-linear decision trees to solve the \(k\)-SUM problem when
\(s > k\). In particular, they give an additional proof for the
$\Omega(n^{\ceil{\frac{k}{2}}})$ lower bound of Erickson~\cite{E99} and
generalize the lower bound for the \(s\)-linear decision tree model when \(s >
k\). Note that the exact lower bound given by Erickson~\cite{E99} for \(s = k\) is
$\Omega({(nk^{-k})}^{\ceil{\frac{k}{2}}})$ while the one given by
Ailon and Chazelle~\cite{AC05} is $\Omega({(nk^{-3})}^{\ceil{\frac{k}{2}}})$. Their result
improves therefore the lower bound for \(s = k\) when \(k\) is large.
The lower bound they prove for \(s > k\) is the following
\begin{theorem}[Ailon and Chazelle~\cite{AC05}]
The depth of an $s$-linear decision tree solving the \(k\)-LDT\ problem is
$$
\Omega\mleft(\group{nk^{-3}}^{\frac{2k-s}{2\lceil\frac{s-k+1}{2}\rceil}
(1-\epsilon_k) }\mright),
$$
where \(\epsilon_k > 0\) tends to \(0\) as \(k \to\infty\).
\end{theorem}
This lower bound breaks down when
\(k = \Omega(n^{\sfrac{1}{3}})\) or \(s \ge 2 k\) and the cases where \(k < 6\)
give trivial lower bounds. For example, in the case
of 3SUM with \(s = 4\) we only get an $\Omega(n)$ lower bound.

As for upper bounds, Baran et al.~\cite{BDP08} gave subquadratic Las Vegas
algorithms for 3SUM on integer and
rational numbers in the circuit RAM, word RAM, external memory, and
cache-oblivious models of computation. The idea of their approach is to exploit
the parallelism of the models, using linear and universal hashing.

More recently, Gr{\o}nlund and Pettie~\cite{GP14} proved the existence of a linear decision tree
solving the 3SUM problem using a strongly subquadratic number of linear queries.
The classical quadratic algorithm for 3SUM uses \(3\)-linear queries
while the decision tree of Gr{\o}nlund and Pettie uses \(4\)-linear queries and
requires $O(n^{\sfrac{3}{2}} \sqrt{\log n})$ of them.
Moreover, they show that their decision tree can be used to get better upper
bounds for \(k\)-SUM when \(k\) is odd.

They also provide two subquadratic 3SUM
algorithms. A deterministic one running in
$O(n^2/{(\log n/\log \log n)}^{\sfrac{2}{3}})$
time and a randomized one running in
$O(n^2 {(\log \log n)}^2 / \log n)$ time with high probability.
These results refuted the long-lived conjecture that
3SUM cannot be solved in subquadratic time in the RAM model.

Freund~\cite{F15} and Gold and Sharir~\cite{GS15} later gave improvements on the
results of Gr{\o}nlund and Pettie~\cite{GP14}. Freund~\cite{F15} gave a deterministic algorithm for
3SUM running in \(O( {n^2\log \log n}/{\log n})\) time.
Gold and Sharir~\cite{GS15} gave another deterministic algorithm for 3SUM
with the same running time and shaved off the $\sqrt{\log n}$ factor in the
decision tree complexities of 3SUM and \(k\)-SUM given by Gr{\o}nlund and Pettie.

Meyer auf der Heide~\cite{M84} gave the first point location algorithm to solve the knapsack
problem in the linear decision tree model in polynomial time. He thereby
answers a question raised by Dobkin and Lipton~\cite{DL74,DL78}, Yao~\cite{Y82}
and others. However, if one uses this algorithm to locate a point in an
arbitrary arrangement of hyperplanes the running time is increased by a factor
linear in the greatest coefficient in the equations of all hyperplanes.
On the other hand, the complexity of Meiser's point location algorithm is
polynomial in the dimension, logarithmic in the number of hyperplanes and
does not depend on the value of the coefficients in the equations of the
hyperplanes. A useful complete description of the latter is also given by
Bürgisser et al.~\cite{B97} (Section 3.4).


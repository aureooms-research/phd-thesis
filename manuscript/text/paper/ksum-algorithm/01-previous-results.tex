\section{Uniform and Nonuniform Algorithms}

It has been long known that the \(k\)-SUM problem can be solved in time
$O(n^{\frac{k}{2}}\log n)$ for even $k$, and $O(n^{\frac{k+1}{2}})$ for odd
$k$. Erickson~\cite{Er99a} proved a near-matching lower bound in the $k$-linear
decision tree model. In this model, the complexity is measured by the depth of
a decision tree, every node of which corresponds to a query of the form
$q_{i_1} + q_{i_2} + \cdots + q_{i_k} \ask{\le} 0$, where $q_1, q_2, \ldots, q_n$ are the
input numbers. In a recent breakthrough paper, Gr\o nlund and
Pettie~\cite{GP18} showed that in the $(2k-2)$-linear decision tree model,
where queries test the sign of weighted sums of up to $2k-2$ input numbers, only
$O(n^\frac{k}{2}\sqrt{\log n})$ queries are required for odd values of $k$. In
particular, there exists a $4$-linear decision tree for 3SUM of depth
$\tilde{O}(n^\frac{3}{2})$, while every 3-linear decision tree has depth $\Omega
(n^2)$~\cite{Er99a}. This indicates that increasing the size of the queries,
defined as the maximum number of input numbers involved in a query, can yield
significant improvements on the depth of the minimal-height decision tree.

Ailon and
Chazelle~\cite{AC05} slightly extended the range of query sizes for which a
nontrivial lower bound could be established, elaborating on Erickson's
technique.

Ailon and Chazelle~\cite{AC05} study \(s\)-linear decision trees to solve the \(k\)-SUM problem when
\(s > k\). In particular, they give an additional proof for the
$\Omega(n^{\ceil{\frac{k}{2}}})$ lower bound of Erickson~\cite{Er99a} and
generalize the lower bound for the \(s\)-linear decision tree model when \(s >
k\). Note that the exact lower bound given by Erickson~\cite{Er99a} for \(s = k\) is
$\Omega({(nk^{-k})}^{\ceil{\frac{k}{2}}})$ while the one given by
Ailon and Chazelle~\cite{AC05} is $\Omega({(nk^{-3})}^{\ceil{\frac{k}{2}}})$. Their result
improves therefore the lower bound for \(s = k\) when \(k\) is large.
The lower bound they prove for \(s > k\) is the following
\begin{theorem}[Ailon and Chazelle~\cite{AC05}]
The depth of an $s$-linear decision tree solving the \(k\)-LDT\ problem is
$$
\Omega\mleft(\group{nk^{-3}}^{\frac{2k-s}{2\lceil\frac{s-k+1}{2}\rceil}
(1-\epsilon_k) }\mright),
$$
where \(\epsilon_k > 0\) tends to \(0\) as \(k \to\infty\).
\end{theorem}
This lower bound breaks down when
\(k = \Omega(n^{\sfrac{1}{3}})\) or \(s \ge 2 k\) and the cases where \(k < 6\)
give trivial lower bounds. For example, in the case
of 3SUM with \(s = 4\) we only get an $\Omega(n)$ lower bound.

It has been well established that there exist nonuniform
polynomial-time algorithms for the subset-sum problem. One of them was
described by Meiser~\cite{M93}, and is derived from a data structure for point
location in arrangements of hyperplanes using the bottom vertex decomposition.
This algorithm can be cast as the construction of a linear decision tree in which
the queries have non-constant size.

Meyer auf der Heide~\cite{M84} gave the first point location algorithm to solve the knapsack
problem in the linear decision tree model in polynomial time. He thereby
answers a question raised by Dobkin and Lipton~\cite{DL74,DL78}, Yao~\cite{Y82}
and others. However, if one uses this algorithm to locate a point in an
arbitrary arrangement of hyperplanes the running time is increased by a factor
linear in the greatest coefficient in the equations of all hyperplanes.
On the other hand, the complexity of Meiser's point location algorithm is
polynomial in the dimension, logarithmic in the number of hyperplanes and
does not depend on the value of the coefficients in the equations of the
hyperplanes. A useful complete description of the latter is also given by
Bürgisser et al.~\cite{B97} (Section 3.4).

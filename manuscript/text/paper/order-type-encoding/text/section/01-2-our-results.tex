\subsection*{\iftitlecase%
Our Results\else%
Our results\fi}\label{sec:results}

In this contribution, we are interested in \emph{compact} encodings for
order types: we wish to design data structures using as few bits as possible
that can be used to quickly answer orientation queries of a given abstract or
realizable order type.
\ifeurocg%
\begin{definition}[label=def:encoding,restate=DefinitionEncoding]
  For fixed \(k\) and given a function \(f : {[n]}^k \to [O(1)]\), we define
  a \((S(n),Q(n))\)-encoding of \(f\) to be a string of \(S(n)\) bits such
  that, given this string and any \(i \in {[n]}^k\), we can compute \(f(i)\)
  in \(Q(n)\) time in the word-RAM model with word size \(w \geq \log
  n\).
\end{definition}

\fi%
%
\ifjournal%
	In Section~\ref{sec:preliminaries} we give the tools necessary to
	attain our main result. This section can be skipped entirely if deemed
	unnecessary.
\fi%
\ifeurocg%
We \else%
In Section~\ref{sec:lines-and-pseudolines}, we \fi%
give the first optimal encoding for abstract
order types that allows efficient query of the orientation of any triple: the
encoding is a data structure that uses \( O(n^2) \) bits of space with queries
taking \(O(\log n)\) time in the word-RAM model with word size \(w \geq \log
n\).
\ifeurocg%
\begin{theorem}\label{thm:abstract}
  All abstract order types have an \((O(n^2), O(\log n))\)-encoding.
\end{theorem}

\fi%
%
Our encoding is far from being space-optimal for realizable order types.
We show that its construction can be easily tuned to only require \(O(n^2
{(\log{\log{n}})}^2 / \log{n})\) bits in this case.
\ifeurocg%
\begin{contribution}[label=thm:realizable,restate=TheoremGPTRealizable]
  All realizable order types have
  a
  \(O(\frac{n^2 {(\log \log n)}^2}{\log n})\)-bits
  encoding
  allowing for queries in \(O(\log n)\) time.
  %All realizable order types have a
  %\(O(\frac{n^2 {(\log \log n)}^2}{\log n})\)-bits
  %\(O(\log n)\)-querytime encoding.
\end{contribution}

\fi%
%
\ifeurocg%
We \else%
In Section~\ref{sec:query-time}, we \fi%
further refine our encoding to
reduce the query time to \(O(\log{n}/\log{\log{n}})\).
\ifeurocg%
\begin{contribution}[label=thm:abstract-loglog,restate=TheoremGPTAbstractLogLog]
  All abstract order types have an encoding
  using \(O(n^2)\) bits of space
  and allowing for queries in \(O(\frac{\log n}{\log \log n})\) time.
  %(\(O(n^2)\), \(O(\frac{\log n}{\log \log n})\))-encoding.
  %\(O(n^2)\)-bits
  %\(O(\frac{\log{n}}{\log{\log{n}}}))\)-querytime
  %encoding.
\end{contribution}

\begin{contribution}[
  name={
    \(o(n^2)\)-bits
    \(o(\log n)\)-querytime
    encodings for realizable OT
  },%
  label=thm:realizable-loglog,%
  restate=TheoremGPTRealizableLogLog%
]
  All realizable order types have
  a \(O(\frac{n^2 \log^\epsilon n}{\log n})\)-bits encoding
  allowing for queries in \(O(\frac{\log n}{\log \log n})\) time.
  %All realizable order types have a
  %\(O(\frac{n^2\log^\epsilon n}{\log n})\)-bits
  %\(O(\frac{\log{n}}{\log{\log{n}}}))\)-querytime encoding.
\end{contribution}

\fi%
%
In the realizable case, we give quadratic upper bounds on the
preprocessing time required to compute an encoding in the real-RAM model.
\ifeurocg%
\begin{theorem}\label{thm:preprocessing-loglog}
  In the real-RAM model and the constant-degree algebraic decision tree model,
  given \(n\) real-coordinate input points in \(\mathbb{R}^2\) we can compute
  the encoding of their order type as in
  Theorems~\ref{thm:abstract-loglog}~and~\ref{thm:realizable-loglog} in
  \(O(n^2)\) time.
\end{theorem}

\fi%
%
\ifeurocg%
We \else\ifsocg%
In the full version of the paper, we \else%
In \appref~\ref{sec:hyperplanes} we \fi\fi%
generalize our encodings to chirotopes of
point sets in higher dimension\ifsocg~\cite{CCILO18}\fi.
\ifeurocg%
\begin{contribution}[
  name={
    \(o(n^{k-1})\)-bits
    \(o(\log n)\)-querytime
    encodings for realizable chirotopes of rank \(k\)
  },%
  label=thm:realizable-d,%
  restate=TheoremGPTRealizableD%
]
  All realizable chirotopes of rank \(k \geq 4\) have
  an encoding using
  \(O(\frac{n^{k-1}{(\log{\log{n}})}^2}{\log n})\) bits of space
  and allowing for queries in
  \(O(\frac{\log{n}}{\log{\log{n}}})\) time.
\end{contribution}

\fi%
\ifeurocg%
\begin{contribution}[label=thm:preprocessing-d,restate=TheoremGPTPreprocessingD]
  In the real-RAM model and the constant-degree algebraic decision tree model,
  given \(n\) real-coordinate input points in \(\mathbb{R}^d\) we can compute
  the encoding of their chirotope as in Theorem~\ref{thm:realizable-d} in \(
  O(n^{d}) \) time.
\end{contribution}

\fi%

\subsection*{\iftitlecase%
A Remark\else%
A remark\fi}\label{sec:a-remark}

Our data structure is the first subquadratic encoding for realizable order
types that allows efficient query of the orientation of any triple. It is not
known whether a subquadratic algebraic computation tree exists for
identifying the order type of a given point set.
Any such computation tree would yield another subquadratic encoding for
realizable order types. We see the design of compact encodings for realizable
order types as a subgoal towards subquadratic nonuniform algorithms for this
related problem, a major open problem in Computational Geometry. Note that
pushing the preprocessing time below quadratic would yield such an algorithm.

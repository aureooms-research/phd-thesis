\paragraph*{Encoding}
Given \(n\) pseudolines in the plane and some fixed
parameters \(r\) and \(\ell\), compute a \(\ell\)-levels hierarchical cutting
of parameter \(r\) for those
pseudolines as in Lemma~\ref{lem:hierarchical-cutting-2}.
This hierarchical cutting consists of \( \ell \) levels
labeled \(0,1,\ldots,\ell-1\). Level \(i\) has \(O(r^{2i})\)
cells. Each of those cells is further partitioned into
\(O(r^2)\) subcells. The \(O(r^{2(i+1)})\) subcells of level \(i\)
are the cells of level \(i+1\). Each cell of level \(i\) is intersected by at
most \(\frac{n}{r^i}\) pseudolines, and hence each subcell is intersected by at
most \(\frac{n}{r^{i+1}}\) pseudolines.

We compute and store a combinatorial representation of the hierarchical cutting
as follows: For each level of the hierarchy, for each cell in that level, for
each pseudoline intersecting that cell, for each subcell of that cell, we store
two bits to indicate the location of the pseudoline with respect to that
subcell\ifeurocg\else, that is, whether the pseudoline lies above
(\texttt{00}), lies below (\texttt{01}), intersects the interior of that
subcell (\texttt{10}), or contains the subcell (\texttt{11})\fi.
When a pseudoline intersects the interior of a 2-dimensional subcell, we also
store the two indices of that pseudoline
in the cyclic permutation of that subcell, beginning at an
arbitrary location in, say, clockwise order.
\ifeurocg%
Location in a non-full-dimensional subcell can be encoded similarily.
\else%
If the intersected subcell is
1-dimensional instead, we store the index of the pseudoline in the acyclic
permutation of that subcell, beginning at an arbitrary endpoint.
If two pseudolines intersect in the interior of a 1-dimensional subcell or on
the boundary of a 2-dimensional subcell, they share the same index in the
permutation of that subcell.
\fi%

\ifeurocg\else
This representation takes \(O(\frac{n}{r^i} + \frac{n}{r^{i+1}}
\log{\frac{n}{r^{i+1}}})\) bits per subcell of level \(i\) by storing for each
pseudoline its location and, when needed, the permutation indices of its
intersections with the subcell. At the last level of the hierarchy, let \(t =
\frac{n}{r^\ell}\) denote an upper bound on the number of pseudolines
intersecting each subcell. For each of those \(O(r^{2 \ell}) =
O(\frac{n^2}{t^2})\) subcells we store a pointer to a lookup table of size
\(O(t^3)\) that allows to answer the query of the orientation of any
triple of pseudolines intersecting that subcell.
\fi

\ifeurocg
The hierarchy is such that each subcell of the last level is intersected by no
more than \(t = \frac{n}{r^\ell}\) pseudolines.
For those subcells, we answer the queries by table lookup.
The use of hierarchical cuttings essentially guarantees we get quadratic
preprocessing time, quadratic space, and logarithmic query time in the abstract
case. In the realizable case, we know there can only be \(2^{O(t \log t)}\)
distinct lookup tables. Choosing the right superconstant \(t\) leads to
subquadratic space in that case.
\fi

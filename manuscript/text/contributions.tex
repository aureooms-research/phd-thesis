\section{Efficient Application of Meiser's Algorithm to \(k\)-SUM}

\section{Generalization of Gr\o nlund and Pettie's Results}

We study two different generalizations of 3SUM\@. In the first generalization,
which we call the 3POL problem, we replace the sum function by a trivariate
polynomial of constant degree:
\begin{problem}[3POL]
Let $F \in \mathbb{R}[x,y,z]$ be a trivariate polynomial of constant degree,
given three sets $A$, $B$, and $C$, each containing $n$ real numbers, decide
whether there exist $a \in A$, $b \in B$, and $c \in C$ such that
$F(a,b,c)=0$.
\end{problem}
The second generalization is a special case of the 3POL problem where we
restrict the trivariate polynomial $F$ to have the form $F(a,b,c) = f(a,b) -
c$. We call it the explicit 3POL problem because the dependency on the third
variable is explicitly given:
\begin{problem}[explicit 3POL]
Let $f \in \mathbb{R}[x,y]$ be a bivariate polynomial of constant degree,
given three sets $A$, $B$, and $C$, each containing $n$ real numbers, decide
whether there exist $a \in A$, $b \in B$, and $c \in C$ such that $c=f(a,b)$.
\end{problem}
We will first design algorithms for this easier problem. The techniques used can
then be adapted to work for the more general 3POL problem. We design both
uniform and nonuniform algorithms.
%
In \S\ref{sec:algo:explicit:nonuniform},
we give a $O(n^{12/7+\varepsilon})$-depth bounded-degree
algebraic decision tree for explicit 3POL, and in
\S\ref{sec:algo:explicit:uniform}, we adapt this decision tree
to run in $O(n^2 {(\log \log n)}^{3/2} / {(\log n)}^{1/2})$-time
in the real-RAM model.
%
In \S\ref{sec:algo:implicit:nonuniform}, we generalize the decision tree from
\S\ref{sec:algo:explicit:nonuniform} to work for 3POL with the same depth, up
to constant factors.
%
Finally, in \S\ref{sec:algo:implicit:uniform}, we give a real-RAM
implementation of this second decision tree to solve 3POL as fast as
explicit 3POL, up to constant factors.

\section{Subquadratic Encodings for GPT}

\section{Strongly Subquadratic Encodings for 3SUM}
